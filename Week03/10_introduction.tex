\section{Introduction}

\subsection{Previous Week}
\begin{frame}{Last Week's Review}
  \begin{itemize}
  \item \structure{Linear Structures (Arrays, Vectors)}
    \begin{itemize}
    \item Simple, but effective, often used in programming challenges;
    \item Learn array functions in the library: sorting, binary search, etc;
    \end{itemize}

    \medskip

  \item \structure{Tree Structures (Map, Set)}
    \begin{itemize}
    \item Are fast for querying data;
    \item Hard to implement by hand, but available on the library too;
    \end{itemize}

    \medskip

  \item \structure{Union-Find Disjoint Set (UFDS)}
    \begin{itemize}
    \item Efficiently assign data items into different groups;
    \item Implementation and algorithm is very simple;
    \end{itemize}

    \medskip

  \item \structure{Segment Trees}
    \begin{itemize}
    \item Good for Max/Min Range Query in dynamic data
    \item Hard to implement!
    \end{itemize}
  \end{itemize}
\end{frame}


\begin{frame}[fragile]
  \frametitle{Java Speed Hints}

  \begin{itemize}
      \item Don't add strings in a loop:
    {\small
\begin{verbatim}
String a,b;
for (int i = 0; i < N; i++) { a = a + b; } // SLOW!
\end{verbatim}
    }
  \item Use \emph{StringBuilder} instead:
    {\small
\begin{verbatim}
StringBuilder sb; String a,b;
for (int i = 0; i < N; i++) { sb.append(b); }
a = sb.toString();
\end{verbatim}
    }

  \item Java \emph{Arraylist}'s {\bf contain()} is O(n)\footnote{\url{http://stackoverflow.com/questions/10196343/hash-set-and-array-list-performances}}
    {\small
\begin{verbatim}
ArrayList<Int> a;
for (int i = 0; i < N; i++) { a.contain(b); }
\end{verbatim}
    }
  \item Use \emph{HashMap} instead (O(1)):
    {\small
\begin{verbatim}
HashMap<Int> a;
for (int i = 0; i < N; i++) { a.contain(b); }
\end{verbatim}
    }
  \end{itemize}
\end{frame}

\begin{frame}
  \frametitle{Python Speed Hints}
  \begin{block}{}
    Sometimes an algorithm that is accepted with C++ or Java receives "time limit exceeded" when implemented in Python (example: CD).
  \end{block}
  \bigskip

  \begin{itemize}
  \item Python can be 10 times slower than C++ on the worst case! So you need to make sure your algorithm is well implemented, and prune as much as possible.\bigskip

  \item Read the problem carefully, and discover if it is possible to optimize the program;\bigskip

  \item For example, in the problem CD the input is ordered. It is possible to use this information to prune the search for repeated number greatly! 
  \end{itemize}
\end{frame}


\begin{frame}{Outline for this week}
  This week, we study "Search-based" approaches to solving programming challenges:
  \bigskip

  \begin{itemize}
    \item Complete Search (Brute Force)\bigskip

    \item Divide and Conquer\bigskip

    \item Greedy Search\bigskip

    \item Dynamic Programming (next week!)
  \end{itemize}
\end{frame}
