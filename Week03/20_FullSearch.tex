\section{Complete Search}

\begin{frame}{}{}
  \begin{center}
    {\bf Part II: Complete Search}
  \end{center}
\end{frame}

% Full search - Talk about the problems listed, then "calculatingDartScores" (recursive)
% - remove hints



\begin{frame}{Idea of Complete Search}

  In the Complete Search algorithm, we loop through all possibilities, until
  we find the correct answer.\bigskip

  Some people call this algorithm {\bf Brute Force}. This name is negative,
  but sometimes a Complete Search is very useful.\bigskip

  \begin{block}{How to think of a complete search algorithm}
    \begin{itemize}
      \item Find the loop that checks all possibilities; (or a recursive function)
      \item Write the program to check if a possibility is correct or not;
      \item {\bf PRUNING}: Remove from the loop cases that are impossible;
      \begin{itemize}
        \item Use "break", "continue" or good loop limits.
      \end{itemize}
    \end{itemize}
  \end{block}
\end{frame}

\subsection{Example 1}
\begin{frame}{Complete Search Example: Division}

  \begin{block}{Problem Summary}
    Input: One integer $N$.\\
    Output: All pairs of numbers with 5 digits ($abcde$ and $fghij$) that satisfy:
    \begin{enumerate}
      \item $fghij / abcde = N$
      \item $a,b,c,d,e,f,g,h,i,j$ are all different.
    \end{enumerate}
  \end{block}\bigskip

  {\bf Example:} $N = 62$
  \medskip

  79546 / 01283 = 62\\
  94736 / 01528 = 62\\
  \bigskip

  \begin{block}{}
    Think: How can you solve this problem using {\bf Complete Search}?
  \end{block}
\end{frame}

\begin{frame}[fragile]{Complete Search Example: Division}{Naive Solution}
  \begin{block}{}
    Let abcde = $x$, fghij = $y$. Loop all $x$ from 0 to 99999, calculate $y = x*n$.\\
    Test if all digits of $x$ and $y$ are different.
  \end{block}

{\smaller
\begin{verbatim}
for (int x = 0; x <= 99999; x++) {
  y = x*n;
  digits = count_digits(x,y);
  if (digits == 1<<10 - 1) printf("%0.5d/%0.5d=%d\n",y,x,N);
}

int count_digits(int x, int y) {
  int used = (x < 10000); % bit array: each bit mark one digit
  int tmp;
  tmp = x; while (tmp) {used |= 1 << (tmp%10); tmp /= 10; }
  tmp = y; while (tmp) {used |= 1 << (tmp%10); tmp /= 10; }
  return used;
}
\end{verbatim}
  }
\end{frame}

\begin{frame}{Complete Search Example: Division}{Prunning the Loop}
  The naive algorithm tests many numbers that {\bf will never be the answer}. (eg: x = 00021).\bigskip

  We can {\bf prune} the loop, to test a smaller number of cases.\bigskip

  \begin{itemize}
  \item Prune 1: Minimum and Maximum values of $x$:\\
    \hspace{1cm} $x_{min}: 01234$, $x_{max}: 98765$.\hfill Other values have repeated digits!
    \bigskip

  \item Prune 2: Consider $y = x * N$. The maximum of $y$ is also 98765, so maximum of $x$ is:\\
    \hspace{1cm} $x_{max}: 98765 / N$
    \bigskip

  \item Can you think of other ways to prune?
  \end{itemize}
\end{frame}

\subsection{Reflection}
\begin{frame}
  \frametitle{Notes about Complete Search algorithms}
  \begin{itemize}
  \item A Complete Search should always give the correct solution\\
    \begin{itemize}
    \item It checks all cases, so it should always find the correct one;
    \end{itemize}

    \bigskip

  \item For some problems, the Complete Search is the right algorithm.
    \begin{itemize}
    \item If the problem is small, use Complete search, no need for complex algorithm;
    \item Prune, prune, prune to make the problem small;
    \end{itemize}

    \bigskip

  \item For a very hard problem, a Complete Search can give you a starting point;
    \begin{itemize}
    \item Even if Complete Search is "Time Limited Exceeded",\\
    you can use it to check test cases;
    \end{itemize}
  \end{itemize}
\end{frame}

\subsection{Example 2}
\begin{frame}
  \frametitle{Complete Search Example 2: Simple Equations}
  \begin{block}{Problem Summary}
    {\bf Input:} A, B, C, $1 \leq A,B,C \leq 10000$.\bigskip

    Find $x,y,z$ so that:
    \begin{itemize}
    \item $x+y+z=A$,
    \item $x*y*z=B$,
    \item $x^2+y^2+z^2=C$,
    \end{itemize}

    \bigskip
  \end{block}
  \bigskip

  To solve this problem we can loop and test on of x, y, z (3-nested loop).
  \bigskip

  But what should be the minimum and maximum value of the loops?
\end{frame}

% First take x^2+y^2+z^2 = C. Since maximum C is 10000, and X,Y,Z must be
% different, the maximum range of x is -100 to 100. The Reasoning goes for
% Y and Z. With this we can do a triple loop with about 8M operations.

\begin{frame}[fragile]{Complete Search Example 2: Simple Equations}{Initial Pruning}
  \begin{block}{}
    Consider $x^2 + y^2 + z^2 = C$.\medskip

    Since $C \leq 10000$, and $x^2,y^2,z^2 \geq 0$, if $y =
    z = 0$ then the range for $x$ must be $-100, 100$.
  \end{block}

{\smaller
\begin{verbatim}
int x,y,z;
for (x = -100; x <= 100 && !sol; x++)
  for (y = -100; y <= 100 && !sol; y++)
    for (z = -100; z <= 100 && !sol; z++)
      if (y != x && z != x && z != y &&
          x + y + z == A && x * y * z == B &&
          x*x + y*y + z*z == C) {
             printf("%d %d %d\n", x,y,z);
             exit(0);
          }
\end{verbatim}}

\begin{block}{}
  QUESTION: How can we prune this loop even more?
\end{block}
\end{frame}

\begin{frame}{Complete Search Example 2: Simple Equations}{More Pruning}
  There are many other ways that we can prune the loop:

  \medskip

  \begin{itemize}
  \item We can change the range using the actual input values of $A,B,C$
  \item We only need one solution. We can break the loop once we find it.
  \item We can use equation 2 to think of other ways to prune.
  \end{itemize}

  \vfill

  % \begin{alertblock}{}
  %   The problem: ``Simple Equations -- Extreme!'' has a much
  %   higher range for $A,B,C$. You need a lot of pruning to avoid a TLE!
  % \end{alertblock}
\end{frame}

\subsection{Example 3}
\begin{frame}{Complete Search Example 3: Calculating Dart Scores}
  \begin{block}{Problem Summary}
    \begin{itemize}
      \item Given a score $S$, $1\leq S \leq 180$, find three dart scores that add to $S$.
      \item One dart score can be $1$ to $20$, "double" and "triple"
    \end{itemize}
  \end{block}
  \vfill

  \begin{itemize}
    \item For this program, maybe it is easier to use recursion than loop.
    \item Function: "findscore(S, darts)". Print the score when you return.
    \item How to prune this search?
  \end{itemize}
\end{frame}
