\section{Introduction}

\begin{frame}{Lecture Outline}
\end{frame}

\subsection{Grade Discussion}

\begin{frame}
  \frametitle{Grade Dates}

  \begin{itemize}
  \item Final Submission Date: 7/10
    \bigskip

  \item Grade Announcement: 7/14
    \bigskip

  \item Grade Registration: 7/20
  \end{itemize}
\end{frame}

\subsection{Course Summary}

\subsection{Solving Problems}
\begin{frame}
  \frametitle{Course Summary -- Solving a problem}

  \begin{block}{}
    In this course, we studied and practice many ways
    to solve problems using computer algorithms. Many
    problems can be imagined as \emph{searches}.
  \end{block}

  \vfill

  General Problem Solving:
  \begin{itemize}
  \item Identify the \structure{full search approach}
  \item Think about edge and special cases
  \item See if a better algorithm \structure{is needed}
  \end{itemize}

\end{frame}

\begin{frame}
  \frametitle{Course Summary -- Topics Approached}

  \begin{exampleblock}{}
    In this course we also saw many examples of \structure{specific}
    algorithms for problems.
  \end{exampleblock}

  \vfill

  \begin{itemize}
  \item Graphs (Minimum spanning tree, Bellman-Ford APSP, $\ldots$)
  \item Mathematics (Eristhenes Sieve, Prime Factoring)
  \item Computational Geometry (Convex Hull)
  \item String (Knuth-Morris-Prat, suffix trie)
  \end{itemize}

  % Course lessons and representative algorithms
\end{frame}

\begin{frame}{Today's Lecture: Multi-Problems}

  The most interesting problems are those that \structure{mix two or
    more} different algorithms. Or \structure{require variations} of
  standard algorithms.

  \bigskip

  This week, we will try to solve together some of these more
  interesting problems.
\end{frame}
