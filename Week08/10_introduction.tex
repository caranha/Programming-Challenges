\section{Introduction}

\begin{frame}{New Deadlines (More time to solve problems)}
  The deadlines for the final three classes, and the extra deadline for late exercises is the following:\bigskip

  \begin{itemize}
    \item Lecture 8: Lecture: 6/16, Deadline: 6/25 (10 days)
    \item Lecture 9: Lecture: 6/23, Deadline: 7/02 (10 days)
    \item Lecture 10: Lecture: 6/30, Deadline: 7/09 (10 days)
    \item Late Submission Time: 7/10 to 7/21 (11 days)
  \end{itemize}\bigskip

  If you have too many reports right now, please use the Late Submission Time!  
\end{frame}

\subsection{Outline}
\begin{frame}{Math Problems: Lecture Outline}

  Every computer program requires some amount of mathematics. So what does {\bf "Math Problems"} mean in Programming Challenges?\bigskip

  Here we describe two kinds of problems as {\bf "Math Problems"}:

  \begin{block}{The Challenge is The Implementation of Mathematical Concepts}
    \begin{itemize}
      \item Problems with Big Numbers (above variable limits)
      \item Problems with Geometry (next lecture!)
    \end{itemize}
  \end{block}
  \begin{exampleblock}{The Challenge Requires Mathematical Planning Before Programming}
    In this case, it is sometimes possible to solve the entire problem in paper and quickly implement a solution to the problem.
    \begin{itemize}
      \item Number Theory (primality testing, factorization, rings)
      \item Combinatorics (sequences, counting, recurrences)
    \end{itemize}
  \end{exampleblock}
\end{frame}
