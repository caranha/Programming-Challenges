
\section{Primality}
\begin{frame}
  \begin{center}
  {\bf Part I -- Primality}
  \end{center}
\end{frame}


\subsection{Primality Testing}
\begin{frame}
  \frametitle{Primality Testing}

    {\bf Question:} How do you write a (simple) program to test if $N$ is prime?\bigskip

    \begin{itemize}

      \item Complete Search: For each $d \in 2..N-1$, test if $N\%d == 0$.
      \begin{itemize}
        \item This requires $O(N)$ divisions.\medskip
      \end{itemize}

      \item Pruning (reducing) the Complete Search:
      \begin{itemize}
        \item Search only from 2 to $\sqrt{N}$: $O(\sqrt{N})$
        \item Search only 2, and {\bf odd} numbers from 3 to $\sqrt{N}$: $O(\frac{\sqrt{N}}{2})$
        \item Search only {\bf prime} numbers from 2 to $\sqrt{N}$:
        $O(\frac{\sqrt{N}}{\ln(\sqrt{N})})$
      \end{itemize}\medskip

      \item How do we quickly calculate a set of small prime numbers?
    \end{itemize}
\end{frame}

\begin{frame}{Primality Testing: Finding {\bf Sets} of primes}

  \begin{block}{The Prime Number Theorem (simplified)}
    There are approximately $\frac{N}{\log{N}-1}$ prime numbers between 1 and $N$
  \end{block}\bigskip

  \begin{itemize}
    \item Number of prime numbers between 1 and $\sqrt{10^6}$ = 168
    \item Number of prime numbers between 1 and $\sqrt{10^{10}} \approx 9500$
  \end{itemize}\bigskip

  With a list of small prime numbers, we can test the primality of large numbers quickly.\bigskip


  A simple algorithm to find a list of primes is {\bf Sieve of Eratosthenes}.
\end{frame}



\begin{frame}[fragile]{Sieve of Eratosthenes}

    \begin{enumerate}
      \item Initialize Vector "sieve" of size $\sqrt{N}$, all TRUE; Loop on Vector.
      \item If sieve[i] is TRUE, add $i$ to prime list;
      \item Set all multiples of $i$, sieve[$i*m$] to FALSE;
    \end{enumerate}

    {\smaller
  \begin{exampleblock}{}
\begin{verbatim}
def sieve(k):                 ## Find all primes up to k
   primes = []                ## List of primes found
   sieve = [1]*(k+1)          ## set all elements of "sieve" to true;
   sieve[0] = sieve[1] = 0    ## 0,1 trivially not primes
   for i in range(k+1):       ## Loops on the sieve;
      if (sieve[i] == 1):     ## Found a new prime
         primes.append(i)     ## Add to prime list
         j = i*i              ## Remove multiples of i (Quiz: Why not i*2?)
         while (j < k+1):     ## Costs O(loglogN)
            sieve[j] = 0      ## Remove multiples from sieve
            j += i
   return primes              ## list of primes
\end{verbatim}
  \end{exampleblock}
  }
\end{frame}

\begin{frame}
  \frametitle{Sieve of Eratosthenes: Computation Cost}

    \begin{itemize}
      \item The cost of calculating the Sieve for $k$ is $O(k\log\log k)$
      \item The cost of full search for $N$ is $O(\sqrt{N}/2)$
      \item Why use sieve and not the full search?
    \end{itemize}

    \begin{block}{Amortized Complexity}
      Do a complex calculation once, use result many times:
      \begin{itemize}
      \item If we are only testing {\bf ONE PRIME}, the full search is better.
      \item But, if the problem requires many primes to be tested, the sieve is better.
        \begin{itemize}
        \item If $N$ < $k$, checking the sieve table costs $O(1)$.
        \item We can pre-calculate the sieve table when initalizing the program;
        \end{itemize}
      \end{itemize}
    \end{block}\bigskip

    When do we need to calculate multiple primes? Prime factorization!
\end{frame}

\subsection{Prime Factorization}
\begin{frame}
  \frametitle{Prime Factorization}

  Every natural number $N$ can be written as a {\bf unique multiplication of primes}\footnote{Fundamental Theorem of Arithmetics}. Example:

  \begin{equation*}
    1200 = 2\times2\times2\times2\times3\times5\times5 = 2^4\times3\times5^2
  \end{equation*}

  In other words, for $N$, the prime number factorization of $N$ is:
    \begin{equation*}
      N=p_1^{e_1}p_2^{e_2}\ldots p_n^{e_n}, p_i \text{ is prime}
    \end{equation*}

  (Prime) Factorization is a key issue in Cryptography, so fast factorization is an important research problem. For programming challenges, we use two simple approaches:\bigskip

  \begin{itemize}
    \item {\bf Full search}: create a list of primes (with sieve) and test if each of them divides $N$.
    \item {\bf Divide and Conquer:} Find the smallest prime $p_i$ from sieve that divides $N$. Replace $N$ with $N|p_i$. Repeat until $p_i > \sqrt{N}$.
  \end{itemize}
\end{frame}

\begin{frame}[fragile]
  \frametitle{Prime factorization: Divide and conquer approach}

  This algorithm is reasonably fast if $N$ is composed of several small prime factors.

  {\smaller
  \begin{exampleblock}{}
\begin{verbatim}
vector<int> primeFactors(ll N) {
  vector<int> factors;
  long PF_idx = 0, PF = sieve[PF_idx];   // sieve is a precomputed prime list
  while (PF * PF <= N) {                 // remember, N gets smaller;
    while (N % PF == 0) {                // Remove PF^x from N
      N /= PF;
      factors.push_back(PF);
    }
    PF = primes[PF_idx++];               // only consider primes!
  }
  if (N != 1) factors.push_back(N);      // special case: N is prime
  return factors;
}
\end{verbatim}
  \end{exampleblock}}
\end{frame}

\begin{frame}{Full Factorization}
  In some cases, we want to know {\bf all} numbers that divide a certain number $N$.\bigskip

  We can calculate the full factorization of $N$ from its prime factorization.\\
  In fact, the full factorization of $N$ is the set of all unique combinations of prime factors.\bigskip

  Example:
  \begin{itemize}
    \item $1200 = 2\times2\times2\times2\times3\times5\times5 = 2^4\times3\times5^2$
    \item Number of factors of 1200: $5(2^4)\times2(3^1)\times3(5^2) = 30$
    \begin{itemize}
      \item $2^0 \times 3^0 \times 5^1 = 5$,
      \item $2^0 \times 3^0 \times 5^2 = 25$,
      \item $2^0 \times 3^1 \times 5^0 = 3$,
      \item $2^0 \times 3^1 \times 5^1 = 15$,
      \item $2^0 \times 3^1 \times 5^2 = 75$,
      \item $\ldots$
    \end{itemize}
  \end{itemize}
\end{frame}

\begin{frame}
  \frametitle{Factorization Problem Example: 10139 -- Factovisors}

    \begin{block}{Problem summary}
      Check if $m$ divides $n!$ ($1 \leq m,n \leq 2^{31}-1$)
    \end{block}

    The factorial of $n \leq 2^{31}-1$ is a HUGE number. Fortunately, it is not necessary to calculate this number at all. Consider that:

    \begin{itemize}
    \item $F_m$: primefactors(m)
    \item $F_{n!}$: $\cup$(primefactors(1), primefactors(2) $\ldots$, primefactors(n))
    \end{itemize}

    We can say that $m$ divides $n!$ iff $F_m \subset F_{n!}$.\bigskip

    Examples:
    \begin{itemize}
    \item $m = 48, n=6, n! = 2\times3\times4\times5\times6$\\
      $F_m = \{2,2,2,2,3\}, F_{n!} = \{2,3,2,2,5,2,3\}$

  \medskip

    \item $m = 25, n=6, n! = 2\times3\times4\times5\times6$\\
      $F_m = \{5,5\}, F_{n!} = \{2,3,2,2,5,2,3\}$

    \end{itemize}
\end{frame}
