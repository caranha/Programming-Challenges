
\section{Large Numbers}

\begin{frame}{Math Problems Part I: Large Numbers}

  Some programming challenges, in particular challenges involving combinatoric analysis, require operations on very large numbers.\bigskip

  In this section, we will review some ways to deal with these numbers:
  \begin{itemize}
    \item "BigNum" libraries;
    \item Modulo Operations;
  \end{itemize}
\end{frame}

\subsection{Bignum}
\begin{frame}
  \frametitle{Dealing with Large Numbers}

  In this lecture, we call "Large Numbers" (also sometimes {\bf Bignum}) integers that do not fit in the standard variable types in programming languages (ex: long, long long, unsigned long, etc).\bigskip

  This is very common in problems and algorithms involving factorials. For example: $25! = 15511210043330985984000000 > 10^{26}$.\bigskip

  \begin{block}{BigNum in Different Languages}
    \begin{itemize}
      \item {\bf C++ STL} does not have native support to bignum. You have to program yourself;
      \begin{itemize}
        \item unsigned long long: $2^{64} < 10^20$
      \end{itemize}
      \item {\bf Java} has the "BigInteger" class, which contains several useful operations on large numbers;
      \item {\bf Python} handles BigNums natively, so a special class is not necessary;
    \end{itemize}
  \end{block}
\end{frame}

\begin{frame}[fragile]{Example of Java's "Big Integer" class}
  {10925 -- Krakovia: Calculate sum and division using large integers}

  {\smaller
\begin{block}{}
\begin{verbatim}
import java.util.Scanner;
import java.math.BigInteger;
class Main {
  public static void main(String[] args) {
    Scanner sc = new Scanner(System.in);
    int caseNo = 1;
    while (true) {
      int N = sc.nextInt(), F = sc.nextInt();
      if (N == 0 && F == 0) break;
      BigInteger sum = BigInteger.ZERO;     // Bignum Constant
      for (int i = 0; i < N; i++) {
        BigInteger V = sc.nextBigInteger(); // Bignum I/O
        sum = sum.add(V); }
      System.out.println("Bill #" + (caseNo++)
        + " costs " + sum + ": each friend should pay "
        + sum.divide(BigInteger.valueOf(F)) + "\n" );}
  }
}
\end{verbatim}
  \end{block}}
\end{frame}


\begin{frame}[fragile]
  \frametitle{Useful functions in Java.math.BigInteger}
  Besides dealing with arbitrarily large numbers, the BigInteger class also has some other useful mathematical functions:

{\smaller
  \begin{block}{Algebraic functions}
    BigInteger.add(), .subtract(), .multiply(), .divide(),
    .pow(), .mod(), .remainder()
  \end{block}

  \begin{block}{Changing Number Base}
\begin{verbatim}
BI = BigInteger(10); System.println(BI.toString(2))
// Result: 1010
\end{verbatim}
  \end{block}

  \begin{block}{Probabilistic Primality Test}
\begin{verbatim}
isPrime = BI.isProbablePrime(int certainty)
// Chance of being correct is 1 - (1/2)^certainty
\end{verbatim}
  \end{block}


\begin{block}{Other functions}
  BigInteger.gcd(BI)
  BigInteger.modPow(BI exponent, BI m)
\end{block}}
\end{frame}

\subsection{Modulo Arithmetic}
\begin{frame}
  \frametitle{Modulo Arithmetic}

  Another way to operate in very large numbers is to use {\bf Modulo Arithmetic}.\bigskip

  For some problems, the final result is small (modulo $n$) but the intermediate results are too large. In these cases, we can use modulo arithmetic to avoid storing these large intermediate results.\bigskip

  \begin{block}{Modulo Arithmetic Reminder}
    \begin{enumerate}
    \item $(a+b)\%s = ((a\%s)+(b\%s)+s)\%s$
    \item $(a*b)\%s = ((a\%s)*(b\%s))\%s$
    \item $(a^n)\%s = ((a^{n/2}\%s)*(a^{n/2}\%s)*(a^{n\%2}\%s))\%s$
    \end{enumerate}
  \end{block}
\end{frame}

\subsection{Example Problem}
\begin{frame}{Example Problem: 10176, Ocean Deep! Make it Small}
  \begin{block}{Problem summary}
    Your receive as input a {\bf large binary number} (up to 100 digits). You need to calculate if the number is divisible by 131071 (a prime number).
  \end{block}\bigskip

  \begin{itemize}
  \item Problem: Input and store a large $n$, and calculate $n\%131071$.
  \item Alternatives. Which is easier?
  \begin{itemize}
    \item Use some BigNum data structure to store $n$?
    \item Use modulo arithmetic to calculate the result without bignum?
  \end{itemize}

  \end{itemize}
\end{frame}

% \subsection{Precision}
% TODO: printing with precision in C, Java, Python
% Dealing with very small numbers

\section{Number Theory}
\begin{frame}{Part II: Number Theory}

  Number Theory is the part of mathematics that studies the relationships between {\bf integer numbers}.\bigskip

  It is a large and fascinating field of study, but for the purposes of programming contests, in this lecture we will focus on three topics:\bigskip

  \begin{itemize}
  \item Primality: How to decide if a number is prime;
  \item Division and Remainders: The division relationship between integers;
  \item Sequences: Recurrence relations between sets of numbers;
  \end{itemize}
\end{frame}

\section{Primality}
\subsection{Primality Testing}
\begin{frame}
  \frametitle{Primality Testing}

    {\bf Prime Numbers} are natural numbers ($\geq 0$) that are only divisible by 1 and by themselves: $2,3,5,7,11,13,\ldots$\bigskip

    {\bf Question:} How do you write a (simple) program to test if $N$ is prime?
    \begin{itemize}

      \item Complete Search: For each $d \in 2..N-1$, test if $N\%d == 0$. This requires $O(N)$ divisions.\medskip

      \item Pruning the search:
      \begin{itemize}
        \item Search only numbers between 2 and $\sqrt{N}$: $O(\sqrt{N})$
        \item Search only {\bf odd} numbers between 2,3 and $\sqrt{N}$: $O(\frac{\sqrt{N}}{2})$
        \item Search only {\bf PRIME} numbers between 2 and $\sqrt{N}$:
        $O(\frac{\sqrt{N}}{\ln(\sqrt{N})})$
      \end{itemize}\medskip

      \item Can we calculate these primes easily?
    \end{itemize}
\end{frame}

\begin{frame}{Primality Testing: Finding {\bf Sets} of primes}

  \begin{block}{The Prime Number Theorem (simplified)}
    There are approximately $\frac{N}{\log{N}-1}$ prime numbers between 1 and $N$
  \end{block}\bigskip

  \begin{itemize}
    \item Number of prime numbers between 1 and $\sqrt{10^6}$ = 168
    \item Number of prime numbers between 1 and $\sqrt{10^{10}} \approx 9500$
  \end{itemize}\bigskip

  So if we can find these primes, we can calculate the primality of "programming contest"-sized numbers quickly!\bigskip

  A simple algorithm for finding sets of primes quickly is the {\bf Sieve of Eratosthenes}.
\end{frame}



\begin{frame}[fragile]{Sieve of Eratosthenes}

    \begin{block}{Idea}
      \begin{itemize}
      \item Initialize "Sieve" vector of size $\sqrt{N}$, all TRUE;
      \item Loop on Sieve. If Sieve[i] is TRUE, add $i$ to prime list
      \item Remove {\bf ALL $i\times m$} multiples of $i$ from Sieve;
      \end{itemize}
    \end{block}

    {\smaller
  \begin{exampleblock}{}
\begin{verbatim}
def sieve(k):                 ## Find all primes up to k
   primes = []                ## List of primes found
   sieve = [1]*(k+1)          ## all numbers start in the list
   sieve[0] = sieve[1] = 0    ## 0,1 trivially not primes
   for i in range(k+1):       ## Linear search
      if (sieve[i] == 1):     ## Found a new prime
         primes.append(i)     ## Add to prime list
         j = i*i              ## Optimization. Why not i*2?
         while (j < k+1):     ## Costs O(loglogN)
            sieve[j] = 0      ## Remove multiples from sieve
            j += i
   return primes              ## list of primes
\end{verbatim}
  \end{exampleblock}
  }
\end{frame}

\begin{frame}
  \frametitle{Sieve of Eratosthenes: Computation Cost}

    \begin{itemize}
      \item The cost of calculating the Sieve for $k$ is $O(k\log\log k)$
      \item The cost of full search for $N$ is $O(\sqrt{N}/2)$
      \item Why use sieve and not the full search?
    \end{itemize}

    \begin{block}{Amortized Complexity}
      Do a complex calculation once, use result many times:
      \begin{itemize}
      \item If we are only testing {\bf ONE PRIME}, the full search is better.
      \item But, if the problem requires many primes to be tested, the sieve is better.
        \begin{itemize}
        \item If $N$ < $k$, checking the sieve table costs $O(1)$.
        \item We can pre-calculate the sieve table when initalizing the program;
        \end{itemize}
      \end{itemize}
    \end{block}\bigskip

    When do we need to calculate multiple primes? Prime factorization!
\end{frame}

\subsection{Prime Factorization}
\begin{frame}
  \frametitle{Prime Factorization}

  {\bf Fundamental Theorem of Arithmetic:} Every natural number $N$ can be written as a {\bf unique multiplication of primes}. Example:

  \begin{equation*}
    1200 = 2\times2\times2\times2\times3\times5\times5 = 2^4\times3\times5^2
  \end{equation*}

  In other words, for $N$, the prime number factorization of $N$ is:
    \begin{equation*}
      N=p_1^{e_1}p_2^{e_2}\ldots p_n^{e_n}, p_i \text{ is prime}
    \end{equation*}

  (Prime) Factorization is a key issue in Cryptography, so fast factorization is an important research problem. In this course, we focus on simpler algorithms:\bigskip

  \begin{itemize}
    \item {\bf Full search}: create a list of primes (with sieve) and test if each of them divides $N$.
    \item {\bf Divide and Conquer:} Find the smallest prime $p_i$ from sieve that divides $N$. Replace $N$ with $N|p_i$. Repeat until $p_i > \sqrt{N}$.
  \end{itemize}
\end{frame}

\begin{frame}[fragile]
  \frametitle{Prime factorization: Divide and conquer approach}

  This algorithm is reasonably fast if $N$ is composed of several small prime factors.

  {\smaller
  \begin{exampleblock}{}
\begin{verbatim}
vi primeFactors(ll N) {   // remember: vi is vector<int>
  vi factors;             //           ll is long long
  ll PF_idx = 0, PF = sieve[PF_idx];     // sieve is prime list
  while (PF * PF <= N) {           // remember, N gets smaller;
    while (N % PF == 0) {          // Remove PF^x from N
      N /= PF; factors.push_back(PF);
    }
    PF = primes[PF_idx++];             // only consider primes!
  }
  if (N != 1) factors.push_back(N); // special case: N is prime
  return factors;
}
\end{verbatim}
  \end{exampleblock}}
\end{frame}

\begin{frame}{Full Factorization}
  In some cases, we want to know {\bf all} numbers that divide a certain number $N$.\bigskip

  We can calculate the full factorization of $N$ from its prime factorization. In fact, the full factorization of $N$ is the set of all unique combinations of prime factors.\bigskip

  Example:
  \begin{itemize}
    \item $1200 = 2\times2\times2\times2\times3\times5\times5 = 2^4\times3\times5^2$
    \item Number of factors of 1200: $5\times2\times3 = 30$
    \item Examples:
    \begin{itemize}
      \item $2^3 \times 3^0 \times 5^1 = 40$,
      \item $2^2 \times 3^1 \times 5^2 = 300$,
      \item $2^1 \times 3^1 \times 5^1 = 30$,
      \item $2^4 \times 3^1 \times 5^0 = 48$,
      \item $\ldots$
    \end{itemize}
  \end{itemize}
\end{frame}

\begin{frame}
  \frametitle{Factorization Problem Example: 10139 -- Factovisors}

    \begin{block}{Problem summary}
      Check if $m$ divides $n!$ ($1 \leq m,n \leq 2^{31}-1$)
    \end{block}

    The factorial of $n \leq 2^{31}-1$ is a HUGE number. Fortunately, it is not necessary to calculate this number at all. Consider that:

    \begin{itemize}
    \item $F_m$: primefactors(m)
    \item $F_{n!}$: $\cup$(primefactors(1), primefactors(2) $\ldots$, primefactors(n))
    \end{itemize}

    We can say that $m$ divides $n!$ iff $F_m \subset F_{n!}$.\bigskip

    Examples:
    \begin{itemize}
    \item $m = 48, n=6, n! = 2\times3\times4\times5\times6$\\
      $F_m = \{2,2,2,2,3\}, F_{n!} = \{2,3,2,2,5,2,3\}$

  \medskip

    \item $m = 25, n=6, n! = 2\times3\times4\times5\times6$\\
      $F_m = \{5,5\}, F_{n!} = \{2,3,2,2,5,2,3\}$

    \end{itemize}
\end{frame}


\subsection{Greatest Common Divisor}
\begin{frame}[fragile]
  \frametitle{Extended Euclid Algorithm}

  For integers $a$ and $b$, the {\bf greatest common divisor} GCD(a,b) is the largest integer $d$ so that $d|a$ and $d|b$. Euclid's algorithm can quickly calculate $d$ for a,b ($O(\log_{10}a)$).\bigskip

  The {\bf Extended Euclid's Algorithm}\footnote{Also called "The Pulverizer"}, calculate's $x_0$ and $y_0$ so that $a\times x_0 + b\times y_0 = d$.

{\smaller
    \begin{exampleblock}{}
\begin{verbatim}
int gcdExtended(int a, int b, int *x, int *y)  {
  if (a == 0) { *x = 0; *y = 1; return b; }

  int x1, y1; // To store results of recursive call
  int gcd = gcdExtended(b%a, a, &x1, &y1);

  *x = y1 - (b/a) * x1; *y = x1;      // Update x,y

  return gcd;
}
\end{verbatim}
    \end{exampleblock}
}
\end{frame}

\begin{frame}{Extended GCD and the Diophantine Equation}

  One very useful property of $d =$ GCD$(a,b)$ is that {\bf $d$ divides every integer combination of $a$ and $b$}. In other words: For every $ax+by = c$, if x and y are integers, then $d|c$.\footnote{The proof for this is very cool}.\bigskip

  We can use this property to calculate the integer solutions of the {\bf Diophantine Equation}: $xa+yb = c$\bigskip

  \begin{itemize}
    \item If $d|c$ is not true, there are no integer solutions.
    \item If $d|c$ is true, there are infinite integer solutions:
    \begin{itemize}
      \item The first solution $(x_0, y_0)$ is calculated from the extended GCD.
      \item Other solutions $(x_n,y_n)$ can be derived as: $x_n = x_0 + (b/d)n, y_n = y_0 - (a/d)n$, where $n$ is an integer.
    \end{itemize}
  \end{itemize}


\end{frame}

\begin{frame}{Diophantine Equation Problem Example}
    \begin{block}{Problem Example}
      With 839 yens, you want to buy Chocolate Candy and Vanilla Candy.
      \begin{itemize}
        \item Chocolate Candy costs $x=25$ yens.
        \item Vanilla Candy costs $y=18$ yens.
      \end{itemize}
      How many candies can you buy?
    \end{block}

    \begin{enumerate}
      \item Calculate $d, x_0, y_0$ from extended GCD:
      \begin{itemize}
        \item $d = 1, x_0 = -5, y_0 = 7$ or $25\times(-5) + 18\times(7) = 1$
      \end{itemize}
      \item Is $d|c$? {\bf Yes}. Continue.
      \item Multiply both sides of the equation by $\frac{c}{d}$:
      \begin{itemize}
        \item $25 \times (-4195) + 18\times(5873) = 839$
      \end{itemize}
      \item It is impossible to buy negative candies, so we iterate on $n$ to find
      \begin{itemize}
        \item $x_n = x_0 + (b/d)n$ and $y_n = y_0 - (a/d)n$
      \end{itemize}
      \item At $n = 234$ we find: $25 \times 17 + 18 \times 23 = 839$
    \end{enumerate}
\end{frame}

%%%%%%%%%%%%%%%%%%%%%%%%%%%%%%%%%%%%%%

\section{Combinatorics}
\begin{frame}
  \frametitle{Combinatoric Problems}
    Combinatorics is the area of mathematics that studies {\bf countable discrete structures} (integers, sets, etc).\bigskip

    For our focus on competitive programming, combinatoric problems involves calculating the values of {\bf numeric sequences}. This requires programming the {\bf Recurrent Form} or the {\bf Closed Form} of the sequence.\bigskip


    \begin{itemize}
    \item {\bf Recurrent Form}: The recurrent form of a sequence $F$ calculates $F_n$ based on its antecessor values: $F_{n-1}, F_{n-2},\ldots$.
    \begin{itemize}
      \item Recurrent forms are usually implemented using {\bf Dynamic Programming} and {\bf Memoization};
    \end{itemize}

    \item {\bf Closed Form}: The closed form of a sequence $F$ calculates $F_n$ {\bf without} using the antecessor values of the sequence.
    \end{itemize}
\end{frame}

\subsection{Sequence Examples}
\begin{frame}
  \frametitle{Sequence Example: Triangular Numbers}
  \begin{block}{Definition}
    {\bf Triangular Numbers} is the sequence where $T_n$ is the sum of all inegers from $1$ to $n$. Example:\medskip

    $T_1=1, T_2=1+2=3, \ldots, T_7=1+2+3+4+5+6+7=28$\\
    Trivial, right?
  \end{block}\bigskip

  \begin{itemize}
    \item {\bf Recurrent Form:} $T(n) = T(n-1) + n$
    \begin{itemize}
      \item The recurrent form can be calculated with a loop or recursion;
    \end{itemize}\bigskip
    \item {\bf Closed Form:} $T(n) = \frac{n(n+1)}{2}$
    \begin{itemize}
      \item The closed form can be calculated at once;
      \item It can be used to estimate how fast a sequence grows. In this case, $T_n$ is $O(N^2)$
    \end{itemize}
  \end{itemize}
\end{frame}

\begin{frame}
  \frametitle{A more famous sequence: Fibonacci Numbers}

  \begin{block}{Definition}
    The Fibonacci number $F_n$ is the sum of the two numbers before it.\medskip

    $0, 1, 1, 2, 3, 5, 8, 13, 21, 34, \ldots$
  \end{block}\bigskip

  \begin{itemize}
    \item Recurrent Form:
    \begin{itemize}
      \item Starting Values: $F_0 = 0$, $F_1 = 1$
      \item Recurrence: $F_n = F_{n-1} + F_{n-2}$
    \end{itemize}\bigskip

    \item Be careful when implementing recurrences with multiple terms;
    \begin{itemize}
      \item If using recursive functions, {\bf memoization/DP} is necessary to avoid wasted calculation;
      \item In general, each term in a recurrence requires a starting value;
    \end{itemize}
  \end{itemize}
\end{frame}

\begin{frame}[fragile]
  \frametitle{Bonus: Fibonacci Facts}
  \begin{block}{Closed Form for the Fibonacci Numbers:}
    \begin{equation*}
      F(n) = \frac{1}{\sqrt{5}}\left(\left(\frac{1+\sqrt{5}}{2}\right)^n-\left(\frac{1-\sqrt{5}}{2}\right)^n\right)
    \end{equation*}
    The second term of the closed form tends to 0 when $n$ is large!
  \end{block}

  \begin{block}{Pisano's period}
    The last digits of the Fibonacci sequence repeat with a fixed period!\bigskip
{\smaller
\begin{verbatim}
 Digits        | Period      || Digits        | Period
 last digit    | 60 numbers  || last 3 digits | 1500 numbers
 last 2 digits | 300 numbers || last 4 digits | 15000 numbers

F(6)   =                             8
F(66)  =                27777890035288
F(366) =  1380356 ... 8899086435571688
\end{verbatim}}
  \end{block}
\end{frame}

% What are binomial numbers (equations)
% How do we calculate them (closed form and DP)
% Where do we use them? (Number of Choices)

\begin{frame}{Binomial Coefficient}
  \begin{block}{Definition}
    {\bf Binomial Coefficients} are the set of numbers that correspond to the expansion of a binomial:\bigskip

    \begin{itemize}
      \item $B_3 = (a+b)^3 = 1a^3 + 3a^2b + 3ab^2 + b^3 = \{1,3,3,1\}$
      \item $B_5 = (a+b)^5 = 1a^5 + 5a^4b + 10a^3b^2 + 10a^2b^3 + 5ab^4 + 1b^5 = \{1,5,10,10,5,1\}$
    \end{itemize}\bigskip

    Many times, we are interested in the k-th number of the n-binomial, written as $C(n,k)$ or $^nC_k$. Example: $C(5,2) = 10$.
  \end{block}
\end{frame}

\begin{frame}[fragile]{Binomial Coefficient}{Interpretation and Recurrent Form}
  The common interpretation of $C(n,k)$ is "I have to select A or B $n$ times, how many different ways can I choose A $k$ times?"
  \begin{itemize}
    \item How many binary strings with $n$ digits have $k$ ones?
    \item How many paths exist
  \end{itemize}\bigskip

  Using this definition, we can define the recurrent form of the Binomial:
  \begin{itemize}
    \item I have to choose A $k$ times out of $n$
    \begin{itemize}
      \item If I choose A $k-1$ times out of $n-1$, I choose A again.
      \item If I choose A $k$ times out of $n-1$, I choose B.
    \end{itemize}
    \item $C(n,k) = C(n-1,k-1) + C(n-1,k)$
    \item Don't forget to use DP to implement this!
  \end{itemize}\bigskip
\end{frame}

\begin{frame}[fragile]{Pascal's Triangle}
  The recurrent form of the binomials:
  \begin{equation*}
    C(n,k) = C(n-1,k-1) + C(n-1,k)
  \end{equation*}

  Can also be observed by laying out the numbers:

\begin{verbatim}
1

1 1

1 2 1

1 3 3 1

1 4 6 4 1

1 5 10 10 5 1
\end{verbatim}
\end{frame}

\begin{frame}{Closed Form for the Binomial}
  The closed form for $C(n,k)$ is:\bigskip

    \begin{equation*}
      C(n,k) = \frac{n!}{(n-k)!k!}
    \end{equation*}\bigskip

  Be careful! As you remember, the value of $n!$ can become very big,
  very fast. It might be better to calculate the binomial using the recurrent
  form, to avoid overflow.
\end{frame}

\begin{frame}{The Catalan Numbers}
  \begin{block}{Motivating Problem}
    Given $n$ pairs of parenthesis, how many different balanced expressions can you create?\bigskip

    \begin{itemize}
      \item n = 0: . = 1
      \item n = 1: () = 1
      \item n = 2: ()(), (()) = 2
      \item n = 3: ((())), ()(()), (())(), (()()), ()()() = 5
      \item n = 4: 14
      \item n = 5: 42
    \end{itemize}
  \end{block}\bigskip

  This sequence is known as the {\bf Catalan Numbers}, and it appears in
  several recursive combinatory problems.
\end{frame}

\begin{frame}{The Catalan Numbers}{Recurrent Form}
  The {\bf Recurrent form} of the catalan number can be derived from the parenthesis definition:\bigskip

  \begin{itemize}
    \item If we define $c_k$ as an expression with k parenthesis, we can break it down into: $c_k = (c_a)c_b$, where $k = a + b + 1$. \medskip

    \item Varying the values of $a$ and $b$, and counting all possible variations gives us the recurrent form:\medskip

    \item $c_{n+1} = \sum_{i=0}^n c_ic_{n-i}$
  \end{itemize}
\end{frame}


\begin{frame}{Closed Form and Usage}
  The closed form of the Catalan Numbers is:
  \begin{equation*}
    c_n = \frac{1}{n+1}C(2n,n)
  \end{equation*}
  Be careful of calculating factorials in $C(2n,n)$\bigskip

  \begin{block}{Other uses of Catalan Numbers}
    \begin{itemize}
    \item Number of ways you can triangulate a poligon with $n+2$ sides;
    \item Number of monotonic paths on an $nxn$ grid that do not pass above
      the diagonal.
    \item Number of distinct binary trees with $n$ vertices
    \item Etc...
    \end{itemize}
  \end{block}
\end{frame}

% TODO: Check if this makes sense
% \begin{frame}
%   \frametitle{Integer Partition}
%   \begin{block}{}
%     f(5,5) = (5),(4,1),(3,2),(3,1,1),(2,2,1),(2,1,1,1),(1,1,1,1,1)
%   \end{block}
%   \begin{block}{Definition and calculation}
%     $f(n,k)$ -- number of ways that we can sum $n$, using integers
%     equal or less than $k$.
%
%     \bigskip
%
%     \structure{Recurrence:}
%     \begin{itemize}
%     \item $f(n,k) = f(n-k,k) + f(n, k+1)$
%     \item $f(1,1) = 1$; $f(n,k) = 0$ if $k > n$
%     \end{itemize}
%   \end{block}
% \end{frame}



%%% FIXME: Probability Section
%%% Probably remove for good, there are very few
%%% Probability problems in programming challenges.

% \begin{frame}
%   \frametitle{Ad Hoc Example: Probability problems}
%
%   {\smaller
%     \begin{block}{Dice Throwing}
%       If you have $n$ dice, what is the chance of rolling a total above $m$?
%     \end{block}
%
%     \begin{itemize}
%     \item \structure{Example:} For $n=3$, $m=16$, what is the probability?
%     \end{itemize}
%   }
% \end{frame}
%
% \begin{frame}
%   \frametitle{Ad Hoc Example: Probability problems}
%
%   {\smaller
%     \begin{block}{Dice Throwing}
%       If you have $n$ dice, what is the chance of rolling a total above $m$?
%     \end{block}
%
%     \begin{itemize}
%     \item \structure{Example:} For $n=3$, $m=16$, the chance is $10/216$
%
%       \bigskip
%
%     \item All combinations of 3 dice: $6*6*6 = 216$
%     \item Combinations above 16:
%     \end{itemize}
%
%     \begin{columns}[T]
%       \column{0.3\textwidth}
%       \begin{itemize}
%       \item 6,6,6
%       \item 6,6,5
%       \item 6,5,6
%       \item 5,6,6
%       \end{itemize}
%       \column{0.3\textwidth}
%       \begin{itemize}
%       \item 6,5,5
%       \item 5,6,5
%       \item 5,5,6
%       \end{itemize}
%       \column{0.3\textwidth}
%       \begin{itemize}
%       \item 4,6,6
%       \item 6,4,6
%       \item 6,6,4
%       \end{itemize}
%     \end{columns}
%
%     \medskip
%
%     \begin{itemize}
%     \item What algorithm do you use?
%     \end{itemize}
%   }
% \end{frame}
%
% \begin{frame}
%   \frametitle{Ad Hoc example: Probabilty Problems}
%
%   {\smaller
%   \begin{block}{The dice problem}
%     If I have $n$ dice, what is the chance of rolling a total above $m$?
%   \end{block}
%
%   \medskip
%
%   Solving with DP
%
%   \medskip
%
%   \begin{itemize}
%   \item For $n=0$, we have only one result: $r=0$
%   \item For $n=1$, we have 6 results: $r = \{1,2,3,4,5,6\}$
%   \item The result for $n=i$ and $r_{n-1}=k$ is $r_n = k + \{1,2,3,4,5,6\}$
%
%     \bigskip
%
%   \item With a state table (dice,result), we can count the number of
%     dice combination above a certain value;
%
%   \end{itemize}
%   }
% \end{frame}
%
% \begin{frame}[fragile]
%   \frametitle{Ad Hoc example: Probability Problems}
%   \begin{exampleblock}{Example Code}
% {\small
% \begin{verbatim}
% int count(int dice_left, int score_left) {
%    if (score_left < 1) return pow(6,dice);
%    if (dice_left == 0) return 0;
%    if (result[dice_left][score_left] != -1)
%       return result[dice_left][score_left];
%    int sum = 0;
%    for (int i = 0; i < 6; i++)
%       sum += count(dice_left-1, score_left-(i+1))
%    result[dice_left][score_left] = sum;
%    return sum;
% }
%
% prob = count(n,m)/6**n;
%
% \end{verbatim}
% }
%   \end{exampleblock}
% \end{frame}

%% TODO: Cycle Finding (Hare and Tortoise, Halim's Book)
%% TODO: Game Theory (NIM, Halim's Book)
