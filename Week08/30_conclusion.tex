\section{Conclusion}

\subsection{Summary}
\begin{frame}{Class Summary}

In this lecture, we discussed challenges in math-focused problems:

\begin{itemize}
  \item Large Integers and Log Operations;
  \item Number Theory:
  \begin{itemize}
    \item Primality Testing and Prime Number Sieve;
    \item Factorization;
    \item Diaphantyne Equation and Linear Combinations;
  \end{itemize}
  \item Common Combinatorics Sequences in Programming Challenges;
\end{itemize}
\bigskip

Next Week we will discuss geometry problems!
\end{frame}

\subsection{Problem Discussion}
\begin{frame}{Problems for this Week}
  \begin{itemize}
    \item Ocean Deep! - Make it Shallow!!
    \item Sum of Consecutive Prime Numbers
    \item Divisibility of Factors
    \item Summation of Four Primes
    \item How Many Trees?
    \item Triangle Counting
    \item Self-Describing sequence
    \item Marbles
  \end{itemize}
\end{frame}

\begin{frame}{10176 -- Ocean Deep! -- Make it Shallow!!}{Discussed in the Lecture}

  \begin{block}{Outline}
    You receive many binary numbers (up to 100 digits), and you must determine if each number is divisible by 131071. Example:
    \bigskip

    \begin{itemize}
      \item 0 -- YES (0)
      \item 1010101 -- NO (85)
    \end{itemize}
  \end{block}\bigskip

  \begin{itemize}
    \item You can use some bignum library;
    \item Or you can use mod division too;
  \end{itemize}
\end{frame}

\begin{frame}{Sum of Consecutive Primes}
  \begin{block}{Outline}
    For a number $N \leq 10000$, determine how many different ways you can write $N$ as a sum of consecutive primes ($p_i + p_{i+1} + \ldots + p_{i+k}$).
  \end{block}\bigskip

  \begin{itemize}
    \item You have to solve for many numbers, but the primes are always the same, so you should pre-calculate the primes.
    \item Remember that the primes are consecutive, so you should be able to search without backtracking.
  \end{itemize}
\end{frame}

\begin{frame}{Divisibility of Factors}
  \begin{block}{Outline}
    Given $N$ and $d$, count how many factors of $N!$ are divisible by $d$.
  \end{block}\bigskip

  \begin{itemize}
    \item Hint 1: You don't need to calculate $N!$, just the factorization of $N!$
    \item Hint 2: Think about the relationship between {\bf Prime Factorization} and {\bf Divisibility}
  \end{itemize}
\end{frame}

\begin{frame}{Summation of Four Primes}
  \begin{block}{Outline}
    For a given number $N$, find four primes that add up to $N$.
  \end{block}\bigskip

  \begin{itemize}
    \item Unlike the previous problem, the four primes do not need to be consecutive;
    \item However, you only need to find one solution;
    \item This is a search problem, but you can use mathematical properties to prune your search!
  \end{itemize}
\end{frame}

\begin{frame}{How Many Trees?}
  \begin{block}{Outline}
    Given a number of nodes with increasing labels, how many {\bf Binary Search Trees} can you make?
  \end{block}\bigskip

  \begin{itemize}
    \item Easy combinatoric problem. Which sequence describes this situation?
    \item Note that the output might be a large integer.
  \end{itemize}
\end{frame}

\begin{frame}{Triangle Counting}
  \begin{block}{Outline}
    Given an integer $N$, how many triangles can you make by choosing three {\bf different} sizes $\leq N$?\bigskip

    {\bf Example:} $N = 5$, triangles: 2,3,4; 2,4,5; 3,4,5;
  \end{block}\bigskip

  \begin{itemize}
    \item Note that testing all pairs can be too slow for large $N$
    \item You should try to find the recurrence on paper first;
    \begin{itemize}
      \item When you add a new $n$ in the end, how many new triangles can you make with $n$?
    \end{itemize}
  \end{itemize}
\end{frame}

\begin{frame}[fragile]{Self Describing Sequence}
  \begin{block}{Outline}
    In the {\bf self describing sequence}, the value $f(n)$ indicates how many times $n$ appears in the sequence. For example, the first few numbers are:

    \begin{verbatim}
 n  :  1  2  3  4  5  6  7  8  9  10  11  12
f(n):  1  2  2  3  3  4  4  4  5   5   5   6
    \end{verbatim}

    Given a value of $n \leq 2\times10^9$, calculate $f(n)$.
  \end{block}\bigskip

  \begin{itemize}
    \item To calculate $f(n)$, is it necessary to calculate every value between $f(1)$ and $f(n-1)$?
    \item Can we skip some values?
  \end{itemize}
\end{frame}

\begin{frame}{Marbles}
  \begin{block}{Outline}
    You have $n$ marbles to put in boxes. Box 1 fits $n_1$ marbles and costs $c_1$. Box 2 fits $n_2$ marbles and costs $c_2$. What is the minimum cost to put all $n$ marbles in boxes?
  \end{block}\bigskip

  \begin{itemize}
    \item This is equivalent to the "candies" problem, but you also have to think about cost.
    \item Remember, that there are multiple linear combinations that satisfy $n = b_1n_1 + b_2n_2$.
    \item After you calculate one pair $b_1,b_2$, how do you find other pairs with possibly smaller cost?
  \end{itemize}
\end{frame}


% - Self Describing Sequence (combinatorics, hard)
%
% Diaphantyne Equation:
% - Marbles - Diaphantyne equation: search through linear combinations
