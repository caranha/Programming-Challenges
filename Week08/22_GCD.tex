
\section{GCD and Friends}
\begin{frame}
  \begin{center}
  {\bf Part II -- GCD}
  \end{center}
\end{frame}


\begin{frame}
  \frametitle{Modulo Arithmetic}

  Modulo Arithmetic is a way to operate in very large number without using bignum.\bigskip

  For some problems, the final result is small (modulo $n$) but the intermediate results are too large. In these cases, we use modulo arithmetic to avoid storing these large intermediate results.\bigskip

  \begin{block}{Modulo Arithmetic Reminder}
    \begin{enumerate}
    \item $(a+b)\%n = ((a\%n)+(b\%n)+n)\%n$
    \item $(a*b)\%n = ((a\%n)*(b\%n))\%n$
    \item $(a^p)\%n = ((a^{p/2}\%n)*(a^{p/2}\%n)*(a^{p\%2}\%n))\%n$
    \end{enumerate}
  \end{block}
\end{frame}

\subsection{Example Problem}
\begin{frame}{Example Problem}
  \begin{block}{}
    Your receive as input a {\bf large binary number} (up to 100 digits). You need to calculate if the number is divisible by 131071 (a prime number).
  \end{block}\bigskip

  \begin{itemize}
  \item Problem: Input and store a large $n$, and calculate $n\%131071$.
  \bigskip

  \item Two approaches:
  \begin{itemize}
    \item Use a BigNum data structure to store $n$, and calculate.
    \item Use modulo arithmetic to calculate the result {\bf without} BigNum.
  \end{itemize}

  \end{itemize}
\end{frame}

\begin{frame}{Modular Inverse}
  The {\bf Modular Inverse} of $a$ is the number $a^{-1}$ so that $a\times a^{-1} \equiv 1 \mod n$. \medskip

  How do we find $a^{-1}\mod n$?
  \begin{itemize}
    \item If $n$ is prime, then $b^{-1} \equiv b^{n-2}\mod n$
    \item If $n$ is not prime, but gcd($n,b$) = 1, then $b^{-1} \equiv b^{\Phi(n)-1} \mod n$
  \end{itemize}\bigskip

  We can use the extended GCD to calculate this.

\end{frame}


\subsection{Greatest Common Divisor}
\begin{frame}[fragile]
  \frametitle{Extended Euclid Algorithm}

  For integers $a$ and $b$, the {\bf greatest common divisor} GCD(a,b) is the largest integer $d$ so that $d|a$ and $d|b$. Euclid's algorithm can quickly calculate $d$ for a,b ($O(\log_{10}a)$).\bigskip

  The {\bf Extended Euclid's Algorithm}\footnote{Also called "The Pulverizer"}, calculate's $x_0$ and $y_0$ so that $a\times x_0 + b\times y_0 = d$.

{\smaller
    \begin{exampleblock}{}
\begin{verbatim}
int extEuclid(int a, int b, int &x, int &y)  {
  int xx = y = 0; int yy = x = 1;
  while (b) {
    int q = a/b;
    int t = b; b = a%b; a = t;
    t = xx; xx = x - q*xx; x = t;
    t = yy; yy = y - q*yy; y = t;
  }
  return a; // GCD, xa + by = d;
}
\end{verbatim}
    \end{exampleblock}
}
\end{frame}


\begin{frame}{Extended GCD and the Diophantine Equation}

  One very useful property of $d =$ GCD$(a,b)$ is that {\bf $d$ divides every integer combination of $a$ and $b$}. In other words: For every $ax+by = c$, if x and y are integers, then $d|c$.\footnote{The proof for this is very cool}.\bigskip

  We can use this property to calculate the integer solutions of the {\bf Diophantine Equation}: $xa+yb = c$\bigskip

  \begin{itemize}
    \item If $d|c$ is not true, there are no integer solutions.
    \item If $d|c$ is true, there are infinite integer solutions:
    \begin{itemize}
      \item The first solution $(x_0, y_0)$ is calculated from the extended GCD.
      \item Other solutions $(x_n,y_n)$ can be derived as: $x_n = x_0 + (b/d)n, y_n = y_0 - (a/d)n$, where $n$ is an integer.
    \end{itemize}
  \end{itemize}

\end{frame}

\begin{frame}{Diophantine Equation Problem Example}
    \begin{block}{Problem Example}
      With 839 yens, you want to buy Candy X and Candy Y.
      \begin{itemize}
        \item Candy X costs $x=25$ yens.
        \item Candy Y costs $y=18$ yens.
      \end{itemize}
      How many candies can you buy?
    \end{block}

    \begin{enumerate}
      \item Calculate $d, x_0, y_0$ from extended GCD:
      \begin{itemize}
        \item $d = 1, x_0 = -5, y_0 = 7$. This means that $25\times(-5) + 18\times(7) = 1$
      \end{itemize}
      \item Is $d|c$? {\bf Yes}. Continue.
      \item Multiply both sides of the equation by $\frac{c}{d}$:
      \begin{itemize}
        \item $25 \times (-5 \times 839) + 18\times(7 \times 839) = 839$
      \end{itemize}
      \item It is impossible to buy negative candies, so we iterate on $n$ to find
      \begin{itemize}
        \item $x_n = x_0 + (y/d)n$ and $y_n = y_0 - (x/d)n$
      \end{itemize}
      \item At $n = 234$ we find: $25 \times 17 + 18 \times 23 = 839$
    \end{enumerate}
\end{frame}

\begin{frame}[fragile]{Extended GCD to calculate modular inverse}
  Let's calculate $x$ so that $b\times x \equiv 1 \mod n$.\medskip

  This is equivalent to $bx = 1 + ny \to bx - ny = 1$, for any $y$. We feed
  these values to the extended GCD.
  \begin{block}{}
\begin{verbatim}
int mod(int a, int m) { return ((a%m) + m)%m; }

int modInverse(int b, int m) {
  int x, y;
  int d = extEuclid(b, m, x, y);
  if (d != 1) return -1;  // inverse only exists if gcd(b,m) = 1;

  // b*x + m*y == 1, so apply (mod m) to x to obtain b^-1
  return mod(x, m);
}
\end{verbatim}
  \end{block}
\end{frame}
