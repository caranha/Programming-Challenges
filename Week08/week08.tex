\documentclass[aspectratio=169]{beamer}
\usepackage{tikz}
\usetikzlibrary{arrows,shapes}
\tikzstyle{vertex}=[circle,fill=black!25,minimum size=10pt,inner sep=0pt]
\tikzstyle{blue vertex}=[circle,fill=blue!100,minimum size=10pt,inner sep=0pt]
\tikzstyle{red vertex}=[circle,fill=red!100,minimum size=10pt,inner sep=0pt]
%\tikzstyle{label}=[thin, draw=black, align=center,minimum width=0.5cm, minimum height=0.5cm,fill=white]
\tikzstyle{edge} = [draw,thick,-]
\tikzstyle{red edge} = [draw, line width=5pt,-,red!50]
\tikzstyle{black edge} = [draw, line width=5pt,-,black!20]
\tikzstyle{weight} = [font=\smaller]


\usepackage{amssymb,amsmath}
\usepackage{graphicx}
\usepackage{url}
\usepackage{color}
\usepackage{relsize}		% For \smaller
\usepackage{url}			% For \url
\usepackage{epstopdf}	% Included EPS files automatically converted to PDF to include with pdflatex
\usepackage{pagenote}[continuous,page]

%For MindMaps
% \usepackage{tikz}%
% \usetikzlibrary{mindmap,trees,arrows}%

%%% Color Definitions %%%%%%%%%%%%%%%%%%%%%%%%%%%%%%%%%%%%%%%%%%%%%%%%%%%%%%%%%
%\definecolor{bordercol}{RGB}{40,40,40}
%\definecolor{headercol1}{RGB}{186,215,230}
%\definecolor{headercol2}{RGB}{80,80,80}
%\definecolor{headerfontcol}{RGB}{0,0,0}
%\definecolor{boxcolor}{RGB}{186,215,230}

%%% Save space in lists. Use this after the opening of the list %%%%%%%%%%%%%%%%
%\newcommand{\compresslist}{
%	\setlength{\itemsep}{1pt}
%	\setlength{\parskip}{0pt}
%	\setlength{\parsep}{0pt}
%}

%\setbeameroption{show notes on top}

% You should run 'pdflatex' TWICE, because of TOC issues.

% Rename this file.  A common temptation for first-time slide makers
% is to name it something like ``my_talk.tex'' or
% ``john_doe_talk.tex'' or even ``discrete_math_seminar_talk.tex''.
% You really won't like any of these titles the second time you give a
% talk.  Try naming your tex file something more descriptive, like
% ``riemann_hypothesis_short_proof_talk.tex''.  Even better (in case
% you recycle 99% of a talk, but still want to change a little, and
% retain copies of each), how about
% ``riemann_hypothesis_short_proof_MIT-Colloquium.2000-01-01.tex''?

\mode<presentation>
{
  % A tip: pick a theme you like first, and THEN modify the color theme, and then add math content.
  % Warsaw is the theme selected by default in Beamer's installation sample files.

  %%%%%%%%%%%%%%%%%%%%%%%%%%%% THEME
  %\usetheme{Madrid}		% No subsection
  \usetheme{AnnArbor}  % Subsection on top, no color


  %\usetheme{Antibes}
  %\usetheme{Bergen}
  %\usetheme{Berkeley}		% bem bacana - menu esquerdo
  %\usetheme{Berlin}
  %\usetheme{Boadilla}
  %\usetheme{boxes}
  %\usetheme{CambridgeUS}		% bem bacana - menu superior
  %\usetheme{Copenhagen}
  %\usetheme{Darmstadt}
  %\usetheme{default}
  %\usetheme{Dresden}
  %\usetheme{Frankfurt}
  %\usetheme{Goettingen}
  %\usetheme{Hannover}		% bem bacana - menu esquerdo
  %\usetheme{Ilmenau}
  %\usetheme{JuanLesPins}
  %\usetheme{Luebeck}
  %\usetheme{Malmoe}
  %\usetheme{Marburg}		% bem bacana - menu direito
  %\usetheme{Montpellier}
  %\usetheme{PaloAlto}		% bem bacana - menu esquerdo
  %\usetheme{Pittsburgh}
  %\usetheme{Rochester}		%bacana
  %\usetheme{Singapore}
  %\usetheme{Szeged}
  %\usetheme{Warsaw}

  %%%%%%%%%%%%%%%%%%%%%%%%%%%% COLOR THEME
  %\usecolortheme{default}		% branco, azul clarinho
  \usecolortheme{crane}		% Very yellow (ok)

  %\usecolortheme{albatross}		% azul escuro, massa
  %\usecolortheme{beetle}		% cinza, menu azul
  %\usecolortheme{dolphin}		% azul e branco, legal
  %\usecolortheme{dove}			% cinza e branco, feio
  %\usecolortheme{fly}			% todo cinza, horrível
  %\usecolortheme{lily}			% parece o default
  %\usecolortheme{orchid}		% azul e branco, ok
  %\usecolortheme{rose}			% branco e violeta-claro, bonito
  %\usecolortheme{seagull}		% cinza, feio
  %\usecolortheme{seahorse}		% nhé, meio feio
  %\usecolortheme{sidebartab}		% Azul, branco, destaque na tab, interessante
  %\usecolortheme{structure}		% bichado
  %\usecolortheme{whale}		% Azul e branco, bem bonito

  %%%%%%%%%%%%%%%%%%%%%%%%%%%% OUTER THEME
  \useoutertheme{default}
  %\useoutertheme{infolines}
  %\useoutertheme{miniframes}
  %\useoutertheme{shadow}
  %\useoutertheme{sidebar}
  %\useoutertheme{smoothbars}
  %\useoutertheme{smoothtree}
  %\useoutertheme{split}
  %\useoutertheme{tree}

  %%%%%%%%%%%%%%%%%%%%%%%%%%%% INNER THEME
  \useinnertheme{circles}
  %\useinnertheme{default}
  %\useinnertheme{inmargin}
  %\useinnertheme{rectangles}
  %\useinnertheme{rounded}

  %%%%%%%%%%%%%%%%%%%%%%%%%%%%%%%%%%%

  \setbeamercovered{invisible} % or whatever (possibly just delete it)
  % To change behavior of \uncover from graying out to totally
  % invisible, can change \setbeamercovered to invisible instead of
  % transparent. apparently there are also 'dynamic' modes that make
  % the amount of graying depend on how long it'll take until the
  % thing is uncovered.

}


% Get rid of nav bar
\beamertemplatenavigationsymbolsempty

% Use short top
%\usepackage[headheight=12pt,footheight=12pt]{beamerthemeboxes}
%\addheadboxtemplate{\color{black}}{
%\hskip0.5cm
%\color{white}
%\insertshortauthor \ \ \ \
%\insertframenumber \ \ \ \ \ \ \
%\insertsection \ \ \ \ \ \ \ \ \ \ \ \ \ \ \ \ \  \insertsubsection
%\hskip0.5cm}
%\addheadboxtemplate{\color{black}}{
%\color{white}
%\ \ \ \
%\insertsection
%}
%\addheadboxtemplate{\color{black}}{
%\color{white}
%\ \ \ \
%\insertsubsection
%}

% Insert frame number at bottom of the page.
% \usefoottemplate{\hfil\tiny{\color{black!90}\insertframenumber}}

%% makes the ppagenote command for figure references at the end.

\usepackage[english]{babel}
%qq\usepackage[latin1]{inputenc}
\usepackage{CJKutf8}
\usepackage{subfigure}

\usepackage{times}
\usepackage[T1]{fontenc}

\makepagenote
\renewcommand{\notenumintext}[1]{}
\newcommand{\ppagenote}[1]{\pagenote[Page \insertframenumber]{#1}}

\title[Programming Challenges]{GB20602 - Programming Challenges}
\author[Claus Aranha]{Claus Aranha\\{\footnotesize caranha@cs.tsukuba.ac.jp}}
\institute[U. Tsukuba]{University of Tsukuba, Department of Computer Sciences}


\subtitle[Week 8: Mathematics]{Week 8 - Mathematics}
\date[]{{\smaller(last updated: \today)}}

\begin{document}
\begin{CJK}{UTF8}{ipxm}

\begin{frame}
\maketitle
\vfill

\hfill Version 2021.1
\end{frame}

\section{Introduction}

\subsection{Outline}
\begin{frame}{Math Problems: Lecture Outline}

  Every computer program requires some amount of mathematics. So what does {\bf "Math Problems"} mean in Programming Challenges?\bigskip

  Here we describe two kinds of problems as {\bf "Math Problems"}:

  \begin{block}{The Challenge is The Implementation of Mathematical Concepts}
    \begin{itemize}
      \item Problems with Big Numbers (above variable limits)
      \item Problems with Geometry (next lecture!)
    \end{itemize}
  \end{block}
  \begin{exampleblock}{The Challenge Requires Mathematical Planning Before Programming}
    In this case, it is sometimes possible to solve the entire problem in paper and quickly implement a solution to the problem.
    \begin{itemize}
      \item Number Theory (primality testing, factorization, rings)
      \item Combinatorics (sequences, counting, recurrences)
    \end{itemize}
  \end{exampleblock}
\end{frame}


\section{Large Numbers}

\begin{frame}{Math Problems Part I: Large Numbers}

  Some programming challenges, in particular challenges involving combinatoric analysis, require operations on very large numbers.\bigskip

  In this section, we will review some ways to deal with these numbers:
  \begin{itemize}
    \item "BigNum" libraries;
    \item Modulo Operations;
  \end{itemize}
\end{frame}

\subsection{Bignum}
\begin{frame}
  \frametitle{Dealing with Large Numbers}

  In this lecture, we call "Large Numbers" (also sometimes {\bf Bignum}) integers that do not fit in the standard variable types in programming languages (ex: long, long long, unsigned long, etc).\bigskip

  This is very common in problems and algorithms involving factorials. For example: $25! = 15511210043330985984000000 > 10^{26}$.\bigskip

  \begin{block}{BigNum in Different Languages}
    \begin{itemize}
      \item {\bf C++ STL} does not have native support to bignum. You have to program yourself;
      \begin{itemize}
        \item unsigned long long: $2^{64} < 10^{20}$
      \end{itemize}
      \item {\bf Java} has the "BigInteger" class, which contains several useful operations on large numbers;
      \item {\bf Python} handles BigNums natively, so a special class is not necessary;
    \end{itemize}
  \end{block}
\end{frame}

\begin{frame}[fragile]{Sum and Division using Java's "Big Integer" class}

{\smaller
\begin{block}{}
\begin{verbatim}
import java.util.Scanner; import java.math.BigInteger;
class Main {
  public static void main(String[] args) {
    Scanner sc = new Scanner(System.in);
    while (true) {
      int N = sc.nextInt(), F = sc.nextInt();
      if (N == 0 && F == 0) break;
      BigInteger sum = BigInteger.ZERO;                  // Bignum Constant
      for (int i = 0; i < N; i++) {
        BigInteger V = sc.nextBigInteger();              // Bignum I/O
        sum = sum.add(V); }                              // Bignum Addition
      System.out.println(
        "Total " + sum + ": Division: "
        + sum.divide(BigInteger.valueOf(F)) + "\n" );}   // Bignum division.
  }
}
\end{verbatim}
  \end{block}}
\end{frame}


\begin{frame}[fragile]
  \frametitle{Useful functions in Java.math.BigInteger}
  Besides dealing with arbitrarily large numbers, the BigInteger class also has some other useful mathematical functions:

{\smaller
  \begin{block}{Algebraic functions}
    BigInteger.add(), .subtract(), .multiply(), .divide(),
    .pow(), .mod(), .remainder()
  \end{block}

  \begin{block}{Changing Number Base}
\begin{verbatim}
BI = BigInteger(10); System.println(BI.toString(2))
// Result: 1010
\end{verbatim}
  \end{block}

  \begin{block}{Probabilistic Primality Test}
\begin{verbatim}
isPrime = BI.isProbablePrime(int certainty)
// Chance of being correct is 1 - (1/2)^certainty
\end{verbatim}
  \end{block}


\begin{block}{Other functions}
  BigInteger.gcd(BI)
  BigInteger.modPow(BI exponent, BI m)
\end{block}}
\end{frame}

\subsection{Modulo Arithmetic}
\begin{frame}
  \frametitle{Modulo Arithmetic}

  Another way to operate in very large numbers is to use {\bf Modulo Arithmetic}.\bigskip

  For some problems, the final result is small (modulo $n$) but the intermediate results are too large. In these cases, we can use modulo arithmetic to avoid storing these large intermediate results.\bigskip

  \begin{block}{Modulo Arithmetic Reminder}
    \begin{enumerate}
    \item $(a+b)\%s = ((a\%s)+(b\%s)+s)\%s$
    \item $(a*b)\%s = ((a\%s)*(b\%s))\%s$
    \item $(a^n)\%s = ((a^{n/2}\%s)*(a^{n/2}\%s)*(a^{n\%2}\%s))\%s$
    \end{enumerate}
  \end{block}
\end{frame}

\subsection{Example Problem}
\begin{frame}{Example Problem: 10176, Ocean Deep! Make it Small}
  \begin{block}{Problem summary}
    Your receive as input a {\bf large binary number} (up to 100 digits). You need to calculate if the number is divisible by 131071 (a prime number).
  \end{block}\bigskip

  \begin{itemize}
  \item Problem: Input and store a large $n$, and calculate $n\%131071$.
  \bigskip

  \item Two approaches:
  \begin{itemize}
    \item Use a BigNum data structure to store $n$, and calculate.
    \item Use modulo arithmetic to calculate the result {\bf without} BigNum.
  \end{itemize}

  \end{itemize}
\end{frame}

% \subsection{Precision}
% TODO: printing with precision in C, Java, Python
% Dealing with very small numbers

\section{Number Theory}
\begin{frame}{Part II: Number Theory}

  Number Theory studies the relationships between {\bf integer numbers}.\bigskip

  It is a large and fascinating field of study, but for the purposes of programming contests, in this lecture we will focus on three topics:\bigskip

  \begin{itemize}
  \item {\bf Primality}: How to decide if a number is prime;
  \item {\bf Division and Remainders}: The division relationship between integers;
  \item {\bf Sequences}: Recurrence relations between sets of numbers;
  \end{itemize}
\end{frame}

\section{Primality}
\subsection{Primality Testing}
\begin{frame}
  \frametitle{Primality Testing}

    {\bf Prime Numbers} are integers ($> 1$) that are only divisible by 1 and by themselves: $2,3,5,7,11,13,\ldots$\bigskip

    {\bf Question:} How do you write a (simple) program to test if $N$ is prime?
    \begin{itemize}

      \item Complete Search: For each $d \in 2..N-1$, test if $N\%d == 0$.
      \begin{itemize}
        \item This requires $O(N)$ divisions.\medskip
      \end{itemize}

      \item Pruning the search:
      \begin{itemize}
        \item Search only numbers between 2 and $\sqrt{N}$: $O(\sqrt{N})$
        \item Search only {\bf odd} numbers between 2,3 and $\sqrt{N}$: $O(\frac{\sqrt{N}}{2})$
        \item Search only {\bf PRIME} numbers between 2 and $\sqrt{N}$:
        $O(\frac{\sqrt{N}}{\ln(\sqrt{N})})$
      \end{itemize}\medskip

      \item Can we calculate all primes between 2 and $\sqrt{N}$ easily?
    \end{itemize}
\end{frame}

\begin{frame}{Primality Testing: Finding {\bf Sets} of primes}

  \begin{block}{The Prime Number Theorem (simplified)}
    There are approximately $\frac{N}{\log{N}-1}$ prime numbers between 1 and $N$
  \end{block}\bigskip

  \begin{itemize}
    \item Number of prime numbers between 1 and $\sqrt{10^6}$ = 168
    \item Number of prime numbers between 1 and $\sqrt{10^{10}} \approx 9500$
  \end{itemize}\bigskip

  If we have a "list of prime numbers", we can calculate primality of
  many large numbers very quickly.\bigskip


  A simple algorithm to find a list of primes is {\bf Sieve of Eratosthenes}.
\end{frame}



\begin{frame}[fragile]{Sieve of Eratosthenes}

    \begin{block}{}
      \begin{itemize}
      \item Initialize "Sieve" vector of size $\sqrt{N}$, all TRUE;
      \item Loop on Sieve. If Sieve[i] is TRUE, add $i$ to prime list
      \item Remove {\bf ALL $i\times m$} multiples of $i$ from Sieve;
      \end{itemize}
    \end{block}

    {\smaller
  \begin{exampleblock}{}
\begin{verbatim}
def sieve(k):                 ## Find all primes up to k
   primes = []                ## List of primes found
   sieve = [1]*(k+1)          ## all numbers start in the list
   sieve[0] = sieve[1] = 0    ## 0,1 trivially not primes
   for i in range(k+1):       ## Linear search
      if (sieve[i] == 1):     ## Found a new prime
         primes.append(i)     ## Add to prime list
         j = i*i              ## Optimization. Why not i*2?
         while (j < k+1):     ## Costs O(loglogN)
            sieve[j] = 0      ## Remove multiples from sieve
            j += i
   return primes              ## list of primes
\end{verbatim}
  \end{exampleblock}
  }
\end{frame}

\begin{frame}
  \frametitle{Sieve of Eratosthenes: Computation Cost}

    \begin{itemize}
      \item The cost of calculating the Sieve for $k$ is $O(k\log\log k)$
      \item The cost of full search for $N$ is $O(\sqrt{N}/2)$
      \item Why use sieve and not the full search?
    \end{itemize}

    \begin{block}{Amortized Complexity}
      Do a complex calculation once, use result many times:
      \begin{itemize}
      \item If we are only testing {\bf ONE PRIME}, the full search is better.
      \item But, if the problem requires many primes to be tested, the sieve is better.
        \begin{itemize}
        \item If $N$ < $k$, checking the sieve table costs $O(1)$.
        \item We can pre-calculate the sieve table when initalizing the program;
        \end{itemize}
      \end{itemize}
    \end{block}\bigskip

    When do we need to calculate multiple primes? Prime factorization!
\end{frame}

\subsection{Prime Factorization}
\begin{frame}
  \frametitle{Prime Factorization}

  Every natural number $N$ can be written as a {\bf unique multiplication of primes}\footnote{Fundamental Theorem of Arithmetics}. Example:

  \begin{equation*}
    1200 = 2\times2\times2\times2\times3\times5\times5 = 2^4\times3\times5^2
  \end{equation*}

  In other words, for $N$, the prime number factorization of $N$ is:
    \begin{equation*}
      N=p_1^{e_1}p_2^{e_2}\ldots p_n^{e_n}, p_i \text{ is prime}
    \end{equation*}

  (Prime) Factorization is a key issue in Cryptography, so fast factorization is an important research problem. For programming challenges, we use two simple approaches:\bigskip

  \begin{itemize}
    \item {\bf Full search}: create a list of primes (with sieve) and test if each of them divides $N$.
    \item {\bf Divide and Conquer:} Find the smallest prime $p_i$ from sieve that divides $N$. Replace $N$ with $N|p_i$. Repeat until $p_i > \sqrt{N}$.
  \end{itemize}
\end{frame}

\begin{frame}[fragile]
  \frametitle{Prime factorization: Divide and conquer approach}

  This algorithm is reasonably fast if $N$ is composed of several small prime factors.

  {\smaller
  \begin{exampleblock}{}
\begin{verbatim}
vector<int> primeFactors(ll N) {
  vector<int> factors;
  long PF_idx = 0, PF = sieve[PF_idx];   // sieve is a precomputed prime list
  while (PF * PF <= N) {                 // remember, N gets smaller;
    while (N % PF == 0) {                // Remove PF^x from N
      N /= PF; factors.push_back(PF);
    }
    PF = primes[PF_idx++];               // only consider primes!
  }
  if (N != 1) factors.push_back(N);      // special case: N is prime
  return factors;
}
\end{verbatim}
  \end{exampleblock}}
\end{frame}

\begin{frame}{Full Factorization}
  In some cases, we want to know {\bf all} numbers that divide a certain number $N$.\bigskip

  We can calculate the full factorization of $N$ from its prime factorization.\\
  In fact, the full factorization of $N$ is the set of all unique combinations of prime factors.\bigskip

  Example:
  \begin{itemize}
    \item $1200 = 2\times2\times2\times2\times3\times5\times5 = 2^4\times3\times5^2$
    \item Number of factors of 1200: $5(2^4)\times2(3^1)\times3(5^2) = 30$
    \begin{itemize}
      \item $2^0 \times 3^0 \times 5^1 = 5$,
      \item $2^0 \times 3^0 \times 5^2 = 25$,
      \item $2^0 \times 3^1 \times 5^0 = 3$,
      \item $2^0 \times 3^1 \times 5^1 = 15$,
      \item $2^0 \times 3^1 \times 5^2 = 75$,
      \item $\ldots$
    \end{itemize}
  \end{itemize}
\end{frame}

\begin{frame}
  \frametitle{Factorization Problem Example: 10139 -- Factovisors}

    \begin{block}{Problem summary}
      Check if $m$ divides $n!$ ($1 \leq m,n \leq 2^{31}-1$)
    \end{block}

    The factorial of $n \leq 2^{31}-1$ is a HUGE number. Fortunately, it is not necessary to calculate this number at all. Consider that:

    \begin{itemize}
    \item $F_m$: primefactors(m)
    \item $F_{n!}$: $\cup$(primefactors(1), primefactors(2) $\ldots$, primefactors(n))
    \end{itemize}

    We can say that $m$ divides $n!$ iff $F_m \subset F_{n!}$.\bigskip

    Examples:
    \begin{itemize}
    \item $m = 48, n=6, n! = 2\times3\times4\times5\times6$\\
      $F_m = \{2,2,2,2,3\}, F_{n!} = \{2,3,2,2,5,2,3\}$

  \medskip

    \item $m = 25, n=6, n! = 2\times3\times4\times5\times6$\\
      $F_m = \{5,5\}, F_{n!} = \{2,3,2,2,5,2,3\}$

    \end{itemize}
\end{frame}


\subsection{Greatest Common Divisor}
\begin{frame}[fragile]
  \frametitle{Extended Euclid Algorithm}

  For integers $a$ and $b$, the {\bf greatest common divisor} GCD(a,b) is the largest integer $d$ so that $d|a$ and $d|b$. Euclid's algorithm can quickly calculate $d$ for a,b ($O(\log_{10}a)$).\bigskip

  The {\bf Extended Euclid's Algorithm}\footnote{Also called "The Pulverizer"}, calculate's $x_0$ and $y_0$ so that $a\times x_0 + b\times y_0 = d$.

{\smaller
    \begin{exampleblock}{}
\begin{verbatim}
int gcdExtended(int a, int b, int *x, int *y)  {
  if (a == 0) { *x = 0; *y = 1; return b; }

  int x1, y1; // To store results of recursive call
  int gcd = gcdExtended(b%a, a, &x1, &y1);

  *x = y1 - (b/a) * x1; *y = x1;      // Update x,y

  return gcd;
}
\end{verbatim}
    \end{exampleblock}
}
\end{frame}

\begin{frame}{Extended GCD and the Diophantine Equation}

  One very useful property of $d =$ GCD$(a,b)$ is that {\bf $d$ divides every integer combination of $a$ and $b$}. In other words: For every $ax+by = c$, if x and y are integers, then $d|c$.\footnote{The proof for this is very cool}.\bigskip

  We can use this property to calculate the integer solutions of the {\bf Diophantine Equation}: $xa+yb = c$\bigskip

  \begin{itemize}
    \item If $d|c$ is not true, there are no integer solutions.
    \item If $d|c$ is true, there are infinite integer solutions:
    \begin{itemize}
      \item The first solution $(x_0, y_0)$ is calculated from the extended GCD.
      \item Other solutions $(x_n,y_n)$ can be derived as: $x_n = x_0 + (b/d)n, y_n = y_0 - (a/d)n$, where $n$ is an integer.
    \end{itemize}
  \end{itemize}


\end{frame}

\begin{frame}{Diophantine Equation Problem Example}
    \begin{block}{Problem Example}
      With 839 yens, you want to buy Candy X and Candy Y.
      \begin{itemize}
        \item Candy X costs $x=25$ yens.
        \item Candy Y costs $y=18$ yens.
      \end{itemize}
      How many candies can you buy?
    \end{block}

    \begin{enumerate}
      \item Calculate $d, x_0, y_0$ from extended GCD:
      \begin{itemize}
        \item $d = 1, x_0 = -5, y_0 = 7$. This means that $25\times(-5) + 18\times(7) = 1$
      \end{itemize}
      \item Is $d|c$? {\bf Yes}. Continue.
      \item Multiply both sides of the equation by $\frac{c}{d}$:
      \begin{itemize}
        \item $25 \times (-5 \times 839) + 18\times(7 \times 839) = 839$
      \end{itemize}
      \item It is impossible to buy negative candies, so we iterate on $n$ to find
      \begin{itemize}
        \item $x_n = x_0 + (y/d)n$ and $y_n = y_0 - (x/d)n$
      \end{itemize}
      \item At $n = 234$ we find: $25 \times 17 + 18 \times 23 = 839$
    \end{enumerate}
\end{frame}

%%%%%%%%%%%%%%%%%%%%%%%%%%%%%%%%%%%%%%

\section{Combinatorics}
\begin{frame}
  \frametitle{Combinatoric Problems}
    Combinatorics is the area of mathematics that studies {\bf countable discrete structures} (integers, sets, etc).\bigskip

    For our focus on competitive programming, combinatoric problems involves calculating the values of {\bf numeric sequences}. This requires programming the {\bf Recurrent Form} or the {\bf Closed Form} of the sequence.\bigskip


    \begin{itemize}
    \item {\bf Recurrent Form}: The recurrent form of a sequence $F$ calculates $F_n$ based on its antecessor values: $F_{n-1}, F_{n-2},\ldots$.
    \begin{itemize}
      \item Recurrent forms are usually implemented using {\bf Dynamic Programming} and {\bf Memoization};
    \end{itemize}

    \item {\bf Closed Form}: The closed form of a sequence $F$ calculates $F_n$ {\bf without} using the antecessor values of the sequence.
    \end{itemize}
\end{frame}

\subsection{Sequence Examples}
\begin{frame}
  \frametitle{Sequence Example: Triangular Numbers}
  \begin{block}{Definition}
    {\bf Triangular Numbers} is the sequence where $T_n$ is the sum of all inegers from $1$ to $n$. Example:\medskip

    $T_1=1, T_2=1+2=3, \ldots, T_7=1+2+3+4+5+6+7=28$\\
    Trivial, right?
  \end{block}\bigskip

  \begin{itemize}
    \item {\bf Recurrent Form:} $T(n) = T(n-1) + n$
    \begin{itemize}
      \item The recurrent form can be calculated with a loop or recursion;
    \end{itemize}\bigskip
    \item {\bf Closed Form:} $T(n) = \frac{n(n+1)}{2}$
    \begin{itemize}
      \item The closed form can be calculated at once;
      \item It can be used to estimate how fast a sequence grows. In this case, $T_n$ is $O(N^2)$
    \end{itemize}
  \end{itemize}
\end{frame}

\begin{frame}
  \frametitle{A more famous sequence: Fibonacci Numbers}

  \begin{block}{Definition}
    The Fibonacci number $F_n$ is the sum of the two numbers before it.\medskip

    $0, 1, 1, 2, 3, 5, 8, 13, 21, 34, \ldots$
  \end{block}\bigskip

  \begin{itemize}
    \item Recurrent Form:
    \begin{itemize}
      \item Starting Values: $F_0 = 0$, $F_1 = 1$
      \item Recurrence: $F_n = F_{n-1} + F_{n-2}$
    \end{itemize}\bigskip

    \item Be careful when implementing recurrences with multiple terms;
    \begin{itemize}
      \item If using recursive functions, {\bf memoization/DP} is necessary to avoid wasted calculation;
      \item In general, each term in a recurrence requires a starting value;
    \end{itemize}
  \end{itemize}
\end{frame}

\begin{frame}[fragile]
  \frametitle{Bonus: Fibonacci Facts}
  \begin{block}{Closed Form for the Fibonacci Numbers:}
    \begin{equation*}
      F(n) = \frac{1}{\sqrt{5}}\left(\left(\frac{1+\sqrt{5}}{2}\right)^n-\left(\frac{1-\sqrt{5}}{2}\right)^n\right)
    \end{equation*}
    The second term of the closed form tends to 0 when $n$ is large!
  \end{block}

  \begin{block}{Pisano's period}
    The last digits of the Fibonacci sequence repeat with a fixed period!\smallskip
{\smaller
\begin{verbatim}
 Digits        | Period      || Digits        | Period
 last digit    | 60 numbers  || last 3 digits | 1500 numbers
 last 2 digits | 300 numbers || last 4 digits | 15000 numbers
F(6)   =                             8
F(66)  =                27777890035288
F(366) =  1380356 ... 8899086435571688
\end{verbatim}}
  \end{block}
\end{frame}

% What are binomial numbers (equations)
% How do we calculate them (closed form and DP)
% Where do we use them? (Number of Choices)

\begin{frame}{Binomial Coefficient}
  \begin{block}{Definition}
    {\bf Binomial Coefficients} are the set of numbers that correspond to the expansion of a binomial:\bigskip

    \begin{itemize}
      \item $B_3 = (a+b)^3 = 1a^3 + 3a^2b + 3ab^2 + b^3 = \{1,3,3,1\}$
      \item $B_5 = (a+b)^5 = 1a^5 + 5a^4b + 10a^3b^2 + 10a^2b^3 + 5ab^4 + 1b^5 = \{1,5,10,10,5,1\}$
    \end{itemize}\bigskip

    Many times, we are interested in the k-th number of the n-binomial,\\ written as $C(n,k)$ or $^nC_k$. Example: $C(5,2) = 10$.
  \end{block}
\end{frame}

\begin{frame}[fragile]{Binomial Coefficient}{Interpretation and Recurrent Form}
  The common interpretation of $C(n,k)$ is "I have to select A or B $n$ times, how many different ways can I choose A $k$ times?"
  \begin{itemize}
    \item How many binary strings with $n$ digits have $k$ ones?
    \item How many paths exist
  \end{itemize}\bigskip

  Using this definition, we can define the recurrent form of the Binomial:
  \begin{itemize}
    \item I have to choose A $k$ times out of $n$
    \begin{itemize}
      \item If I choose A $k-1$ times out of $n-1$, I choose A again.
      \item If I choose A $k$ times out of $n-1$, I choose B.
    \end{itemize}
    \item $C(n,k) = C(n-1,k-1) + C(n-1,k)$
    \item Don't forget to use DP to implement this!
  \end{itemize}\bigskip
\end{frame}

\begin{frame}[fragile]{Pascal's Triangle}
  The recurrent form of the binomials:
  \begin{equation*}
    C(n,k) = C(n-1,k-1) + C(n-1,k)
  \end{equation*}

  Can also be observed by laying out the numbers:

\begin{verbatim}
1
1 1
1 2 1
1 3 3 1
1 4 6 4 1
1 5 10 10 5 1
1 6 15 20 15 6 1
\end{verbatim}
\end{frame}

\begin{frame}{Closed Form for the Binomial}
  The closed form for $C(n,k)$ is:\bigskip

    \begin{equation*}
      C(n,k) = \frac{n!}{(n-k)!k!}
    \end{equation*}\bigskip

  Be careful! As you remember, the value of $n!$ can become very big,
  very fast. It might be better to calculate the binomial using the recurrent
  form, to avoid overflow.
\end{frame}

\begin{frame}{The Catalan Numbers}
  \begin{block}{Motivating Problem}
    Given $n$ pairs of parenthesis, how many different balanced expressions can you create?\bigskip

    \begin{itemize}
      \item n = 0: . = 1
      \item n = 1: () = 1
      \item n = 2: ()(), (()) = 2
      \item n = 3: ((())), ()(()), (())(), (()()), ()()() = 5
      \item n = 4: 14
      \item n = 5: 42
    \end{itemize}
  \end{block}\bigskip

  This sequence is known as the {\bf Catalan Numbers}, and it appears in
  several recursive combinatory problems.
\end{frame}

\begin{frame}{The Catalan Numbers}{Recurrent Form}
  The {\bf Recurrent form} of the catalan number can be derived from the parenthesis definition:\bigskip

  \begin{itemize}
    \item If we define $c_k$ as an expression with k parenthesis, we can break it down into: $c_k = (c_a)c_b$, where $k = a + b + 1$. \medskip

    \item Varying the values of $a$ and $b$, and counting all possible variations gives us the recurrent form:\medskip

    \item $c_{n+1} = \sum_{i=0}^n c_ic_{n-i}$
  \end{itemize}
\end{frame}


\begin{frame}{Closed Form and Usage}
  The closed form of the Catalan Numbers is:
  \begin{equation*}
    c_n = \frac{1}{n+1}C(2n,n)
  \end{equation*}
  Be careful of calculating factorials in $C(2n,n)$\bigskip

  \begin{block}{Other uses of Catalan Numbers}
    \begin{itemize}
    \item Number of ways you can triangulate a poligon with $n+2$ sides;
    \item Number of monotonic paths on an $nxn$ grid that do not pass above
      the diagonal.
    \item Number of distinct binary trees with $n$ vertices
    \item Etc...
    \end{itemize}
  \end{block}
\end{frame}

\subsection{Summary}
\begin{frame}{Class Summary}

In this lecture, we discussed challenges in math-focused problems:

\begin{itemize}
  \item Large Integers and Log Operations;
  \item Number Theory:
  \begin{itemize}
    \item Primality Testing and Prime Number Sieve;
    \item Factorization;
    \item Diaphantyne Equation and Linear Combinations;
  \end{itemize}
  \item Common Combinatorics Sequences in Programming Challenges;
\end{itemize}
\bigskip

Next Week we will discuss geometry problems!
\end{frame}


% TODO: Check if this makes sense
% \begin{frame}
%   \frametitle{Integer Partition}
%   \begin{block}{}
%     f(5,5) = (5),(4,1),(3,2),(3,1,1),(2,2,1),(2,1,1,1),(1,1,1,1,1)
%   \end{block}
%   \begin{block}{Definition and calculation}
%     $f(n,k)$ -- number of ways that we can sum $n$, using integers
%     equal or less than $k$.
%
%     \bigskip
%
%     \structure{Recurrence:}
%     \begin{itemize}
%     \item $f(n,k) = f(n-k,k) + f(n, k+1)$
%     \item $f(1,1) = 1$; $f(n,k) = 0$ if $k > n$
%     \end{itemize}
%   \end{block}
% \end{frame}



%%% FIXME: Probability Section
%%% Probably remove for good, there are very few
%%% Probability problems in programming challenges.

% \begin{frame}
%   \frametitle{Ad Hoc Example: Probability problems}
%
%   {\smaller
%     \begin{block}{Dice Throwing}
%       If you have $n$ dice, what is the chance of rolling a total above $m$?
%     \end{block}
%
%     \begin{itemize}
%     \item \structure{Example:} For $n=3$, $m=16$, what is the probability?
%     \end{itemize}
%   }
% \end{frame}
%
% \begin{frame}
%   \frametitle{Ad Hoc Example: Probability problems}
%
%   {\smaller
%     \begin{block}{Dice Throwing}
%       If you have $n$ dice, what is the chance of rolling a total above $m$?
%     \end{block}
%
%     \begin{itemize}
%     \item \structure{Example:} For $n=3$, $m=16$, the chance is $10/216$
%
%       \bigskip
%
%     \item All combinations of 3 dice: $6*6*6 = 216$
%     \item Combinations above 16:
%     \end{itemize}
%
%     \begin{columns}[T]
%       \column{0.3\textwidth}
%       \begin{itemize}
%       \item 6,6,6
%       \item 6,6,5
%       \item 6,5,6
%       \item 5,6,6
%       \end{itemize}
%       \column{0.3\textwidth}
%       \begin{itemize}
%       \item 6,5,5
%       \item 5,6,5
%       \item 5,5,6
%       \end{itemize}
%       \column{0.3\textwidth}
%       \begin{itemize}
%       \item 4,6,6
%       \item 6,4,6
%       \item 6,6,4
%       \end{itemize}
%     \end{columns}
%
%     \medskip
%
%     \begin{itemize}
%     \item What algorithm do you use?
%     \end{itemize}
%   }
% \end{frame}
%
% \begin{frame}
%   \frametitle{Ad Hoc example: Probabilty Problems}
%
%   {\smaller
%   \begin{block}{The dice problem}
%     If I have $n$ dice, what is the chance of rolling a total above $m$?
%   \end{block}
%
%   \medskip
%
%   Solving with DP
%
%   \medskip
%
%   \begin{itemize}
%   \item For $n=0$, we have only one result: $r=0$
%   \item For $n=1$, we have 6 results: $r = \{1,2,3,4,5,6\}$
%   \item The result for $n=i$ and $r_{n-1}=k$ is $r_n = k + \{1,2,3,4,5,6\}$
%
%     \bigskip
%
%   \item With a state table (dice,result), we can count the number of
%     dice combination above a certain value;
%
%   \end{itemize}
%   }
% \end{frame}
%
% \begin{frame}[fragile]
%   \frametitle{Ad Hoc example: Probability Problems}
%   \begin{exampleblock}{Example Code}
% {\small
% \begin{verbatim}
% int count(int dice_left, int score_left) {
%    if (score_left < 1) return pow(6,dice);
%    if (dice_left == 0) return 0;
%    if (result[dice_left][score_left] != -1)
%       return result[dice_left][score_left];
%    int sum = 0;
%    for (int i = 0; i < 6; i++)
%       sum += count(dice_left-1, score_left-(i+1))
%    result[dice_left][score_left] = sum;
%    return sum;
% }
%
% prob = count(n,m)/6**n;
%
% \end{verbatim}
% }
%   \end{exampleblock}
% \end{frame}

%% TODO: Cycle Finding (Hare and Tortoise, Halim's Book)
%% TODO: Game Theory (NIM, Halim's Book)

\section{Conclusion}

\begin{frame}{Summary}
  \begin{itemize}
    \item Search Algorithms will check every possible answer in a problem, until they find the solution;
    \item The set of "every possible answer" ({\bf the search space}) depends on the data structure, algorithm, and smart pruning;
    \item The time performance of Search Algorithms depends on the size of the search space;
    \item {\bf Complete Search} takes a lot of time, but it will always find the correct answer (if the search space contains it);
    \item {\bf Binary Search} and {\bf Greedy Search} are much faster, because they discard a large part of the search space. They require special conditions for the problem;
  \end{itemize}
\end{frame}

\subsection{Search Algorithms in Research}
%% TODO: Improve the Search Algorithms in CS research ... Next time!

\begin{frame}
  \frametitle{Search Algorithms in CS Research}

  Search algortihms (including Greedy and Binary search) sound simple, but they have a very important place in CS research.\bigskip

  The key idea of search algorithms is central to the definition of NP-completeness: a solution to an NP-complete problem can be {\bf checked} in polynomial time. This implies that the approach to solve many NP-Complete problems is to define the search space, and systematically check the answers in this space: Just what we're doing!\bigskip

  This also to other problems where we do not have complete information, and/or we do not know efficient algorithms.\bigskip

  There are, of course, more complex approaches to search algorithms:
  \begin{itemize}
    \item Heuristic Search;
    \item Meta-heuristic Search;
  \end{itemize}
\end{frame}

\begin{frame}{Heuristic Search}
  {\bf Heuristic Search} is a search algorithm guided by a {\bf Heuristic function}, which is a function that {\bf estimates} the distance of an answer to the optimal solution.\bigskip

  One famous example of a heuristic algorithm is A* search, which is often used in path-finding and game AI.

  \[\includegraphics[width=.45\textwidth]{../img/wikipedia_pathfinding_astar}\]
  \ppagenote{A* pathing image by dbenzhuser, CC-BY-SA 2.5}
\end{frame}

\begin{frame}
  \frametitle{Meta-heuristic Search}
  Meta-heuristic search are a more general form of heuristic search. While a Heuristic search uses a function that guides the search for {\bf one specific problem}, Meta-heuristic search guides the search towards an {\bf entire category of search spaces}.\bigskip

  Meta-heuristic search are used in real world industrial optimization problems, and are an active area of research.\bigskip

  Some examples of meta-heuristics:
  \begin{itemize}
    \item Evolutionary Algorithms / Genetic Algorithms;
    \item Hill Climbing;
    \item Swarm Algorithms;
    \item Etc...
  \end{itemize}
\end{frame}

\subsection{Problem Discussion}
\begin{frame}{Problem Discussion}

  \[Finally, let's quickly overview the problems for this week: \]
\end{frame}


\begin{frame}
\frametitle{1- Dragon of Loowater}
  A dragon with many heads is attacking the kingdom, and the king is wants to hire knights to slay the dragon.
  \begin{itemize}
    \item You need one knight per head;
    \item A knight can only defeat the head if he is bigger than the head;
    \item A knight charges a reward equals to his size;
  \end{itemize}

  \begin{block}{Input}
    \begin{itemize}
      \item List of Dragon heads and their sizes
      \item List of Knights and their sizes
    \end{itemize}
  \end{block}

  \begin{exampleblock}{Output}
    \begin{itemize}
      \item Find the minimum cost necessary to defeat the dragon;
      \item Or write if it is impossible to defeat the dragon;
    \end{itemize}
  \end{exampleblock}

  {\bf Hint:} Data pre-processing is important here.
\end{frame}

\begin{frame}
  \frametitle{2- Stern-Brocot Number}
  This problem describes a tree-structure that can generate all fractions of rational numbers. For any given fraction, you must find the path in this tree that leads to that fraction.\bigskip

  \begin{block}{Input}
    \begin{itemize}
    \item Two numbers that make a fraction. e.g.: 5, 7;
    \end{itemize}
  \end{block}

  \begin{exampleblock}{Output}
    \begin{itemize}
      The path to that fraction in the tree;
    \end{itemize}
  \end{exampleblock}

  {\bf Hint}: Try to do it by hand a few times!
\end{frame}

\begin{frame}
  \frametitle{3- Bars}

  This is just the famous {\bf knapsack problem}.

  \begin{block}{Input}
    Size of the knapsack, and size of the items.
  \end{block}
  \begin{exampleblock}{Output}
    "YES" if you can solve the knapsack, "NO" if you can't.
  \end{exampleblock}

  {\bf Hint:} The knapsack problem is a permutation problem, so pruning is important! (every item you prune, the search space is cut in half)\\
\end{frame}

\begin{frame}
  \frametitle{4- Rat Attack}
    \begin{block}{}
      \begin{itemize}
      \item You have an $n\times n$ matrix (max 1024) with some rats in each cell;
      \item You have a rat trap (bomb) that kills all rats in a matrix $2d+1 \times 2d+1$
      \item Where can you put the trap to kill the largest number of rats?
      \end{itemize}
    \end{block}\bigskip

    {\bf Hint:} Search all positions for the trap, and counting all the rats at that position will take too much time. But you can change the data structure to reduce this time.
\end{frame}

\begin{frame}
  \frametitle{5- Simple Equations}

  \begin{block}{}
    Given the numbers $A, B, C$, you need to find $x, y, z$ that complete the following equations:
    \begin{itemize}
      \item $x + y + z = A$
      \item $xyz = B$
      \item $x^2 + y^2 + z^2 = C$
    \end{itemize}
  \end{block}

  {\bf Hint:} You need to search all values of $x,y,z$ (triple loop). Pruning is important;\\
  {\bf Hint:} What are the maximum and minimum values of $x,y,z$? Equation "C" is a good place to start.
\end{frame}

\begin{frame}
  \frametitle{6- Through the Desert}
  {\small
    \begin{block}{}
      \begin{itemize}
      \item Simulate a car going through the desert;
      \item Find the least amount of starting fuel needed to win.
      \end{itemize}
    \end{block}

    \begin{itemize}
    \item You could try to calculate the amount of fuel needed in each
      section of the trip (between gas stations);
    \item Or you could just simulate the trip, and make a binary
      search based on the amount of fuel left;
    \end{itemize}
  }
\end{frame}

\begin{frame}
  \frametitle{7- Zones}
  A cellphone company has a plan for $N$ towers, but will only build $M$ of them. ($M \leq N$). You know how many people are served by each tower, and you have to choose the towers that {\bf maximize} the number of people.\bigskip

  {\bf Hint:} This is a "select the maximum" subset problem. Can it be solved by greedy seach?\\
  {\bf Hint:} One big problem is the overlap between the towers. Be careful!
\end{frame}

\begin{frame}
  \frametitle{8- Little Bishops}

  \begin{itemize}
    \item Like 8 queens, but with bishops!
    \item The number of bishops and the size of the board can be very big. Be careful of TLE!
  \end{itemize}
\end{frame}


%%%%%%%%%%%%%%%%%%%%%%%%%%%%%%%%%%%%%%%%%%%%%%%%%%%%
\section{Backmatter}
\begin{frame}{About these Slides}
  These slides were made by Claus Aranha, 2020. You are welcome to copy, re-use and modify this material.
  \bigskip

  Individual images in some slides might have been made by other
  authors. Please see the references in each slide for those cases.
\end{frame}

\begin{frame}[allowframebreaks]{Image Credits}
  \printnotes
\end{frame}

\end{CJK}
\end{document}
