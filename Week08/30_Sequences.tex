
\section{Sequences}
\begin{frame}
  \begin{center}
  {\bf Part III -- Sequences}
  \end{center}
\end{frame}

\begin{frame}
  \frametitle{Sequences}
    Some programming challenges involves the calculation of well known number sequences.\bigskip

    We usually focus this calculation on two forms:
    \begin{itemize}
    \item {\bf Recurrent Form}: The recurrent form of a sequence $F$ calculates $F_n$ based on its antecessor values: $F_{n-1}, F_{n-2},\ldots$.
    \begin{itemize}
      \item Recurrent forms are usually implemented using {\bf Dynamic Programming};\bigskip
    \end{itemize}

    \item {\bf Closed Form}: The closed form of a sequence $F$ calculates $F_n$ {\bf without} using the antecessor values of the sequence.
    \begin{itemize}
      \item Formula for F(n);
    \end{itemize}
    \end{itemize}
\end{frame}

\subsection{Sequence Examples}
\begin{frame}
  \frametitle{Sequence Example: Triangular Numbers}
  \begin{block}{Definition}
    {\bf Triangular Numbers} is the sequence where $T_n$ is the sum of all inegers from $1$ to $n$. Example:\medskip

    $T_1=1, T_2=1+2=3, \ldots, T_7=1+2+3+4+5+6+7=28$\\
    Trivial, right?
  \end{block}\bigskip

  \begin{itemize}
    \item {\bf Recurrent Form:} $T(n) = T(n-1) + n$
    \begin{itemize}
      \item The recurrent form can be calculated with a loop or recursion;
    \end{itemize}\bigskip
    \item {\bf Closed Form:} $T(n) = \frac{n(n+1)}{2}$
    \begin{itemize}
      \item The closed form can be calculated at once;
      \item It can be used to estimate how fast a sequence grows. In this case, $T_n$ is $O(N^2)$
    \end{itemize}
  \end{itemize}
\end{frame}

\begin{frame}
  \frametitle{A more famous sequence: Fibonacci Numbers}

  \begin{block}{Definition}
    The Fibonacci number $F_n$ is the sum of the two numbers before it.\medskip

    $0, 1, 1, 2, 3, 5, 8, 13, 21, 34, \ldots$
  \end{block}\bigskip

  \begin{itemize}
    \item Recurrent Form:
    \begin{itemize}
      \item Starting Values: $F_0 = 0$, $F_1 = 1$
      \item Recurrence: $F_n = F_{n-1} + F_{n-2}$
    \end{itemize}\bigskip

    \item Be careful when implementing recurrences with multiple terms;
    \begin{itemize}
      \item If using recursive functions, {\bf memoization/DP} is necessary to avoid wasted calculation;
      \item In general, each term in a recurrence requires a starting value;
    \end{itemize}
  \end{itemize}
\end{frame}

\begin{frame}[fragile]
  \frametitle{Bonus: Fibonacci Facts}
  \begin{block}{Closed Form for the Fibonacci Numbers:}
    \begin{equation*}
      F(n) = \frac{1}{\sqrt{5}}\left(\left(\frac{1+\sqrt{5}}{2}\right)^n-\left(\frac{1-\sqrt{5}}{2}\right)^n\right)
    \end{equation*}
    The second term of the closed form tends to 0 when $n$ is large!
  \end{block}

  \begin{block}{Pisano's period}
    The last digits of the Fibonacci sequence repeat with a fixed period!\smallskip
{\smaller
\begin{verbatim}
 Digits        | Period      || Digits        | Period
 last digit    | 60 numbers  || last 3 digits | 1500 numbers
 last 2 digits | 300 numbers || last 4 digits | 15000 numbers
F(6)   =                             8
F(66)  =                27777890035288
F(366) =  1380356 ... 8899086435571688
\end{verbatim}}
  \end{block}
\end{frame}

% What are binomial numbers (equations)
% How do we calculate them (closed form and DP)
% Where do we use them? (Number of Choices)

\begin{frame}{Binomial Coefficient}
  \begin{block}{Definition}
    {\bf Binomial Coefficients} are the set of numbers that correspond to the expansion of a binomial:\bigskip

    \begin{itemize}
      \item $B_3 = (a+b)^3 = 1a^3 + 3a^2b + 3ab^2 + b^3 = \{1,3,3,1\}$
      \item $B_5 = (a+b)^5 = 1a^5 + 5a^4b + 10a^3b^2 + 10a^2b^3 + 5ab^4 + 1b^5 = \{1,5,10,10,5,1\}$
    \end{itemize}\bigskip

    Many times, we are interested in the k-th number of the n-binomial,\\ written as $C(n,k)$ or $^nC_k$. Example: $C(5,2) = 10$.
  \end{block}
\end{frame}

\begin{frame}[fragile]{Binomial Coefficient}{Interpretation and Recurrent Form}
  The common interpretation of $C(n,k)$ is "I have to select A or B $n$ times, how many different ways can I choose A $k$ times?"
  \begin{itemize}
    \item How many binary strings with $n$ digits have $k$ ones?
    \item How many paths exist
  \end{itemize}\bigskip

  Using this definition, we can define the recurrent form of the Binomial:
  \begin{itemize}
    \item I have to choose A $k$ times out of $n$
    \begin{itemize}
      \item If I choose A $k-1$ times out of $n-1$, I choose A again.
      \item If I choose A $k$ times out of $n-1$, I choose B.
    \end{itemize}
    \item $C(n,k) = C(n-1,k-1) + C(n-1,k)$
    \item Don't forget to use DP to implement this!
  \end{itemize}\bigskip
\end{frame}

\begin{frame}[fragile]{Pascal's Triangle}
  The recurrent form of the binomials:
  \begin{equation*}
    C(n,k) = C(n-1,k-1) + C(n-1,k)
  \end{equation*}

  Can also be observed by laying out the numbers:

\begin{verbatim}
1
1 1
1 2 1
1 3 3 1
1 4 6 4 1
1 5 10 10 5 1
1 6 15 20 15 6 1
\end{verbatim}
\end{frame}

\begin{frame}{Closed Form for the Binomial}
  The closed form for $C(n,k)$ is:\bigskip

    \begin{equation*}
      C(n,k) = \frac{n!}{(n-k)!k!}
    \end{equation*}\bigskip

  Be careful! As you remember, the value of $n!$ can become very big,
  very fast. It might be better to calculate the binomial using the recurrent
  form, to avoid overflow.
\end{frame}

\begin{frame}{The Catalan Numbers}
  \begin{block}{Motivating Problem}
    Given $n$ pairs of parenthesis, how many different balanced expressions can you create?\bigskip

    \begin{itemize}
      \item n = 0: . = 1
      \item n = 1: () = 1
      \item n = 2: ()(), (()) = 2
      \item n = 3: ((())), ()(()), (())(), (()()), ()()() = 5
      \item n = 4: 14
      \item n = 5: 42
    \end{itemize}
  \end{block}\bigskip

  This sequence is known as the {\bf Catalan Numbers}, and it appears in
  several recursive combinatory problems.
\end{frame}

\begin{frame}{The Catalan Numbers}{Recurrent Form}
  The {\bf Recurrent form} of the catalan number can be derived from the parenthesis definition:\bigskip

  \begin{itemize}
    \item If we define $c_k$ as an expression with k parenthesis, we can break it down into: $c_k = (c_a)c_b$, where $k = a + b + 1$. \medskip

    \item Varying the values of $a$ and $b$, and counting all possible variations gives us the recurrent form:\medskip

    \item $c_{n+1} = \sum_{i=0}^n c_ic_{n-i}$
  \end{itemize}
\end{frame}


\begin{frame}{Closed Form and Usage}
  The closed form of the Catalan Numbers is:
  \begin{equation*}
    c_n = \frac{1}{n+1}C(2n,n)
  \end{equation*}
  Be careful of calculating factorials in $C(2n,n)$\bigskip

  \begin{block}{Other uses of Catalan Numbers}
    \begin{itemize}
    \item Number of ways you can triangulate a poligon with $n+2$ sides;
    \item Number of monotonic paths on an $nxn$ grid that do not pass above
      the diagonal.
    \item Number of distinct binary trees with $n$ vertices
    \item Etc...
    \end{itemize}
  \end{block}
\end{frame}


% TODO: Check if this makes sense
% \begin{frame}
%   \frametitle{Integer Partition}
%   \begin{block}{}
%     f(5,5) = (5),(4,1),(3,2),(3,1,1),(2,2,1),(2,1,1,1),(1,1,1,1,1)
%   \end{block}
%   \begin{block}{Definition and calculation}
%     $f(n,k)$ -- number of ways that we can sum $n$, using integers
%     equal or less than $k$.
%
%     \bigskip
%
%     \structure{Recurrence:}
%     \begin{itemize}
%     \item $f(n,k) = f(n-k,k) + f(n, k+1)$
%     \item $f(1,1) = 1$; $f(n,k) = 0$ if $k > n$
%     \end{itemize}
%   \end{block}
% \end{frame}
