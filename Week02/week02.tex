\documentclass{beamer}

\usepackage{amssymb,amsmath}
\usepackage{graphicx}
\usepackage{url}
\usepackage{color}
\usepackage{relsize}		% For \smaller
\usepackage{url}			% For \url
\usepackage{epstopdf}	% Included EPS files automatically converted to PDF to include with pdflatex
\usepackage{pagenote}[continuous,page]

%For MindMaps
% \usepackage{tikz}%
% \usetikzlibrary{mindmap,trees,arrows}%

%%% Color Definitions %%%%%%%%%%%%%%%%%%%%%%%%%%%%%%%%%%%%%%%%%%%%%%%%%%%%%%%%%
%\definecolor{bordercol}{RGB}{40,40,40}
%\definecolor{headercol1}{RGB}{186,215,230}
%\definecolor{headercol2}{RGB}{80,80,80}
%\definecolor{headerfontcol}{RGB}{0,0,0}
%\definecolor{boxcolor}{RGB}{186,215,230}

%%% Save space in lists. Use this after the opening of the list %%%%%%%%%%%%%%%%
%\newcommand{\compresslist}{
%	\setlength{\itemsep}{1pt}
%	\setlength{\parskip}{0pt}
%	\setlength{\parsep}{0pt}
%}

%\setbeameroption{show notes on top}

% You should run 'pdflatex' TWICE, because of TOC issues.

% Rename this file.  A common temptation for first-time slide makers
% is to name it something like ``my_talk.tex'' or
% ``john_doe_talk.tex'' or even ``discrete_math_seminar_talk.tex''.
% You really won't like any of these titles the second time you give a
% talk.  Try naming your tex file something more descriptive, like
% ``riemann_hypothesis_short_proof_talk.tex''.  Even better (in case
% you recycle 99% of a talk, but still want to change a little, and
% retain copies of each), how about
% ``riemann_hypothesis_short_proof_MIT-Colloquium.2000-01-01.tex''?

\mode<presentation>
{
  % A tip: pick a theme you like first, and THEN modify the color theme, and then add math content.
  % Warsaw is the theme selected by default in Beamer's installation sample files.

  %%%%%%%%%%%%%%%%%%%%%%%%%%%% THEME
  %\usetheme{Madrid}		% No subsection
  \usetheme{AnnArbor}  % Subsection on top, no color


  %\usetheme{Antibes}
  %\usetheme{Bergen}
  %\usetheme{Berkeley}		% bem bacana - menu esquerdo
  %\usetheme{Berlin}
  %\usetheme{Boadilla}
  %\usetheme{boxes}
  %\usetheme{CambridgeUS}		% bem bacana - menu superior
  %\usetheme{Copenhagen}
  %\usetheme{Darmstadt}
  %\usetheme{default}
  %\usetheme{Dresden}
  %\usetheme{Frankfurt}
  %\usetheme{Goettingen}
  %\usetheme{Hannover}		% bem bacana - menu esquerdo
  %\usetheme{Ilmenau}
  %\usetheme{JuanLesPins}
  %\usetheme{Luebeck}
  %\usetheme{Malmoe}
  %\usetheme{Marburg}		% bem bacana - menu direito
  %\usetheme{Montpellier}
  %\usetheme{PaloAlto}		% bem bacana - menu esquerdo
  %\usetheme{Pittsburgh}
  %\usetheme{Rochester}		%bacana
  %\usetheme{Singapore}
  %\usetheme{Szeged}
  %\usetheme{Warsaw}

  %%%%%%%%%%%%%%%%%%%%%%%%%%%% COLOR THEME
  %\usecolortheme{default}		% branco, azul clarinho
  \usecolortheme{crane}		% Very yellow (ok)

  %\usecolortheme{albatross}		% azul escuro, massa
  %\usecolortheme{beetle}		% cinza, menu azul
  %\usecolortheme{dolphin}		% azul e branco, legal
  %\usecolortheme{dove}			% cinza e branco, feio
  %\usecolortheme{fly}			% todo cinza, horrível
  %\usecolortheme{lily}			% parece o default
  %\usecolortheme{orchid}		% azul e branco, ok
  %\usecolortheme{rose}			% branco e violeta-claro, bonito
  %\usecolortheme{seagull}		% cinza, feio
  %\usecolortheme{seahorse}		% nhé, meio feio
  %\usecolortheme{sidebartab}		% Azul, branco, destaque na tab, interessante
  %\usecolortheme{structure}		% bichado
  %\usecolortheme{whale}		% Azul e branco, bem bonito

  %%%%%%%%%%%%%%%%%%%%%%%%%%%% OUTER THEME
  \useoutertheme{default}
  %\useoutertheme{infolines}
  %\useoutertheme{miniframes}
  %\useoutertheme{shadow}
  %\useoutertheme{sidebar}
  %\useoutertheme{smoothbars}
  %\useoutertheme{smoothtree}
  %\useoutertheme{split}
  %\useoutertheme{tree}

  %%%%%%%%%%%%%%%%%%%%%%%%%%%% INNER THEME
  \useinnertheme{circles}
  %\useinnertheme{default}
  %\useinnertheme{inmargin}
  %\useinnertheme{rectangles}
  %\useinnertheme{rounded}

  %%%%%%%%%%%%%%%%%%%%%%%%%%%%%%%%%%%

  \setbeamercovered{invisible} % or whatever (possibly just delete it)
  % To change behavior of \uncover from graying out to totally
  % invisible, can change \setbeamercovered to invisible instead of
  % transparent. apparently there are also 'dynamic' modes that make
  % the amount of graying depend on how long it'll take until the
  % thing is uncovered.

}


% Get rid of nav bar
\beamertemplatenavigationsymbolsempty

% Use short top
%\usepackage[headheight=12pt,footheight=12pt]{beamerthemeboxes}
%\addheadboxtemplate{\color{black}}{
%\hskip0.5cm
%\color{white}
%\insertshortauthor \ \ \ \
%\insertframenumber \ \ \ \ \ \ \
%\insertsection \ \ \ \ \ \ \ \ \ \ \ \ \ \ \ \ \  \insertsubsection
%\hskip0.5cm}
%\addheadboxtemplate{\color{black}}{
%\color{white}
%\ \ \ \
%\insertsection
%}
%\addheadboxtemplate{\color{black}}{
%\color{white}
%\ \ \ \
%\insertsubsection
%}

% Insert frame number at bottom of the page.
% \usefoottemplate{\hfil\tiny{\color{black!90}\insertframenumber}}

%% makes the ppagenote command for figure references at the end.

\usepackage[english]{babel}
%qq\usepackage[latin1]{inputenc}
\usepackage{CJKutf8}
\usepackage{subfigure}

\usepackage{times}
\usepackage[T1]{fontenc}

\makepagenote
\renewcommand{\notenumintext}[1]{}
\newcommand{\ppagenote}[1]{\pagenote[Page \insertframenumber]{#1}}

\title[Programming Challenges]{GB20602 - Programming Challenges}
\author[Claus Aranha]{Claus Aranha\\{\footnotesize caranha@cs.tsukuba.ac.jp}}
\institute[U. Tsukuba]{University of Tsukuba, Department of Computer Sciences}


\subtitle[Week 2]{Week 2 - Data Structures}
\date[2020/5/12]{2020/5/12\\{\smaller(last updated: \today)}}

\begin{document}
\begin{CJK}{UTF8}{ipxm}

\begin{frame}
\maketitle
\vfill

\hfill Version 2020.1
\end{frame}

\section{Introduction}
\subsection{Outline}

\begin{frame}
  \begin{center}
    {\large Lecture 02 -- Data Structures\\Part I -- Introduction}\\
  \end{center}
\end{frame}

\begin{frame}
  \frametitle{Outline}

  When writing any program (and not just programming challenges!) the right data structure makes a great difference in \structure{how easy the program is to write} and \structure{how time and memory efficient the algorithm is}.\bigskip

  \begin{itemize}
    \item In this lecture, we will review some data structures that commonly appear in programming challenges;\medskip

    \item This lecture covers Chapter 2 of the "Competitive Programming" book;\medskip

    \item In this lecture, we focus the {\bf description and implementation} of data structures more than on their theoretical analysis.
  \end{itemize}
\end{frame}

\begin{frame}
  \frametitle{Comments on the last week:}

  \begin{itemize}
    \item Numbers of Problems solved;\medskip

    \item Orders of Problems;\medskip

    \item Lecture Calendar for this week;
  \end{itemize}
  \vfill

  \begin{block}{Motivating Problems}
    To introduce the topics of this class, let's show three problems where the choice of data structure can make a big difference;
  \end{block}
\end{frame}

\subsection{Motivating Problems}


\begin{frame}[fragile]
  \frametitle{Example 1: 8 Queen Problem (UVA 750)}
  \includegraphics[width=0.25\textwidth]{../img/8queen}\\
  \ppagenote{Chessboard by Lee Daniel Crocker. CC-BY-SA 3.0}

  For a board of size $n \times n$, you have to find \alert{how many} safe
  configurations of $n$ queens exist.
  \bigskip

  Because you need to count \alert{how many} configurations exist, it is
  necessary to test {\bf all} valid configurations.

  \bigskip

\begin{verbatim}
for (int i = 0; i < #configurations; i++)
  if configurationIsSafe(i) sum++
return(sum)
\end{verbatim}
\end{frame}

\begin{frame}[fragile]
  \frametitle{Example 1: 8 Queen Problem (UVA 750)}

  Last lecture we talked about how {\bf pruning} can be used to reduce the problem size. This time we review this concept more concretely.\bigskip

  Consider how we store information about all the configurations. Imagine that we have an array, \emph{conf}, which contains all configurations that we want to test.
  \bigskip

  Approach 1: For each queen, we store the pair $(\text{col},\text{row})$.
\begin{verbatim}
conf[0]   = {{a,1}, {a,1}, {a,1}, ... {a,1}, {a,1}}
conf[1]   = {{a,1}, {a,1}, {a,1}, ... {a,1}, {a,2}}
conf[2]   = {{a,1}, {a,1}, {a,1}, ... {a,1}, {a,3}}
  ...
conf[k]   = {{a,1}, {b,2}, {b,2}, ... {c,8}, {d,8}}
conf[k+1] = {{a,1}, {b,2}, {b,2}, ... {c,8}, {e,1}}
  ...
\end{verbatim}

Looping through all options: $n^{n^2}$ steps
\end{frame}

\begin{frame}[fragile]
  \frametitle{Example 1: 8 Queen Problem (UVA 750)}
  Approach 2: We fix each queen on a column (a,b,c,d...). Our data structure only needs to represent the row of each queen. \bigskip

  We store an array of arrays, containing 8 integers representing the row:\bigskip
\begin{verbatim}
conf[0]   = {0,0,0,0,0,0,0,0}
conf[1]   = {0,0,0,0,0,0,0,1}
conf[2]   = {0,0,0,0,0,0,0,2}
  ...
conf[k]   = {0,0,0,3,3,6,7,7}
conf[k+1] = {0,0,0,3,3,7,0,0}
  ...
\end{verbatim}
Looping through all options: $n^n$ steps
\end{frame}

\begin{frame}[fragile]
  \frametitle{Example 1: 8 Queen Problem (UVA 750)}
  Approach 3: We fix each queen on a column (a,b,c,d...), and each configuration
  is a permutation of rows where we place the queens. \bigskip

  We store a string of rows, and each configuration is a permutation accessed using "next\_permutation" function from C++ stl's "algorithm" header. \bigskip

\begin{verbatim}
conf[0] = "01234567"
conf[1] = "01234576"
conf[2] = "01234657"
  ...
\end{verbatim}
\end{frame}

\begin{frame}
  \frametitle{Example 2: The Towers of Hanoi}

  \begin{center}
    \includegraphics[width=0.5\textwidth]{img/hanoi}
  \end{center}
  \medskip

  {\small
    \begin{itemize}
    \item You have $N$ disks and $K$ poles. Each disk has unique size $s_i$.
    \item A disk $i$ can be moved from one pole to another.
    \item A move of disk $i$ to pole $k$ is only valid if $k$ has no disks smaller than $i$
    \item Find the list of moves to move all disks from pole 1 to pole $K$.
    \end{itemize}
  }

  \vfill

  How do you represent the data in this problem?
\end{frame}

\begin{frame}
  \frametitle{Example 2: The Towers of Hanoi}
  A string with ``n'' disks, from smaller to larger.
  \begin{center}
    \includegraphics[width=0.65\textwidth]{img/hanoi_graph}
  \end{center}
  \ppagenote{Tower of Hanoi's graph image by nonenmac}
\end{frame}

\begin{frame}[fragile]
  \frametitle{Example 3: Army Buddies (UVA 12356)}
  \framesubtitle{Problem Description}

  \begin{block}{}
    \begin{itemize}
    \item There is a line of $S$ soldiers: $0,1,2,3,4,...,S$
    \item There are $Q$ queries that remove soldiers from $i$ to $j$:
\begin{verbatim}
Q1: 2,4           (removes soldiers 2, 3, 4)
Q2: 6,7           (removes soldiers 6, 7)
Q3: 1,1           (removes soldier 1)
\end{verbatim}
    \item For each query, list the soldier to the \alert{left} and to the \alert{right}
\begin{verbatim}
A1: 1,5       1 x x x 5 6 7
A2: 5,*       1 - - - 5 x x
A3: *,5       x - - - 5 - -
\end{verbatim}
    \end{itemize}
  \end{block}

  \bigskip

  How do we solve this problem?
\end{frame}

\begin{frame}
  \frametitle{Example 2: Army Buddies (UVA 12356)}
  \framesubtitle{Idea 1: Linked Lists}

  \begin{columns}
    \column{0.5\textwidth}
    For each query, we find the first soldier, and we remove each soldier
    until we find the second soldier.\bigskip

    We use the linked list to reduce the size of the list after each query.
    \column{0.5\textwidth}
    \begin{center}
      \includegraphics[width=1\textwidth]{img/army-list}
    \end{center}
  \end{columns}

  \begin{itemize}
  \item Represent the line as a linked list.
  \item Find the 1\textsuperscript{st} soldier and 2\textsuperscript{nd} soldiers \hfill \structure{($O(n)$ steps)}
  \item Repeat the operation above for each query. \hfill \structure{($O(nm)$ steps)}
  \end{itemize}
  \bigskip

\end{frame}

\begin{frame}
  \frametitle{Example 2: Army Buddies (UVA 12356)}
  \framesubtitle{A solution using linked lists}

  \begin{center}
    \includegraphics[width=0.4\textwidth]{img/army-list}
  \end{center}

  \alert{Problem!} The input is too big, and $O(nm)$ takes too much time.

  \begin{itemize}
  \item $1 \leq S \leq B \leq 10^5$;\hfill \structure{($O(10^5\times 10^5)) = 10^{10}$}
  \item Also \alert{multiple cases};\hfill \structure{($O(n^2k)) = 10^{10}k$}
  \end{itemize}

  \bigskip

  Let's think of a different solution! (Before looking at the next slide)
\end{frame}


\begin{frame}[t]
  \frametitle{Example 2: Army Buddies (UVA 12356)}
  \framesubtitle{A solution using arrays}

  The problem with last solution is that it costs $n$ to search the soldiers. We need to access the sodier position in $O(1)$ using an {\bf index}. We also need to keep track of neighbors when removing soldiers.

  \begin{itemize}
  \item \alert{Idea}: To use {\bf two} Neighbor Arrays
    \begin{itemize}
    \item Let {\bf R} be: {\bf Int} Array of Right neighbors
    \item Let {\bf L} be: {\bf Int} Array of Left neighbors
      \includegraphics[width=0.7\textwidth]{img/army-array}
    \end{itemize}
  \item \alert{Question}: how do we update R and L after query $(r,l)$?\\
  \end{itemize}
\end{frame}

\subsection{Thinking about Data Structures}
\begin{frame}{Motivating Data Structure}
  As you can see, the choice of data structure and problem representation is very important.\bigskip

  \begin{itemize}
    \item Choosing the right data structure:
    \begin{itemize}
      \item Changes the time or memory complexity of the implementation;
      \item Makes the programming task simpler or more complex;
    \end{itemize}\medskip

    \item Hints for programming contests;
    \begin{itemize}
      \item Avoid using pointers (source of bugs, programming overhead);
      \item Prefer multiple variables, instead of complex structs;
      \item In larger programs (not challenges) you want more complex structures;
    \end{itemize}\medskip

    \item Learn the library tools of your language (STL, java.utils, etc);
  \end{itemize}\bigskip

  \hfill {\bf End of part I}
\end{frame}

\section{The Array Data Structure}

\begin{frame}
  \begin{center}
    {\large Lecture 02 -- Data Structures\\Part II -- The Array Data Structure}\\
  \end{center}
\end{frame}

\subsection{Basics}

\begin{frame}
  \frametitle{Introducing the simple array!}

  Arrays are the simplest data structure, but also the ones most often used for programming challenges.

  \bigskip

  {\bf Merits}
  \begin{itemize}
  \item They are easy to implement and manipulate (no pointers);
  \item Random access is usually very fast;
  \item Pointers can be \emph{simulated} using index operations;
  \item Many library functions for array manipulation;
  \end{itemize}

  \bigskip

  {\bf Concerns}
  \begin{itemize}
  \item Inserting many items in the middle of an array can be expensive;
  \end{itemize}
\end{frame}


\begin{frame}[fragile]
  \frametitle{Implementing arrays/vectors (C++)}
  {\small
\begin{verbatim}
#include <vector>

int arr[5] = {7,7,7};     // arr = {7,7,7,0,0}
vector<int> v(5, 5);      // v = {5,5,5,5,5}

int x = arr[2] + v[2];    // x = 12

arr[5] = 5;               // Runtime error
cout << v[7];             // 0 !! Be careful.

v.push_back(6);           // v = {5,5,5,5,5,6}
\end{verbatim}
  }

  \begin{alertblock}{}
    Trying to access indexes outside of an array is a common source of
    Runtime Errors (RTE)
  \end{alertblock}

\end{frame}

\begin{frame}[fragile]
  \frametitle{How do you reset an array?}
  \framesubtitle{Implementation matters}
{\smaller
\begin{verbatim}
#include <vector>
#include <string.h>
vector<int> v(10000,7)

memset(v, 0, 10000*__SIZEOF_INT__);       // Method 1
fill(v.begin(), v.end(), 0);              // Method 2
for (int i = 0; i < 10000; i++) v[i] = 0; // Method 3
v.assign(v.size(), 0);                    // Method 4

Method      |  executable size  |  Time Taken (in sec) |
            |  -O0    |  -O3    |  -O0      |  -O3     |
------------|---------|---------|-----------|----------|
1. memset   | 17 kB   | 8.6 kB  | 0.125     | 0.124    |
2. fill     | 19 kB   | 8.6 kB  | 13.4      | 0.124    |
3. manual   | 19 kB   | 8.6 kB  | 14.5      | 0.124    |
4. assign   | 24 kB   | 9.0 kB  | 1.9       | 0.591    |
\end{verbatim}
}
\end{frame}

\subsection{Sorting and Searching}

\begin{frame}{Operations in Arrays}{Problem Example}

  \begin{block}{Example -- Vito's Family (UVA 10041)}
    Vito wants to move to an address that is closest to his entire family.
  \end{block}
  \bigskip

  {\bf Input:} A list of integers (street addresses):\\
  10, 20, 10, 10, 40, 80, 30, 90, 20, 55, 20
  \bigskip

  {\bf Output:} The address (integer) with \structure{minimal} distance to all others.
  \begin{itemize}
    \item {\bf 10}: $0+10+0+0+30+70+20+80+10+45+10 = 275$
    \item {\bf 40}: $30+20+30+30+0+40+10+50+20+15+20 = 265$
    \item {\bf 20}: $10+0+10+10+20+60+10+70+0+35+0 = 225$
  \end{itemize}
  \bigskip

  Result: 20!\\
  How would you solve this problem?
\end{frame}

\begin{frame}[fragile]{Operations in Arrays}{Problem Example}

  \begin{itemize}
  \item The solution to this problem is to find de {\bf median} address.
  \item 1- \structure{sort the address array}, 2- select the middle value.
  \end{itemize}

\begin{verbatim}
  #include<iostream>
  #include<algorithm>
  using namespace std;
  int main() {
      int n; int add[100];
      cin >> n;
      for (int i=0; i<n; i++) { cin >> add[i]; }

      sort(add, add+n);
      cout << add[n/2] << endl;
  }
\end{verbatim}
\end{frame}

\begin{frame}{Operations in Arrays}{Sorting}
  In the last problem example, we used sorting to calculate the median. In fact, you can {\bf solve many, many problems using sorting}.\bigskip

  Some examples:
  \begin{itemize}
  \item Finding the Highest $n$ values, Finding duplicate values;
    \bigskip

  \item Binary Search ($O(\log n)$)
    \bigskip

  \item Pre-processing data for other algorithms.
  \end{itemize}
\end{frame}

\begin{frame}[fragile]{Operations in Arrays}{The "algorithm" header: sorting and binary search}
{\small
\begin{block}{}
\begin{verbatim}
#include <iostream>
#include <algorithm>
#include <vector>
using namespace std;
int main () {
  int n, t, search; vector<int> v;
  cin >> n >> search;
  for (int i=0; i<n; i++) { cin >> t; v.push_back(t); }

  sort (v.begin(), v.end());
  vector<int>::iterator low,up;
  low = lower_bound (v.begin(), v.end(), search);
  up  = upper_bound (v.begin(), v.end(), search);
  cout << (low-v.begin()) << " and " << (up-v.begin());
}
\end{verbatim}
\end{block}}
\end{frame}


\begin{frame}[fragile]{Operations in Arrays}{Sorting with specific funtions}
{\small
  In some cases, you need to do a complex sort on several variables.
  \begin{block}{}
\begin{verbatim}
#include <algorithm>
#include <vector>
#include <string>
struct team{ string name; int point; int penal;
             team(string _n, int _po, int _pe) :
               name(_n), point(_p), penal(_g){} };

bool cmp(team a, team b) {      % Sorting Function
  if (a.point != b.point) return a.point > b.point;
  if (a.penal != b.penal) return a.penal < b.penal;
  return strcmp(a.name,b.name); }
vector<team> v;
sort(v.begin(), v.end(), cmp); // sort using cmp
reverse(v.begin(), v.end()); // and reverse
\end{verbatim}
\end{block}}
\end{frame}

% \input{11_bitmask.tex}
\section{Library Structures}
\begin{frame}
  \begin{center}
    {\large Lecture 02 -- Data Structures\\Part III -- Data Structures from Libraries}\\
  \end{center}
\end{frame}

\subsection{Visalgo}
\begin{frame}
  \frametitle{Long Live the STL!}

  \begin{itemize}
    \item The standard library implements many data structures that are
      useful for programming contests.
      \bigskip

    \item Let's review a few of them here.
      \bigskip

    \item The website \url{https://visualgo.net/} has good reviews of many
     data structures;
  \end{itemize}

\end{frame}

%%%%%%%%%%%%%%%%%%%%%%%%%%%%%%%%%%%%%%%%%%%%%%%%%%%%%%%%%
\subsection{Deque, Queue, Stack}

%% TODO: Add Motivating Problem For Queue
\begin{frame}
  \frametitle{Deque, Queue, Stack}

  Sometimes you want special access to the \structure{start} or \structure{end}
  of a vector.

  \bigskip

  \begin{itemize}
    \item \structure{stack}: \emph{pop} and \emph{push} from the front;
    \bigskip

    \item \structure{queue}: \emph{pop} from the back, \emph{push} from the front;
    \bigskip

    \item \structure{deque}: \emph{pop\_front, push\_front, pop\_back, push\_back};
  \end{itemize}
  \bigskip

  \begin{block}{Behind C++}
    Actually, \emph{Queue} and \emph{Stack} are high level constructs,
    \structure{List} or \structure{Deque} are used to implement them.
  \end{block}
\end{frame}

\begin{frame}[fragile]
  \frametitle{Queue and Stacks}

  \begin{block}{}
    Queues and Stacks are useful to simplify common cases of vectors
  \end{block}

  Stack Example: Testing if a set of parenthesis is balanced.
{\small
\begin{verbatim}
#include <stack>
stack<char> s;
char c;

while(cin >> c) {
  if (c == '(') s.push(c);
  else {
    if (s.size() == 0) { s.push('*'); break; }
    s.pop();
  }
}
cout << (s.size() == 0 ? "balanced" : "unbalanced");

\end{verbatim}}
\end{frame}


\subsection{Balanced Search Tree (BST) -- Maps and Sets}

\begin{frame}
  \frametitle{Problem Example: CD -- 11849}

  \begin{block}{}
    {\bf Input:}
    \begin{itemize}
    \item Jack CD collection: Up to $10^6$ CDs, with ID up to $10^9$
    \item Jill CD collection: Up to $10^6$ CDs, with ID up to $10^9$
    \end{itemize}

    {\bf Output:}
    \begin{itemize}
    \item How Many CDs are in both Collections?
    \end{itemize}

  \end{block}
\end{frame}

\begin{frame}
  \frametitle{Problem Example: CD -- 11849}

  Naive Solution:

  \begin{enumerate}
  \item Store all IDs in collection 1 in a Vector (n)
  \item Sort the Vector (nlogn)
  \item For each ID in collection 2, test if it exists in Vector with \structure{Binary Search} (nlogn)
  \end{enumerate}

  Total Cost: $n + n\text{log}n + n\text{log}n$

  \bigskip

  Let's use a \structure{MAP} for $O(\log N)$ search using a \structure{balanced search tree}
\end{frame}

\begin{frame}[fragile]
  \frametitle{Solving CD with a MAP (Approximate Solution)}

{\smaller
  \begin{block}{}
\begin{verbatim}
#include <iostream>
#include <set>
using namespace std;

int main() {
    int N, M, num;
    cin >> N >> M;

    set<int> first, second;
    while (N--) { cin >> num; first.insert(num); }
    while (M--) { cin >> num; second.insert(num); }
    int count = 0;
    for (set<int>::iterator iter = first.begin();
         iter != first.end(); ++iter)
      if (second.find(*iter) != second.end())
        ++count;
      cout << count << '\n';
}
\end{verbatim}
\end{block}}
\end{frame}



\subsection{Balanced Search Trees}

\begin{frame}
  \frametitle{Balanced Search Trees}
  \begin{center}
    \includegraphics[width=0.8\textwidth]{img/BST}
  \end{center}
  \begin{itemize}
  \item \emph{Search Trees} Keep items in an ordered relationship.
  \item For example: Left children always have smaller values, Right
    children always have larger values;
  \item Insertion/Search/Deletion in a tree costs $O(h)$, where $h$ is
    the height of the tree;
  \item For a tree with $n$ elements, the \structure{minimum} height
    is $\text{log}n$
  \item For a balanced tree, the \structure{maximum} height is also
    $\text{log}n$
  \item How to keep the tree balanced?
  \end{itemize}
\end{frame}

\begin{frame}
  \frametitle{Balanced Search Trees}
  \framesubtitle{How to keep the tree balanced?}

  There are many Tree implementations/algorithms for keeping an BST
  balanced, and minimizing the tree height efficiently:
  \begin{itemize}
  \item AVL Tree (Adelson-Velskii-Landis);
  \item Red-Black Tree;
  \item B-Tree;
  \item Splay Tree;
  \end{itemize}
  \bigskip

  However, in a programming context (or even day to day life),
  implementing these trees from scratch is \alert{Dangerous}.

  \bigskip

  Luckly, most standard libraries include some implementation of BST.
\end{frame}

\begin{frame}
  \frametitle{ABLs in C++: Map and Set}

  \begin{itemize}
  \item In C++, the \emph{Map} and \emph{Set} classes are implemented
    using BSTs
  \item \emph{Map} Accept Key-value pairs;
  \item \emph{Set} Accepts only Keys;
  \end{itemize}

\end{frame}

\begin{frame}[fragile]
  \frametitle{Using Map in C++}
  {\small
\begin{verbatim}
#include <map>
map<string, int> ages;   ages.clear();

ages["john"] = 40;
ages["billy"] = 39;
ages["andy"] = 29;
ages["steven"] = 42;
ages["felix"] = 33;

// What is the age of andy?
map<string, int>::iterator it = ages.find("andy");
cout << it->second << endl;

// Which names are between "f" and "m" ??
for (map<string, int>::iterator it =
     age.lower_bound("f");              // finds felix
     it != age.upper_bound("m"); it++)  // finds johm
        cout << " " << ((string)it->first).c_str();
\end{verbatim}}
\end{frame}


\begin{frame}[fragile]
  \frametitle{Using Set in C++}
  {\small
\begin{verbatim}
#include <set>
set<int> CDs;
CDs.clear();

// Adding some values
CDs.insert(1000); CDs.insert(999); CDs.insert(1337);
CDs.insert(1313); CDs.insert(100020);

// Testing if a particular value exists (O(logn))
set<int>::iterator f = used_values.find(79);
if (f == used_values.end())
  cout << "not found!\n";
else
  cout << *f;    // Index!
\end{verbatim}}
\end{frame}

\subsection{Hash Tables}
\begin{frame}[fragile]
  \frametitle{Hash Tables}

  \includegraphics[width=0.95\textwidth]{img/hash}

  \begin{itemize}
  \item Insertion and Search: \structure{O(1)} -- \alert{Slow iteration};
  \item C++ library: \emph{std::unordered\_map};
  \item \structure{Hash} parameter -- Defines Collision results.
  \item Learn more about hash tables here: \url{https://visualgo.net/ja/hashtable}
  \end{itemize}
\end{frame}

\section{Hand-made Data Structures}
\begin{frame}
  \begin{center}
    {\large Lecture 02 -- Data Structures\\Part IV -- Hand-made Data Structures}\\
  \end{center}
\end{frame}

\subsection{Motivation}
\begin{frame}
  \frametitle{Hand-making Data Structures}

  \begin{block}{}
    For certain problems, it is necessary to extend existing data structures;
  \end{block}

  \begin{itemize}
    \item Extensions of Arrays and Vectors for complex data;
    \bigskip

    \item New features for indexing and/or searching;
    \bigskip

    \item Express special relationship between data items (ex: graphs)
  \end{itemize}\bigskip

  We will examine two data structures now: UFDS and Segment Tree
\end{frame}

\subsection{Union-Find}
\begin{frame}
  \frametitle{Union-Find Disjoint Set (UFDS)}
  \framesubtitle{Motivating Problem}

  \begin{block}{Network Connections -- UVA793}
    We define a network with $n$ computers. Using the commands "c" and "q", we {\bf set} and {\bf test} the connection between the computers.
    \bigskip

    {\bf Input:} The number of computers $n$, and a sequence of commands:
    \begin{itemize}
    \item c i j -- Make computer $i$ and $j$ connected.
    \item q i j -- Ask if computer $i$ is connected to computer $j$. (yes/no)
    \end{itemize}

    \bigskip
    {\bf Output:} The number of queries (q) with answer "yes", and the number
    of queries with answer "no".
  \end{block}
\end{frame}

\begin{frame}
  \frametitle{Union-Find Disjoint Set (UFDS)}
  \framesubtitle{Motivating Problem -- Naive answer}

  \begin{block}{Neighborhood Graph}
    \begin{itemize}
    \item Initialize an $n\times n$ matrix with zeros.
    \item For every ``c i j'' input, $N_{i,j}$ and $N_{j,i}$ becomes 1.
    \item For every ``q i j'', we perform a breadth first search on the graph.
    \end{itemize}
  \end{block}


  \bigskip

  How good is this solution?
  \begin{itemize}
  \item Cost to insert a new connection: $O(1)$
  \item Cost to check if ``q i j'': $O(V+E)$
  \end{itemize}

  \bigskip We can do better!
\end{frame}

\begin{frame}
  \frametitle{Union-Find Disjoint Set}

  \begin{center}
    \includegraphics[width=.9\textwidth]{img/ufds1}
  \end{center}

  \begin{itemize}
  \item The UFDS keeps \structure{sets of items}, each is represented by a \structure{parent};
  \item When you join two sets \structure{You join their parents};
  \item When you test the parent of an item \structure{You flatten the tree};
  \item Test\_item and Join\_item are both O(1);\hfill \emph{(amortized)}
  \item Visualization: \url{https://visualgo.net/ja/ufds};
  \end{itemize}
\end{frame}

\begin{frame}[fragile]
  \frametitle{UFDS Implementation using Arrays}

  {\small
\begin{verbatim}
int p[MAX], r[MAX];
                             # which groups x belong to?
int find(int x) { return x == p[x] ? x : p[x]=find(p[x]); }

int join(int x, int y) {     # x and y are the same group
    x = find(x), y = find(y);
    if(x != y) {
        if(r[x] < r[y])     { p[x] = y; r[y] += r[x]; }
        else                { p[y] = x; r[x] += r[y]; }
        return 1;
    }
    return 0;
}
void init() { # Initialize each element as separate group
    for(int i = 0; i < MAX; i++)  { p[i] = i; r[i] = 1; }
}
\end{verbatim}
}

\end{frame}

\begin{frame}
  \frametitle{Union Find Disjoint Set}
  \framesubtitle{Problem II -- War}
  {\small
  \begin{block}{}
    From a set of 10k people, some are friends, other are enemies.
    \begin{itemize}
      \item If A,B are friends, and B,C are friends, then A,C are friends
      \item If A,B are friends, and B,C are enemies, then A,C are enemies
      \item If A,B are enemies, and B,C are enemies, then A,C are friends
    \end{itemize}

    {\bf Input:} A series of commands from the set below:
    \begin{itemize}
    \item SetFriends(i,j) \hspace{1.1cm} SetEnemies(i,j)
    \item TestFriends(i,j) \hspace{1cm} TestEnemies(i,j)
    \end{itemize}

    {\bf Output:}
    \begin{itemize}
    \item If a ``SetFriends'' or ``SetEnemies'' is impossible, output ``-1''
    \item For a ``TestFriends'', ``TestEnemies'', output 0 - false, 1 - true
    \end{itemize}
  \end{block}}

\end{frame}

\begin{frame}
  \frametitle{Union Find Disjoint Set}
  \framesubtitle{Problem II -- War}

  This problem is similar to ``Networking'', but now you need to keep track of {\bf TWO} relations: Friends and Enemies.\bigskip

  There are different ways to implement this:
  \begin{itemize}
    \item Create one UFDS for friends, and one UFDS for enemies?
    \item Add a ``friend/enemy'' flag for each person?
  \end{itemize}
  \bigskip

  What other ideas can you think? Which one is easy/hard to implement?
\end{frame}

\subsection{Segment Tree}
\begin{frame}[fragile]
  \frametitle{Range Maximum Query -- RMQ}

  Suppose you have an array of values:
\begin{verbatim}
Value: 18 17 13 19 15 11 20
Index:  0  1  2  3  4  5  6
\end{verbatim}

\bigskip

The \structure{Range Maximum Query} problem asks you to \structure{find
the index with the maximum value} between two indexes:

\begin{itemize}
  \item RMQ(0,0) = 0
  \item RMQ(0,6) = 6
  \item RMQ(1,4) = 3
\end{itemize}

\bigskip

\alert{Naive Method:} loop from $i$ to $j$, find maximum value. $O(nk)$ steps\\
\medskip

But what is the number of {\bf Values (n)} or {\bf Queries (k)} is too big?
\end{frame}

\begin{frame}
  \frametitle{Segment Tree}

  \begin{itemize}
    \item {\bf Basic idea}: Binary tree with the max index in of each subtree.
    \bigskip

    \item {\bf Operation Costs:}
    \begin{itemize}
      \item Creation of the tree: \structure{$O(n)$}
      \item Query of a segment: \structure{$O(\log n)$}
      \item Update of the tree: \structure{$O(\log n)$} \hfill \alert{Important Part}
    \end{itemize}
    \bigskip

    \item There are many implementations. We will show a vector based heap.
  \end{itemize}
\end{frame}

\begin{frame}
  \frametitle{Segment Tree}
  \begin{center}
    \includegraphics[width=1\textwidth]{img/segment_tree}
  \end{center}

  \bigskip

  Segment Tree animation at VISUALGO: \url{https://visualgo.net/en/segmenttree}
\end{frame}

\begin{frame}[fragile]
  \frametitle{Coding the Segment Tree}
  \framesubtitle{Building the Tree -- add data in array "A", index of biggest value is array "st"}
{\smaller
\begin{block}{}
\begin{verbatim}
typedef vector<int> vi; // vector of ints, we will use this a lot here.

class SegmentTree {     // Object-oriented implementation
private: vi st, A;      // st - Index of biggest, A - value of contents

  int left (int p) { return (p<<1); }     // index of left child;
  int right(int p) { return (p<<1) + 1; } // index of right child;
  void build(int p, int L, int R) {       // Build tree in O(n log n)
    if (L == R)
      st[p] = L;                      // At leaf, largest element is the current.
    else {                            // recursive build the branches.
      build(left(p) , L          , (L+R)/2);   // build left branch
      build(right(p), (L+R)/2 + 1, R      );   // build right branch
      int p1 = st[left(p)], p2 = st[right(p)];
      st[p] = (A[p1] <= A[p2]) ? p1 : p2;      // compare branches.
  } }
\end{verbatim}
\end{block}}
\ppagenote{Segment Tree Code from \url{https://github.com/stevenhalim/cpbook-code}}

\end{frame}

\begin{frame}[fragile]
  \frametitle{Coding the Segment Tree}
  \framesubtitle{Query the Tree -- what is the highest value between i and j?}
{\smaller
\begin{block}{rmq(1, 0, n-1, i, j) -- Query from i to j, bounded by L and R}
\begin{verbatim}
int rmq(int p, int L, int R, int i, int j) // O(log n)
{
  if (i >  R || j <  L)
    return -1;    // query range is outside L/R bounds
  if (L >= i && R <= j)
    return st[p]; // query range is inside L/R bounds

  // compute the highest value in the left and right branches
  int p1 = rmq(left(p) , L        , (L+R)/2, i, j);
  int p2 = rmq(right(p), (L+R)/2+1, R      , i, j);

  if (p1 == -1) return p2;   // left segment outside bounds
  if (p2 == -1) return p1;   // right segment outside bounds
  return (A[p1] <= A[p2]) ? p1 : p2; // return highest of left and right
}
\end{verbatim}
\end{block}}
\end{frame}

\begin{frame}[fragile]
  \frametitle{Coding the Segment Tree}
  \framesubtitle{Update the Tree}

{\smaller
\begin{block}{update(1, 0, n-1, i, v) -- update index i to value v}
\begin{verbatim}
int update(int p, int L, int R, int idx, int new_value) {
  int i = idx, j = idx;
  if (i > R || j < L) return st[p];  // if update ouside interval, return value!

  if (L == i && R == j) {            // if update index matches interval:
    A[i] = new_value;     // update the value array
    return st[p] = L;     // update the leaf index.
  }
  int p1, p2;             // Update left and right branches
  p1=update(left(p) , L        , (L+R)/2, idx, new_value);
  p2=update(right(p), (L+R)/2+1, R      , idx, new_value);

  // Update and return index of current node based on branches.
  return st[p] = (A[p1] <= A[p2]) ? p1 : p2;
}
\end{verbatim}
\end{block}}
\end{frame}


\section{Conclusion}
\begin{frame}{Conclusion}
  \begin{itemize}
    \item The choice of data structure and its implementation has great influence in the algorithm used to solve a programming challenge.
    \bigskip

    \item It is important to be familiar with the main data structures of your programming language (arrays, matrices), and their utility functions.\bigskip

    \item For those structures not available in the library, I recommend that you create a "library" of code you have created. It will come in handy time and time again in your career!
  \end{itemize}

\end{frame}

%%%%%%%%%%%%%%%%%%%%%%%%%%%%%%%%%%%%%%%%%%%%%%%%%%%%
\section{Backmatter}
\begin{frame}{About these Slides}
  These slides were made by Claus Aranha, 2020. You are welcome to copy, re-use and modify this material.
  \bigskip

  Individual images in some slides might have been made by other
  authors. Please see the references in each slide for those cases.
\end{frame}

\begin{frame}[allowframebreaks]{Image Credits}
  \printnotes
\end{frame}

\end{CJK}
\end{document}
