\section{Linear Data Structures}

\begin{frame}
  \begin{center}
    {\large Lecture 02 -- Data Structures\\Part II -- Linear Data Structures}\\
  \end{center}
\end{frame}

\subsection{Basics}

\begin{frame}
  \frametitle{The simple array!}
  Arrays are the simplest data structure, but also the most often used.

  \bigskip

  \structure{Merits}
  \begin{itemize}
  \item Easy to implement! No worries about pointers;
  \item Can simulate pointers using index operations;
  \item Many library Functions;
  \end{itemize}

  \bigskip

  \alert{Concerns}
  \begin{itemize}
  \item Reordering many items can be expensive;
  \end{itemize}
\end{frame}


\begin{frame}[fragile]
  \frametitle{Implementing arrays/vectors (C++)}
  {\small
\begin{verbatim}
#include <vector>

int arr[5] = {7,7,7};     // arr = {7,7,7,0,0}
vector<int> v(5, 5);      // v = {5,5,5,5,5}

int x = arr[2] + v[2];    // x = 12

arr[5] = 5;               // Runtime error
cout << v[7];             // 0 !! Be careful.

v.push_back(6);           // v = {5,5,5,5,5,6}
\end{verbatim}
  }

  \begin{alertblock}{}
    Trying to access indexes outside of an array is a common source of
    Runtime Errors (RTE)
  \end{alertblock}

\end{frame}

\begin{frame}[fragile]
  \frametitle{How do you reset an array?}
  \framesubtitle{Implementation matters}
{\small
\begin{verbatim}
#include <vector>
#include <string.h>
vector<int> v(10000,7)

memset(v, 0, 10000*__SIZEOF_INT__);       // Method 1
fill(v.begin(), v.end(), 0);              // Method 2
for (int i = 0; i < 10000; i++) v[i] = 0; // Method 3
v.assign(v.size(), 0);                    // Method 4


Method      |  executable size  |  Time Taken (in sec) |
            |  -O0    |  -O3    |  -O0      |  -O3     |
------------|---------|---------|-----------|----------|
1. memset   | 17 kB   | 8.6 kB  | 0.125     | 0.124    |
2. fill     | 19 kB   | 8.6 kB  | 13.4      | 0.124    |
3. manual   | 19 kB   | 8.6 kB  | 14.5      | 0.124    |
4. assign   | 24 kB   | 9.0 kB  | 1.9       | 0.591    |
\end{verbatim}
}
\end{frame}

\subsection{Sorting and Searching}

\begin{frame}
  \frametitle{Operations in Arrays}

  \begin{block}{Example -- Vito's Family (UVA 10041)}
    {\bf Input:} A list of integers (street addresses):\\

    \smallskip

    10, 20, 10, 10, 40, 80, 30, 90, 20, 55, 20

    \bigskip

    {\bf Output:} The address (integer) with \structure{minimal} distance to all others.
    \begin{itemize}
      \item {\bf 10}: $0+10+0+0+30+70+20+80+10+45+10 = 275$
      \item {\bf 40}: $30+20+30+30+0+40+10+50+20+15+20 = 265$
      \item {\bf 20}: $10+0+10+10+20+60+10+70+0+35+0 = 225$
    \end{itemize}

    \bigskip

    Result: 20!

  \end{block}

  \bigskip

  How do we solve this problem?

\end{frame}

\begin{frame}[fragile]
  \frametitle{Operations in Arrays}

  \begin{itemize}
  \item Solution: Find the \structure{Median} address.
  \item 1- \structure{sort the address array}, 2- select the middle value.
  \end{itemize}

\begin{verbatim}
  #include<iostream>
  #include<algorithm>
  using namespace std;

  int main() {
      int n; int add[100];
      cin >> n;
      for (int i=0; i<n; i++) { cin >> add[i]; }

      sort(add, add+n);

      cout << add[n/2] << endl;
  }
\end{verbatim}
\end{frame}

\begin{frame}
  \frametitle{Using Sorting}
  Sorting can be used for \alert{many, many} things:

  \bigskip

    \begin{itemize}
    \item Finding the Highest $n$ values, Finding duplicate values;
      \bigskip

    \item Binary Search ($O(\log n)$)
      \bigskip

    \item Pre-processing data for other algorithms.
    \end{itemize}

  \bigskip
  Let's do Sorting!
\end{frame}

\begin{frame}[fragile]
  \frametitle{algorithm, sorting and binary search}
{\small
\begin{block}{}
\begin{verbatim}
#include <iostream>
#include <algorithm>
#include <vector>
using namespace std;

int main () {
  int n, t, search; vector<int> v;
  cin >> n >> search;
  for (int i=0; i<n; i++) { cin >> t; v.push_back(t); }

  sort (v.begin(), v.end());
  vector<int>::iterator low,up;
  low = lower_bound (v.begin(), v.end(), search);
  up  = upper_bound (v.begin(), v.end(), search);
  cout << (low-v.begin()) << " and " << (up-v.begin());
}
\end{verbatim}
\end{block}}
\end{frame}


\begin{frame}[fragile]
  \frametitle{Sorting with specific sorting function}
{\small
  Imagine you need to sort by number of points (bigger is best),
  penalty (smaller is best), and name (alphabetical order)

  \begin{block}{}
\begin{verbatim}
#include <algorithm>
#include <vector>
#include <string>
struct team{ string name; int point; int penal;
             team(string _n, int _po, int _pe) :
               name(_n), point(_p), penal(_g){} };

bool cmp(team a, team b) {
  if (a.point != b.point) return a.point > b.point;
  if (a.penal != b.penal) return a.penal < b.penal;
  return strcmp(a.name,b.name); }

vector<team> v;
sort(v.begin(), v.end(), cmp); // sort using cmp
reverse(v.begin(), v.end()); // and reverse
\end{verbatim}
\end{block}}
\end{frame}
