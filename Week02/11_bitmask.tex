\subsection{Bitmask}

\begin{frame}
  \frametitle{A Permutation Problem}
  \begin{itemize}
    \item \structure{Input}: A set of item costs: $c_1, c_2, c_3, \ldots, c_n$, and your money $m$.
    \bigskip

    \item \structure{Output}: The list of indexes of items you can buy.
    \bigskip

    \item \structure{how to do it?}: Test all possible \alert{item permutations}
    \bigskip

    \item But how?
  \end{itemize}
\end{frame}

\begin{frame}[fragile]
  \frametitle{Looping Permutations}

{\small
  \begin{block}{}
\begin{verbatim}
#include <iostream>
#include <vector>
using namespace std;

int main () {
  int n, t, m; vector<int> p;
  cin >> n >> m;
  for (int i=0; i<n; i++)
    { cin >> t; p.push_back(t); }

  for (int i = 0; i < (1 << n); i++) {
      t = 0;
      for (int j = 0; j < n; j++)
          t += (i & (1 << j) ? p[j] : 0);
      if (t == m)
          cout << "found!" << endl;
}}
\end{verbatim}
\end{block}}
\end{frame}

\begin{frame}[fragile]
  \frametitle{Bitmasks}

\begin{block}{}
\begin{verbatim}
1 --> for (int i = 0; i < (1 << n); i++) {
2 --> t += (i & (1 << j) ? p[j] : 0);
\end{verbatim}
\end{block}

\begin{itemize}
  \item Bitmasks are \structure{sets of booleans} using \alert{integers}.
  \bigskip

  \item They are very useful for representing \alert{sets}
  \begin{itemize}
    \item Set Looping;
    \item Set Union (OR);
    \item Set Intersection (AND);
  \end{itemize}
  \bigskip

  \item They are \structure{Very fast too}
  \end{itemize}
\end{frame}

\begin{frame}[fragile]
  \frametitle{Binary Operatons on Bitmasks (2)}
{\smaller

  \begin{itemize}
  \item Multiply/Divide an integer by two :: shift bits left, right
\begin{verbatim}
S                = 34 =  100010
S = S << 1 = S*2 = 68 = 1000100
S = S >> 2 = S/4 = 17 =   10001
S = S >> 1 = S/2 =  8 =    1000
\end{verbatim}
\bigskip

\item Check if the i-th bit is turned on:
\begin{verbatim}
S          = 34       =  100010
j = 3, 1 << j         =  001000
i = 1, 1 << 1         =  000010
                         ------
Tj= S & ( 1 << j)     =  000000  = 0 # 3 is not set
Ti= S & ( 1 << i)     =  000010 != 0 # 1 is set
\end{verbatim}

  \end{itemize}

}
\end{frame}

\begin{frame}[fragile]
  \frametitle{Binary Operations on Bitmasks (2)}
  {\smaller
  \begin{itemize}
  \item To set/turn on the jth item, use bitwise OR operation S |= (1 << j)
\begin{verbatim}
S          = 34       =  100010
j = 3, 1 << j         =  001000
                         ------ OR (S |= 1 << j)
S          = 42       =  101010
\end{verbatim}
\item To set/turn off the jth item, use bitwise AND operation S \&= ~(1 << j)
\begin{verbatim}
S          = 50       =  110010
j = (1<<5)|(1<<3)     =  101000 # unset items 5,3
~j                    =  010111
                         ------
S &= ~(j)             =  010010 # 18
\end{verbatim}
\bigskip

\item There is the \structure{bitset} class, but it cannot be used as an
\structure{array index}.

  \end{itemize}

  }
\end{frame}
