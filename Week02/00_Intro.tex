\section{Introduction}
\subsection{Outline}

\begin{frame}
  \begin{center}
    {\large
    Introduction\bigskip

    CP4 -- Chapter 2.1}
  \end{center}
\end{frame}

\begin{frame}{Motivation}

  Choosing the right data structure is very important when writing a program:\medskip

  \begin{itemize}
    \item How easy is it to program?
    \item How fast are the operations?
    \item Does it have all the abilities necessary for my algorithm?
  \end{itemize}\bigskip

  In this lecture, we review some data structures that are useful for programming challenges.\medskip

  Most of these data structures are available in the standard library. A few you have to
  program by hand.
\end{frame}

\begin{frame}{How to choose a data structure?}
  {Main Operations of a Data Structure}
  Think about which operations you need for your program:\medskip

  \begin{columns}
    \column{0.5\textwidth}
  \begin{itemize}
    \item Inserting new data once;
    \item Inserting new data after acessing;
    \item Accessing data in order;
    \item Accessing data out of order;
    \item Re-ordering data;
  \end{itemize}
  \column{0.5\textwidth}
  \begin{itemize}
    \item Updating data;
    \item Deleting data;
    \item Finding data by position;
    \item Finding data by content;
    \item Summarizing data;
  \end{itemize}
\end{columns}\bigskip

Different data structures will be better or
worse at these operations.\medskip

You want to use {\bf the simplest} data structure that can do what you need.
\end{frame}

\begin{frame}{Data Structures and Standard Libraries}
  Most data structures that we use on programming contests are available in the {\bf Standard Libraries}. It is important to know how to use the standard library of your language.\bigskip

  However, some specialist data structures are not available, so we have to code them by hand. Sometimes, we also have a modified version of a standard DS.\bigskip

  \begin{block}{Personal Code Library}
    When you write many programs, you will discover that you write similar code many times. It is useful to store this code in a "Personal Library" file.
  \end{block}
\end{frame}

\begin{frame}{Topics we are studying today}
\begin{block}{Linear Data Structures}
  The STL array, sorting on arrays,
  searching, Deques and stacks
\end{block}

\begin{block}{Non-Linear Data Structures}
  Priority Queues and Sets/Hashes
\end{block}
\begin{block}{Hand Crafted Data Structures}
  Union-Find Disjoint Set
  % (TODO) DSes for Dynamic Range Query\\
  % (TODO) Bitmask Indexes\\
\end{block}
\end{frame}

\begin{frame}{Topics we are NOT studying today}
  \begin{alertblock}{Data Structure Theory}
    In this lecture we are interested in \emph{remembering, using and implementing} Data Structures that are useful in Programming Challenges. It is important to know them at a theoretical level, but please review the 2nd year DS lecture.
  \end{alertblock}
  \begin{alertblock}{Big Number}
    In the past, we used to have a module on Big Number. Today, just use Python.
  \end{alertblock}
\end{frame}
