\section{Linear Data Structures}

\begin{frame}
  \begin{center}
    {\large
    Part II -- Linear Data Structures (Arrays)\bigskip

    CP4 -- Section 2.2
    }
  \end{center}
\end{frame}

\subsection{Basics}

\begin{frame}{Arrays are your friend}
  Arrays are the simplest data structure, but they are also the most used one.\bigskip

  \begin{itemize}
    \item They are easy to use;
    \item They are very fast;
    \item They have many abilities;
    \item Complex data structures can be programmed as array + special functions;
  \end{itemize}\bigskip

  It is important to know how to use arrays in your programming language.\medskip

  \begin{block}{}
    Sometimes it is better to use a quick array than implement a complex DS!
  \end{block}
\end{frame}


\begin{frame}[fragile]{1D arrays in C++: Basic Usage}

  {\small
\begin{verbatim}
# include <vector.h>

int arr[5] = {7,7,7};  // arr = {7,7,7,0,0} -- Fixed Size
vector<int> v(5, 5);   // v = {5,5,5,5,5}   -- size increases as necessary
int x = arr[2] + v[2]; // x = 12            -- Access them the same way.

arr[5] = 5;            // Runtime Error - Index Out of Bounds
cout << v[7];          // This returns 0! Be careful!

v.push_back(6);        // v = {5,5,5,5,5,6} - Increase the size (double)
\end{verbatim}
  }

  \begin{alertblock}{}
    \alert{\bf Warning:} "Index out of bounds" is a  common source of Runtime Errors (RTE)
  \end{alertblock}

\end{frame}

\begin{frame}[fragile]
  \frametitle{Understand your library: Many ways to do the same thing.}
  \framesubtitle{Example: How do you reseat an array?}
{\smaller
\begin{verbatim}
#include <vector>
#include <string.h>
vector<int> v(10000,7)

memset(v, 0, 10000*__SIZEOF_INT__);       // Method 1
fill(v.begin(), v.end(), 0);              // Method 2
for (int i = 0; i < 10000; i++) v[i] = 0; // Method 3
v.assign(v.size(), 0);                    // Method 4

Method      |  executable size  |  Time Taken (in sec) |
            |  -O0    |  -O3    |  -O0      |  -O3     |
------------|---------|---------|-----------|----------|
1. memset   | 17 kB   | 8.6 kB  | 0.125     | 0.124    |
2. fill     | 19 kB   | 8.6 kB  | 13.4      | 0.124    |
3. manual   | 19 kB   | 8.6 kB  | 14.5      | 0.124    |
4. assign   | 24 kB   | 9.0 kB  | 1.9       | 0.591    |
\end{verbatim}
}
\end{frame}

\subsection{Sorting and Searching}

\begin{frame}{Sorting in Arrays}

{\bf Sorting an array} is a frequent operation in programming challenges:
\begin{itemize}
  \item Take the highest, lowest, median elements;
  \item Pre-computation step in many algorithms (binary search, greedy, etc);
\end{itemize}\bigskip

How do we sort?
\begin{itemize}
  \item Use the sort function from the standard library -- $O(n\log n)$;\\
  \hfill Most cases.\medskip

  \item Make a simple sort algorithm by hand (bubble/selection sort) -- $O(n^2)$\\
  \hfill When you need to sort with special conditions.\medskip

  \item Special sort for specific data (bucket/radix sort) -- $O(n)$.\\
  \hfill Very large amount of data.
\end{itemize}
\end{frame}


\begin{frame}[fragile]{Sorting using the standard library}

\begin{verbatim}
  #include<iostream>
  #include<algorithm>
  #include<vector>

  using namespace std;
  int main() {
      int n, t; vector<int> values;
      cin >> n;

      for (int i=0; i<n; i++) { cin >> t; values.push_back(t); }

      sort(values.begin(), values.end())
      cout << values[n/2] << endl;       // find the median value
  }
\end{verbatim}
\end{frame}

\begin{frame}[fragile]{Sorting with a specific function}

You can define a comparison function to sort in special cases (multiple variables, etc);


\begin{verbatim}
struct team{ string name; int point; int penal;
             team(string _n, int _po, int _pe) :
               name(_n), point(_p), penalty(_g){} };

bool cmp(team a, team b) {      % Sorting Function
  if (a.point != b.point)     return a.point > b.point;
  if (a.penalty != b.penalty) return a.penalty < b.penalty;
  return strcmp(a.name, b.name);
}

vector<team> v;
sort(v.begin(), v.end(), cmp); // sort using cmp
\end{verbatim}
\end{frame}

\begin{frame}[fragile]{Binary Search in Sorted Arrays -- $O(\log n)$}

{\small
\begin{verbatim}
#include <iostream>
#include <algorithm>
#include <vector>
using namespace std;
int main () {
  int n, t, search; vector<int> v;
  cin >> n >> search;  // Find "search" in an array with "n" elements

  for (int i=0; i<n; i++) { cin >> t; v.push_back(t); }
  sort (v.begin(), v.end()); // Need to sort before binary search!

  vector<int>::iterator low, up;
  low = lower_bound (v.begin(), v.end(), search); // lowest index
  up  = upper_bound (v.begin(), v.end(), search); // highest index
  cout << "Between" << (low-v.begin()) << " and " << (up-v.begin());
}
\end{verbatim}}
\end{frame}

\subsection{Counting Sort}
% TODO: Add implementation of bucket sort
\begin{frame}{Special $O(n + k)$ sorting: Counting Sort}
  If we have to sort a large ($n$) quantity of numbers, but there is a small variation ($k$) of values, we can use {\bf counting sort} (or bucket sort).\medskip

  \begin{enumerate}
    \item Let $A$ be the input array, with possible values $K = \{k_0, k_1, \ldots, k_K\}$
    \item Let $F$ be a \emph{cumulative frequency array}. $F[0]$ is the number of times $k_0$ happens in $A$, $F[1] = F[0] +$ number of times $k_1$ happens in $A$, and so on...

    \item Proccess $A$ element by element, starting from the back.
    \item If $A[j] == k_i$, place $A[j]$ in the $F[i]$ position of the new array, and decrease $F[i]$ by 1.
  \end{enumerate}\bigskip

  Counting sort only works when $k$ is not very big.

\end{frame}


\subsection{Stacks, Queues and Deques}
\begin{frame}
  \frametitle{Stacks, Queues and Deques}

  These are special variants on vectors. They are highly optimized for inserting and removing from the start and the end of the array.\bigskip

  \begin{itemize}
    \item {\bf stack}: \emph{push()} and \emph{pop()}: Add and remove from the top of stack; \emph{top()}: Check top of stack; \emph{empty()}: Check if stack is empty.\medskip

    \item {\bf queue}: \emph{push()} and \emph{pop()}: Add to the back, remove from the front; \emph{front()}, \emph{back()}: check front and back; \emph{empty()}
    \medskip

    \item {\bf deque}: \emph{pop\_front(), push\_front(), pop\_back(), push\_back()}; \emph{front(), back(), empty()}
  \end{itemize}
  \bigskip

  \begin{block}{}
    We will use these variants for many algorithms in the future (graphs, geometry). Check your 2nd year Data Structures material for details!
  \end{block}
\end{frame}

\begin{frame}[fragile]{Example of Using Stack}

  {\bf Input}: A string containing "(" and ")". Example: "(())()(()()())"\\
  {\bf Output}: "balanced" or "unbalanced"

{\small
\begin{verbatim}
#include <stack>
stack<char> s;
char c;

while(cin >> c) {
  if (c == '(') s.push(c);
  else {
    if (s.size() == 0) { s.push('*'); break; }
    s.pop();
  }
}
cout << (s.size() == 0 ? "balanced" : "unbalanced");
\end{verbatim}}
\end{frame}

\subsection{Conclusion}

\begin{frame}{More Code Examples}
  The Github repository of the textbook has more code examples:

  \begin{block}{Basic Arrays}
    \url{https://github.com/stevenhalim/cpbook-code/blob/master/ch2/lineards/resizeable_array.cpp}
  \end{block}

  \begin{block}{Sort, Binary Search and Permutation}
    \url{https://github.com/stevenhalim/cpbook-code/blob/master/ch2/lineards/array_algorithms.cpp}
  \end{block}

  \begin{block}{Stack, Queue, Deque}
    \url{https://github.com/stevenhalim/cpbook-code/blob/master/ch2/lineards/list.cpp}
  \end{block}

\end{frame}
