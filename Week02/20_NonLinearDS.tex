\section{Non Linear Data structures}
\begin{frame}
  \begin{center}
    {\large Part III -- Non Linear Data Structures

    CP4 -- Section 2.3
    }
  \end{center}
\end{frame}

\begin{frame}{Non-Linear Data Structures}

  {\bf Non-Linear Data Structures} refer to Data Structures that are implemented as trees (balanced trees, red-black trees, etc).\bigskip

  They often have some useful property, such as being really fast for ordered insertion, deletion or update.\bigskip

  These Data Structures are usually available as part of the {\bf standard libraries}, so today we will just remember them and check how to use them.\bigskip

  \begin{block}{}
    It is still useful to understand how they work. Review your 2nd year DS material, and check the link at the end of this video.
  \end{block}
\end{frame}

\subsection{Priority Queues}

\begin{frame}[fragile]{Priority Queue}
Access the maximum element in $O(1)$, and insert/delete in $O(\log n)$.

\begin{verbatim}
#include <utility>     // pair
#include <queue>       // priority_queue
using namespace std;
typedef pair<int, string> is;

priority_queue<is> pq;
pq.push(make_pair(100, "john")); // insertion in O(log n)
pq.push({10, "billy"});          // alternative way with {}
pq.push({20, "andy"});
pq.push({2000, "grace"});

is first = pq.top(); pq.pop();  // first = (2000, grace)
is second = pq.top(); pq.pop(); // second = (100, john)
\end{verbatim}
\end{frame}

\begin{frame}{Priority Queue}
  Priority Queue is useful for a large variety of problems.\vfill

  We will use PQ in the future for:
  \begin{itemize}
    \item Dijkstra Algorithm
    \item Minimum Spanning Tree (Prim's, Krushkal's Algorithms)
    \item etc...
  \end{itemize}
\end{frame}

\subsection{Sets and Maps}

\begin{frame}{Sets and Maps: Accessing Very Large sets}

  Imagine that we have two very large sets:
  \begin{itemize}
    \item Set size: $10^7$
    \item Element size: $10^{12}$ (\bf UNIQUE)
  \end{itemize}

  If we want to identify the common elements of the two sets, we would need a data structure that can access large amounts of unique data quickly.

  \begin{block}{C++ Data Structures}
    \begin{itemize}
      \item {\bf unordered\_set}: Store unique keys. Search/Insert/Delete in $O(1)$, no order.
      \item {\bf unordered\_map}: Store key/data, Search/Insert/Delete in $O(1)$, no order.
      \item {\bf set}: Store unique keys. Search/Insert/Delete in $O(1)$.
      \item {\bf map}: Store key/data, Search/Insert/Delete in $O(1)$.
    \end{itemize}
  \end{block}
\end{frame}

\begin{frame}[fragile]{Unordered Map/Set Example}

{\small
\begin{verbatim}
#include <unordered_map>
using namespace std;
unordered_map<string, int> mapper;

mapper["john"]   = 78; mapper["billy"]  = 69; mapper["andy"]   = 80;
mapper["steven"] = 77; mapper["felix"]  = 82; mapper["grace"]  = 75;

for (auto &[key, value] : mapper) // Results are in mixed order
  printf("%s %d\n", key.c_str(), value);

printf("steven's score is %d\n", mapper["steven"]);
if (mapper.find("andy") == mapper.end())
  printf("not found\n");

mapper.clear();
\end{verbatim}}
\end{frame}


\begin{frame}[fragile]{Map/Set Example}

{\smal
\begin{verbatim}
#include <set>
using namespace std;
set<int> uv;

uv.insert(78); uv.insert(69); uv.insert(80);
uv.insert(77); uv.insert(82); uv.insert(75);
printf("%d\n", *uv.find(77));  // O(log n) search

for (auto it = uv.begin(); it != uv.lower_bound(77); it++)
  printf("%d,", *it); // returns [69, 75] (before 77)
for (auto it = uv.lower_bound(77); it != uv.end(); it++)
  printf("%d,", *it); // returns [77, 78, 80, 81, 82]

used_values.clear();
\end{verbatim}}
\end{frame}

\begin{frame}{Some Extra Code}

See more examples in the textbook Github.

\begin{block}{Priority Queue}
\url{https://github.com/stevenhalim/cpbook-code/blob/master/ch2/nonlineards/priority_queue.cpp}
\end{block}

\begin{block}{Maps and Sets}
\url{https://github.com/stevenhalim/cpbook-code/blob/master/ch2/nonlineards/unordered_map_unordered_set.cpp}

\url{https://github.com/stevenhalim/cpbook-code/blob/master/ch2/nonlineards/map_set.cpp}
\end{block}

Visualization of Data Structures: 
\url{https://visualgo.net/}
\end{frame}
