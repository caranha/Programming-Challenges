\section{introduction}

\begin{frame}
  \begin{center}
    {\bf Part I -- Graph Introduction}
  \end{center}
\end{frame}

\begin{frame}
  \frametitle{Graph Algorithms: Week 5 and 6}
  \begin{block}{Graphs Part I (This Week)}
    \begin{itemize}
    \item Graphs Data Structure;
    \item Depth First Search and Breadth First Search;
    \item Graph Search Problems (DFS and BFS);
    \item Minimum Spanning Tree: Kruskal and Prim Algorithms;
    \end{itemize}
  \end{block}
  \begin{block}{Graphs Part II (Next Week)}
    \begin{itemize}
    \item Single Sourse Shortest Path (Djikstra);
    \item All Pairs Shortest Path (Floyd-Warshall);
    \item Network Flow;
    \item Bipartite Graph Matching;
    \end{itemize}
  \end{block}
\end{frame}

\subsection{Definitions}
\begin{frame}
  \frametitle{What is a graph?}

  \begin{columns}[T]
    \column{0.6\textwidth}
    \begin{block}{}
      A graph $G = \{V,E\}$ is composed of a set of {\bf vertices} $V$, which are connected to a set of {\bf edges} $E$. Each edge connects exactly two vertices.
    \end{block}\bigskip

    \begin{itemize}
    \item An edge can be {\bf directed} or {\bf undirected};
    \item An edge or a vertice can have {\bf weights} or {\bf labels};
    \item {\bf Self-edge}: edge between $v_i$ and $v_i$;
    \item {\bf Multi-edge}: two edges with same end-vertices;
    \item A graph can be {\bf connected} or {\bf disconnected};
    \end{itemize}

    \column{0.4\textwidth}
    \begin{tikzpicture}[scale=.8,auto,swap]
      \tikzset{edge/.style = {->,>=latex'}}
      \node[vertex] (a) at (0,0) {};
      \node[vertex] (b) at (2,3) {};
      \node[vertex] (c) at (4,2) {};
      \node[vertex] (d) at (4,0) {};
      \draw[edge] (a) to (b);
      \draw[edge] (a) to[bend left] (c);
      \draw[edge] (c) to[bend left] (a);
      \draw[edge] (d) to (c);
      \draw[edge] (d) to[loop left] (d);
      \draw[edge] (d) to[loop right] (d);
    \end{tikzpicture}

    \vspace{.5cm}

    \begin{tikzpicture}[scale=1,auto,swap]
      \node[vertex] (a) at (0,2) {a};
      \node[vertex] (b) at (0,0) {b};
      \node[vertex] (c) at (2,1) {c};
      \node[vertex] (d) at (4,0) {d};
      \node[vertex] (e) at (4,1) {e};
      \draw[edge] (a) -- node[weight] {$7$} (b);
      \draw[edge] (b) -- node[weight] {$-2$} (c);
      \draw[edge] (c) -- node[weight] {$3$} (a);
      \draw[edge] (c) -- node[weight] {$5$} (d);
    \end{tikzpicture}

  \end{columns}
\end{frame}

\begin{frame}{Graphs in Computer Science}
  Graph Data structures show relationships between data;\\
  They are used in many problems:\bigskip

  \begin{itemize}
    \item Geography and Maps;
    \begin{itemize}
      \item Pathing between locations;
      \item Cycles and Tours;
    \end{itemize}
    \item Human Networks;
    \begin{itemize}
      \item Social Networks;
      \item Citation Clusters;
    \end{itemize}
    \item State Machines;
    \begin{itemize}
      \item Program Pipelines;
      \item Library Requirements;
    \end{itemize}
    \item Natural Language;
    \begin{itemize}
      \item Graph Grammars;
    \end{itemize}
  \end{itemize}
\end{frame}


\begin{frame}{Common graph tasks in an algorithm}
  \begin{itemize}
    \item Test if a path exist between vertice $V_i$ and $V_j$ (test if they are {\bf connected})
    \item Test the shortest path between vertice $V_i$ and $V_j$
    \begin{itemize}
      \item With or without weights
      \item Test if there is more than one path
    \end{itemize}
    \item Add or remove vertices or edges from a graph;
    \item Test some characteristics of a graph;
    \begin{itemize}
      \item Longest path? Shortest path?
      \item Does it have a {\bf Cycle}?
      \item Vertice with maximum number of vertices?
      \item etc...
    \end{itemize}
  \end{itemize}
\end{frame}

\subsection{Example Problem}

\begin{frame}[fragile]{Programming Challenge Example}{Dominator}
  \begin{block}{}
    Definition: A vertice $V_i$ {\bf dominates} $V_j$ if all paths $V_0 \to V_j$ must include $V_i$.
    \begin{itemize}
      \item {\bf input}: A directed graph $\{V,E\}$;
      \item {\bf output}: A table with the DOMINATE relationship
    \end{itemize}
  \end{block}
\begin{columns}[T]
  \column{0.4\textwidth}
    \begin{center}
      \begin{tikzpicture}[scale=1.3,auto,swap]
  \tikzset{edge/.style = {->,>=latex'}}
  \node[vertex] (a) at (0,0) {0};
  \node[vertex] (b) at (1,1) {1};
  \node[vertex] (c) at (1,-1) {2};
  \node[vertex] (d) at (2,0) {3};
  \node[vertex] (e) at (3,0) {4};
  \draw[edge] (a) to (b);
  \draw[edge] (a) to (c);
  \draw[edge] (b) to (d);
  \draw[edge] (c) to (d);
  \draw[edge] (d) to (e);
\end{tikzpicture}

    \end{center}
      \column{0.3\textwidth}
\begin{verbatim}
Input:
       5
       0 1 1 0 0
       0 0 0 1 0
       0 0 0 1 0
       0 0 0 0 1
       0 0 0 0 0
\end{verbatim}
\column{0.3\textwidth}
{\smaller
\begin{verbatim}
Output:
0 -> 0, 1, 2, 3, 4
1 -> 1
2 -> 2
3 -> 3, 4
4 -> 4
\end{verbatim}}
  \end{columns}
\end{frame}

\begin{frame}[fragile]{Programming Challenge Example}{Dominator}
  \begin{block}{}
    \begin{itemize}
      \item Which data structure should be used?
      \item How to calculate the "DOMINATE" status of a vertice?
    \end{itemize}
  \end{block}
  \begin{columns}[T]
    \column{0.4\textwidth}
      \begin{center}
        \begin{tikzpicture}[scale=1.3,auto,swap]
  \tikzset{edge/.style = {->,>=latex'}}
  \node[vertex] (a) at (0,0) {0};
  \node[vertex] (b) at (1,1) {1};
  \node[vertex] (c) at (1,-1) {2};
  \node[vertex] (d) at (2,0) {3};
  \node[vertex] (e) at (3,0) {4};
  \draw[edge] (a) to (b);
  \draw[edge] (a) to (c);
  \draw[edge] (b) to (d);
  \draw[edge] (c) to (d);
  \draw[edge] (d) to (e);
\end{tikzpicture}

      \end{center}
        \column{0.3\textwidth}
\begin{verbatim}
Input:
       5
       0 1 1 0 0
       0 0 0 1 0
       0 0 0 1 0
       0 0 0 0 1
       0 0 0 0 0
\end{verbatim}
\column{0.3\textwidth}
{\smaller
\begin{verbatim}
Output:
0 -> 0, 1, 2, 3, 4
1 -> 1
2 -> 2
3 -> 3, 4
4 -> 4
\end{verbatim}}
    \end{columns}
\end{frame}

\subsection{Graph Data Structure}


\begin{frame}[fragile]{Data Structure for Graph 1}

\begin{block}{Adjacency Matrix: stores the connection between vertices}
\begin{verbatim}
int adj[100][100];

for (int i = 0; i < n; i++)
  for (int j = 0; i < n; j++)
    cin >> adj[i][j]; // 0 if no edge, 1 if edge
\end{verbatim}

  \begin{itemize}
    \item \structure{Pros}:
    \begin{itemize}
      \item Easy to program;
      \item Access to edge $e_{ij}$ is quick;
    \end{itemize}
    \item \alert{Cons}:
      \begin{itemize}
      \item Cannot store multigraph;
      \item Wastes memory with sparse graphs;
      \item Time $O(V)$ to calculate number of neighbors of vertice $v_i$;
      \end{itemize}
    \end{itemize}
  \end{block}
\end{frame}

\begin{frame}[fragile]{Data Structure for Graph 2}

  \begin{block}{Adjacency List: stores edge list for each Vertex}
      {\smaller
\begin{verbatim}
typedef pair<int,int> edge;           // pair: <neighbor, weight>
typedef vector<edge> neighb;          // all neighbors of V_i
vector<neighb> AdjList;               // all V_i
int e;
for (int i = 0; i < n; i++)
  for (int j = 0; j < n; j++)
    cin >> e;
    if (e == 1) { AdjList[i].push_back(pair(j,1)); }
\end{verbatim}}

\begin{itemize}
  \item {\bf Pro}:
    \begin{itemize}
      \item Memory efficient if the graph is sparse;
      \item Can store multigraph;
    \end{itemize}
  \item {\bf Cons}:
  \begin{itemize}
    \item $O(\log(V))$ to test if two vertices are adjacent; (QUIZ: Why log(V)?)
  \end{itemize}
\end{itemize}
  \end{block}
\end{frame}

\begin{frame}[fragile]{Data Structure for Graph 3}
  \begin{block}{Edge List}
    {\smaller
    \begin{verbatim}
    pair <int,int> edge; // Edge between i and j
    vector<pair <int,edge>> Elist; // All edges;

    int e;
    for (int i = 0; i < n; i++)
      for (int j = 0; j < n; j++)
        cin >> e;
        if (e == 1) Elist.push_back(pair(1, pair(i,j)));
    \end{verbatim}}
    \begin{itemize}
      \item Not very common, used in specialized algorithms (ex:MST);
      \item To find if two vertices are neighbors, list must be sorted;
    \end{itemize}
  \end{block}
\end{frame}


\subsection{Graph Search: BFS, DFS}
\begin{frame}
  \frametitle{Graph Search: BFS and DFS}
    \begin{itemize}
    \item Graph Search Question: from vertice $v_s$, can we reach $v_e$?
    \item Many graph algorithms start from a graph search;
    \item Two basic algorithms for search: BFS, DFS;
    \end{itemize}

  \begin{block}{Depth First Search -- DFS}
    \begin{itemize}
      \item Visit the first edge available;
      \item Vertice order is not guaranteed;
      \item Easy to implement with recursion or stack;
    \end{itemize}
  \end{block}

  \begin{block}{Breadth First Search -- BFS}
    \begin{itemize}
      \item First visit the vertices close to the starting point;
      \item Place new vertices on a list, and visit them with a loop;
    \end{itemize}
  \end{block}
\end{frame}

\begin{frame}
  \frametitle{BFS and DFS: Visualize the difference}
  \begin{columns}[t]
    \column{0.5\textwidth}
    \begin{exampleblock}{DFS}
      \vspace{0.1cm}
      \begin{center}
      \begin{tikzpicture}[scale=1.3,auto,swap]
        \node[vertex] (a) at (0,0) {};
        \node[vertex] (b) at (2,0) {};
        \node[vertex] (c) at (0,-1) {};
        \node[vertex] (d) at (1,-2) {};
        \node[vertex] (e) at (0,-2) {};
        \node[vertex] (f) at (2,-2) {};
        \node[vertex] (g) at (2,-3) {};
        \node[vertex] (h) at (1,-4) {};
        \draw[edge] (a) to (b);
        \draw[edge] (a) to (c);
        \draw[edge] (a) to (f);
        \draw[edge] (b) to (f);
        \draw[edge] (c) to (d);
        \draw[edge] (c) to (e);
        \draw[edge] (f) to (h);
        \draw[edge] (e) to (h);
        \draw[edge] (f) to (g);
        \draw<2->[red edge] (a) to (b);
        \draw<3->[red edge] (b) to (f);
        \draw<4->[red edge] (f) to (g);
        \draw<5->[red edge] (f) to (h);
        \draw<6->[red edge] (h) to (e);
        \draw<7->[red edge] (e) to (c);
        \draw<8->[red edge] (c) to (d);
      \end{tikzpicture}
      \end{center}
      \vspace{0.1cm}
    \end{exampleblock}
    \column{0.5\textwidth}
    \begin{block}{BFS}
      \vspace{0.1cm}
      \begin{center}
      \begin{tikzpicture}[scale=1.3,auto,swap]
        \node[vertex] (a) at (0,0) {};
        \node[vertex] (b) at (2,0) {};
        \node[vertex] (c) at (0,-1) {};
        \node[vertex] (d) at (1,-2) {};
        \node[vertex] (e) at (0,-2) {};
        \node[vertex] (f) at (2,-2) {};
        \node[vertex] (g) at (2,-3) {};
        \node[vertex] (h) at (1,-4) {};
        \draw[edge] (a) to (b);
        \draw[edge] (a) to (c);
        \draw[edge] (a) to (f);
        \draw[edge] (b) to (f);
        \draw[edge] (c) to (d);
        \draw[edge] (c) to (e);
        \draw[edge] (f) to (h);
        \draw[edge] (e) to (h);
        \draw[edge] (f) to (g);
        \draw<2->[red edge] (a) to (b);
        \draw<3->[red edge] (a) to (f);
        \draw<4->[red edge] (a) to (c);
        \draw<5->[red edge] (f) to (g);
        \draw<6->[red edge] (f) to (h);
        \draw<7->[red edge] (c) to (d);
        \draw<8->[red edge] (c) to (e);
      \end{tikzpicture}
      \end{center}
      \vspace{0.1cm}
    \end{block}
  \end{columns}
\end{frame}

\begin{frame}[fragile]
  \frametitle{DFS Implementation}
  \begin{exampleblock}{DFS (Using Adjacency List)}
\begin{verbatim}
vector<int> dfs_vis; // visited nodes, init to 0

void dfs(int v) {
   dfs_vis[v] = 1;
   for (int i; i < AdjList[v].size(); i++)
   {
      edge u = AdjList[v][i]; // u = neighb, weight
      // do something...
      if (dfs_vis[u.first] == 0)
         dfs(v.first);
   }
}
dfs(start_vertice);
\end{verbatim}
  \end{exampleblock}
\end{frame}

\begin{frame}[fragile]
  \frametitle{BFS Implementation}
  \begin{exampleblock}{BFS (Using adjacency List)}
\begin{verbatim}
vector<int> bfs_vis;   // visited nodes; init to 0
queue<int> q;          // list of vertices to visit;
q.push(start_vertice); // Start BFS

while(!q.empty()) {
  int u = q.front(); q.pop(); bfs_vis[u] = 1;
  // Do something...
  for (int i = 0; i < AdjList[v].size(); i++) {
    edge e = AdjList[v][i];
    if (bfs_vis[e.first] == 0)   // Check if node is visited
      q.push(e.first);
  }
}
\end{verbatim}
  \end{exampleblock}
\end{frame}

\begin{frame}{BFS and DFS}{Computational Cost}

  In the full BFS and DFS, you need to check every vertice and every edge in the graph:\bigskip

  \begin{itemize}
    \item A BFS/DFS implemented with {\bf Adjacency List}, costs $O(V+E)$.\bigskip

    \item A BFS/DFS implemented with {\bf Adjacency Matrix}, costs $O(V^2)$.
    \begin{itemize}
      \item That's because to visit every edge of a vertice in an Adjacency Matrix, it costs $O(V)$.
    \end{itemize}\bigskip

    \item Adjacency List is faster, {\bf if the graph is sparse (has few edges)}
  \end{itemize}
\end{frame}

\subsection{Solving the Dominator Problem}

\begin{frame}[fragile]{Solving the Dominator Problem with DFS}
  \begin{exampleblock}{}
    \begin{itemize}
      \item $v_j$ is dominated by $v_i$, if all paths from $v_0$ to $v_j$ pass through $v_i$;
      \item In other words, you cannot access $v_j$ from $v_0$, if $v_i$ is not available;
      \item {\bf Algorithm:} Remove $v_i$, and test if you can access $v_j$ from $v_0$;
    \end{itemize}
  \end{exampleblock}\bigskip

  \begin{tikzpicture}[scale=1.3,auto,swap]
  \tikzset{edge/.style = {->,>=latex'}}
  \node[vertex] (a) at (0,0) {0};
  \node[vertex] (b) at (1,1) {1};
  \node[vertex] (c) at (1,-1) {2};
  \node[vertex] (d) at (2,0) {3};
  \node[vertex] (e) at (3,0) {4};
  \draw[edge] (a) to (b);
  \draw[edge] (a) to (c);
  \draw[edge] (b) to (d);
  \draw[edge] (c) to (d);
  \draw[edge] (d) to (e);
\end{tikzpicture}

\end{frame}

\begin{frame}[fragile]{Solving the Dominator Problem with DFS}{Use DFS/BFS N times}

  \begin{columns}
    \column{0.4\textwidth}
    \begin{tikzpicture}[scale=1.3,auto,swap]
  \tikzset{edge/.style = {->,>=latex'}}
  \node[vertex] (a) at (0,0) {0};
  \node[vertex] (b) at (1,1) {1};
  \node[vertex] (c) at (1,-1) {2};
  \node[vertex] (d) at (2,0) {3};
  \node[vertex] (e) at (3,0) {4};
  \draw[edge] (a) to (b);
  \draw[edge] (a) to (c);
  \draw[edge] (b) to (d);
  \draw[edge] (c) to (d);
  \draw[edge] (d) to (e);
\end{tikzpicture}

    \column{0.6\textwidth}

    \begin{exampleblock}{}
{\smaller
\begin{verbatim}
// Modified DFS: does not visit vertex v_i;
boolean DFS2(S,i) {...};

// initialization: which nodes v_0 can reach?
DFS2(0,-1);
for (int j = 0; j < N; j++)
  if (VISITED[j]) { DOMINATED[0][j] = 1; }

// check DOMINATED relationship of each v_i
for (int i = 1; i < N; i++) {
  memset(VISITED,0,sizeof(VISITED));
  DFS2(0,i);
  for (int j = 0; j < N; j++)
    if (!VISITED[j] && DOMINATED[0][j])
      DOMINATED[i][j] = 1;
}
\end{verbatim}}
    \end{exampleblock}
  \end{columns}
\end{frame}


%%%%%%%%%%%%%%%%%%%%%%%%%%%%%%%%%%%%%%%%%%%%%%%%%%%%%%%%%%%%%%%%%%%%%%%%%%%%%%%
%% \subsection{Graph Terms}
%% \begin{frame}
%%   \frametitle{Quick Review of Graph Terms (1)}
%%   {\smaller
%%     \begin{block}{}
%%       You probably know all of these. If not, ask questions!
%%     \end{block}

%%     \begin{columns}[T]
%%       \column{0.6\textwidth}
%%       \begin{itemize}
%%       \item A Graph $G$ is made of a set of \structure{vertices} $V$
%%         and \structure{edges} $E$.
%%       \item Edges can be \structure{directed}\\ (has source and destination vertices);
%%       \item Edges can be \structure{weighted} or not\\ (all weigths = 1);
%%       \item Sets of nodes can be \structure{connected}\\ or \structure{disconnected}
%%       \item Directed Graphs can be \structure{Strongly Connected}
%%       \item Edges can be \structure{self-edges}, and/or \structure{multiple edges}
%%       \end{itemize}
%%       \column{0.4\textwidth}

%%       \begin{tikzpicture}[scale=.8,auto,swap]
%%         \tikzset{edge/.style = {->,>=latex'}}
%%         \node[vertex] (a) at (0,0) {};
%%         \node[vertex] (b) at (2,3) {};
%%         \node[vertex] (c) at (4,2) {};
%%         \node[vertex] (d) at (4,0) {};
%%         \draw[edge] (a) to (b);
%%         \draw[edge] (a) to[bend left] (c);
%%         \draw[edge] (c) to[bend left] (a);
%%         \draw[edge] (d) to (c);
%%         \draw[edge] (d) to[loop left] (d);
%%         \draw[edge] (d) to[loop right] (d);
%%       \end{tikzpicture}

%%       \vspace{.5cm}

%%       \begin{tikzpicture}[scale=1,auto,swap]
%%         \node[vertex] (a) at (0,2) {a};
%%         \node[vertex] (b) at (0,0) {b};
%%         \node[vertex] (c) at (2,1) {c};
%%         \node[vertex] (d) at (4,0) {d};
%%         \node[vertex] (e) at (4,1) {e};
%%         \draw[edge] (a) -- node[weight] {$7$} (b);
%%         \draw[edge] (b) -- node[weight] {$-2$} (c);
%%         \draw[edge] (c) -- node[weight] {$3$} (a);
%%         \draw[edge] (c) -- node[weight] {$5$} (d);
%%       \end{tikzpicture}

%%     \end{columns}
%%   }
%% \end{frame}

%% \begin{frame}
%%   \frametitle{Quick Review of Graph Terms (2)}

%%   {\smaller
%%     \begin{block}{}
%%       You probably know all of these. If not, ask questions!
%%     \end{block}

%%     \begin{columns}[T]
%%       \column{0.6\textwidth}
%%       \begin{itemize}
%%       \item A \structure{path} is a set of vertices connected by edges;
%%       \item A \structure{cycle} is a path with first and last vertices identical;
%%       \item \structure{Labelled} graphs and \structure{Isomorphic} graphs;
%%       \item A \structure{tree} is a acyclical, undirected graph;
%%       \item A \structure{spanning tree} is a subset of edges from E'
%%         that form a tree, connecting all nodes $V \in G$;
%%       \item A \alert{spamming tree} houses very noisy insects in summer;
%%       \end{itemize}
%%       \column{0.4\textwidth}
%%       \begin{tikzpicture}[scale=1,auto,swap]
%%         \node[vertex] (s) at (0,0) {a};
%%         \node[vertex] (a1) at (-1,-1) {b};
%%         \node[vertex] (a2) at (1,-1) {c};
%%         \node[vertex] (b1) at (-1,-2) {d};
%%         \node[vertex] (b2) at (0,-2) {e};
%%         \draw[edge] (s) to (a1);
%%         \draw[edge] (s) to  (a2);
%%         \draw[edge] (a1) to  (b1);
%%         \draw[edge] (a1) to  (b2);
%%         \draw[black edge] (b1) to (b2);
%%         \draw[black edge] (a2) to (b2);
%%       \end{tikzpicture}

%%       \vspace{.5cm}

%%       \begin{tikzpicture}[scale=1,auto,swap]
%%         \node[vertex] (s) at (0,0) {e};
%%         \node[vertex] (a1) at (-1,-1) {d};
%%         \node[vertex] (a2) at (1,-1) {b};
%%         \node[vertex] (b1) at (-1,-2) {c};
%%         \node[vertex] (b2) at (0,-2) {a};
%%         \draw[edge] (s) to (a1);
%%         \draw[edge] (s) to  (a2);
%%         \draw[edge] (a1) to  (b1);
%%         \draw[edge] (a1) to  (b2);
%%         \draw[black edge] (b1) to (b2);
%%         \draw[black edge] (a2) to (b2);
%%       \end{tikzpicture}
%%   \end{columns}}

%% \end{frame}

%% \begin{frame}
%%   \frametitle{Quick Review of Graph Terms (3)}

%%   {\smaller
%%     \begin{block}{}
%%       You probably know all of these. If not, ask questions!
%%     \end{block}

%%     \begin{columns}[T]
%%       \column{0.6\textwidth}
%%       \begin{itemize}
%%       \item The \structure{degree} of a node is the number of edges
%%         connected to it;
%%       \item Directed graphs have \structure{in-degrees} and
%%         \structure{out-degrees};
%%       \item A \structure{bipartite} graph can be divided in two sets
%%         of unconnected vertices;
%%       \item A \structure{Match} or \structure{Pairing} is a set of
%%         edges that connects the nodes in the bipartite graph;
%%       \end{itemize}
%%       \column{0.4\textwidth}
%%       \begin{tikzpicture}[scale=.8,auto,swap]
%%         \node[vertex] (a) at (0,0) {deg: 2};
%%         \node[vertex] (b) at (2,1) {};
%%         \node[vertex] (c) at (2,-1) {};
%%         \node[vertex] (d) at (4,0) {in,out:1};
%%         \draw[edge] (a) to (b);
%%         \draw[edge] (a) to (c);
%%         \tikzset{edge/.style = {->,>=latex'}}
%%         \draw[edge] (c) to (d);
%%         \draw[edge] (d) to (b);
%%       \end{tikzpicture}

%%       \vspace{.5cm}

%%       \begin{tikzpicture}[scale=.8,auto,swap]
%%         \node[vertex] (a1) at (0,0) {};
%%         \node[vertex] (b1) at (0,2) {};
%%         \node[vertex] (a2) at (1,0) {};
%%         \node[vertex] (b2) at (2,2) {};
%%         \node[vertex] (a3) at (3,0) {};
%%         \node[vertex] (b3) at (3,2) {};
%%         \node[vertex] (a4) at (5,0) {};
%%         \node[vertex] (b4) at (4,2) {};
%%         \draw[red edge] (a1) to (b1);
%%         \draw[red edge] (a2) to (b3);
%%         \draw[red edge] (a3) to (b4);
%%         \draw[red edge] (a4) to (b2);
%%         \draw[edge] (a1) to (b1);
%%         \draw[edge] (a1) to (b2);
%%         \draw[edge] (a2) to (b1);
%%         \draw[edge] (a2) to (b3);
%%         \draw[edge] (a2) to (b4);
%%         \draw[edge] (a3) to (b1);
%%         \draw[edge] (a3) to (b4);
%%         \draw[edge] (a4) to (b3);
%%         \draw[edge] (a4) to (b2);
%%       \end{tikzpicture}
%%   \end{columns}}
%% \end{frame}
