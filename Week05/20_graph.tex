

\section{Graph Basics}

\begin{frame}
  \begin{center}
    {\bf Part I -- Graph Basics}
  \end{center}
\end{frame}

\subsection{Definitions}

\begin{frame}
  \frametitle{What is a graph?}

  \begin{columns}[T]
    \column{0.6\textwidth}
    \begin{block}{}
      A graph $G = \{V,E\}$ is composed of a set of {\bf vertices} $V$, which are connected to a set of {\bf edges} $E$. Each edge connects exactly two vertices.
    \end{block}\bigskip

    \begin{itemize}
    \item An edge can be {\bf directed} or {\bf undirected};
    \item An edge or a vertice can have {\bf weights} or {\bf labels};
    \item {\bf Self-edge}: edge between $v_i$ and $v_i$;
    \item {\bf Multi-edge}: two edges with same end-vertices;
    \item A graph can be {\bf connected} or {\bf disconnected};
    \end{itemize}

    \column{0.4\textwidth}
    \begin{tikzpicture}[scale=.8,auto,swap]
      \tikzset{edge/.style = {->,>=latex'}}
      \node[vertex] (a) at (0,0) {};
      \node[vertex] (b) at (2,3) {};
      \node[vertex] (c) at (4,2) {};
      \node[vertex] (d) at (4,0) {};
      \draw[edge] (a) to (b);
      \draw[edge] (a) to[bend left] (c);
      \draw[edge] (c) to[bend left] (a);
      \draw[edge] (d) to (c);
      \draw[edge] (d) to[loop left] (d);
      \draw[edge] (d) to[loop right] (d);
    \end{tikzpicture}

    \vspace{.5cm}

    \begin{tikzpicture}[scale=1,auto,swap]
      \node[vertex] (a) at (0,2) {a};
      \node[vertex] (b) at (0,0) {b};
      \node[vertex] (c) at (2,1) {c};
      \node[vertex] (d) at (4,0) {d};
      \node[vertex] (e) at (4,1) {e};
      \draw[edge] (a) -- node[weight] {$7$} (b);
      \draw[edge] (b) -- node[weight] {$-2$} (c);
      \draw[edge] (c) -- node[weight] {$3$} (a);
      \draw[edge] (c) -- node[weight] {$5$} (d);
    \end{tikzpicture}

  \end{columns}
\end{frame}

\begin{frame}{Graphs in Computer Science}
  Graph Data structures show relationships between data;\\
  They are used in many problems:\bigskip

  \begin{itemize}
    \item Geography and Maps;
    \begin{itemize}
      \item Pathing between locations;
      \item Cycles and Tours;
    \end{itemize}
    \item Human Networks;
    \begin{itemize}
      \item Social Networks;
      \item Citation Clusters;
    \end{itemize}
    \item State Machines;
    \begin{itemize}
      \item Program Pipelines;
      \item Library Requirements;
    \end{itemize}
    \item Natural Language;
    \begin{itemize}
      \item Graph Grammars;
    \end{itemize}
  \end{itemize}
\end{frame}


\begin{frame}{Common graph tasks in an algorithm}
  \begin{itemize}
    \item Test if a path exist between vertice $V_i$ and $V_j$ (test if they are {\bf connected})
    \item Test the shortest path between vertice $V_i$ and $V_j$
    \begin{itemize}
      \item With or without weights
      \item Test if there is more than one path
    \end{itemize}
    \item Add or remove vertices or edges from a graph;
    \item Test some characteristics of a graph;
    \begin{itemize}
      \item Longest path? Shortest path?
      \item Does it have a {\bf Cycle}?
      \item Vertice with maximum number of vertices?
      \item etc...
    \end{itemize}
  \end{itemize}
\end{frame}

\subsection{Example Problem}

\begin{frame}[fragile]{Programming Challenge Example}{Dominator}
  \begin{block}{}
    Definition: A vertice $V_i$ {\bf dominates} $V_j$ if all paths $V_0 \to V_j$ must include $V_i$.
    \begin{itemize}
      \item {\bf input}: A directed graph $\{V,E\}$;
      \item {\bf output}: A table with the DOMINATE relationship
    \end{itemize}
  \end{block}
\begin{columns}[T]
  \column{0.4\textwidth}
    \begin{center}
      \begin{tikzpicture}[scale=1.3,auto,swap]
  \tikzset{edge/.style = {->,>=latex'}}
  \node[vertex] (a) at (0,0) {0};
  \node[vertex] (b) at (1,1) {1};
  \node[vertex] (c) at (1,-1) {2};
  \node[vertex] (d) at (2,0) {3};
  \node[vertex] (e) at (3,0) {4};
  \draw[edge] (a) to (b);
  \draw[edge] (a) to (c);
  \draw[edge] (b) to (d);
  \draw[edge] (c) to (d);
  \draw[edge] (d) to (e);
\end{tikzpicture}

    \end{center}
      \column{0.3\textwidth}
\begin{verbatim}
Input: 1
       5
       0 1 1 0 0
       0 0 0 1 0
       0 0 0 1 0
       0 0 0 0 1
       0 0 0 0 0
\end{verbatim}
\column{0.3\textwidth}
{\smaller
\begin{verbatim}
Output: Case 1:
        +---------+
        |Y|Y|Y|Y|Y|
        +---------+
        |N|Y|N|N|N|
        +---------+
        |N|N|Y|N|N|
        +---------+
        |N|N|N|Y|Y|
        +---------+
        |N|N|N|N|Y|
        +---------+
\end{verbatim}}
  \end{columns}
\end{frame}

\begin{frame}[fragile]{Programming Challenge Example}{Dominator}
  \begin{block}{}
    \begin{itemize}
      \item Which data structure should be used?
      \item How to calculate the "DOMINATE" status of a vertice?
    \end{itemize}
  \end{block}
  \begin{columns}[T]
    \column{0.4\textwidth}
      \begin{center}
        \begin{tikzpicture}[scale=1.3,auto,swap]
  \tikzset{edge/.style = {->,>=latex'}}
  \node[vertex] (a) at (0,0) {0};
  \node[vertex] (b) at (1,1) {1};
  \node[vertex] (c) at (1,-1) {2};
  \node[vertex] (d) at (2,0) {3};
  \node[vertex] (e) at (3,0) {4};
  \draw[edge] (a) to (b);
  \draw[edge] (a) to (c);
  \draw[edge] (b) to (d);
  \draw[edge] (c) to (d);
  \draw[edge] (d) to (e);
\end{tikzpicture}

      \end{center}
        \column{0.3\textwidth}
\begin{verbatim}
Input: 1
       5
       0 1 1 0 0
       0 0 0 1 0
       0 0 0 1 0
       0 0 0 0 1
       0 0 0 0 0
\end{verbatim}
\column{0.3\textwidth}
{\smaller
\begin{verbatim}
Output: Case 1:
        +---------+
        |Y|Y|Y|Y|Y|
        +---------+
        |N|Y|N|N|N|
        +---------+
        |N|N|Y|N|N|
        +---------+
        |N|N|N|Y|Y|
        +---------+
        |N|N|N|N|Y|
        +---------+
\end{verbatim}}
    \end{columns}
\end{frame}

\subsection{Graph Data Structure}


\begin{frame}[fragile]{Data Structure for Graph 1}

\begin{block}{Adjacency Matrix: stores the connection between vertices}
\begin{verbatim}
int adj[100][100];

for (int i = 0; i < n; i++)
  for (int j = 0; i < n; j++)
    cin >> adj[i][j]; // 0 if no edge, 1 if edge
\end{verbatim}

  \begin{itemize}
    \item \structure{Pros}:
    \begin{itemize}
      \item Easy to program;
      \item Access to edge $e_{ij}$ is quick;
    \end{itemize}
    \item \alert{Cons}:
      \begin{itemize}
      \item Cannot store multigraph;
      \item Wastes memory with sparse graphs;
      \item Time $O(V)$ to calculate number of neighbors of vertice $v_i$;
      \end{itemize}
    \end{itemize}
  \end{block}
\end{frame}

\begin{frame}[fragile]{Data Structure for Graph 2}

  \begin{block}{Adjacency List: stores edge list for each Vertex}
      {\smaller
\begin{verbatim}
typedef pair<int,int> edge;           // pair: <neighbor, weight>
typedef vector<edge> neighb;          // all neighbors of V_i
vector<neighb> AdjList;               // all V_i
int e;
for (int i = 0; i < n; i++)
  for (int j = 0; j < n; j++)
    cin >> e;
    if (e == 1) { AdjList[i].push_back(pair(j,1)); }
\end{verbatim}}

\begin{itemize}
  \item {\bf Pro}:
    \begin{itemize}
      \item Memory efficient if the graph is sparse;
      \item Can store multigraph;
    \end{itemize}
  \item {\bf Cons}:
  \begin{itemize}
    \item $O(\log(V))$ to test if two vertices are adjacent; (QUIZ: Why log(V)?)
  \end{itemize}
\end{itemize}
  \end{block}
\end{frame}

\begin{frame}[fragile]{Data Structure for Graph 3}
  \begin{block}{Edge List}
    {\smaller
    \begin{verbatim}
    pair <int,int> edge; // Edge between i and j
    vector<pair <int,edge>> Elist; // All edges;

    int e;
    for (int i = 0; i < n; i++)
      for (int j = 0; j < n; j++)
        cin >> e;
        if (e == 1) Elist.push_back(pair(1, pair(i,j)));
    \end{verbatim}}
    \begin{itemize}
      \item Not very common, used in specialized algorithms (ex:MST);
      \item To find if two vertices are neighbors, list must be sorted;
    \end{itemize}
  \end{block}
\end{frame}


\subsection{Graph Search: BFS, DFS}
\begin{frame}
  \frametitle{Graph Search: BFS and DFS}
    \begin{itemize}
    \item Basic Question: from vertice $v_s$, can we reach $v_e$?
    \item Many graph algorithms start from a graph search;
    \item Two types: BFS, DFS;
    \end{itemize}

  \begin{block}{Depth First Search -- DFS}
    \begin{itemize}
      \item Visit the first edge available;
      \item Vertice order is not guaranteed;
      \item Easy to implement with recursion or stack;
    \end{itemize}
  \end{block}

  \begin{block}{Breadth First Search -- BFS}
    \begin{itemize}
      \item First visit the vertices close to the starting point;
      \item Place new vertices on a list, and visit them with a loop;
    \end{itemize}
  \end{block}
\end{frame}

\begin{frame}
  \frametitle{BFS and DFS: Visualize the difference}
  \begin{columns}[t]
    \column{0.5\textwidth}
    \begin{exampleblock}{DFS}
      \vspace{0.1cm}
      \begin{center}
      \begin{tikzpicture}[scale=1.3,auto,swap]
        \node[vertex] (a) at (0,0) {};
        \node[vertex] (b) at (2,0) {};
        \node[vertex] (c) at (0,-1) {};
        \node[vertex] (d) at (1,-2) {};
        \node[vertex] (e) at (0,-2) {};
        \node[vertex] (f) at (2,-2) {};
        \node[vertex] (g) at (2,-3) {};
        \node[vertex] (h) at (1,-4) {};
        \draw[edge] (a) to (b);
        \draw[edge] (a) to (c);
        \draw[edge] (a) to (f);
        \draw[edge] (b) to (f);
        \draw[edge] (c) to (d);
        \draw[edge] (c) to (e);
        \draw[edge] (f) to (h);
        \draw[edge] (e) to (h);
        \draw[edge] (f) to (g);
        \draw<2->[red edge] (a) to (b);
        \draw<3->[red edge] (b) to (f);
        \draw<4->[red edge] (f) to (g);
        \draw<5->[red edge] (f) to (h);
        \draw<6->[red edge] (h) to (e);
        \draw<7->[red edge] (e) to (c);
        \draw<8->[red edge] (c) to (d);
      \end{tikzpicture}
      \end{center}
      \vspace{0.1cm}
    \end{exampleblock}
    \column{0.5\textwidth}
    \begin{block}{BFS}
      \vspace{0.1cm}
      \begin{center}
      \begin{tikzpicture}[scale=1.3,auto,swap]
        \node[vertex] (a) at (0,0) {};
        \node[vertex] (b) at (2,0) {};
        \node[vertex] (c) at (0,-1) {};
        \node[vertex] (d) at (1,-2) {};
        \node[vertex] (e) at (0,-2) {};
        \node[vertex] (f) at (2,-2) {};
        \node[vertex] (g) at (2,-3) {};
        \node[vertex] (h) at (1,-4) {};
        \draw[edge] (a) to (b);
        \draw[edge] (a) to (c);
        \draw[edge] (a) to (f);
        \draw[edge] (b) to (f);
        \draw[edge] (c) to (d);
        \draw[edge] (c) to (e);
        \draw[edge] (f) to (h);
        \draw[edge] (e) to (h);
        \draw[edge] (f) to (g);
        \draw<2->[red edge] (a) to (b);
        \draw<3->[red edge] (a) to (f);
        \draw<4->[red edge] (a) to (c);
        \draw<5->[red edge] (f) to (g);
        \draw<6->[red edge] (f) to (h);
        \draw<7->[red edge] (c) to (d);
        \draw<8->[red edge] (c) to (e);
      \end{tikzpicture}
      \end{center}
      \vspace{0.1cm}
    \end{block}
  \end{columns}
\end{frame}

\begin{frame}[fragile]
  \frametitle{DFS Implementation}
  \begin{exampleblock}{DFS (Using Adjacency List)}
\begin{verbatim}
vector<int> dfs_vis; // visited nodes, init to 0

void dfs(int v) {
   dfs_vis[v] = 1;
   for (int i; i < AdjList[v].size(); i++)
   {
      edge u = AdjList[v][i]; // u = neighb, weight
      // do something...
      if (dfs_vis[u.first] == 0)
         dfs(v.first);
   }
}
dfs(start_vertice);
\end{verbatim}
  \end{exampleblock}
\end{frame}

\begin{frame}[fragile]
  \frametitle{BFS Implementation}
  \begin{exampleblock}{BFS (Using adjacency List)}
\begin{verbatim}
vector<int> bfs_vis;   // visited nodes; init to 0
queue<int> q;          // list of vertices to visit;
q.push(start_vertice); // Start BFS

while(!q.empty()) {
  int u = q.front(); q.pop(); bfs_vis[u] = 1;
  // Do something...
  for (int i = 0; i < AdjList[v].size(); i++) {
    edge e = AdjList[v][i];
    if (bfs_vis[e.first] == 0)   // Check if node is visited
      q.push(e.first);
  }
}
\end{verbatim}
  \end{exampleblock}
\end{frame}

\begin{frame}{BFS and DFS}{Computational Cost}

  In the full BFS and DFS, you need to check every vertice and every edge in the graph:\bigskip

  \begin{itemize}
    \item A BFS/DFS implemented with {\bf Adjacency List}, costs $O(V+E)$.\bigskip

    \item A BFS/DFS implemented with {\bf Adjacency Matrix}, costs $O(V^2)$.
    \begin{itemize}
      \item That's because to visit every edge of a vertice in an Adjacency Matrix, it costs $O(V)$.
    \end{itemize}\bigskip

    \item Adjacency List is faster, {\bf if the graph is sparse (has few edges)}
  \end{itemize}
\end{frame}

\subsection{Solving the Dominator Problem}

\begin{frame}[fragile]{Solving the Dominator Problem with DFS}
  \begin{exampleblock}{}
    \begin{itemize}
      \item $v_j$ is dominated by $v_j$, if all paths from $v_0$ to $v_j$ pass through $v_i$;
      \item In other words, you cannot access $v_j$ from $v_0$, if $v_i$ is not available;
      \item {\bf Algorithm:} Remove $v_i$, and test if you can access $v_j$ from $v_0$;
    \end{itemize}
  \end{exampleblock}\bigskip

  \begin{tikzpicture}[scale=1.3,auto,swap]
  \tikzset{edge/.style = {->,>=latex'}}
  \node[vertex] (a) at (0,0) {0};
  \node[vertex] (b) at (1,1) {1};
  \node[vertex] (c) at (1,-1) {2};
  \node[vertex] (d) at (2,0) {3};
  \node[vertex] (e) at (3,0) {4};
  \draw[edge] (a) to (b);
  \draw[edge] (a) to (c);
  \draw[edge] (b) to (d);
  \draw[edge] (c) to (d);
  \draw[edge] (d) to (e);
\end{tikzpicture}

\end{frame}

\begin{frame}[fragile]{Solving the Dominator Problem with DFS}{Use DFS/BFS N times}

  \begin{columns}
    \column{0.4\textwidth}
    \begin{tikzpicture}[scale=1.3,auto,swap]
  \tikzset{edge/.style = {->,>=latex'}}
  \node[vertex] (a) at (0,0) {0};
  \node[vertex] (b) at (1,1) {1};
  \node[vertex] (c) at (1,-1) {2};
  \node[vertex] (d) at (2,0) {3};
  \node[vertex] (e) at (3,0) {4};
  \draw[edge] (a) to (b);
  \draw[edge] (a) to (c);
  \draw[edge] (b) to (d);
  \draw[edge] (c) to (d);
  \draw[edge] (d) to (e);
\end{tikzpicture}

    \column{0.6\textwidth}

    \begin{exampleblock}{}
{\smaller
\begin{verbatim}
// Modified DFS: does not visit vertex v_i;
boolean DFS2(S,i) {...};

// initialization: which nodes v_0 can reach?
DFS2(0,-1);
for (int j = 0; j < N; j++)
  if (VISITED[j]) { DOMINATED[0][j] = 1; }

// check DOMINATED relationship of each v_i
for (int i = 1; i < N; i++) {
  memset(VISITED,0,sizeof(VISITED));
  DFS2(0,i);
  for (int j = 0; j < N; j++)
    if (!VISITED[j] && DOMINATED[0][j])
      DOMINATED[i][j] = 1;
}
\end{verbatim}}
    \end{exampleblock}
  \end{columns}
\end{frame}


%%%%%%%%%%%%%%%%%%%%%%%%%%%%%%%%%%%%%%%%%%%%%%%%%%%%%%%%%%%%%%%%%%%%%%%%%%%%%%%
%% \subsection{Graph Terms}
%% \begin{frame}
%%   \frametitle{Quick Review of Graph Terms (1)}
%%   {\smaller
%%     \begin{block}{}
%%       You probably know all of these. If not, ask questions!
%%     \end{block}

%%     \begin{columns}[T]
%%       \column{0.6\textwidth}
%%       \begin{itemize}
%%       \item A Graph $G$ is made of a set of \structure{vertices} $V$
%%         and \structure{edges} $E$.
%%       \item Edges can be \structure{directed}\\ (has source and destination vertices);
%%       \item Edges can be \structure{weighted} or not\\ (all weigths = 1);
%%       \item Sets of nodes can be \structure{connected}\\ or \structure{disconnected}
%%       \item Directed Graphs can be \structure{Strongly Connected}
%%       \item Edges can be \structure{self-edges}, and/or \structure{multiple edges}
%%       \end{itemize}
%%       \column{0.4\textwidth}

%%       \begin{tikzpicture}[scale=.8,auto,swap]
%%         \tikzset{edge/.style = {->,>=latex'}}
%%         \node[vertex] (a) at (0,0) {};
%%         \node[vertex] (b) at (2,3) {};
%%         \node[vertex] (c) at (4,2) {};
%%         \node[vertex] (d) at (4,0) {};
%%         \draw[edge] (a) to (b);
%%         \draw[edge] (a) to[bend left] (c);
%%         \draw[edge] (c) to[bend left] (a);
%%         \draw[edge] (d) to (c);
%%         \draw[edge] (d) to[loop left] (d);
%%         \draw[edge] (d) to[loop right] (d);
%%       \end{tikzpicture}

%%       \vspace{.5cm}

%%       \begin{tikzpicture}[scale=1,auto,swap]
%%         \node[vertex] (a) at (0,2) {a};
%%         \node[vertex] (b) at (0,0) {b};
%%         \node[vertex] (c) at (2,1) {c};
%%         \node[vertex] (d) at (4,0) {d};
%%         \node[vertex] (e) at (4,1) {e};
%%         \draw[edge] (a) -- node[weight] {$7$} (b);
%%         \draw[edge] (b) -- node[weight] {$-2$} (c);
%%         \draw[edge] (c) -- node[weight] {$3$} (a);
%%         \draw[edge] (c) -- node[weight] {$5$} (d);
%%       \end{tikzpicture}

%%     \end{columns}
%%   }
%% \end{frame}

%% \begin{frame}
%%   \frametitle{Quick Review of Graph Terms (2)}

%%   {\smaller
%%     \begin{block}{}
%%       You probably know all of these. If not, ask questions!
%%     \end{block}

%%     \begin{columns}[T]
%%       \column{0.6\textwidth}
%%       \begin{itemize}
%%       \item A \structure{path} is a set of vertices connected by edges;
%%       \item A \structure{cycle} is a path with first and last vertices identical;
%%       \item \structure{Labelled} graphs and \structure{Isomorphic} graphs;
%%       \item A \structure{tree} is a acyclical, undirected graph;
%%       \item A \structure{spanning tree} is a subset of edges from E'
%%         that form a tree, connecting all nodes $V \in G$;
%%       \item A \alert{spamming tree} houses very noisy insects in summer;
%%       \end{itemize}
%%       \column{0.4\textwidth}
%%       \begin{tikzpicture}[scale=1,auto,swap]
%%         \node[vertex] (s) at (0,0) {a};
%%         \node[vertex] (a1) at (-1,-1) {b};
%%         \node[vertex] (a2) at (1,-1) {c};
%%         \node[vertex] (b1) at (-1,-2) {d};
%%         \node[vertex] (b2) at (0,-2) {e};
%%         \draw[edge] (s) to (a1);
%%         \draw[edge] (s) to  (a2);
%%         \draw[edge] (a1) to  (b1);
%%         \draw[edge] (a1) to  (b2);
%%         \draw[black edge] (b1) to (b2);
%%         \draw[black edge] (a2) to (b2);
%%       \end{tikzpicture}

%%       \vspace{.5cm}

%%       \begin{tikzpicture}[scale=1,auto,swap]
%%         \node[vertex] (s) at (0,0) {e};
%%         \node[vertex] (a1) at (-1,-1) {d};
%%         \node[vertex] (a2) at (1,-1) {b};
%%         \node[vertex] (b1) at (-1,-2) {c};
%%         \node[vertex] (b2) at (0,-2) {a};
%%         \draw[edge] (s) to (a1);
%%         \draw[edge] (s) to  (a2);
%%         \draw[edge] (a1) to  (b1);
%%         \draw[edge] (a1) to  (b2);
%%         \draw[black edge] (b1) to (b2);
%%         \draw[black edge] (a2) to (b2);
%%       \end{tikzpicture}
%%   \end{columns}}

%% \end{frame}

%% \begin{frame}
%%   \frametitle{Quick Review of Graph Terms (3)}

%%   {\smaller
%%     \begin{block}{}
%%       You probably know all of these. If not, ask questions!
%%     \end{block}

%%     \begin{columns}[T]
%%       \column{0.6\textwidth}
%%       \begin{itemize}
%%       \item The \structure{degree} of a node is the number of edges
%%         connected to it;
%%       \item Directed graphs have \structure{in-degrees} and
%%         \structure{out-degrees};
%%       \item A \structure{bipartite} graph can be divided in two sets
%%         of unconnected vertices;
%%       \item A \structure{Match} or \structure{Pairing} is a set of
%%         edges that connects the nodes in the bipartite graph;
%%       \end{itemize}
%%       \column{0.4\textwidth}
%%       \begin{tikzpicture}[scale=.8,auto,swap]
%%         \node[vertex] (a) at (0,0) {deg: 2};
%%         \node[vertex] (b) at (2,1) {};
%%         \node[vertex] (c) at (2,-1) {};
%%         \node[vertex] (d) at (4,0) {in,out:1};
%%         \draw[edge] (a) to (b);
%%         \draw[edge] (a) to (c);
%%         \tikzset{edge/.style = {->,>=latex'}}
%%         \draw[edge] (c) to (d);
%%         \draw[edge] (d) to (b);
%%       \end{tikzpicture}

%%       \vspace{.5cm}

%%       \begin{tikzpicture}[scale=.8,auto,swap]
%%         \node[vertex] (a1) at (0,0) {};
%%         \node[vertex] (b1) at (0,2) {};
%%         \node[vertex] (a2) at (1,0) {};
%%         \node[vertex] (b2) at (2,2) {};
%%         \node[vertex] (a3) at (3,0) {};
%%         \node[vertex] (b3) at (3,2) {};
%%         \node[vertex] (a4) at (5,0) {};
%%         \node[vertex] (b4) at (4,2) {};
%%         \draw[red edge] (a1) to (b1);
%%         \draw[red edge] (a2) to (b3);
%%         \draw[red edge] (a3) to (b4);
%%         \draw[red edge] (a4) to (b2);
%%         \draw[edge] (a1) to (b1);
%%         \draw[edge] (a1) to (b2);
%%         \draw[edge] (a2) to (b1);
%%         \draw[edge] (a2) to (b3);
%%         \draw[edge] (a2) to (b4);
%%         \draw[edge] (a3) to (b1);
%%         \draw[edge] (a3) to (b4);
%%         \draw[edge] (a4) to (b3);
%%         \draw[edge] (a4) to (b2);
%%       \end{tikzpicture}
%%   \end{columns}}
%% \end{frame}


\section{Common Graph Problems}

\begin{frame}{}
  \begin{center}
    {\bf Part II: Common Graph Problems}
  \end{center}
\end{frame}

\begin{frame}{Common Graph Problems in Competitive Programming}

  Most of these can be solved with small modifications to DFS or BFS.\bigskip

  \begin{itemize}
    \item Connected Components;
    \item Flood Fill;
    \item Topological Sort;
    \item Bipartite Checking;\bigskip

    \item Articulation Vertices; \hfill Next Video
    \item Strongly Connected Components; \hfill Next Video
  \end{itemize}
\end{frame}


\subsection{Connected Components}
\begin{frame}{Connected Components (undirected graph)}
  \begin{block}{Definition}
    A {\bf connected component} of a graph is a subset of vertices $C^k$ where every pair of vertices $v_i, v_j \in C^k$ is connected.\bigskip
  \end{block}

  \vfill
  \begin{center}
    \begin{tikzpicture}[scale=1,auto,swap]
      \node[vertex] (a) at (0,2) {a};
      \node[vertex] (b) at (0,0) {b};
      \node[vertex] (c) at (2,1) {c};
      \node[vertex] (d) at (4,0) {d};
      \node[vertex] (e) at (6,1) {e};
      \node[vertex] (f) at (8,2) {f};
      \node[vertex] (g) at (9,0) {g};
      \draw[edge] (a) to (b);
      \draw[edge] (b) to (c);
      \draw[edge] (c) to (a);
      \draw[edge] (c) to (d);
      \draw[edge] (f) to (g);
    \end{tikzpicture}
  \end{center}
\end{frame}

\begin{frame}{Connected Components}{Example Problem}
  \begin{block}{Problem Example: Extra cables}
    There is a network of $N$ computers. Some of the computers are connected by cables. Computers connected by cables, even if indirectly, are said to be on the {\bf same network}.
    \bigskip

    What is the minimum number of cables that you need to make sure that all $N$ computers are part of the same network?
  \end{block}\bigskip

  {\bf Solution:} Count the number of Connected Components ($C$), the answer is $C-1$.\bigskip

  {\bf Quiz:} How do you implement this?
\end{frame}

\begin{frame}[fragile]{Connected Components}{Finding Connected Components using BFS/DFS}
  We can find all connected components by looping through all vertices, and running BFS/DFS on each unvisited vertice;

\begin{columns}
  \column{0.7\textwidth}
  \begin{exampleblock}{}
\begin{verbatim}
int dfs_vis[];          // visited vertices

int cables = 0;
for (int = 0; i < N; i++)
   if (dfs_vis[i] == 0) // found new component
   {
      dfs(i);           // visit more vertices
      cables += 1;
   }
cout << "Need "<< cables - 1 <<".\n";
\end{verbatim}
  \end{exampleblock}
  \column{0.3\textwidth}
  \begin{center}
  \begin{tikzpicture}[scale=1,auto,swap]
    \node[vertex] (a) at (0,0) {0};
    \node[vertex] (b) at (2,0) {1};
    \node[vertex] (c) at (0,2) {2};
    \node[vertex] (d) at (2,2) {3};
    \node[vertex] (e) at (0,4) {4};
    \node[vertex] (f) at (2,4) {5};
    \draw[edge] (a) to (b);
    \draw[edge] (a) to (c);
    \draw[edge] (b) to (c);
    \draw[edge] (d) to (f);
  \end{tikzpicture}
\end{center}
\end{columns}

\end{frame}

%% Maybe add back later
% \begin{frame}[fragile]{Connected Components}{UFDS Variant}
%
%   You could also count Connected Components using the {\bf UFDS} data structure from lecture 2.\bigskip
%
%   It costs $O(E)$ to build the UFDS, and $O(V)$ to count the number of components.\bigskip
%
%   If your problem is dynamic and includes several additions to the graph edges, this might be a good choice, because it is cheaper to recalculate the CCs.
% \end{frame}

\subsection{Flood Fill}

\begin{frame}[fragile]{Flood Fill}
  \begin{block}{Problem: Find The Biggest Island}
    You want to find the biggest island in a game map to build a castle.

    {\bf Input:} A 2D representation of the map:
\begin{verbatim}
....................................
.###.......###.....#.....###.####...
.#####....#####.##.#####.##....#....
.###........###..#...##..#....###...
......###.......###...####...##.....
....####.............######.....###.
....####.......#.......###......###.
....................................
\end{verbatim}
  \end{block}
  \hfill Can we solve this as a graph problem?
\end{frame}

\begin{frame}{Implicit Graphs}
  \begin{columns}
    \column{0.7\textwidth}
    \begin{itemize}
      \item {\bf Implict Graphs} are data that suggest graph organization. Examples:
      \begin{itemize}
        \item grids (NSWE connections)
        \item maps (distance = weights)
      \end{itemize}\bigskip

      \item In some problems, it is not necessary to store the entire graph from the beginning;\bigskip

      \item {\bf Grid Floodfill}: Painting images, Walkable tiles in videogames, etc;
      \item Algorithm is just BFS/DFS with vertex labels;
    \end{itemize}
    \column{0.3\textwidth}
    \begin{tikzpicture}[scale=1,auto,swap]
      \node[vertex] (00) at (0,0) {};
      \node[vertex] (01) at (0,1) {};
      \node[vertex] (02) at (0,2) {};
      \node[vertex] (03) at (0,3) {};
      \node[vertex] (04) at (0,4) {};
      \node[vertex] (05) at (0,5) {};
      \node[vertex] (10) at (1,0) {};
      \node[vertex] (11) at (1,1) {};
      \node[vertex] (12) at (1,2) {};
      \node[vertex] (13) at (1,3) {};
      \node[vertex] (14) at (1,4) {};
      \node[vertex] (15) at (1,5) {};
      \node[vertex] (20) at (2,0) {};
      \node[vertex] (21) at (2,1) {};
      \node[vertex] (22) at (2,2) {};
      \node[vertex] (23) at (2,3) {};
      \node[vertex] (24) at (2,4) {};
      \node[vertex] (25) at (2,5) {};
      \draw[edge] (00) to (01);\draw[edge] (00) to (10);
      \draw[edge] (01) to (02);\draw[edge] (01) to (11);
      \draw[edge] (02) to (03);\draw[edge] (02) to (12);
      \draw[edge] (03) to (04);\draw[edge] (03) to (13);
      \draw[edge] (04) to (05);\draw[edge] (04) to (14);
      \draw[edge] (10) to (11);\draw[edge] (10) to (20);
      \draw[edge] (11) to (12);\draw[edge] (11) to (21);
      \draw[edge] (12) to (13);\draw[edge] (12) to (22);
      \draw[edge] (13) to (14);\draw[edge] (13) to (23);
      \draw[edge] (14) to (15);\draw[edge] (14) to (24);
      \draw[edge] (20) to (21);\draw[edge] (21) to (22);
      \draw[edge] (22) to (23);\draw[edge] (23) to (24);
      \draw[edge] (24) to (25);\draw[edge] (15) to (25);
      \draw[edge] (05) to (15);
    \end{tikzpicture}
  \end{columns}

\end{frame}


\begin{frame}[fragile]{Flood Fill}{Finding the "Biggest Island" with BFS/DFS and modifying labels}

  {\smaller
  \begin{exampleblock}{}
\begin{verbatim}
int dr[] = {1,1,0,-1,-1,-1,0,1}; // neighbors for a grid
int dc[] = {0,1,1,1,0,-1,-1,-1}; // with diagonals;

int floodfill(int y, int x) {    // size of one position
  if (y < 0 || y >= R || x < 0 || x >= C) return 0;
  if (grid[y][x] != '#') return 0;
  int size = 1;
  grid[y][x] = '.';              // Change the map to mark visited nodes
  for (int d = 0; d < 8; d++)
     size += floodfill(y+dr[d], x+dc[d]);
  return ans;
}
biggest = 0;
for (int i = 0; i < C; i++)
  for (int j = 0; j < R; j++)
    biggest = max(biggest, floodfill(i,j));
\end{verbatim}
  \end{exampleblock}
  }
\end{frame}

\subsection{Topological Sort}

\begin{frame}[fragile]{Topological Sort}
  \begin{block}{Example Problem: Preparing a Curriculum}
    You have a list of courses and requisites.\\
    Choose an {\bf ordering} of topics that respect all requisites.
    \bigskip

    {\bf Input}: list M topics, and N pairs of topics;\\
    {\bf Output}: Sorted list of all topics;
  \end{block}

{\smaller
\begin{verbatim}
** Example Input:
5 4 Graphs DP Search Flow Programming
Programming -> Search
Search -> DP
Graph -> Flow
Search -> Graph

** Example Output:
Course: Programming -> Search -> DP -> Graph -> Flow
\end{verbatim}

  }
\end{frame}

\begin{frame}{Topological Sort Definition}
  A {\bf topological sort} is an ordering of vertices where $v_i \prec v_j$ only if there is no path $v_j \to v_i$.\bigskip
  \begin{center}
    \begin{tikzpicture}[scale=.8,auto,swap]
  \node[vertex] (a) at (0,0) {a};
  \node[vertex] (b) at (2,1) {b};
  \node[vertex] (c) at (2,-1) {c};
  \node[vertex] (d) at (4,0) {d};
  \node[vertex] (e) at (6,0) {e};
  \tikzset{edge/.style = {->,>=latex'}}
  \draw[edge] (a) to (b);
  \draw[edge] (a) to (c);
  \draw[edge] (c) to (d);
  \draw[edge] (b) to (d);
  \draw[edge] (d) to (e);
\end{tikzpicture}

  \end{center}
  For this graph, one possible topological sort is $a \prec b \prec c \prec d \prec e$.\bigskip

  \begin{itemize}
    \item Toposorts are {\bf not unique}:
    \begin{itemize}
      \item $a \prec c \prec b \prec d \prec e$ is also a toposort.
    \end{itemize}
    \item A graph only has a toposort if it has {\bf no cycles}.
    \item To find the toposort, we use {\bf in-degrees and out-degrees} of each vertex:
    \begin{itemize}
      \item $a$ -- In-deg: 0; Out-deg: 2;
      \item $d$ -- In-deg: 2; Out-deg: 1;
      \item $e$ -- In-deg: 1; Out-deg: 0;
    \end{itemize}
  \end{itemize}
\end{frame}

\begin{frame}[fragile]{Finding Topological Sort -- Khan's Algorithm}

Modified BFS: Vertices are only added to the queue if they in-degree is 0.

\begin{exampleblock}{}
  {\smaller
\begin{verbatim}
queue<int> q; vector<int> toposort;
vector<int> in-deg;                 // initialize to 0 for all N;

for (int i = 0; i < EdgeList.size(); i++)
  in-deg[EdgeList[i].second]++;     // calculate in-degrees based on edge list.
for (int i = 0; i < N; i++)
  if (in-deg[i] == 0) q.push(i);    // add vertices with in-deg = 0 to queue

while (!q.empty()) {
  u = q.front(); q.pop(); toposort.push_back(u); // Add top of queue to toposort
  for (int i = 0; i < EdgeList[u].size(); i++) {
    d = EdgeList[u][i].first; in-deg[d]--;       // remove edges from visited.
    if (in-deg[d] == 0) q.push(d); // queue in-deg = 0;
  }
}
\end{verbatim}}
  \end{exampleblock}
\end{frame}

\begin{frame}{Khan's Algorithm}{Simulation}
  \begin{center}
    \begin{tikzpicture}[scale=.8,auto,swap]
  \node[vertex] (a) at (0,0) {a};
  \node[vertex] (b) at (2,1) {b};
  \node[vertex] (c) at (2,-1) {c};
  \node[vertex] (d) at (4,0) {d};
  \node[vertex] (e) at (6,0) {e};
  \tikzset{edge/.style = {->,>=latex'}}
  \draw[edge] (a) to (b);
  \draw[edge] (a) to (c);
  \draw[edge] (c) to (d);
  \draw[edge] (b) to (d);
  \draw[edge] (d) to (e);
\end{tikzpicture}

  \end{center}
  {\bf In-deg list:}
  \begin{itemize}
    \item<2-> iteration 1: (a,0), (b,1), (c,1), (d,2), (e,1)\hfill visit a
    \item<3-> iteration 2: (b,0), (c,0), (d,2), (e,1)\hfill visit b
    \item<4-> iteration 3: (c,0), (d,1), (e,1), \hfill visit c
    \item<5-> iteration 4: (d,0), (e,1)\hfill visit d
    \item<6-> iteration 5: (e,0)\hfill visit e
  \end{itemize}
  {\bf Toposort: \only<2->{a,} \only<3->{b,} \only<4->{c,} \only<5->{d,} \only<6->{e}}
\end{frame}

% \begin{frame}
%   \frametitle{Topological Sort and Bottom-Up Dynamic Programming}
%
%   %TODO
%   What is the relationship between Topological Sort and Bottom-up DP?
%
%   \bigskip
%
%   Bottom-up DP are Topological sorts on Tables!
% \end{frame}

\subsection{Bipartite Checking}
\begin{frame}{Bipartite Graphs}{Definition}
  \begin{columns}
    \column{.7\textwidth}
      Intuitively, a {\bf Bipartite Graph} is one that we can separate between a "left" side and a "right" side.\bigskip

      More generally, a graph $(V,E)$ is bipartite if you can completely partition its vertices in two subsets: $V_1$ and $V_2$, so that {\bf there are no edges} connecting two vertices in the same subset.\bigskip

      Bipartite graphs appear in a large number of algorithms. In particular, {\bf flow graphs} (next week) are bipartite graphs.\bigskip

      Most neural networks are bipartite graphs too!\\
      {\bf Quiz:} How do you test if a graph is bipartite?
    \column{.3\textwidth}
    \begin{tikzpicture}[scale=1.3,auto,swap]
  \node[red vertex] (a) at (0,0) {};
  \node[blue vertex] (b) at (2,0) {};
  \node[blue vertex] (c) at (2,-1) {};
  \node[red vertex] (d) at (0,-1) {};
  \node[red vertex] (e) at (0,-2) {};
  \node[red vertex] (f) at (0,-3) {};
  \node[blue vertex] (g) at (2,-2) {};
  \node[blue vertex] (h) at (2,-3) {};
  \draw[edge] (a) to (b);
  \draw[edge] (a) to (c);
  \draw[edge] (b) to (f);
  \draw[edge] (c) to (d);
  \draw[edge] (c) to (e);
  \draw[edge] (f) to (h);
  \draw[edge] (e) to (h);
  \draw[edge] (f) to (g);
\end{tikzpicture}

  \end{columns}
\end{frame}

\begin{frame}[fragile]
  \frametitle{Bipartite Check Algorithm}
  Visit all vertices using BFS/DFS. Every time we visit a vertice, we mark it "0" or "1". If two adjacent vertices are of the same colors, the graph is not bipartite.

  {\smaller
  \begin{exampleblock}{}
\begin{verbatim}
queue<int> q; q.push(s);
vector<int> color(V, -1); color[s] = 0; // Starting vertex
bool isBipartite = True;

while (!q.empty() && isBipartite) {
   int u = q.front(); q.pop();
   for (int j=0; j < adj_list[u].size(); j++) {
      v = adj_list[u][j].first;
      if (color[v] == -1) {
         color[v] = 1 - color[i];        // Coloring new vertex
         q.push(v.first);}
      else if (color[v.first] == color[u]) {
         isBipartite = False;            // Bipartite collision
}}}
\end{verbatim}
  \end{exampleblock}
  }
\end{frame}

\begin{frame}
  \frametitle{Bipartite Check -- Visualization}
  \begin{columns}[t]
    \column{0.5\textwidth}
    \begin{exampleblock}{Testing Bipartite property}
      \vspace{0.1cm}
      \begin{center}
        \begin{tikzpicture}[scale=1.3,auto,swap]
          \node[vertex] (a) at (0,0) {};
          \node[vertex] (b) at (2,0) {};
          \node[vertex] (c) at (0,-1) {};
          \node[vertex] (d) at (1,-2) {};
          \node[vertex] (e) at (0,-2) {};
          \node[vertex] (f) at (2,-2) {};
          \node[vertex] (g) at (2,-3) {};
          \node[vertex] (h) at (1,-4) {};
          \draw[edge] (a) to (b);
          \draw[edge] (a) to (c);
          \draw[edge] (b) to (f);
          \draw[edge] (c) to (d);
          \draw[edge] (c) to (e);
          \draw[edge] (f) to (h);
          \draw[edge] (e) to (h);
          \draw[edge] (f) to (g);
          \uncover<2->{\node[red vertex] (a1) at (0,0) {};}
          \uncover<3->{\node[blue vertex] (b1) at (2,0) {};}
          \uncover<3->{\node[blue vertex] (c1) at (0,-1) {};}
          \uncover<4->{\node[red vertex] (d1) at (1,-2) {};}
          \uncover<4->{\node[red vertex] (e1) at (0,-2) {};}
          \uncover<4->{\node[red vertex] (f1) at (2,-2) {};}
          \uncover<5->{\node[blue vertex] (g1) at (2,-3) {};}
          \uncover<5->{\node[blue vertex] (h1) at (1,-4) {};}
        \end{tikzpicture}
      \end{center}
      \vspace{0.1cm}
    \end{exampleblock}
    \column{0.5\textwidth}
    \uncover<6->{
    \begin{exampleblock}{Rearranging the nodes}
      \vspace{0.1cm}
      \begin{center}
        \begin{tikzpicture}[scale=1.3,auto,swap]
  \node[red vertex] (a) at (0,0) {};
  \node[blue vertex] (b) at (2,0) {};
  \node[blue vertex] (c) at (2,-1) {};
  \node[red vertex] (d) at (0,-1) {};
  \node[red vertex] (e) at (0,-2) {};
  \node[red vertex] (f) at (0,-3) {};
  \node[blue vertex] (g) at (2,-2) {};
  \node[blue vertex] (h) at (2,-3) {};
  \draw[edge] (a) to (b);
  \draw[edge] (a) to (c);
  \draw[edge] (b) to (f);
  \draw[edge] (c) to (d);
  \draw[edge] (c) to (e);
  \draw[edge] (f) to (h);
  \draw[edge] (e) to (h);
  \draw[edge] (f) to (g);
\end{tikzpicture}

      \end{center}
      \vspace{0.1cm}
    \end{exampleblock}}
  \end{columns}
\end{frame}

\section{Articulation Points}
\begin{frame}
  \begin{center}
    {\bf Part III -- Articulation Points}
  \end{center}
\end{frame}

\begin{frame}
  \frametitle{Articulation Points and Bridges}
    \begin{block}{Definition: In a graph $G$}
      \begin{itemize}
      \item Vertex $v_i$ is an {\bf Articulation Point} if removing $v_i$ makes $G$ disconnected.
      \item Edge $e_{i,j}$ is a \structure{Bridge} if removing $e_{i,j}$ makes $G$ disconnected.
      \end{itemize}
    \end{block}
    \begin{center}
        \begin{tikzpicture}[scale=1.3,auto,swap]
          \node[vertex] (a) at (0,0) {};
          \node[vertex] (b) at (2,0) {};
          \node[red vertex] (c) at (0,2) {};
          \node[red vertex] (d) at (2,2) {};
          \node[red vertex] (e) at (3,1) {};
          \node[vertex] (f) at (4,0) {};
          \node[vertex] (g) at (4,2) {};
          \node[blue vertex] (h) at (3,3) {};
          \draw[edge] (a) to (b);
          \draw[edge] (a) to (c);
          \draw[edge] (c) to (d);
          \draw[edge] (d) to (b);
          \draw[red edge] (d) to (e);
          \draw[edge] (e) to (f);
          \draw[edge] (f) to (g);
          \draw[edge] (g) to (e);
          \draw[red edge] (c) to (h);
        \end{tikzpicture}
      \end{center}
\end{frame}

\begin{frame}
  \frametitle{Problems and Naive Algorithm}
  \begin{exampleblock}{Example Problems}
    \begin{itemize}
      \item Find vertices that can be removed from a graph to "break" it;
      \item Add extra edges to "reinforce" a graph;
      \item Measure the reliability of a network, etc;
    \end{itemize}
  \end{exampleblock}\medskip

  \begin{block}{Complete Search algorithm to find Articulation Points: $O(V\times(V+E)) = O(V^2+VE)$}
  \begin{enumerate}
    \item Run DFS/BFS, and count the number of CC in the graph;
    \item For each vertex $v_i$, remove $v_i$ and run DFS/BFS again;
    \item If the number of CC increases, $v_i$ is an articulation point;
  \end{enumerate}
  \end{block}
\end{frame}

\begin{frame}{Tarjan's DFS variant for Articulation point (O(V+E))}
  \begin{exampleblock}{Find Articulation Points/Bridges in a single DFS pass: $O(V+E)$}
    Main idea: Track loops to detect articulations:
    \begin{itemize}
    \item {\bf dfs\_num[i]}: vector with visitation order from DFS;
    \item {\bf dfs\_low[i]}: vector with lowest dfs\_num reachable from $v_i$;\hfill
    \end{itemize}\bigskip

    For any $u,v$, if low[$v$] >= num[$u$], then $u$ is an articulation node (except root)\bigskip

    For any $u,v$, if low[$v$] > num[$u$], $e_{u,v}$ is a bridge; (articulation edge)
  \end{exampleblock}
\end{frame}

\begin{frame}{Tarjan's DFS variant for Articulation point (O(V+E))}

  \begin{center}
    \includegraphics[width=0.9\textwidth]{../img/graph_articulation}
  \end{center}
\end{frame}

\begin{frame}{Tarjan's Algorithm for Articulation Point}{Simulation}
  \begin{columns}
    \column{.55\textwidth}
    \begin{tikzpicture}[scale=1.3,auto,swap]
      \node[vertex] (a) at (0,0) {0};
      \node[vertex] (b) at (2,0) {1};
      \node[vertex] (c) at (0,2) {3};
      \node[vertex] (d) at (2,2) {2};
      \node[vertex] (e) at (3,1) {5};
      \node[vertex] (f) at (4,0) {6};
      \node[vertex] (g) at (4,2) {7};
      \node[vertex] (h) at (3,3) {4};
      \draw[edge] (a) to (b);
      \draw[edge] (a) to (c);
      \draw[edge] (c) to (d);
      \draw[edge] (d) to (b);
      \draw[edge] (d) to (e);
      \draw[edge] (e) to (f);
      \draw[edge] (f) to (g);
      \draw[edge] (g) to (e);
      \draw[edge] (c) to (h);
    \end{tikzpicture}
    \column{.45\textwidth}
    \begin{itemize}
      \item dfs\_num: 0; dfs\_low: 0
      \item<2-> dfs\_num: 1; dfs\_low: 0
      \item<3-> dfs\_num: 2; dfs\_low: 0
      \item<4-> dfs\_num: 3; dfs\_low: 0
      \item<5-> dfs\_num: 4; dfs\_low: 4
      \item<6-> dfs\_num: 5; dfs\_low: 5
      \item<7-> dfs\_num: 6; dfs\_low: 5
      \item<8-> dfs\_num: 7; dfs\_low: 5
    \end{itemize}
  \end{columns}
\end{frame}

\begin{frame}[fragile]{Tarjan's Algorithm for Articulation Point}{Implementation}
{\smaller
  \begin{exampleblock}{}
\begin{verbatim}
void articulation(u){
   dfs_num[u] = dfs_low[u] = IterationCounter++; // update num[u], init low[u]
   for (int i = 0; i < AdjList[u].size(); i++){  // Do DFS on each edge from u
      v = AdjList[u][i];
      if (dfs_num[v.first] == UNVISITED) {       // DFS tree edge
         dfs_parent[v.first] = u;                // store parent
         if (u == 0) rootTreeEdge++;             // special case for root vertex
         articulation(v.first);                  // visit next vertex

         // After we finish the DFS from u, we check if u is articulation.
         if (dfs_low[v.first] >= dfs_num[u])
            articulation_vertex[u] = true;       // u is articulation
         dfs_low[u] = min(dfs_low[u],dfs_low[v.first])
      }
      else if (v.first != dfs_parent[u])         // found a cycle edge
         dfs_low[u] = min(dfs_low[u],dfs_num[v.first]);
}  }
\end{verbatim}
  \end{exampleblock}}
\end{frame}

\subsection{Strongly Connected Components}

\begin{frame}{Strongly Connected Components}
    \begin{block}{Definition}
      Given a {\bf directed} graph $G(V,E)$, a {\bf Strongly Connected Component (SCC)} is a subset of vertices $V_1$ where for every pair of vertices $v_i, v_i \in V_1$, there is both a path $v_i \to v_j$ and a path $v_j \to v_i$.
    \end{block}


\begin{columns}[t]
    \column{0.5\textwidth}
    \begin{exampleblock}{One Connected Component (undirected)}
      \vspace{0.1cm}
      \begin{center}
        \begin{tikzpicture}[scale=1.1,auto,swap]
          \node[vertex] (a) at (0,0) {};
          \node[vertex] (b) at (1,0) {};
          \node[vertex] (c) at (2,0) {};
          \node[vertex] (d) at (1,1) {};
          \node[vertex] (e) at (3,0) {};
          \node[vertex] (f) at (4,0) {};
          \node[vertex] (g) at (4,1) {};
          \node[vertex] (h) at (3,1) {};
          \draw[edge] (a) to (b);
          \draw[edge] (b) to (c);
          \draw[edge] (c) to (d);
          \draw[edge] (d) to (b);
          \draw[edge] (c) to (e);
          \draw[edge] (e) to (f);
          \draw[edge] (f) to (g);
          \draw[edge] (g) to (h);
          \draw[edge] (h) to (e);
        \end{tikzpicture}
      \end{center}
      \vspace{0.1cm}
    \end{exampleblock}
    \column{0.5\textwidth}
    \begin{exampleblock}{Three SCC (directed)}
      \vspace{0.1cm}
      \begin{center}
        \begin{tikzpicture}[scale=1.1,auto,swap]
          \tikzset{edge/.style = {->,>=latex'}}
          \node[vertex] (a) at (0,0) {};
          \node[blue vertex] (b) at (1,0) {};
          \node[blue vertex] (c) at (2,0) {};
          \node[blue vertex] (d) at (1,1) {};
          \node[red vertex] (e) at (3,0) {};
          \node[red vertex] (f) at (4,0) {};
          \node[red vertex] (g) at (4,1) {};
          \node[red vertex] (h) at (3,1) {};
          \draw[edge] (a) to (b);
          \draw[edge] (b) to (c);
          \draw[edge] (c) to (d);
          \draw[edge] (d) to (b);
          \draw[edge] (c) to (e);
          \draw[edge] (e) to (f);
          \draw[edge] (f) to (g);
          \draw[edge] (g) to (h);
          \draw[edge] (h) to (e);
        \end{tikzpicture}
      \end{center}
      \vspace{0.1cm}
    \end{exampleblock}
  \end{columns}
\end{frame}

\begin{frame}{Algorithm for Finding SCCs}

  We can modify Tarjan's algorithm (for articulation points and bridges) to find Strongly Connected Components:\bigskip

  \begin{block}{}
  \begin{itemize}
    \item Every time we visit a new vertex $u$, we put $u$ in a stack $S$;
    \item Only update dfs\_low for vertices with the "visited" flag = 1;
    \item After visiting all edges of $u$, check if "dfs\_num[$u$] == dfs\_min[$u$]";
    \item If the condition is true, $u$ is the root of a new SCC.
    \item Pop all vertices in $S$ until (and including) $u$;
    \item Add all popped vertices to the SCC.
  \end{itemize}
  \end{block}
\end{frame}

\begin{frame}[fragile]{Algorithm for Finding SCCs}{Do this simulation yourself!}
  \begin{center}
    \begin{tikzpicture}[scale=1.1,auto,swap]
      \tikzset{edge/.style = {->,>=latex'}}
      \node[vertex] (a) at (0,0) {0};
      \node[vertex] (b) at (1,0) {1};
      \node[vertex] (c) at (2,0) {2};
      \node[vertex] (d) at (1,1) {3};
      \node[vertex] (e) at (3,0) {4};
      \node[vertex] (f) at (4,0) {5};
      \node[vertex] (g) at (4,1) {6};
      \node[vertex] (h) at (3,1) {7};
      \draw[edge] (a) to (b);
      \draw[edge] (b) to (c);
      \draw[edge] (c) to (d);
      \draw[edge] (d) to (b);
      \draw[edge] (c) to (e);
      \draw[edge] (e) to (f);
      \draw[edge] (f) to (g);
      \draw[edge] (g) to (h);
      \draw[edge] (h) to (e);
    \end{tikzpicture}
  \end{center}
  \bigskip
  {\bf SCC Stack:}\bigskip

\begin{verbatim}
          0   1   2   3   4   5   6   7

dfs_low

dfs_sum
\end{verbatim}

\end{frame}

%%%%%%%%%%%%%%%%%%%%%%%%%%%%%%%%%%%%%%%%%%%%%%%

\begin{frame}
  \begin{center}
    {\bf Part 4: Minimum Spanning Tree}
  \end{center}
\end{frame}

\subsection{Spanning Tree}
\begin{frame}
  \frametitle{Minimum Spanning Trees (MST) -- Definition}

  \begin{block}{}
    A {\bf Spanning Tree} is a subset $E'$ from graph $G$ so
    that all vertices are connected without cycles.

    \medskip

    A \structure{Minimum Spanning Tree} is a spanning tree where the
    sum of edge's weights is minimal.
  \end{block}

  \begin{columns}[T]
    Graph\\
    \column{0.3\textwidth}
    \begin{tikzpicture}[transform shape,label/.style={thin, draw=black, align=center,fill=white,font=\smaller},scale=1.1]
      \node[vertex] (a) at (0,0) {};
      \node[vertex] (b) at (1,1) {};
      \node[vertex] (c) at (2,0) {};
      \node[vertex] (d) at (1,-1) {};
      \node[vertex] (e) at (0,-2) {};
      \draw[edge] (a) -- node[label] {$4$} (b);
      \draw[edge] (b) -- node[label] {$2$} (c);
      \draw[edge] (a) -- node[label] {$4$} (c);
      \draw[edge] (a) -- node[label] {$6$} (d);
      \draw[edge] (c) -- node[label] {$8$} (d);
      \draw[edge] (a) -- node[label] {$6$} (e);
      \draw[edge] (d) -- node[label] {$9$} (e);
    \end{tikzpicture}
    \column{0.3\textwidth}
    Spanning Tree
    \begin{tikzpicture}[transform shape,label/.style={thin, draw=black, align=center,fill=white,font=\smaller},scale=1.1]
      \node[vertex] (a) at (0,0) {};
      \node[vertex] (b) at (1,1) {};
      \node[vertex] (c) at (2,0) {};
      \node[vertex] (d) at (1,-1) {};
      \node[vertex] (e) at (0,-2) {};
      \draw[red edge] (a) -- node[label] {$4$} (b);
      \draw[red edge] (b) -- node[label] {$2$} (c);
      \draw[edge] (a) -- node[label] {$4$} (c);
      \draw[edge] (a) -- node[label] {$6$} (d);
      \draw[red edge] (c) -- node[label] {$8$} (d);
      \draw[edge] (a) -- node[label] {$6$} (e);
      \draw[red edge] (d) -- node[label] {$9$} (e);
    \end{tikzpicture}
    \column{0.3\textwidth}
    Minimum Spanning Tree
    \begin{tikzpicture}[transform shape,label/.style={thin, draw=black, align=center,fill=white,font=\smaller},scale=1.1]
      \node[vertex] (a) at (0,0) {};
      \node[vertex] (b) at (1,1) {};
      \node[vertex] (c) at (2,0) {};
      \node[vertex] (d) at (1,-1) {};
      \node[vertex] (e) at (0,-2) {};
      \draw[red edge] (a) -- node[label] {$4$} (b);
      \draw[red edge] (b) -- node[label] {$2$} (c);
      \draw[edge] (a) -- node[label] {$4$} (c);
      \draw[red edge] (a) -- node[label] {$6$} (d);
      \draw[edge] (c) -- node[label] {$8$} (d);
      \draw[red edge] (a) -- node[label] {$6$} (e);
      \draw[edge] (d) -- node[label] {$9$} (e);
    \end{tikzpicture}
  \end{columns}
\end{frame}

\begin{frame}{Usage Cases for Minimum Spanning Trees}
  \begin{block}{}
    \begin{itemize}
      \item Problems with MST often ask for a minimal cost to connect all elements in a graph (e.g. minimal infrastructure cost).\medskip

      \item {\bf Variations:} Maximum Spanning Tree, Spanning Forest, Force some edges in advance;
    \end{itemize}
  \end{block}

  \begin{exampleblock}{Main algorithms for MST}
    Two greedy algorithms that add edges to MST:
    \begin{itemize}
      \item {\bf Kruskal Algorithm}: based on edge list;
      \item {\bf Prim's Algorithm}: based on vertex list;
    \end{itemize}
  \end{exampleblock}
\end{frame}

\begin{frame}
  \frametitle{Kruskal's Algorithm}
  \begin{block}{Outline}
    Kruskal's algorithms sorts all edges by their weight, and try to add each edge to the MST, checking whether adding that edge would create a cycle.
  \end{block}

  \begin{columns}[T]
    \column{0.5\textwidth}
    \begin{enumerate}
    \item Sort all edges;
    \item If smallest edge does not create a cycle, add to MST;
    \item If smallest edge creates a cycle, remove it from list;
    \item Go to 2;
    \end{enumerate}
    \column{0.5\textwidth}
    \begin{tikzpicture}[transform shape,label/.style={thin, draw=black, align=center,fill=white,font=\smaller},scale=1.1]
      \node[vertex] (a) at (0,0) {};
      \node[vertex] (b) at (1,1) {};
      \node[vertex] (c) at (2,0) {};
      \node[vertex] (d) at (1,-1) {};
      \node[vertex] (e) at (0,-2) {};
      \draw[edge] (a) -- node[label] {$4$} (b);
      \draw[edge] (b) -- node[label] {$2$} (c);
      \draw[edge] (a) -- node[label] {$4$} (c);
      \draw[edge] (a) -- node[label] {$6$} (d);
      \draw[edge] (c) -- node[label] {$8$} (d);
      \draw[edge] (a) -- node[label] {$6$} (e);
      \draw[edge] (d) -- node[label] {$9$} (e);
      \draw<2->[red edge] (b) -- node[label] {$2$} (c);
      \draw<3->[red edge] (a) -- node[label] {$4$} (b);
      \draw<4->[black edge] (a) -- node[label] {$4$} (c);
      \draw<4->[red edge] (a) -- node[label] {$6$} (d);
      \draw<5->[red edge] (a) -- node[label] {$6$} (e);
    \end{tikzpicture}
  \end{columns}
\end{frame}

\begin{frame}[fragile]
  \frametitle{Kruskal's Algorithm -- Implementation}
{\smaller
\begin{exampleblock}{}
\begin{verbatim}
vector<pair<int, pair<int,int>> Edgelist;
sort(Edgelist.begin(),Edgelist.end());
int mst_cost = 0;
UnionFind UF(V);
  // note 1: Pair object has built-in comparison;
  // note 2: Need to implement UnionSet class;

for (int i = 0; i < Edgelist.size(); i++) {
   pair <int, pair <int,int>> front = Edgelist[i];
   if (!UF.isSameSet(front.second.first,
                     front.second.second)) {
      mst_cost += front.first;
      UF.unionSet(front.second.first,front.second.second)
   }}

cout << "MST Cost: " << mst_cost << "\n"
\end{verbatim}
\end{exampleblock}
}
\end{frame}

\begin{frame}
  \frametitle{Prim's Algorithm}
  {\small
  \begin{block}{Outline}
    Prim's algorith adds nodes to the MST one at a time, and keeps the
    edges connected to those nodes in a \structure{priority queue}. It
    then tests each edge in the priority queue to add more nodes to
    the MST, avoiding cycles.
  \end{block}

  \begin{columns}[T]
    \column{0.5\textwidth}
    \begin{enumerate}
    \item Add node 0 to MST;
    \item Add all edges from new node to Priority Queue;
    \item Visit smallest edge in Queue;
    \item If the edge leades to a new node, add it to MST;
    \item Add new edges to Queue;
    \item Go to 3;
    \end{enumerate}
    \column{0.5\textwidth}
    \begin{tikzpicture}[transform shape,label/.style={thin, draw=black, align=center,fill=white,font=\smaller},scale=1.1]
      \node[vertex] (a) at (0,0) {};
      \node[vertex] (b) at (1,1) {};
      \node[vertex] (c) at (2,0) {};
      \node[vertex] (d) at (1,-1) {};
      \node[vertex] (e) at (0,-2) {};
      \draw[edge] (a) -- node[label] {$4$} (b);
      \draw[edge] (b) -- node[label] {$2$} (c);
      \draw[edge] (a) -- node[label] {$4$} (c);
      \draw[edge] (a) -- node[label] {$6$} (d);
      \draw[edge] (c) -- node[label] {$8$} (d);
      \draw[edge] (a) -- node[label] {$6$} (e);
      \draw[edge] (d) -- node[label] {$9$} (e);
      \uncover<2->{
      \node[red vertex] (a1) at (0,0) {};
      \draw[black edge] (a) -- node[label] {$4$} (b);
      \draw[black edge] (a) -- node[label] {$4$} (c);
      \draw[black edge] (a) -- node[label] {$6$} (d);
      \draw[black edge] (a) -- node[label] {$6$} (e);
      }

      \uncover<3->{
        \node[red vertex] (c1) at (2,0) {};
        \draw[red edge] (a) -- node[label] {$4$} (c);
        \draw[black edge] (c) -- node[label] {$8$} (d);
        \draw[black edge] (b) -- node[label] {$2$} (c);
      }
      \uncover<4->{
        \node[red vertex] (b1) at (1,1) {};
        \draw[red edge] (b) -- node[label] {$2$} (c);
      }

      \uncover<5->{
        \node[red vertex] (d1) at (1,-1) {};
        \draw[red edge] (a) -- node[label] {$6$} (d);
        \draw[black edge] (d) -- node[label] {$9$} (e);
      }

      \uncover<6->{
        \node[red vertex] (e1) at (0,-2) {};
        \draw[red edge] (a) -- node[label] {$6$} (e);
      }
    \end{tikzpicture}
  \end{columns}
  }
\end{frame}

\begin{frame}[fragile]
  \frametitle{Prim's Algorithm -- Implementation}
{\smaller
\begin{exampleblock}{}
\begin{verbatim}
vector <int> taken; priority_queue <pair <int,int>> pq;

void process (int v) {
   taken[v] = 1;
   for (int j = 0; j < (int)AdjList[v].size(); j++) {
      pair <int,int> ve = AdjList[v][j];
      if (!taken[ve.first])
         pq.push(pair <int,int> (ve.first, ve.second))
}}
taken.assign(V,0); process(0);
mst_cost = 0;

while (!pq.empty()) {
  vector <int,int> pq.top(); pq.pop();
  u = front.first, w = front.second;
  if (!taken[u]) mst_cost += w, process(u);
}
\end{verbatim}
\end{exampleblock}
}
\end{frame}

\begin{frame}
  \frametitle{MST variant 1 -- Maximum Spanning tree}

    \begin{block}{}
      The \structure{Maximum Spanning Tree} variant requires the spanning tree to have maximum possible weight.\bigskip


      It is very easy to implement the Maximum MST:
      \begin{itemize}
        \item {\bf Kruskal}: Reverse the sort of the edge list;
        \item {\bf Prim}: Invert the weight of the priority queue;
      \end{itemize}
    \end{block}

  \medskip

  \begin{tikzpicture}[transform shape,label/.style={thin, draw=black, align=center,fill=white,font=\smaller},scale=1.1]
      \node[vertex] (a) at (0,0) {};
      \node[vertex] (b) at (1,1) {};
      \node[vertex] (c) at (0,2) {};
      \node[vertex] (d) at (-1,1) {};
      \node[vertex] (e) at (-2,0) {};
      \draw[red edge] (a) -- node[label] {$4$} (b);
      \draw[edge] (b) -- node[label] {$2$} (c);
      \draw[edge] (a) -- node[label] {$4$} (c);
      \draw[edge] (a) -- node[label] {$6$} (d);
      \draw[red edge] (c) -- node[label] {$8$} (d);
      \draw[red edge] (a) -- node[label] {$6$} (e);
      \draw[red edge] (d) -- node[label] {$9$} (e);
  \end{tikzpicture}
\end{frame}

\begin{frame}
  \frametitle{MST variant 2 -- Minimum Spanning Subgraph, Forest}
    \begin{block}{}
      In this variant, a subset of edges or vertices are pre-selected.

      \begin{itemize}
      \item In the case of pre-selected vertices, add them to the
        ``taken'' list in Kruskal's algorithm before starting;
      \item In the case of edges, add the end vertices to the
        ``taken'' list;
      \end{itemize}
    \end{block}\bigskip

    \begin{center}
      \begin{tikzpicture}[transform shape,label/.style={thin, draw=black, align=center,fill=white,font=\smaller},scale=1.1]
        \node[blue vertex] (a) at (0,0) {};
        \node[vertex] (b) at (1,1) {};
        \node[vertex] (c) at (2,2) {};
        \node[vertex] (d) at (3,1) {};
        \node[vertex] (e) at (4,0) {};
        \node[blue vertex] (f) at (4,2) {};
        \node[vertex] (g) at (5,0) {};
        \node[vertex] (h) at (6,2) {};
        \node[blue vertex] (i) at (6,1) {};
        \draw[red edge] (a) -- node[label] {$2$} (b);
        \draw[red edge] (a) -- node[label] {$1$} (d);
        \draw[edge] (d) -- node[label] {$4$} (c);
        \draw[edge] (d) -- node[label] {$8$} (e);
        \draw[edge] (d) -- node[label] {$3$} (f);
        \draw[red edge] (c) -- node[label] {$3$} (f);
        \draw[edge] (f) -- node[label] {$6$} (g);
        \draw[red edge] (e) -- node[label] {$4$} (g);
        \draw[red edge] (f) -- node[label] {$1$} (h);
        \draw[red edge] (g) -- node[label] {$3$} (i);
      \end{tikzpicture}
    \end{center}
\end{frame}

\begin{frame}
  \frametitle{MST Variant 3 -- Second Best MST}
  \begin{block}{Problem Definition}
    Suppose that you are required to calculate an alternative solution to an MST problem. In this case, you need to find the second cheapest spanning tree.
  \end{block}
  \bigskip

  Simple Algorithm:
  \begin{itemize}
    \item Calculate the MST (using Kruskal or Prim);
    \item For every edge $e_i$ in the MST:
    \begin{itemize}
      \item Remove $e_i$ from $E$;
      \item Calculate a new MST;
    \end{itemize}
    \item Choose the best among the new MSTs as the second-best MST.
  \end{itemize}
  \bigskip

  {\bf QUIZ}: How to generalize this algorithm for the n-th best spanning tree?
\end{frame}

\begin{frame}
  \frametitle{MST Variant 4 -- Minmax path cost}

  \begin{center}
      \begin{tikzpicture}[transform shape,label/.style={thin, draw=black, align=center,fill=white,font=\tiny},scale=1.2]
        \node[red vertex] (a) at (0,0) {0};
        \node[vertex] (b) at (1,1) {1};
        \node[vertex] (c) at (1,-1) {2};
        \node[vertex] (d) at (2,0) {3};
        \node[vertex] (e) at (3,1) {4};
        \node[vertex] (f) at (3,-1) {5};
        \node[red vertex] (g) at (4,0) {6};
        \draw[black edge] (a) -- node[label] {$50$} (b);
        \draw[red edge] (a) -- node[label] {$60$} (c);
        \draw[edge] (b) -- node[label] {$120$} (d);
        \draw[black edge] (b) -- node[label] {$90$} (e);
        \draw[red edge] (c) -- node[label] {$50$} (f);
        \draw[red edge] (d) -- node[label] {$80$} (f);
        \draw[red edge] (d) -- node[label] {$70$} (g);
        \draw[black edge] (e) -- node[label] {$40$} (g);
        \draw[edge] (f) -- node[label] {$140$} (g);
      \end{tikzpicture}
    \end{center}

  \begin{block}{Problem Definition}
    {\bf Regular Cost} for a path is the sum of weights of all edges in the path.\bigskip

    {\bf Minmax Cost} for a path is the maximum weight among all its edges.\bigskip

    Find the path $v_i \to v_j$ with the smallest {\bf minmax cost}
  \end{block}
\end{frame}

\begin{frame}
  \frametitle{Finding the Minmax path with MST}

  \begin{center}
      \begin{tikzpicture}[transform shape,label/.style={thin, draw=black, align=center,fill=white,font=\tiny},scale=1.2]
        \node[red vertex] (a) at (0,0) {0};
        \node[vertex] (b) at (1,1) {1};
        \node[vertex] (c) at (1,-1) {2};
        \node[vertex] (d) at (2,0) {3};
        \node[vertex] (e) at (3,1) {4};
        \node[vertex] (f) at (3,-1) {5};
        \node[red vertex] (g) at (4,0) {6};
        \draw[red edge] (a) -- node[label] {$50$} (b);
        \draw[red edge] (a) -- node[label] {$60$} (c);
        \draw[edge] (b) -- node[label] {$120$} (d);
        \draw[edge] (b) -- node[label] {$90$} (e);
        \draw[red edge] (c) -- node[label] {$50$} (f);
        \draw[red edge] (d) -- node[label] {$80$} (f);
        \draw[red edge] (d) -- node[label] {$70$} (g);
        \draw[red edge] (e) -- node[label] {$40$} (g);
        \draw[edge] (f) -- node[label] {$140$} (g);
      \end{tikzpicture}
    \end{center}

  \begin{exampleblock}{Algorithm}
    \begin{itemize}
      \item Generate the MST for the graph $G$.
      \item Find the path $v_i \to v_j$ inside the MST.
    \end{itemize}\bigskip

    That's it!
  \end{exampleblock}
\end{frame}
