

\section{Outline}

\begin{frame}
  \centering
  {\huge
    Programming Challenges Tips and Tricks
  }
\end{frame}

\begin{frame}{Hints and Tricks for Programming Challenges}
  In the first week of this course, the problems are very easy.\vfill

  \begin{itemize}
    \item I want you to get used with the style of problem;
    \item I want you to get used to the online judge;
  \end{itemize}\vfill

  So in the lecture today, I want to give some {\bf general hints} to solve Programming Challenges.\bigskip

  These hints will be useful for the entire course.
\end{frame}

\begin{frame}{Topics for today}
  \begin{itemize}
    \item Hint 1: Reading the problem;
    \item Hint 2: Problem size and algorithm analysis;
    \item Hint 3: Writing test cases;
    \item Hint 4: Input/Output;
    \item Hint 5: How to ask for help;
  \end{itemize}
\end{frame}

\section{Hint 1: Reading the Problem}
\begin{frame}{Hint 1: Reading, Thinking, then Coding}

  \begin{columns}
    \column{0.7\textwidth}
    \begin{itemize}
    \item Before you start coding, think about the problem, and write the solution on paper.\medskip


    \item Draw the problem on paper, and answer some input examples by hand.\medskip

    \item {\bf Do not start programming until you understand the problem}.\medskip

    \item It is hard to fix a wrong idea after you start programming.\medskip

    \item There are 8 problems every week -- you don't need to solve them in order!
    \end{itemize}
    \column{0.3\textwidth}
    \includegraphics[width=\textwidth]{../img/textmarker}
  \end{columns}

  \vfill

  \hrulefill\\
  \hfill {\tiny Image by Guido ``random'' Alvarez, released as CC-BY-2.0}
\end{frame}

\begin{frame}
  \frametitle{Tricks for reading the problem}

  \begin{alertblock}{}
    Reading problems in English is hard :-(
  \end{alertblock}

  \begin{itemize}
    \item Don't worry: deepl and google translate can help!
    \begin{itemize}
      \item Be careful, sometimes they are wrong...
    \end{itemize}\medskip

    \item Separate {\bf story} from {\bf problem}
    \begin{itemize}
      \item Look for keywords: Maximum, Total, Cost, Rules, Number, Constraint, etc...
    \end{itemize}\medskip

    \item Read the {\bf Input} and {\bf Output} section first!
    \begin{itemize}
      \item Sometimes you can understand the problem just from input/output.
      \item But be careful of special rules!
    \end{itemize}\medskip

    \item Understanding a little English is good for Computer Science;

  \end{itemize}
\end{frame}

\section{Hint 2: Problem Size and Algorithm Analysis}

\begin{frame}{Hint 2: Be careful of problem size!}
  One common case of wrong problem: Your program is too slow.\bigskip

  Before writing the program, you need to think about the {\bf input size}.\bigskip

  If the input size is {\bf too big}, the wrong algorithm can be {\bf too slow}.\bigskip

  Let's see two examples.
\end{frame}


%\item Hint 2: Problem size and algorithm analysis;

\subsection{Revisiting the 3n+1 problem}

\begin{frame}[fragile]{Problem Size Example: The 3n+1 problem}{Problem outline}

  \begin{block}{}
    Calculate the {\bf Maximum Cycle Length} between any two numbers $i$ and $j$.
  \end{block}
  \smallskip

\begin{verbatim}
Cycle of (n)
1. print n
2. if n == 1 then STOP
3. if n is odd then n = 3n + 1
4. else n = n/2
5. GOTO 2

Example: Cycle of (22)
22 11 34 17 52 26 13 40 20 10 5 16 8 4 2 1 -- Cycle Length = 16
\end{verbatim}
For {\bf ONE} cycle, the problem is small and easy.
\end{frame}

\begin{frame}[fragile]
  \frametitle{A simple solution for the 3n+1 problem}
  \begin{columns}
    \column{0.4\textwidth}
    Easy solution: loop until n == 1 and count the loops (lines 9-12).\bigskip

    However, This program can be really slow,
    if you have to {\bf Calculate the solution for many different cycles.}\bigskip

    Remember: The problem wants the {\bf MAXIMUM} cycle length between $i$ and $j$
    \column{0.6\textwidth}
  {\smaller
\begin{verbatim}
 1 while true:
 2  try:
 3    line = input()
 4    max = 0
 5    tk = line.split()
 6    i, j = int(tk[0]), int(tk[1])
 7    for n in range(i, j+1):
 8      count = 1
 9      while n != 1:
10        if n % 2 == 1: n = 3 * n + 1
11        else: n = n / 2
12        count += 1
13      if count > max: max = count
14    print (line, max)
15  except EOFError: break
\end{verbatim}}
\end{columns}
\end{frame}

%%%%%%%%%%%
\begin{frame}[fragile]
  \frametitle{Large input in the 3n+1 problem}

    \begin{itemize}
      \item Objective: Calculate the largest cycle between $i$ and $j$. For a large input:
    \begin{verbatim}
Input: 1 10000 <-- Largest cycle between 1 and 10000?
Output: 262    <-- Largest cycle is 262 steps
    \end{verbatim}

      \item 262 steps for 1 cycle is short. But for 10000 cycles: $262\times10000 = 2,000,000$ steps!\medskip

      \item Can we make it faster?


    \end{itemize}
\end{frame}

\begin{frame}[fragile]
  \frametitle{Make 3n+1 faster by removing repetition}

  \begin{itemize}
    \item Some examples of cycle calculation (A(n)) -- Note the repetition:
\begin{verbatim}
A(22):  22 11 34 17 52 26 13 40 20 10 5 16 8 4 2 1
A(11):  11 34 17 52 26 13 40 20 10 5 16 8 4 2 1
A(17):  17 52 26 13 40 20 10 5 16 8 4 2 1
A(13):  13 40 20 10 5 16 8 4 2 1
A(10):  10 5 16 8 4 2 1
A(8) :  8 4 2 1
\end{verbatim}\medskip

    \item To avoid the repetition, \alert{we store partial results in an array}.
    \medskip

    \item This technique is called {\bf Memoization}. It is very useful for many problems!
  \end{itemize}

\end{frame}

\begin{frame}[fragile]{Memoization for 3n+1}
  Use a list or dictionary to memorize past results ({\bf memo table}). Check the memo table before calculating results.
\begin{verbatim}
  memotable = {}
  memotable[1] = 1

  def A(n):
    if n in table.keys(): return table[n]
    else:
      if n % 2 == 1: table[n] = 1 + A(3*n + 1)
      else: table[n] = 1 + A(n/2)
    return table[n]
\end{verbatim}
More about Memo Tables in Lecture 4 (Dynamic Programming)
\end{frame}


\subsection{Problem Size Example 2: 8-Queens}

\begin{frame}[fragile]{Problem Size Example: 8 Queen Problem}
  \begin{columns}
    \column{0.8\textwidth}
    8-Queen problem: Place 8 queens in a chess board, so that {\bf all columns, all rows, and all diagonals are different}\bigskip

    {\bf Algorithm (1)}: Eight loops size 64: Try all rows and all columns for queen 1, queen 2, queen 3, queen 4, ... etc.
    \column{0.2\textwidth}
    \includegraphics[width=1\textwidth]{img/8queen}
  \end{columns}
\begin{verbatim}
Choices for Queen1: a1, a2, a3, a4, ..., h6, h7, h8
Choices for Queen2: a1, a2, a3, a4, ..., h6, h7, h8
Choices for Queen3: a1, a2, a3, a4, ..., h6, h7, h8
...
Choices for Queen8: a1, a2, a3, a4, ..., h6, h7, h8

Total Number of choices: 64 x 64 x 64 x 64 = 64^8 ~ 10^14
                                        TOO MANY!
\end{verbatim}
\end{frame}

\begin{frame}[fragile]{Reducing the size of 8-Queen (1)}

  \begin{columns}
    \column{0.8\textwidth}
    To reduce the number of choices, we force each queen to be on a different row, and only choose the column.\bigskip

    {\bf Algorithm (2)}: Eight loops size 8: Try all columns for queen 1 (fix row 1), all columns for queen 2 (fix row 2), ..., etc.
    \column{0.2\textwidth}
    \includegraphics[width=1\textwidth]{img/8queen}
  \end{columns}
\begin{verbatim}
Choices for Queen1: a1, b1, c1, d1, e1, f1, g1 or h1
Choices for Queen1: a2, b2, c2, d2, e2, f2, g2 or h2
Choices for Queen1: a3, b3, c3, d3, e3, f3, g3 or h3
...
Choices for Queen8: a8, b8, c8, d8, e8, f8, g8 or h8

Total Solutions: 8 x 8 x 8 ... = 8^8 ~ 10^7
                                    Still Big!
\end{verbatim}
\end{frame}

\begin{frame}[fragile]{Reducing the size of 8-Queen (2)}
  \begin{columns}
    \column{0.8\textwidth}
      We can reduce the number of choices even more, if we force the columnd and the row of each queen to be different.\bigskip

    {\bf Algorithm (3)}: Fix the row for each queen, 1 loop through the \alert{permutation} of columns.
    \column{0.2\textwidth}
    \includegraphics[width=1\textwidth]{img/8queen}
  \end{columns}
\begin{verbatim}
Choice 1: {a1, b2, c3, d4, e5, f6, g7, h8}
Choice 2: {a1, b2, c3, d4, e5, f6, g8, h7}
Choice 3: {a1, b2, c3, d4, e5, f7, g6, h8}
Choice 4: {a1, b2, c3, d4, e5, f7, g8, h6}
...

Total Solutions: 8 x 7 x 6 x 5 ... = 40320  --  OK!
\end{verbatim}
\end{frame}

\begin{frame}{Reducing the size of 8-Queen: Summary}

  \hfill \includegraphics[width=0.2\textwidth]{img/8queen}
  \begin{itemize}
    \item Algorithm 1: All rows and columns: $10^{14}$ steps;
    \item Algorithm 2: Fix rows, All columns: $10^{7}$ steps;
    \item Algorithm 3: Permutation: 40320 steps;
  \end{itemize}
\end{frame}

\begin{frame}{How many steps is too slow?}{$lt$ 10.000.000 steps is a good goal}

  \begin{block}{n $<$ 24}
    Almost anything works: Brute force! (Maybe not factorial)
  \end{block}

  \begin{block}{n = 500}
    Cubic algorithms are too big (O($n^3$) = 125.000.000).
    Maybe O($n^2\text{log}n$).
  \end{block}

  \begin{block}{n = 10.000}
    A square algorithm (O($n^2$)) is still okay. Be careful of big constants!
  \end{block}

  \begin{block}{n = 1.000.000}
    O($n$log$n$) = 13.000.000. We might need a linear algorithm!
  \end{block}
\end{frame}



%%%%%%%%%%%%%%%%%%%%%%%%%%%
%\item Hint 3: Writing test cases;
\section{Hint 3: Writing test cases}

\begin{frame}{Hint 3: Write testcases!}
  The description of a problem includes \alert{Example Input}.\bigskip

  However, to be accepted, your program also has to solve the {\bf Hidden Input}.\bigskip

  Many times, your program will not be accepted because {\bf You do not solve Hidden Input Case correctly}.

  \begin{exampleblock}{Example: 3n+1 Problem}
    In the Example Input, $i$ is always $< j$.\medskip

    But in the Hidden Input, sometimes $i > j$!\medskip
  \end{exampleblock}
\end{frame}

% Extend basic Cases
% Write extreme cases (maxcase, etc)
% write cases where you think a bug might be
% write random cases

\begin{frame}{Examples of Trick Hidden Input}

    \begin{itemize}
      \item The input is negative; the input is zero, the input is the maximum value;
      \item The input is out of order, the input is in reversed order;
      \item The input is repeated;\medskip

      \item Graph is disconnected; graph is fully connected;
      \item Lines are parallel; Points are in the same place;
      \item Angle is 0; Area is 0;\medskip

      \item A string input is very long;
      \item A string input has space or newlines;
    \end{itemize}

    \begin{block}{}
      Read the input rule carefully. If it is not forbidden, then it is allowed!
    \end{block}
\end{frame}

\begin{frame}
  \frametitle{Hidden Input: Generating your own Input}

  Of course, you cannot see the hidden input. But you can {\bf create your own input!}\medskip

  I highly recommend that you create your own input to test your program.\medskip

  \begin{exampleblock}{Input Creating List}
    \begin{itemize}
      \item Simple Input ;
      \item Maximum Value Input;
      \item Tricky Case Input;
      \item Random Input (using scripts)
    \end{itemize}
  \end{exampleblock}

  Creating test data for your programs is in research and in work.\\
  Let's learn how to create test data in this course!
\end{frame}

%%%%%%%%%%%%%%%%%%%%%%%%%%%%%%%%%%%%%%%%%%%%%

%\item Hint 4: Input/Output;

% # Dealing with IO
% ## Standard IO Types (https://github.com/stevenhalim/cpbook-code/tree/master/ch1/ch1IO.cpp)
%
% ## Basic String Skills(https://github.com/stevenhalim/cpbook-code/tree/master/ch1/basic_string.cpp)
%
% ## Be careful of output:
% - Case
% - Number of decimal places: Round up or down
% - Tie breakers

\section{Hint 4: Input and Output}
\begin{frame}
  \frametitle{Hint 4: Read the Input and Output carefully}
  Reading the Input is \alert{very important} (as we saw in 3n+1)
  \bigskip

  \begin{itemize}
  \item What is the stop condition?
  \begin{itemize}
    \item The number of inputs is given in the beginning;
    \item Special value to indicate the end of the input;
    \item Input ends at the end of the file (EOF);
  \end{itemize}
  \medskip

  \item What is the format of the input/output?
  \begin{itemize}
    \item Very important for output (floating point precision)
  \end{itemize}
  \medskip

  \item What is the size of the input?
  \begin{itemize}
    \item What is the maximum number of inputs?
    \item What are the maximum and minimum values?
    \item Are there special conditions?
  \end{itemize}
  \end{itemize}
\end{frame}


\begin{frame}[fragile]
  \frametitle{There are 3 common standards for input}

  \begin{enumerate}
  \item Read $N$, then read $N$ queries;
\begin{verbatim}
int n; cin >> n;
for (; n > 0; n--) { // Do something }
\end{verbatim}
\item Read until a special condition; (example, until $n = 0$)
\begin{verbatim}
int n; cin >> n;
while (n != 0) { // Do something
cin >> n; }
\end{verbatim}
  \item Continue reading until no more Input\hfil (Until EOF);
\begin{verbatim}
int a, b;
while (cin >> a >> b;) { // Do something! }
\end{verbatim}
  \end{enumerate}
\end{frame}

\begin{frame}
  \frametitle{Output: You need to return EXACTLY what the problem requires}
  \begin{columns}
    \column{0.2\textwidth}
  \includegraphics[width=1\textwidth]{../img/angryclient}
    \column{0.8\textwidth}

  Be careful with the common mistakes:\medskip
  \begin{itemize}
    \item REMOVE PRINT DEBUG.\medskip

    \item UPPERCASE vs lowercase
    \item Singular vs plural
    \item Start with 1 vs start with 0
    \medskip

    \item Number of float digits: 0.34 x 0.340
    \item Round up or round down?: 0.34 x 0.35
    \medskip

    \item Solution order;\medskip

    \item etc...
  \end{itemize}
  \end{columns}
\end{frame}

%%%%%%%%%%%%%%%%%%%%%%%%%%%%%%%%%%%%%%%%%%%%

\section{Hint 5: How to ask for help}

\begin{frame}{Hint 5: How to ask for help}
  \begin{block}{}
  When you are stuck, it is okay to ask for help to your friend, or to the teacher.\medskip

  If you give information about your problem, it is easier for other people to help you.
  \end{block}\bigskip

  \begin{itemize}
    \item Send your current code. (Do not send screenshot!)\medskip

    \item Explain the idea of your code.\medskip

    \item Send the Test Input you created.\medskip

    \item Explain the idea of the test input your created.
  \end{itemize}
\end{frame}

\section{Conclusion}

\begin{frame}\frametitle{Final message}
  Thank you for listening!\vfill

  Have fun writing programs.\vfill

  Be faster than your friends!\vfill

  Ask questions on manaba.
\end{frame}
