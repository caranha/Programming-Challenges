
\section{Class System}
\subsection{Outline}

\begin{frame}
  \centering
  {\huge
    Week 1 -- Part 2: How this lecture is organized
  }
\end{frame}

\begin{frame}
  \frametitle{Outline}
  \begin{enumerate}
    \item Class Schedule
    \item Class Materials
    \item How to submit problems
    \item Grading
    \item Office Hours and Teacher Communication
    \item \alert{Special} Distance Learning in 2020
  \end{enumerate}
\end{frame}

\subsection{Class Schedule}
\begin{frame}{What you will do every week}
  \begin{description}[(manaba)]
    \item[(manaba)] Get the week PDF and study the lecture;
    \medskip
    \item[(manaba)] Watch the lecture video;
    \medskip
    \item[(URI)] Check the Programming Homework Exercises;
    \medskip
    \item[(TEAMS)] Ask questions to the professor;
    \medskip
    \item[(URI)] Submit your programs at the URI page;
    \medskip
    \item[(manaba)] Complete the Homework Survey;
  \end{description}
\end{frame}

\begin{frame}{Class Dates and Deadlines}
  \begin{block}{Class Dates}
    \begin{itemize}
    \item 4/13, 4/20, 4/27, 5/11, 5/18, 5/25, 6/1, 6/8, 6/15, 6/22;
    \item No final exam;
    \item On-demand lecture + Office hours on Tue-34;
    \end{itemize}
  \end{block}
  \begin{block}{Deadlines}
    \begin{itemize}
      \item The deadline for homework: {\bf Monday every week}
      \item The deadline for late homework: 7/5 (one extra week)
      \item Final Grades will be published around 7/9
    \end{itemize}
  \end{block}
  Dates subject to changes.
\end{frame}

\subsection{Class Materials}
\begin{frame}{Where to find the material?}
  \begin{block}{manaba and streams}
    \begin{itemize}
      \item Official place for lecture materials and links;
      \item Use Forum for questions;
      \item Please read announcements; Please answer surveys;
      \item Self-registration Code for non-credit students: 4754254
    \end{itemize}
  \end{block}
  \begin{block}{github}
    \begin{itemize}
      \item Lecture materials is also available on github:
      \item URL: \url{https://caranha.github.io/Programming-Challenges/}
      \item Not-official. Includes material from last year.
      \item Access if manaba is not available.
    \end{itemize}
  \end{block}
\end{frame}

\begin{frame}{Websites for the programming homework}
  \begin{block}{URI Online Judge -- check / submit howework here}
    \begin{itemize}
      \item \url{https://www.urionlinejudge.com.br}
      \item Discipline {\bf ID: 007272} (name: Programming Challenges, Spring 2021)
      \item Key: {\bf B8xVp90}
      \bigskip

      \item {\bf Important!}
      \item After you create your account, submit the "ID survey" in manaba;
      \item After you complete each homework, submit the "Homework survey" in manaba;
    \end{itemize}
  \end{block}
  For more information about how to use URI, please see the "URI Howto" video, or read the outline on manaba
\end{frame}

\begin{frame}{About Course Language}
  \begin{block}{Natural Language}
    \begin{itemize}
      \item Weekly Materials and Homework: English
      \item Weekly Video and manaba: Japanese
      \item E-mail, feedback: Any language you want;
      \item If you want to help me translate the homework, contact me!
    \end{itemize}
  \end{block}

  \begin{block}{Programming Language}
    \begin{itemize}
      \item Officially, we only support C and C++;
      \item The Judge accepts: C, C++, Java, Python;
      \item If you want to use another language, contact me; No promises.
    \end{itemize}
  \end{block}
\end{frame}

\begin{frame}{Reference Books}
  Textbook:
  \begin{itemize}
    \item {\bf textbook:} Steven Halim, Felix Halim,"Competitive Programming", 3rd edition. \url{https://cpbook.net/}
  \end{itemize}
  \bigskip

  Other references:
  \begin{itemize}
    \item Steven S. Skiena, Miguel A. Revilla,"Programming Challenges", Springer, 2003
    \item 秋葉拓哉、 岩田陽一、 北川宜稔,『プログラミングコンテストチャレンジブック』
    \item 渡部有隆、『オンラインチャレンジではじめるC/C++プログラミング入門、Online Programming Challenge!』 (ISBN978-4-8399-5110-8)
    \item 渡部有隆、『プログラミングコンテスト攻略のためのアルゴリズムとデータ構造』(ISBN978-4-8399-5295-2)
  \end{itemize}
\end{frame}

\subsection{Grading}
\begin{frame}{Grading Rules}{Base Grade}

  Your {\bf base grade} is based on the \structure{number of accepted homework programs} you submit:

  \begin{itemize}
    \item {\bf A grade:} 4+ accepted minimum every week;
    \item {\bf B grade:} 3+ accepted minimum every week;
    \item {\bf C grade:} 2+ accepted minimum every week;
    \item {\bf D grade:} less than 2 accepted every week.
  \end{itemize}\bigskip

  \begin{alertblock}{Important!!}
    \begin{itemize}
      \item Only "Accepted" programs count. "Wrong Answer", "Time limit" does not count;
      \item The number above is {\bf minimum every week}. NOT AVERAGE.
    \end{itemize}
  \end{alertblock}
\end{frame}

\begin{frame}{Grading Rules}{Best Bonus}
  For each grade group, 10\% of the students with the {\bf most total problems}
  receive a bonus grade. The bonus grade raise 1 step: C to B, B to A, A to A+, etc.

  \begin{block}{Example:}
  15 students are in base grade A (4 problems per week):
  \begin{itemize}
    \item 8 students with 40 problems
    \item 1 student with 41 problems
    \item 2 students with 50 problems
    \item 1 student with 65 problems
    \item 3 students with 72 problems
  \end{itemize}\bigskip

  The 3 students with 72 problems increase their grade from A to A+.
  \end{block}
\end{frame}


\begin{frame}{Grading Rules}{Late Penalty}
  If you do not meet the deadline, you can submit your homework after the deadline.
  If you submit too many programs after the deadline, you will receive a grade penalty.\bigskip

  If the number of late programs $\ge$ 25\% of total programs, your grade will lower 1 step.\bigskip

  {\bf You will not fail the course for late programs.}

  \begin{exampleblock}{Example:}
    \begin{itemize}
      \item Student (1) submitted 40 problems, minimum 4 problems per exercise. 5 problems are late. $5 \le 40*0.25$, no penalty. Grade A.
      \item Student (2) submitted 44 problems, minimum 4 problems per exercise. 16 problems are late. $16 \ge 44*0.25$, {\bf penalty}. Grade B.
      \item Student (3) submitted 24 problems, minimum 2 problems per exercise. 10 problems are late. $10 \ge 24*0.25$, {\bf penalty}. Grade C (will not fail).
    \end{itemize}
  \end{exampleblock}
\end{frame}


\begin{frame}{Grading}{Plagiarism}
  The assignments are \alert{individual}. You must write your
  programs by yourself.

  \begin{exampleblock}{You can do this}
    \begin{itemize}
    \item Ask for ideas to your friends;
    \item Ask for ideas in the MANABA forum;
    \item Ask for help with a bug;
    \end{itemize}
  \end{exampleblock}

  \begin{alertblock}{You can NOT do this}
    \begin{itemize}
    \item Copy a solution from the internet;
    \item Copy a solution from your friends;
    \item Give your code to a friend;
    \end{itemize}
  \end{alertblock}

  Students who do plagiarism will fail the course, and suffer penalties from the university.
\end{frame}
