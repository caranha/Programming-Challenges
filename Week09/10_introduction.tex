
\section{Introduction}

\subsection{Motivation}
\begin{frame}[t]{What is Computational Geometry?}
  In programming challenges, Computational Geometry problems involve answering questions about {\bf lines, points and angles}. Some examples of pure Comp. Geometry problems:

  \begin{block}{Example 1}
    Given a set of $N$ points $(s_1, s_2, s_3, \ldots, s_N)$, what is
    the area of the smallest polygon that covers all points in the set?
  \end{block}

  \centering
  \includegraphics[width=0.8\textwidth]{img/sampleproblem_1.png}
\end{frame}

\begin{frame}[t]{What is Computational Geometry?}
  In programming challenges, Computational Geometry problems involve answering questions about {\bf lines, points and angles}. Some examples of pure Comp. Geometry problems:

  \begin{block}{Example 2}
    Given $N$ rectangles, $\{x_1,y_1,w_1,h_1\}; \ldots; \{x_N, y_N, w_N, h_N\}$, what is the smallest length of line segments needed to connect them?
  \end{block}

  \centering
  \includegraphics[width=0.6\textwidth]{img/sampleproblem_2.png}
\end{frame}

\begin{frame}[t]{What is Computational Geometry?}
  In programming challenges, Computational Geometry problems involve answering questions about {\bf lines, points and angles}. Some examples of pure Comp. Geometry problems:

  \begin{block}{Example 3}
    Given a polygon and a set of $N$ points, find a line that divides the polygon in equal areas, with the same number of points in each area?
  \end{block}

  \begin{center}
    \includegraphics[width=0.4\textwidth]{img/sampleproblem_3.png}
  \end{center}
\end{frame}


\begin{frame}
  \frametitle{Computational Geometry}
  \framesubtitle{The good and the bad}

  Computational Geometry problems have some merits and demerits when compared to other problems that we studied until now.

  \begin{block}{Positive Points}
    \begin{itemize}
      \item Geometry problems are fun, and you draw pretty pictures when thinking about them (ok, maybe this one is a bit personal);
      \item A large part of geometry problems can be solved with algorithms and techniques that you learned in high school;
      \item The code for techniques is highly re-usable;
    \end{itemize}
  \end{block}

  \begin{alertblock}{Negative Points}
    \begin{itemize}
      \item You have to write a lot of code (in the beginning, at least);
      \item Easy to get WE for small mistakes;
      \item Many special cases in the input data;
    \end{itemize}
  \end{alertblock}
\end{frame}


\begin{frame}
  \frametitle{Common Mistakes in Geometry Problems}

    \begin{block}{Errors because of special cases in input data}
      \begin{itemize}
        \item Multiple points in the same position;
        \item Collinear points (three points in the same line);
        \item Vertical lines (bad tangent value, division by 0);
        \item Parallel Lines (bad intersection value);
        \item Intersection at end of a segment;
        \item etc;
      \end{itemize}
    \end{block}

    \begin{block}{Floating Number Precision Errors}
      \begin{itemize}
        \item Wrong Answer because of poor rounding of final result;
        \item Error Propagation inside functions (multiplication, division);
      \end{itemize}
    \end{block}
\end{frame}


\begin{frame}[fragile]
  \frametitle{Common Mistakes in Geometry Problems}

  How to avoid these mistakes in Geometry Problems?\bigskip

  \begin{itemize}
    \item {\bf Special Cases}:
    \begin{itemize}
      \item Make sure to think which special cases affect your tecnique, and add checks for these cases;
      \item When testing your problem, include input with the special case;
    \end{itemize}\bigskip

    \item {\bf Precision Errors}:
    \begin{itemize}
      \item If possible, convert values to integers before calculation;
      \item When testing equality of two values, use an {\bf Epsilon Constant}:
\begin{verbatim}
if (float.1 == float.2) then            // NO
if (fabs(float.1 - float.2) < EPS) then // YES!
\end{verbatim}
    \end{itemize}
  \end{itemize}
\end{frame}

\subsection{Class Outline}
\begin{frame}
  \frametitle{Class Outline}
  In this lecture, we will focus on:
  \begin{itemize}
    \item Discussion of implementation of geometric operations;
    \item Discussion of problem examples;
  \end{itemize}\bigskip

  Specific topics will be:
  \begin{itemize}
    \item Intersection and Rotation of points and lines;\medskip
    \item Circle representation and components;\medskip
    \item Triangles (area, angles, triangles and circles);\medskip
    \item Polygon representation and Convex Hull;
  \end{itemize}\medskip
\end{frame}
