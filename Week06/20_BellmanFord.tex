\section{Negative Loops}

\begin{frame}
  \begin{center}
    {\bf Part II - Bellman-Ford (Negative Loops);}
  \end{center}
\end{frame}

\begin{frame}
  \frametitle{A Problem with Dijkstra}

    \begin{block}{}
      Djikstra can have difficulty when the graph includes a {\bf negative loop}!
    \end{block}

    \begin{center}
      \begin{tikzpicture}[transform shape,label/.style={thin, draw=black, align=center,fill=white,font=\smaller},scale=1.1]
        \node[red vertex] (s) at (0,0) {s};
        \node[vertex] (0) at (1,1) {a};
        \node[vertex] (1) at (2,0) {b};
        \node[vertex] (e) at (4,0) {e};
        \draw[edge] (s) -- node[label] {-2} (0);
        \draw[edge] (s) -- node[label] {2} (1);
        \draw[edge] (1) -- node[label] {-2} (0);
      \draw[edge] (1) -- node[label] {2} (e);
      \end{tikzpicture}
    \end{center}

    \begin{itemize}
    \item Our Dijkstra implementation will add smaller and smaller costs to the priority queue:
    \begin{itemize}
      \item $s \to a$: -2, -4, -6, -8...
    \end{itemize}

    \item Other implementations will have different problems because negative loop breaks the {\bf Greedy Property}.
    \begin{itemize}
      \item But what is the size of the path when a negative loop exists?
    \end{itemize}

    \item {\bf Bellman Ford's algorithm} is a slower SSSP algorithm that {\bf detects negative loops};
    \end{itemize}
\end{frame}

\begin{frame}[fragile]
  \frametitle{Bellman Ford's Algorithm -- ($O(VE)$)}
  {\smaller
  \begin{itemize}
    \item The main idea is to propagate the weight of every edge $i\to j$, $V-1$ times.

    \item The vector of distances from $s$, \emph{dist}, starts with dist[s]=0, and dist[!s]=INF;

    \item Each iteration, non-inf values of dist propagate;

    \item Because the algorithm has a finite number of loops, it always terminates;
    \item Algorithm stabilizes at iteration $V-1$. If dist changes after that, we have detected an infinite loop.
  \end{itemize}

  \begin{exampleblock}{Pseudocode (uses EdgeList data structure)}
\begin{verbatim}
vector<int> dist(V, INF); dist[s] = 0;  // Start Condition
int edges[E][3];                        // Edge list (i,j,w)
for (int i = 0; i < V - 1; i++)         // repeat V-1 times
 for (int u = 0; u < E; u++) {          // for all edges
  dist[edges[u][1]] = min(dist[edges[u][1]],
                          dist[edges[u][0]]+edges[u][2]);
 }
\end{verbatim}
\end{exampleblock}}
\end{frame}

\begin{frame}
  \frametitle{Bellman Ford Simulation: Regular Graph}
  \begin{center}
    \begin{tikzpicture}[transform shape,label/.style={thin, draw=black, align=center,fill=white,font=\smaller},scale=1.1]
      \node[red vertex] (s) at (0,0) {s};
      \node[vertex] (0) at (2,1) {a};
      \node[vertex] (1) at (4,1) {b};
      \node[vertex] (2) at (3,-1) {c};
      \node[vertex] (e) at (6,0) {t};
      \tikzset{edge/.style = {->, >=latex'}}
      \draw[edge] (2) to node[label] {$1$} (s);
      \draw[edge] (0) to node[label] {$2$} (2);
      \draw[edge] (s) to node[label] {$1$} (0);
      \draw[edge] (0) to node[label] {$3$} (1);
      \draw[edge] (1) to node[label] {$2$} (e);
      \draw[edge] (2) to node[label] {$2$} (e);
    \end{tikzpicture}
  \end{center}
  \vspace{10cm}
\end{frame}

\begin{frame}
  \frametitle{Bellman Ford Simulation: Negative Loop}
  \begin{center}
    \begin{tikzpicture}[transform shape,label/.style={thin, draw=black, align=center,fill=white,font=\smaller},scale=1.1]
      \node[red vertex] (s) at (0,0) {s};
      \node[vertex] (0) at (2,1) {a};
      \node[vertex] (1) at (4,1) {b};
      \node[vertex] (2) at (3,-1) {c};
      \node[vertex] (e) at (6,0) {t};
      \tikzset{edge/.style = {->, >=latex'}}
      \draw[edge] (2) to node[label] {$-4$} (s);
      \draw[edge] (0) to node[label] {$2$} (2);
      \draw[edge] (s) to node[label] {$1$} (0);
      \draw[edge] (0) to node[label] {$3$} (1);
      \draw[edge] (1) to node[label] {$2$} (e);
      \draw[edge] (2) to node[label] {$2$} (e);
    \end{tikzpicture}
  \end{center}
  \vspace{10cm}
\end{frame}
