\section{Conclusion}

\begin{frame}
  \frametitle{Lecture Summary}

  Graph Algorithms for Path Finding and Maximum Flow:\bigskip

  \begin{itemize}
    \item Single Source Shortest Path:
    \begin{itemize}
      \item In an unweighted graph, use BFS;
      \item For a weighted graph, use Dijkstra;
      \item If the graph has negative loops, Bellman-ford will terminate;
    \end{itemize}\medskip
    \item All Pairs Shortest path:
    \begin{itemize}
      \item Floyd-Warshall is very easy to program, but costs $O(V^3)$;
      \item You could also just repeat Dijkstra $V$ times;
    \end{itemize}\medskip
    \item Maximum Flow:
    \begin{itemize}
      \item The Ford-Fergusson Method describes how to find the maximum Flow;
      \item Edmond-Karp implements FF using BFS on the residual graph to find minimum paths;
    \end{itemize}
  \end{itemize}\bigskip

  The most important skill to learn for graph problems is {\bf how to transform the problem graph}.
\end{frame}


\subsection{Problem Discussion}

\begin{frame}
  \frametitle{This Week's Problems}
  \begin{itemize}
  \item Wormholes;
  \item Meeting Professor Miguel;
  \item Full Tank?;
  \item Degrees of Separation;
  \item Avoiding your Boss;
  \item Software Allocation;
  \item Sabotage;
  \item Gopher II;
  \end{itemize}
\end{frame}

\begin{frame}
  \frametitle{Problem Hints}
  \begin{block}{Wormholes}
    \begin{itemize}
    \item {\bf Problem goal:} Find a negative weight loop in the graph
    \item {\bf Hint:} Just follow the suggestions from the class
    \end{itemize}
  \end{block}

  \begin{exampleblock}{Meeting Professor Miguel}
    \begin{itemize}
    \item {\bf Problem goal:} Find the shortest path from the student to the professor.
    \item {\bf Trick:} Some edges only the student can walk, some
      edges only the professor can walk;
    \item {\bf Hint 1:} There are really two graphs: One for the student, one for the
      professor;
    \item {\bf Hint 2:} The graphs are really small;
    \end{itemize}
  \end{exampleblock}
\end{frame}

\begin{frame}
  \frametitle{Problem Hints}
  \begin{block}{Full Tank?}
    \begin{itemize}
    \item Discussed in class;
    \item Good practice on modifying graphs to solve problems;
    \end{itemize}
  \end{block}
  \begin{exampleblock}{Degrees of Separation}
    \begin{itemize}
    \item {\bf Problem Outline:} Given a network of relationship, define the "Degree of Separation" of the network. "Degree of separation" is the {\bf largest shortest path} in the network.\bigskip

    \item {\bf Hint 1:} Don't forget the special case of a disconnected Graph;
    \item {\bf Hint 2:} The "largest shortest path" of a graph is also known as the {\bf diameter}, and it is an important property of graphs;
    \end{itemize}
  \end{exampleblock}
\end{frame}

\begin{frame}
  \frametitle{Problem Hints}
  \begin{block}{Avoiding Your Boss}
    \begin{itemize}
    \item {\bf Problem Goal:} You are going from your house to the market. Your boss is going from her house to the office. Can you find a path that does not meet your boss' path?\bigskip

    \item {\bf Hint:} How do you modify the graph so that you can avoid your boss?
    \end{itemize}
  \end{block}

  \begin{exampleblock}{Software Allocation}
    \begin{itemize}
    \item Discussed in Class;
    \item Use this problem to practice "Max Flow";
    \end{itemize}
  \end{exampleblock}
\end{frame}

\begin{frame}
  \frametitle{Problem Hints}
  \begin{block}{Sabotage}
    \begin{itemize}
    \item {\bf Problem Goal:} Find the cost of minimum cut;
    \item Discussed in class, use Max Flow to find the minimum cut set;
    \end{itemize}
  \end{block}

  \begin{exampleblock}{Gopher II}
    {\bf Problem Outline:}
    \begin{itemize}
    \item You receive the position of $N$ gophers and $M$ gopher holes.
    \item Each hole can save 1 Gopher, Gophers run at speed $v$.
    \item Hawks will eat the Gophers after $t$ seconds;
    \item How many gophers are saved? How many are eatern?
    \end{itemize}
    {\bf Hints}:
    \begin{itemize}
      \item Think of an {\bf Allocation} of gophers to holes;
      \item Graph is implicit: How do you define the edges?
    \end{itemize}
  \end{exampleblock}
\end{frame}
