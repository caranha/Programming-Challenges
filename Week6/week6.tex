\documentclass{beamer}
% TODO: Add code, review class
% TODO: new change


\usepackage{amssymb,amsmath}
\usepackage{graphicx}
\usepackage{url}
\usepackage{color}
\usepackage{relsize}		% For \smaller
\usepackage{url}			% For \url
\usepackage{epstopdf}	% Included EPS files automatically converted to PDF to include with pdflatex
\usepackage{pagenote}[continuous,page]

%For MindMaps
% \usepackage{tikz}%
% \usetikzlibrary{mindmap,trees,arrows}%

%%% Color Definitions %%%%%%%%%%%%%%%%%%%%%%%%%%%%%%%%%%%%%%%%%%%%%%%%%%%%%%%%%
%\definecolor{bordercol}{RGB}{40,40,40}
%\definecolor{headercol1}{RGB}{186,215,230}
%\definecolor{headercol2}{RGB}{80,80,80}
%\definecolor{headerfontcol}{RGB}{0,0,0}
%\definecolor{boxcolor}{RGB}{186,215,230}

%%% Save space in lists. Use this after the opening of the list %%%%%%%%%%%%%%%%
%\newcommand{\compresslist}{
%	\setlength{\itemsep}{1pt}
%	\setlength{\parskip}{0pt}
%	\setlength{\parsep}{0pt}
%}

%\setbeameroption{show notes on top}

% You should run 'pdflatex' TWICE, because of TOC issues.

% Rename this file.  A common temptation for first-time slide makers
% is to name it something like ``my_talk.tex'' or
% ``john_doe_talk.tex'' or even ``discrete_math_seminar_talk.tex''.
% You really won't like any of these titles the second time you give a
% talk.  Try naming your tex file something more descriptive, like
% ``riemann_hypothesis_short_proof_talk.tex''.  Even better (in case
% you recycle 99% of a talk, but still want to change a little, and
% retain copies of each), how about
% ``riemann_hypothesis_short_proof_MIT-Colloquium.2000-01-01.tex''?

\mode<presentation>
{
  % A tip: pick a theme you like first, and THEN modify the color theme, and then add math content.
  % Warsaw is the theme selected by default in Beamer's installation sample files.

  %%%%%%%%%%%%%%%%%%%%%%%%%%%% THEME
  %\usetheme{Madrid}		% No subsection
  \usetheme{AnnArbor}  % Subsection on top, no color


  %\usetheme{Antibes}
  %\usetheme{Bergen}
  %\usetheme{Berkeley}		% bem bacana - menu esquerdo
  %\usetheme{Berlin}
  %\usetheme{Boadilla}
  %\usetheme{boxes}
  %\usetheme{CambridgeUS}		% bem bacana - menu superior
  %\usetheme{Copenhagen}
  %\usetheme{Darmstadt}
  %\usetheme{default}
  %\usetheme{Dresden}
  %\usetheme{Frankfurt}
  %\usetheme{Goettingen}
  %\usetheme{Hannover}		% bem bacana - menu esquerdo
  %\usetheme{Ilmenau}
  %\usetheme{JuanLesPins}
  %\usetheme{Luebeck}
  %\usetheme{Malmoe}
  %\usetheme{Marburg}		% bem bacana - menu direito
  %\usetheme{Montpellier}
  %\usetheme{PaloAlto}		% bem bacana - menu esquerdo
  %\usetheme{Pittsburgh}
  %\usetheme{Rochester}		%bacana
  %\usetheme{Singapore}
  %\usetheme{Szeged}
  %\usetheme{Warsaw}

  %%%%%%%%%%%%%%%%%%%%%%%%%%%% COLOR THEME
  %\usecolortheme{default}		% branco, azul clarinho
  \usecolortheme{crane}		% Very yellow (ok)

  %\usecolortheme{albatross}		% azul escuro, massa
  %\usecolortheme{beetle}		% cinza, menu azul
  %\usecolortheme{dolphin}		% azul e branco, legal
  %\usecolortheme{dove}			% cinza e branco, feio
  %\usecolortheme{fly}			% todo cinza, horrível
  %\usecolortheme{lily}			% parece o default
  %\usecolortheme{orchid}		% azul e branco, ok
  %\usecolortheme{rose}			% branco e violeta-claro, bonito
  %\usecolortheme{seagull}		% cinza, feio
  %\usecolortheme{seahorse}		% nhé, meio feio
  %\usecolortheme{sidebartab}		% Azul, branco, destaque na tab, interessante
  %\usecolortheme{structure}		% bichado
  %\usecolortheme{whale}		% Azul e branco, bem bonito

  %%%%%%%%%%%%%%%%%%%%%%%%%%%% OUTER THEME
  \useoutertheme{default}
  %\useoutertheme{infolines}
  %\useoutertheme{miniframes}
  %\useoutertheme{shadow}
  %\useoutertheme{sidebar}
  %\useoutertheme{smoothbars}
  %\useoutertheme{smoothtree}
  %\useoutertheme{split}
  %\useoutertheme{tree}

  %%%%%%%%%%%%%%%%%%%%%%%%%%%% INNER THEME
  \useinnertheme{circles}
  %\useinnertheme{default}
  %\useinnertheme{inmargin}
  %\useinnertheme{rectangles}
  %\useinnertheme{rounded}

  %%%%%%%%%%%%%%%%%%%%%%%%%%%%%%%%%%%

  \setbeamercovered{invisible} % or whatever (possibly just delete it)
  % To change behavior of \uncover from graying out to totally
  % invisible, can change \setbeamercovered to invisible instead of
  % transparent. apparently there are also 'dynamic' modes that make
  % the amount of graying depend on how long it'll take until the
  % thing is uncovered.

}


% Get rid of nav bar
\beamertemplatenavigationsymbolsempty

% Use short top
%\usepackage[headheight=12pt,footheight=12pt]{beamerthemeboxes}
%\addheadboxtemplate{\color{black}}{
%\hskip0.5cm
%\color{white}
%\insertshortauthor \ \ \ \
%\insertframenumber \ \ \ \ \ \ \
%\insertsection \ \ \ \ \ \ \ \ \ \ \ \ \ \ \ \ \  \insertsubsection
%\hskip0.5cm}
%\addheadboxtemplate{\color{black}}{
%\color{white}
%\ \ \ \
%\insertsection
%}
%\addheadboxtemplate{\color{black}}{
%\color{white}
%\ \ \ \
%\insertsubsection
%}

% Insert frame number at bottom of the page.
% \usefoottemplate{\hfil\tiny{\color{black!90}\insertframenumber}}

%% makes the ppagenote command for figure references at the end.

\usepackage[english]{babel}
%qq\usepackage[latin1]{inputenc}
\usepackage{CJKutf8}
\usepackage{subfigure}

\usepackage{times}
\usepackage[T1]{fontenc}

\makepagenote
\renewcommand{\notenumintext}[1]{}
\newcommand{\ppagenote}[1]{\pagenote[Page \insertframenumber]{#1}}

\title[Programming Challenges]{GB20602 - Programming Challenges}
\author[Claus Aranha]{Claus Aranha\\{\footnotesize caranha@cs.tsukuba.ac.jp}}
\institute[U. Tsukuba]{University of Tsukuba, Department of Computer Sciences}


\title[]{Programming Challenges}
\subtitle[]{Week 6 - Dynamic Programming}
\author[Claus Aranha]{Claus Aranha\\{\footnotesize caranha\@@cs.tsukuba.ac.jp}}
\institute{College of Information Siences}
\date{2015-05-29\\{\tiny Last updatet \today}}

\begin{document}

\begin{frame}
\maketitle
\end{frame}

\section{Dynamic Programming}
\subsection{Introduction}

\begin{frame}
  \frametitle{Dynamic Programming -- Outline}
  \begin{block}{}
    \begin{itemize}
    \item Extremely common on programming challenges;
    \item The basic idea is quite simple, but sometimes it is hard to
      grasp;
    \item Let's use an example based approach;
    \end{itemize}
  \end{block}
\end{frame}

\subsection{Coint to a sum}

%% Simples example: Coins-to-a-sum
\begin{frame}
  \frametitle{First Example: Coins to a Sum}
  \begin{block}{Sample Problem}
    We have many coins worth \structure{1}, {3} and {5} yenes. How can
    we combine these coins to add $S$ using the minimum amount of
    coins?
  \end{block}
  \medskip
  {\small ... No, using modulo does not count}
\end{frame}

\begin{frame}
  \frametitle{First Example: Coins to a Sum}
  \begin{center}
    \begin{tabular}{|c|c|c|c|c|c|c|c|c|c|c|c|}
      \hline
          {\bf S} & 1 & 2 & 3 & 4 & 5 & 6 & 7 & 8 & 9 & 10 & 11\\
          \hline
              {\bf C} & \only<2->{1} & \only<2->{2}& \only<3->{1} & \only<3->{2} & \only<3->{1} & \only<4->{2} & \only<4->{3} & \only<4->{2} & \only<4->{3} &\only<4->{2} &\only<4->{3} \\
              \hline
    \end{tabular}
  \end{center}
  \begin{block}{}
    \begin{itemize}
    \item How many coins are needed for $S=1$? for $S=2$? for $S=3$? for $S=i$?
    \item<5-> If we know the values of $S_{i-1}, S_{i-3}, S_{i-5}$, we
      can use those to calculate $S_i$;
    \end{itemize}
  \end{block}
\end{frame}

\begin{frame}
  \frametitle{Basic Idea for Dynamic Programming}
  \begin{block}{Table of Partial Results}
    We want to create a table with all partial results up to the
    desired value.
  \end{block}
  \medskip
  \begin{itemize}
  \item How to define the table?
  \item How to fill the table's values?
  \end{itemize}
\end{frame}

\begin{frame}
  \frametitle{More Coins to a Sum}
  \begin{itemize}
  \item Filling the table forward:
    \begin{onlyenv}<1>
    \begin{center}
      \begin{tabular}{c|c|c|c|c|c|c|c|c|c|c|c|c}
        \hline
            {\bf S} & 0 & 1 & 2 & 3 & 4 & 5 & 6 & 7 & 8 & 9 & 10 & 11\\
            \hline
                {\bf C} & - & 1 & 2 & 1 & 2 & 1 & 2 & 3 & 2 & 3 & 2 & 3\\
                \hline
      \end{tabular}
    \end{center}
    \medskip
    \end{onlyenv}

  \item Filling the table backwards: 
    \begin{onlyenv}<2>
    \begin{center}
      \begin{tabular}{c|c|c|c|c|c|c|c|c|c|c|c|c}
        \hline
            {\bf S} & 0 & 1 & 2 & 3 & 4 & 5 & 6 & 7 & 8 & 9 & 10 & 11\\
            \hline
                {\bf C} & 3 & 2 & 3 & 2 & 3 & 2 & 1 & 2 & 1 & 2 & 1 & -\\
                \hline
      \end{tabular}
    \end{center}
    \medskip
    \end{onlyenv}

  \item Filling the table by coin:
    \begin{onlyenv}<3->
    \begin{center}
      \begin{tabular}{c|c|c|c|c|c|c|c|c|c|c|c|c}
        \hline
            {\bf S} & 0 & 1 & 2 & 3 & 4 & 5 & 6 & 7 & 8 & 9 & 10 & 11\\
            \hline
            {\bf C1} & 0 & 1 & 2 & 3 & 4 & 5 & 6 & 7 & 8 & 9 & 10 & 11\\
            \only<4->{{\bf C2} & 0 & 1 & 2 & 1 & 2 & 3 & 2 & 3 & 4 & 3 & 4 & 5}\\
            \only<5>{{\bf C3} & 0 & 1 & 2 & 1 & 2 & 1 & 2 & 3 & 2 & 3 & 2 & 3}\\
            \hline
      \end{tabular}
    \end{center}
    \end{onlyenv}
  \end{itemize}

\end{frame}

\begin{frame}
  \frametitle{States and Decisions}
  \begin{block}{States}
    A state is a partial solution. Each state is
    \structure{independent}: you don't need to know how the program
    arrived at that state, just its current value.
  \end{block}
  \begin{block}{Decisions}
    Each state has a number of possible \structure{transitions} from
    itself to other states. These are the actions that can be taken in
    the problem.
  \end{block}
  \medskip
  The Key for Dynamic programming is being able to define what are the
  states, and what are the decisions in a problem.
\end{frame}


\subsection{Longest Subsequence}
\begin{frame}
  \frametitle{Example 2: Longest Subsequence}
  \begin{block}{Problem Description}
    Given a sequence $S$ of integers, find the largest subsequence $Z$
    where $z_0 < z_1 < z_2 < \dots < z_n$
  \end{block}
  \bigskip

  Values:\\
  \begin{tabular}{cccccccc}
    2 & 4 & 3 & 8 & 5 & 7 & 4 & 9\\
  \end{tabular}
  
  \begin{onlyenv}<2>
    Length:\\
    \begin{tabular}{cccccccc}
      1 & 2 & 2 &  &  &  &  & \\
    \end{tabular}\\
    Last:\\
    \begin{tabular}{cccccccc}
      0 & 1 & 1 &  &  &  &  & \\
    \end{tabular}\\
  \end{onlyenv}
  \begin{onlyenv}<3>
    Length\\
    \begin{tabular}{cccccccc}
      1 & 2 & 2 & 3 & 3 &  &  & \\
    \end{tabular}\\
    Last:\\
    \begin{tabular}{cccccccc}
      0 & 1 & 1 & 3 & 3 &  &  & \\
    \end{tabular}\\
  \end{onlyenv}
  \begin{onlyenv}<4>
    Length\\
    \begin{tabular}{cccccccc}
      1 & 2 & 2 & 3 & 3 & 4 & 3 & 5 \\
    \end{tabular}\\
    Last:\\
    \begin{tabular}{cccccccc}
      0 & 1 & 1 & 3 & 3 & 5 & 3 & 6 \\
    \end{tabular}\\
  \end{onlyenv}
\end{frame}

\begin{frame}
  \frametitle{Example 2: Longest Subsequence}
  \begin{block}{Algorithm}
    \begin{itemize}
    \item State 0: Length = 1, Parent = 0;
    \item State i: 
      \begin{itemize}
      \item Length $S_i$ = 1; Parent = 0;
      \item For (j = 0 to i-1): 
      \item if ($S_j < S_i$) and Length($S_j$) >= Length($S_i$):
      \item Length($S_i$) = Length($S_j$)+1
      \item Parent($S_i$) = $S_j$
      \end{itemize}
    \end{itemize}
  \end{block}
\end{frame}

\begin{frame}
  \frametitle{Another kind of Recursion}
  \begin{block}{}
    Dynamic Programming problems use a very similar mindset as
    recursive problems: You want to find a recurrence function, and
    use it to calculate new solutions.
  \end{block}
  \bigskip

  \begin{itemize}
    \item It might be useful to store a ``parent'' and ``decision''
      arrays, to restore the solution that was found.
  \end{itemize}
\end{frame}

\subsection{Apples}

\begin{frame}
  \frametitle{When is Dynamic Programming Useful?}
  \begin{itemize}
  \item A ``quick'' way to list all possible solution combinations;
  \item States are mostly independent;
  \item Combination of States/Transitions is polinomial;
  \item Table makes it easier to prune repetitions;
    \bigskip
  \item \alert{Keep in mind the number of transitions!}
  \end{itemize}
\end{frame}


\begin{frame}
  \frametitle{Example 3: Apples}
  \begin{block}{Problem Description}
    A farm has apples spread on a grid. You want to find the
    South/West path on the grid with the maximum number of apples.
    \medskip
    
    Start: Top Left, End: Bottom Right;
  \end{block}
  
  \begin{center}
    \begin{tabular}{|c|c|c|c|c|c|c|c|c|}
      \hline
      2 & 3 & & 3 & 5 & 5 & & 1 & \\
      \hline
       & & 1 & 2 & 2 & 3 & & & 2 \\
      \hline
      2 & & 3 & & 2 & & 2 & 2 & \\
      \hline
       & 3 & 2 & & & & & 3 & 4 \\
      \hline
       & 2 & & 3 & & 3 & & & 3\\
      \hline
       & 2 & & & 3 & & & 3 & \\
      \hline      
    \end{tabular}
  \end{center}
\end{frame}

\begin{frame}
  \frametitle{Example 3: Apples}
  \begin{block}{Solving the Problem}
    \begin{itemize}
    \item What are the states?
    \item What are the transitions?
    \end{itemize}
  \end{block}
  \medskip
  
  Who can solve this first?\\
  \begin{center}
    \begin{tabular}{|c|c|c|c|c|c|c|c|c|}
      \hline
      2 & 3 & & 3 & 5 & 5 & & 1 & \\
      \hline
       & & 1 & 2 & 2 & 3 & & & 2 \\
      \hline
      2 & & 3 & & 2 & & 2 & 2 & \\
      \hline
       & 3 & 2 & & & & & 3 & 4 \\
      \hline
       & 2 & & 3 & & 3 & & & 3\\
      \hline
       & 2 & & & 3 & & & 3 & \\
      \hline      
    \end{tabular}
  \end{center}
\end{frame}

\begin{frame}
  \frametitle{Example 3.5: SUPER Apples}
  \begin{block}{Problem Description}
    The Problem is the same as before, but now you have to cross the
    field three times:
    \begin{enumerate}
      \item Top Left to Bottom Right;
      \item Bottom Right to Top Left;
      \item Top Left to Bottom Right;
    \end{enumerate}
  \end{block}
  \medskip
  \begin{center}
    Give me your ideas!
  \end{center}
\end{frame}

\begin{frame}
  \frametitle{Example 3.5: SUPER Apples}
  \begin{block}{}
    Let's try to solve it!
  \end{block}
  \begin{center}
    \begin{tabular}{|c|c|c|c|c|c|c|c|c|}
      \hline
      2 & 3 & & 3 & 5 & 5 & & 1 & \\
      \hline
      & & 1 & 2 & 2 & 3 & & & 2 \\
      \hline
    2 & & 3 & & 2 & & 2 & 2 & \\
    \hline
    & 3 & 2 & & & & & 3 & 4 \\
    \hline
    & 2 & & 3 & & 3 & & & 3\\
    \hline
    & 2 & & & 3 & & & 3 & \\
    \hline      
    \end{tabular}
  \end{center}
\end{frame}

\begin{frame}
  \frametitle{Example 3.5: SUPER Apples}
  \begin{block}{Hint 1}
    \only<2->{Does direction make a difference?}
  \end{block}

  \begin{block}{Hint 2}
    \only<3->{Does time make a difference?}
  \end{block}
  
  \begin{block}{Hint 3}
    \only<4->{Can/should the 3 pathes cross?}
  \end{block}
\end{frame}

\begin{frame}
  \frametitle{Example 3.5: SUPER Apples}
  \begin{block}{}
    Let's try to solve it!
  \end{block}
  \begin{center}
    \begin{tabular}{|c|c|c|c|c|c|c|c|c|}
      \hline
      2 & 3 & & 3 & 5 & 5 & & 1 & \\
      \hline
      & & 1 & 2 & 2 & 3 & & & 2 \\
      \hline
    2 & & 3 & & 2 & & 2 & 2 & \\
    \hline
    & 3 & 2 & & & & & 3 & 4 \\
    \hline
    & 2 & & 3 & & 3 & & & 3\\
    \hline
    & 2 & & & 3 & & & 3 & \\
    \hline      
    \end{tabular}
  \end{center}
\end{frame}

\begin{frame}
  \frametitle{Another Idea about Dynamic Programming}
  \begin{block}{}
    If the problem sounds too complex, try to change the states
    around, or reduce it to a simpler problem.
  \end{block}
\end{frame}

\subsection{More Examples}

\begin{frame}
  \frametitle{Example 4: Money Djikstra}
  \begin{block}{Problem Definition}
    We want to go from $A_0$ to $A_n$ in an undirected graph. Every
    vertice has a cost $C_i$ to enter that vertice.\\
    \medskip

    We start with total money $S_0$, and we cannot enter a vertice if
    the cost of that vertice is bigger than our current money.\\
    \medskip

    Find the shortest path in the graph. If more than one path have
    the same length, choose the one that spends the least money.
  \end{block}
\end{frame}

\begin{frame}
  \frametitle{Example 5: Edit Distance}
  \begin{block}{Problem Definition}
    Calculate the minimum cost to transform string $a$ into string
    $b$, using the operations ``INSERT'', ``DELETE'', ``CHANGE'' and
    ``NOOP''. The NOOP operation has cost 0, the other operations have
    cost 1.
  \end{block}
\end{frame}

\begin{frame}
  \frametitle{This week's problems}
  \begin{itemize}
  \item Direct Subsequences        
  \item Adventures in Moving IV
  \item Chopsticks
  \item Unidirectional TSP
  \end{itemize}
\end{frame}
\end{document}
