\documentclass{beamer}

\usepackage{amssymb,amsmath}
\usepackage{graphicx}
\usepackage{url}
\usepackage{color}
\usepackage{relsize}		% For \smaller
\usepackage{url}			% For \url
\usepackage{epstopdf}	% Included EPS files automatically converted to PDF to include with pdflatex

%For MindMaps
% \usepackage{tikz}%
% \usetikzlibrary{mindmap,trees,arrows}%

%%% Color Definitions %%%%%%%%%%%%%%%%%%%%%%%%%%%%%%%%%%%%%%%%%%%%%%%%%%%%%%%%%
%\definecolor{bordercol}{RGB}{40,40,40}
%\definecolor{headercol1}{RGB}{186,215,230}
%\definecolor{headercol2}{RGB}{80,80,80}
%\definecolor{headerfontcol}{RGB}{0,0,0}
%\definecolor{boxcolor}{RGB}{186,215,230}

%%% Save space in lists. Use this after the opening of the list %%%%%%%%%%%%%%%%
%\newcommand{\compresslist}{
%	\setlength{\itemsep}{1pt}
%	\setlength{\parskip}{0pt}
%	\setlength{\parsep}{0pt}
%}

%\setbeameroption{show notes on top}

% You should run 'pdflatex' TWICE, because of TOC issues.

% Rename this file.  A common temptation for first-time slide makers
% is to name it something like ``my_talk.tex'' or
% ``john_doe_talk.tex'' or even ``discrete_math_seminar_talk.tex''.
% You really won't like any of these titles the second time you give a
% talk.  Try naming your tex file something more descriptive, like
% ``riemann_hypothesis_short_proof_talk.tex''.  Even better (in case
% you recycle 99% of a talk, but still want to change a little, and
% retain copies of each), how about
% ``riemann_hypothesis_short_proof_MIT-Colloquium.2000-01-01.tex''?

\mode<presentation>
{
  % A tip: pick a theme you like first, and THEN modify the color theme, and then add math content.
  % Warsaw is the theme selected by default in Beamer's installation sample files.

  %%%%%%%%%%%%%%%%%%%%%%%%%%%% THEME
  %\usetheme{AnnArbor}
  %\usetheme{Antibes}
  %\usetheme{Bergen}
  %\usetheme{Berkeley}		% bem bacana - menu esquerdo
  %\usetheme{Berlin}
  %\usetheme{Boadilla}
  %\usetheme{boxes}
  %\usetheme{CambridgeUS}		% bem bacana - menu superior
  %\usetheme{Copenhagen}
  %\usetheme{Darmstadt}
  %\usetheme{default}
  %\usetheme{Dresden}
  \usetheme{Frankfurt}
  %\usetheme{Goettingen}
  %\usetheme{Hannover}		% bem bacana - menu esquerdo
  %\usetheme{Ilmenau}
  %\usetheme{JuanLesPins}
  %\usetheme{Luebeck}
  %\usetheme{Madrid}		%bacana
  %\usetheme{Malmoe}
  %\usetheme{Marburg}		% bem bacana - menu direito
  %\usetheme{Montpellier}
  %\usetheme{PaloAlto}		% bem bacana - menu esquerdo
  %\usetheme{Pittsburgh}
  %\usetheme{Rochester}		%bacana
  %\usetheme{Singapore}
  %\usetheme{Szeged}
  %\usetheme{Warsaw}

  %%%%%%%%%%%%%%%%%%%%%%%%%%%% COLOR THEME
  %\usecolortheme{albatross}		% azul escuro, massa
  %\usecolortheme{beetle}		% cinza, menu azul
  %\usecolortheme{crane}		% branco e amarelo, massa
  \usecolortheme{default}		% branco, azul clarinho
  %\usecolortheme{dolphin}		% azul e branco, legal
  %\usecolortheme{dove}			% cinza e branco, feio
  %\usecolortheme{fly}			% todo cinza, horrível
  %\usecolortheme{lily}			% parece o default
  %\usecolortheme{orchid}		% azul e branco, ok
  %\usecolortheme{rose}			% branco e violeta-claro, bonito
  %\usecolortheme{seagull}		% cinza, feio
  %\usecolortheme{seahorse}		% nhé, meio feio
  %\usecolortheme{sidebartab}		% Azul, branco, destaque na tab, interessante
  %\usecolortheme{structure}		% bichado
  %\usecolortheme{whale}		% Azul e branco, bem bonito

  %%%%%%%%%%%%%%%%%%%%%%%%%%%% OUTER THEME
  \useoutertheme{default}
  %\useoutertheme{infolines}
  %\useoutertheme{miniframes}
  %\useoutertheme{shadow}
  %\useoutertheme{sidebar}
  %\useoutertheme{smoothbars}
  %\useoutertheme{smoothtree}
  %\useoutertheme{split}
  %\useoutertheme{tree}

  %%%%%%%%%%%%%%%%%%%%%%%%%%%% INNER THEME
  \useinnertheme{circles}
  %\useinnertheme{default}
  %\useinnertheme{inmargin}
  %\useinnertheme{rectangles}
  %\useinnertheme{rounded}

  %%%%%%%%%%%%%%%%%%%%%%%%%%%%%%%%%%%

  \setbeamercovered{invisible} % or whatever (possibly just delete it)
  % To change behavior of \uncover from graying out to totally
  % invisible, can change \setbeamercovered to invisible instead of
  % transparent. apparently there are also 'dynamic' modes that make
  % the amount of graying depend on how long it'll take until the
  % thing is uncovered.

}


% Get rid of nav bar
\beamertemplatenavigationsymbolsempty

% Use short top
%\usepackage[headheight=12pt,footheight=12pt]{beamerthemeboxes}
%\addheadboxtemplate{\color{black}}{
%\hskip0.5cm
%\color{white}
%\insertshortauthor \ \ \ \ 
%\insertframenumber \ \ \ \ \ \ \ 
%\insertsection \ \ \ \ \ \ \ \ \ \ \ \ \ \ \ \ \  \insertsubsection
%\hskip0.5cm}
%\addheadboxtemplate{\color{black}}{
%\color{white}
%\ \ \ \ 
%\insertsection
%}
%\addheadboxtemplate{\color{black}}{
%\color{white}
%\ \ \ \ 
%\insertsubsection
%}

% Insert frame number at bottom of the page.
% \usefoottemplate{\hfil\tiny{\color{black!90}\insertframenumber}} 

\usepackage[english]{babel}
\usepackage[latin1]{inputenc}
\usepackage{subfigure}

\usepackage{times}
\usepackage[T1]{fontenc}

\usepackage{tikz}
\usetikzlibrary{arrows,shapes}
% Latex Graph Example:
% https://www.overleaf.com/5297501zrjzfm#/16716638/


\tikzstyle{vertex}=[circle,fill=black!25,minimum size=10pt,inner sep=0pt]
\tikzstyle{blue vertex}=[circle,fill=blue!100,minimum size=10pt,inner sep=0pt]
\tikzstyle{red vertex}=[circle,fill=red!100,minimum size=10pt,inner sep=0pt]
\tikzstyle{edge} = [draw,thick,-]
\tikzstyle{red edge} = [draw, line width=5pt,-,red!50]
\tikzstyle{black edge} = [draw, line width=5pt,-,black!20]
\tikzstyle{weight} = [font=\smaller]

\title[GB21802]{GB21802 - Programming Challenges}
\subtitle[]{Week 6 - Graph Problems (Part II)}
\author[Claus Aranha]{Claus Aranha\\{\footnotesize caranha@cs.tsukuba.ac.jp}}
\institute{College of Information Science}
\date{2015-06-03,6\\{\tiny Last updated \today}}

\begin{document}

\section{Introduction}
\subsection{Title}
\begin{frame}
\maketitle
\end{frame}

\subsection{Notes and Warnings}

\begin{frame}
  \frametitle{Last Week Results}
  \begin{columns}[T]
    \column{0.5\textwidth}
    \begin{block}{Week 5 - Graph I}
      \begin{itemize}
      \item Jill Rides Again - 15/32
      \end{itemize}
    \end{block}
    \column{0.5\textwidth}
  \end{columns}
\end{frame}

\begin{frame}
  \frametitle{Special Notes}
\end{frame}

\begin{frame}
  \frametitle{Week 5 and 6 -- Outline}
  {\smaller
  \begin{block}{This Week - Graph I}
    \begin{itemize}
    \item Graph Basics review: Concepts and Data Structure;
    \item Depth First Search and Breadth First Search;
    \item Problems you solve with DFS and BFS;
    \item Minimum Spanning Tree: Kruskal and Prim Algorithms \alert{(Monday)};
     \end{itemize}
  \end{block}
  \begin{block}{Next Week - Graph II}
    \begin{itemize}
    \item Single Sourse Shortest Path (Djikstra);
    \item All Pairs Shortest Path (Floyd Warshall);   
    \item Network Flow and related Problems;
    \item Bipartite Graph Matching and related Problems;
    \end{itemize}
  \end{block}}
  Many variations in graph problems!
\end{frame}


% NEXT WEEK
%\section{Shortest Paths}
%\subsection{Introduction}
%\begin{frame}
%  \frametitle{Finding Paths in Graphs}
%\end{frame}

%\subsection{Single Source Shortest Paths - SSSP}
%\begin{frame}
%  \frametitle{Single Source Shortest Path}
%\end{frame}

%% Single Source Shortest Paths (Djikstra) -- Djikstra Original Paper
%% does not specify an implementaton.  Bellman Ford Algorithm for
%% shortest path with negative loop (slower than djikstra)

%% All pairs Shortest path (Floyd Warshall) ## Main attractiveness:
%% Very simple to program Explain the idea of Floyd Warshall (DP!)
%% Tricks with all pairs shortest path: Minimum Cycle/Negative Cycle:
%% Check the diagonal for the cost of i,i Diameter of a graph: Maximum
%% shortest path between any i,j of a graph

\section{Conclusion}
\subsection{Conclusion}
\begin{frame}
  \frametitle{Summary}
\end{frame}

\begin{frame}
  \frametitle{This Week's Problems}
  \begin{itemize}
  \item Dominator;
  \item Knight in a War grid;
  \item Wetlands in Florida;
  \item Battleships;
  \item Pick up Sticks;
  \item Place the Guards;
  \item Street Directions;
  \item Dominos;
  \item Freckles;
  \item Artic Network;
  \end{itemize}
\end{frame}

\begin{frame}
  \frametitle{Next Week}
  More Graphs!
  \begin{itemize}
  \item Network Flow (and related problems);
  \item Graph Matching (bipartite matching, etc) (and related problems);
  \end{itemize}
\end{frame}

%% TODO: Extra: Why study graph problems
% Human Networks (Scale-free Networks)
% State Machines (Computation models, theory)
% Dependence Graphs (Compilers, Pipelining, Scheduling)
% Grammar Graphs (Compilers, NLP)
\end{document}

