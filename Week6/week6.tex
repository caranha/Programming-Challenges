\documentclass{beamer}

\usepackage{amssymb,amsmath}
\usepackage{graphicx}
\usepackage{url}
\usepackage{color}
\usepackage{relsize}		% For \smaller
\usepackage{url}			% For \url
\usepackage{epstopdf}	% Included EPS files automatically converted to PDF to include with pdflatex
\usepackage{pagenote}[continuous,page]

%For MindMaps
% \usepackage{tikz}%
% \usetikzlibrary{mindmap,trees,arrows}%

%%% Color Definitions %%%%%%%%%%%%%%%%%%%%%%%%%%%%%%%%%%%%%%%%%%%%%%%%%%%%%%%%%
%\definecolor{bordercol}{RGB}{40,40,40}
%\definecolor{headercol1}{RGB}{186,215,230}
%\definecolor{headercol2}{RGB}{80,80,80}
%\definecolor{headerfontcol}{RGB}{0,0,0}
%\definecolor{boxcolor}{RGB}{186,215,230}

%%% Save space in lists. Use this after the opening of the list %%%%%%%%%%%%%%%%
%\newcommand{\compresslist}{
%	\setlength{\itemsep}{1pt}
%	\setlength{\parskip}{0pt}
%	\setlength{\parsep}{0pt}
%}

%\setbeameroption{show notes on top}

% You should run 'pdflatex' TWICE, because of TOC issues.

% Rename this file.  A common temptation for first-time slide makers
% is to name it something like ``my_talk.tex'' or
% ``john_doe_talk.tex'' or even ``discrete_math_seminar_talk.tex''.
% You really won't like any of these titles the second time you give a
% talk.  Try naming your tex file something more descriptive, like
% ``riemann_hypothesis_short_proof_talk.tex''.  Even better (in case
% you recycle 99% of a talk, but still want to change a little, and
% retain copies of each), how about
% ``riemann_hypothesis_short_proof_MIT-Colloquium.2000-01-01.tex''?

\mode<presentation>
{
  % A tip: pick a theme you like first, and THEN modify the color theme, and then add math content.
  % Warsaw is the theme selected by default in Beamer's installation sample files.

  %%%%%%%%%%%%%%%%%%%%%%%%%%%% THEME
  %\usetheme{Madrid}		% No subsection
  \usetheme{AnnArbor}  % Subsection on top, no color


  %\usetheme{Antibes}
  %\usetheme{Bergen}
  %\usetheme{Berkeley}		% bem bacana - menu esquerdo
  %\usetheme{Berlin}
  %\usetheme{Boadilla}
  %\usetheme{boxes}
  %\usetheme{CambridgeUS}		% bem bacana - menu superior
  %\usetheme{Copenhagen}
  %\usetheme{Darmstadt}
  %\usetheme{default}
  %\usetheme{Dresden}
  %\usetheme{Frankfurt}
  %\usetheme{Goettingen}
  %\usetheme{Hannover}		% bem bacana - menu esquerdo
  %\usetheme{Ilmenau}
  %\usetheme{JuanLesPins}
  %\usetheme{Luebeck}
  %\usetheme{Malmoe}
  %\usetheme{Marburg}		% bem bacana - menu direito
  %\usetheme{Montpellier}
  %\usetheme{PaloAlto}		% bem bacana - menu esquerdo
  %\usetheme{Pittsburgh}
  %\usetheme{Rochester}		%bacana
  %\usetheme{Singapore}
  %\usetheme{Szeged}
  %\usetheme{Warsaw}

  %%%%%%%%%%%%%%%%%%%%%%%%%%%% COLOR THEME
  %\usecolortheme{default}		% branco, azul clarinho
  \usecolortheme{crane}		% Very yellow (ok)

  %\usecolortheme{albatross}		% azul escuro, massa
  %\usecolortheme{beetle}		% cinza, menu azul
  %\usecolortheme{dolphin}		% azul e branco, legal
  %\usecolortheme{dove}			% cinza e branco, feio
  %\usecolortheme{fly}			% todo cinza, horrível
  %\usecolortheme{lily}			% parece o default
  %\usecolortheme{orchid}		% azul e branco, ok
  %\usecolortheme{rose}			% branco e violeta-claro, bonito
  %\usecolortheme{seagull}		% cinza, feio
  %\usecolortheme{seahorse}		% nhé, meio feio
  %\usecolortheme{sidebartab}		% Azul, branco, destaque na tab, interessante
  %\usecolortheme{structure}		% bichado
  %\usecolortheme{whale}		% Azul e branco, bem bonito

  %%%%%%%%%%%%%%%%%%%%%%%%%%%% OUTER THEME
  \useoutertheme{default}
  %\useoutertheme{infolines}
  %\useoutertheme{miniframes}
  %\useoutertheme{shadow}
  %\useoutertheme{sidebar}
  %\useoutertheme{smoothbars}
  %\useoutertheme{smoothtree}
  %\useoutertheme{split}
  %\useoutertheme{tree}

  %%%%%%%%%%%%%%%%%%%%%%%%%%%% INNER THEME
  \useinnertheme{circles}
  %\useinnertheme{default}
  %\useinnertheme{inmargin}
  %\useinnertheme{rectangles}
  %\useinnertheme{rounded}

  %%%%%%%%%%%%%%%%%%%%%%%%%%%%%%%%%%%

  \setbeamercovered{invisible} % or whatever (possibly just delete it)
  % To change behavior of \uncover from graying out to totally
  % invisible, can change \setbeamercovered to invisible instead of
  % transparent. apparently there are also 'dynamic' modes that make
  % the amount of graying depend on how long it'll take until the
  % thing is uncovered.

}


% Get rid of nav bar
\beamertemplatenavigationsymbolsempty

% Use short top
%\usepackage[headheight=12pt,footheight=12pt]{beamerthemeboxes}
%\addheadboxtemplate{\color{black}}{
%\hskip0.5cm
%\color{white}
%\insertshortauthor \ \ \ \
%\insertframenumber \ \ \ \ \ \ \
%\insertsection \ \ \ \ \ \ \ \ \ \ \ \ \ \ \ \ \  \insertsubsection
%\hskip0.5cm}
%\addheadboxtemplate{\color{black}}{
%\color{white}
%\ \ \ \
%\insertsection
%}
%\addheadboxtemplate{\color{black}}{
%\color{white}
%\ \ \ \
%\insertsubsection
%}

% Insert frame number at bottom of the page.
% \usefoottemplate{\hfil\tiny{\color{black!90}\insertframenumber}}

%% makes the ppagenote command for figure references at the end.

\usepackage[english]{babel}
%qq\usepackage[latin1]{inputenc}
\usepackage{CJKutf8}
\usepackage{subfigure}

\usepackage{times}
\usepackage[T1]{fontenc}

\makepagenote
\renewcommand{\notenumintext}[1]{}
\newcommand{\ppagenote}[1]{\pagenote[Page \insertframenumber]{#1}}

\title[Programming Challenges]{GB20602 - Programming Challenges}
\author[Claus Aranha]{Claus Aranha\\{\footnotesize caranha@cs.tsukuba.ac.jp}}
\institute[U. Tsukuba]{University of Tsukuba, Department of Computer Sciences}

\usepackage{tikz}
\usetikzlibrary{arrows,shapes}
% Latex Graph Example:
% https://www.overleaf.com/5297501zrjzfm#/16716638/

% TODO: Silly Makefile

\tikzstyle{vertex}=[circle,fill=black!25,minimum size=10pt,inner sep=0pt]
\tikzstyle{blue vertex}=[circle,fill=blue!100,minimum size=10pt,inner sep=0pt]
\tikzstyle{red vertex}=[circle,fill=red!100,minimum size=10pt,inner sep=0pt]
\tikzstyle{edge} = [draw,thick,-]
\tikzstyle{red edge} = [draw, line width=5pt,-,red!50]
\tikzstyle{black edge} = [draw, line width=5pt,-,black!20]
\tikzstyle{weight} = [font=\smaller]

\title[GB21802]{GB21802 - Programming Challenges}
\subtitle[]{Week 7 - Math Problems}
\author[Claus Aranha]{Claus Aranha\\{\footnotesize caranha@cs.tsukuba.ac.jp}}
\institute{College of Information Science}
\date{2019-05-31,06-03\\{\tiny Last updated \today}}

\begin{document}

\section{Introduction}
\subsection{Title}
\begin{frame}
\maketitle
\end{frame}

% \subsection{Last Week Notes}
%
% \begin{frame}
  \frametitle{Results for the Previous Week}

  \begin{center}
    Here are the results for last week:

    \bigskip
    
    \includegraphics[width=0.8\textwidth]{img/resultsW3}

    \bigskip
    
    Great Results!
    
  \end{center}
\end{frame}

\begin{frame}
  \frametitle{Pre-class Notes (1/2) -- ICPC Dates}

  The dates for the ICPC contest this year are as follows:

  \bigskip
  
  \begin{itemize}
  \item Registration Deadline -- 06/30 (Friday)
  \item National Contest -- 07/14 (Friday)
  \end{itemize}

  \bigskip

  If you want to participate, please talk to me after class or by
  e-mail. (A team need 3 members)
\end{frame}

\begin{frame}
  \frametitle{Pre-class Notes (2/2)}

  \begin{itemize}
  \item I have moved the class {\bf Dynamic Programming II} from
    Week 4 to Week 9;

    \bigskip
    
  \item The idea is that we will use class 9 to mix different
    techniques together: (Maths, Graphs, Geometry, DP)

    \bigskip
    
  \item It will be fun :-)
  \end{itemize}  
\end{frame}



\subsection{Outline}
\begin{frame}
  \frametitle{Outline: Math Problems}

  \begin{itemize}
    \item Math problems in programming competition normally require:
    \begin{itemize}
      \item Simple problem descriptions;
      \item A lot of time thinking;
      \item Not so much time programming;
    \end{itemize}
  \end{itemize}
\end{frame}

\begin{frame}
  \frametitle{Outline: Math Problems}

  Many math problems are {\bf ad hoc}. In this lecture we will study:
  \begin{itemize}
    \item Common Implementation Issues in Math problems: Bignum, precision, etc.
    \item Number Theory Algorithms: Factorization, Primality Testing, GCD;
    \item Combinatory Tricks: Common Sequences, Probability;
  \end{itemize}
\end{frame}


\section{Implementation Tricks}

\begin{frame}
  \frametitle{Implementation Tricks}
  \begin{itemize}
    \item BigNums;
    \item Modulo Operations;
    % \item Precision;
  \end{itemize}
\end{frame}

\subsection{Bignum}
\begin{frame}
  \frametitle{Dealing with Big Numbers}

  Some problems (specially math problems) require using very large numbers.
  For example: $25! = 15511210043330985984000000 > 10^{26}$.\bigskip

  {\bf However}:
  \begin{itemize}
    \item \structure{Maximum C++ unsigned int}: $2^{32} < 10^{11}$
    \item \structure{Maximum C++ unsigned long long}: $2^{64} < 10^{20}$
  \end{itemize}\bigskip


  \begin{block}{}
    I usually recommend to use C++; but Java is better for
    BigNum progchal problems!
  \end{block}
\end{frame}

\begin{frame}[fragile]
  \frametitle{Bignum Example: 10925 -- Krakovia}

  {\smaller
\begin{block}{}
\begin{verbatim}
import java.util.Scanner;
import java.math.BigInteger;
class Main {
  public static void main(String[] args) {
    Scanner sc = new Scanner(System.in);
    int caseNo = 1;
    while (true) {
      int N = sc.nextInt(), F = sc.nextInt();
      if (N == 0 && F == 0) break;
      BigInteger sum = BigInteger.ZERO;     // Bignum Constant
      for (int i = 0; i < N; i++) {
        BigInteger V = sc.nextBigInteger(); // Bignum I/O
        sum = sum.add(V); }
      System.out.println("Bill #" + (caseNo++)
        + " costs " + sum + ": each friend should pay "
        + sum.divide(BigInteger.valueOf(F)) + "\n" );}
  }
}
\end{verbatim}
  \end{block}}
\end{frame}


\begin{frame}[fragile]
  \frametitle{More functions from Java.math.BigInteger}
{\smaller

  \begin{block}{Algebraic functions}
    BigInteger.add(), .subtract(), .multiply(), .divide(),
    .pow(), .mod(), .remainder()
  \end{block}

  \begin{block}{Changing Number Base}
\begin{verbatim}
BI = BigInteger(10); System.println(BI.toString(2))
// Result: 1010
\end{verbatim}
  \end{block}

  \begin{block}{Probabilistic Primality Test}
\begin{verbatim}
isPrime = BI.isProbablePrime(int certainty)
// Chance of being correct is 1 - (1/2)^certainty
\end{verbatim}
  \end{block}


\begin{block}{Other cool functions}
  BigInteger.gcd(BI)
  BigInteger.modPow(BI exponent, BI m)
\end{block}}
\end{frame}

\subsection{Modulo Operations}
\begin{frame}
  \frametitle{Modulo Operation}

  {\smaller
  We can use \structure{modulo arithmetic} to operate on very large
  numbers without calculating the entire number.

  \bigskip

  Remember that:
  \begin{enumerate}
  \item $(a+b)\%s = ((a\%s)+(b\%s)+s)\%s$
  \item $(a*b)\%s = ((a\%s)*(b\%s))\%s$
  \item $(a^n)\%s = ((a^{n/2}\%s)*(a^{n/2}\%s)*(a^{n\%2}\%s))\%s$
  \end{enumerate}

  }
\end{frame}

\begin{frame}
  \frametitle{Modulo Operation -- UVA 10176, Ocean Deep!}
  {\smaller
  \begin{block}{Problem summary}
    Test if a binary number $n$ (up to 100000 digits) is divisible by 131071
  \end{block}

  \begin{itemize}
  \item The problem wants to know if $n\%13107 == 0$
  \item But $n$ is too big!

  \item Use the recurrence in the previous slide to break down each
    digit to a reasonable value.
  \end{itemize}}

\end{frame}

% \subsection{Precision}
% TODO: printing with precision in C, Java, Python
% Dealing with very small numbers

\section{Number Theory}
\subsection{outline}
\begin{frame}
  \frametitle{Number Theory} {\small

    Number Theory studies \structure{the integer numbers} and
    \structure{sets}.

    \bigskip

    \begin{itemize}
    \item Primality;
    \item Division and Remainders;
    \item Sequences of numbers;
    \end{itemize}

    }
\end{frame}

\subsection{Prime Numbers}
\begin{frame}
  \frametitle{Number Theory: Primality Testing}

  {\smaller
    \structure{Prime Numbers}: Only divisible by 1 and itself:

    \medskip

    2,3,5,7,11,13...

    \bigskip

    How do you test if a number $N$ is prime?

  \begin{itemize}

  \item Full search: For each $f \in 2..N-1$, test if $N\%f == 0$\\
    $O(N)$

    \bigskip

  \item A little Pruning: For each $f \in 2..\text{floor}(\sqrt{N})$,
    test if $N\%f == 0$\\
    $O(\sqrt(N))$

  \item Can you do it in $O(\sqrt{n} / \log(n))$?
  \end{itemize}}
\end{frame}

\begin{frame}
  \frametitle{Number Theory: Primality Testing}

  \begin{block}{The Prime Number Theorem (simplified)}
    The probability of $i < N$ is prime is $1 / \log(N)$
  \end{block}

  \hfill

  {\small

  {\bf collorary\footnote{``Collorary'' means ``consequence''} 1}: There are $N/\log(N)$ primes $<N$\\
  {\bf collorary 2}: We just need to test the {\bf primes} between 1 and $\sqrt{N}$

  \hfill

  But how do we find all primes between 1 and $\sqrt{N}$ fast?
  }
\end{frame}



\begin{frame}[fragile]
  \frametitle{Sieve of Eratosthenes}

  {\smaller
    \begin{block}{Idea}
      \begin{itemize}
      \item Start with a set from 2 to $\sqrt{N}$.
      \item Test if each $i$ in the set is prime.
      \item If $i$ is prime, remove all multiples $mi$.
      \end{itemize}
    \end{block}

  \begin{exampleblock}{}
\begin{verbatim}
def sieve(k):                 ## Find all primes up to k
   primes = []
   sieve = [1]*(k+1)    ## all numbers start in the list
   sieve[0] = sieve[1] = 0             ## except 0 and 1
   for i in range(k+1):                          ## O(N)
      if (sieve[i] == 1):
         primes.append(i)             ## new prime found
         j = i*i   ## why can i start from i*i, not i*2?
         while (j < k+1):                  ## O(loglogN)
            sieve[j] = 0
            j += i                      ## next multiple
   return primes
\end{verbatim}
  \end{exampleblock}
  }
\end{frame}

\begin{frame}
  \frametitle{Sieve of Eratosthenes}

  {\smaller
    \begin{block}{Amortized Complexity}
      \begin{itemize}
      \item The complexity of the Sieve is $O(N\log\log N)$

        \medskip

      \item If we do the Sieve every time we test for primes, we are not saving much.

        \medskip

      \item But we can do the Sieve one time, and test many primes later!

      \end{itemize}
    \end{block}

    \hfill

    When we do an expensive operation once, we call it {\bf amortized complexity}
  }
\end{frame}

% Prime number: Sosuu
% Prime factors: Soinsuu
% Factorization: Insuu Bunkai
\begin{frame}
  \frametitle{Finding Prime Factors}

  {\smaller

    Any natural number $N$ can be expressed as a \structure{unique}
    set of prime numbers:

    \begin{equation*}
      N=1p_1^{e_1}p_2^{e_2}\ldots p_n^{e_n}
    \end{equation*}

    These are the \structure{Prime Factors} of $N$. From this set, we
    can also obtain the set of \structure{Factors} of $N$ (all numbers
    $i$ where $i|N$).

    \medskip

    Factorization is a key issue in \structure{cryptography}

    \begin{block}{Very Naive approach -- Test all numbers!}
      For every $i \in 1..N/2$, test $i|N$ and isPrime(i).\\
      \smallskip
      \hfill Very Expensive!
    \end{block}

    \begin{block}{Naive approach -- Test all primes}
      Calculate a list of primes $i$ up to N/2, test if $i|N$.\\
      \smallskip
      \hfill Wrong Answer, why?
    \end{block}
}
\end{frame}

\begin{frame}[fragile]
  \frametitle{Prime factorization: Divide and conquer approach}

  {\smaller
  \begin{block}{Recursive Idea}
    The prime factorization of $N$ is equal to the union of $p_i$ and
    the prime factorization of $N/p_i$, where $p_i$ is the smallest
    prime factor of $N$.

    \bigskip

    The set of all factors is composed of all combinations of the set
    of prime factors (including repetitions).
  \end{block}

  \begin{exampleblock}{}
\begin{verbatim}
def primefactors(n):
   primes = sieve(int(np.sqrt(n))+1)
   c = 0, i = n, factors = []
   while i > 1:
      if (i%primes[c] == 0):
          i = i/primes[c]
          factors.append(primes[c])
      else:
          c = c+1
   return factors
\end{verbatim}
  \end{exampleblock}}
\end{frame}


\begin{frame}
  \frametitle{Working with Prime Factors: 10139 -- Factovisors}

  {\smaller
    \begin{block}{Problem description}
      Calculate whether $m$ divides $n!$ ($1 \leq m,n \leq 2^{31}-1$)
    \end{block}

    Factorial of 22 is already bigint! But we can break down these numbers into their
    factors, which are all $\leq 2^{30}$.

    \begin{itemize}
    \item $F_m$: primefactors(m)
    \item $F_{n!}$: $\cup$(primefactors(1), primefactors(2)...,primefactors(n))
    \end{itemize}

    Having the factor sets, $m$ divides $n!$ if $F_m \subset F_{n!}$.

    \bigskip

    Examples:
    \begin{itemize}
    \item $m = 48$ and $n=6$\\
      $F_m = \{2,2,2,2,3\} F_{n!} = \{2,3,2,2,5,2,3\}$

  \medskip

    \item $m = 25$ and $n = 6$\\
      $F_m = \{5,5\} F_{n!} = \{2,3,2,2,5,2,3\}$

    \end{itemize}
  }
\end{frame}


% Saidaikōyakusū
\subsection{GCD/LCM}
\begin{frame}[fragile]
  \frametitle{Euclid Algorithm and Extended Euclid Algorithm}

  {\smaller
    \begin{itemize}
    \item \structure{Euclid Algorithm} gives us the greatest common divisor $D$ of $a,b$;
    \item \structure{Extended Euclid Algorithm} also gives us $x,y$ so that $ax+by = D$;
    \item Both are extremely simple to code:
    \end{itemize}

    \vfill

    \begin{exampleblock}{}
\begin{verbatim}
int gcd(int a, int b) {return (a == 0?b:gcd(b%a,a));}

int x, y;
int egcd(int a, int b) {
   if (a==0)
      {x = 0; y = 1; return b;}        // stop condition
   int d = egcd(b%a, a);
   int tx = x;                         // gcd recurrence
   x = y - (b/a)*tx; y = tx; return d; }   // update x,y
\end{verbatim}
    \end{exampleblock}
}
\end{frame}

\begin{frame}
  \frametitle{Using EGCD: The Diophantine Equation}
  {\smaller
    \begin{block}{Problem Example (variations of this problem are common)}
      You have 839 yen. \alert{X}hoco candy costs 25 yen,
      \alert{Y}anilla candy costs 18 yen. How many candies can we buy?
    \end{block}

    \bigskip

    The equation $xA+yB=C$ is called the \structure{Linear Diophantine
      Equation}. It has infinite solutions if GCD(A,B)|C, but none if
    it does not.

    \bigskip

    The first solution ($x_0,y_0$) can be derived from the extended
    GCD, and other solutons can be found from:
    expressed as:
    \begin{itemize}
    \item $x = x_0 + (b/d)n$
    \item $y = y_0 - (a/d)n$
    \end{itemize}
    Where $d$ is GCD(A,B) and $n$ is an integer.
  }
\end{frame}

\begin{frame}
  \frametitle{Using EGCD: The Diophantine Equation}
  {\smaller
    \begin{block}{Problem Example (variations of this problem are common)}
      You have 839 yen. \alert{X}hoco candy costs 25 yen,
      \alert{Y}anilla candy costs 18 yen. How many candies can we buy?
    \end{block}

    \begin{itemize}
    \item \structure{EGCD} gives us: $x=-5, y=7, d=1$ or $25(-5)+18(7) = 1$
    \item Multiply both sides by 839: $25(-4195)+18(5873) = 839$
    \item So: $x_n = -4195 + 18n$ and $y_n = 5873 - 25n$
    \item We have to find $n$ so that both $x_n,y_n$ are $> 0$.
    \item $-4195 + 18n \geq 0$ and $5873 - 25n \geq 0$
    \item $n \geq 4195/18$ and $5873/25 \geq n$
    \item $4195/18 \leq n \leq 5873/25$
    \item $233.05 \leq n \leq 234.92$
    \end{itemize}
  }
\end{frame}


\section{Combinatorics}
\subsection{Counting and Closed Forms}
\begin{frame}
  \frametitle{Combinatorics problems}

  {\smaller
    \begin{block}{Definition}
      Combinatorics is the branch of mathematics concerning the study of
      \structure{countable discrete structures}.
    \end{block}

    Combinatory problems involve understanding a sequence, and
    figuring one of:

    \medskip

    \begin{itemize}
    \item \structure{Recurrence}: A formula that calculates the
      $n^{th}$ member of a sequence, based on the value of previous members;

    \item \structure{Closed form}: A formula that calculates the
      $n^{th}$ member of a sequence independently from other members;
    \end{itemize}

  \bigskip

  It is not uncommon to use \structure{Dynamic Programming} or
  \structure{Bignum} to solve combinatoric related problems.
  }
\end{frame}

\begin{frame}
  \frametitle{Example: Triangular Numbers}
  {\smaller
  \begin{block}{Definition}
    The triangular numbers is the sequence where the $n^{th}$ value is
    composed of the sum of all integers from $1$ to $n$
  \end{block}

  \begin{itemize}
  \item S(1) = 1
  \item S(2) = 1+2 = 3
  \item S(3) = 1+2+3 = 6
  \item $\ldots$
  \item S(7) = 1+2+3+4+5+6+7 = 28
  \end{itemize}
  }

  What are the recurrence and the closed form for this sequence?
\end{frame}

\begin{frame}
  \frametitle{Example: Triangular Numbers}
  {\smaller
    \begin{itemize}
    \item S(1) = 1, S(2) = 3, S(3) = 6
    \end{itemize}

    \begin{block}{Recurrence}
      The recursive form of a sequence:
      \begin{equation*}
        S(n) = S(n-1)+n; S(1) = 1
      \end{equation*}
    \end{block}
    \begin{block}{Closed Form}
      The non-recursive form of a sequence:
      \begin{equation*}
        S(n) = \frac{n(n+1)}{2}
      \end{equation*}
    \end{block}
    \alert{Problem:} Calculate the first triangle number with more
    than 500 factors!  }
\end{frame}

\subsection{Fibonacci}

\begin{frame}
  \frametitle{A more famous sequence: Fibonacci Numbers}

  {\smaller
  \begin{block}{Definition -- very famous sequence}
    Each number is the sum of the two numbers before it.

    \medskip

    F() = 0,1,1,2,3,5,8,13,21,34...
  \end{block}

  \begin{block}{The recurrence is well known}
  \begin{equation*}
  F(0) = 0, F(1) = 1, F(n) = F(n-1)+F(n-2)
  \end{equation*}
  When implementing the recurrence, don't forget the memoization
  table!
  \end{block}

  \begin{block}{Closed Form}
    The Fibonacci numbers also have a less well known
    \structure{closed form}:

    \begin{equation*}
      F(n) = \frac{1}{\sqrt{5}}\left(\left(\frac{1+\sqrt{5}}{2}\right)^n-\left(\frac{1-\sqrt{5}}{2}\right)^n\right)
    \end{equation*}
  \end{block}
    Square roots introduce floating point errors. What is the maximum
    $n$ this can calculate with less than 0.1 error?
  }
\end{frame}

\begin{frame}[fragile]
  \frametitle{Fibonacci Facts}
  {\smaller
  \begin{block}{Zeckendorf's theorem}
    Every positive integer can be written in a \structure{unique way} as a sum of
    one or more distinct fibonacci numbers, which are not consecutive.

\begin{verbatim}
def zeckenfy(n):
    fibs = []
    f = greatest fib =< n; fibs.append(f)
    fibs.append(zeckenfy(n-f))
    return fibs
\end{verbatim}
  \end{block}

  \begin{block}{Pisano's period}
    The last digits of the Fibonacci sequence repeat!

    \medskip

    The last one/two/\alert{three/four} digits repeat with a period of
    60/300/\alert{1500/15000}.

    F(6) = 8\\
    F(66) = 27777890035288\\
    F(366) = 1380356705549181797202918793682511 3333650564850089197542855968899086435571688

  \end{block}
  }
\end{frame}

\subsection{Binom}

\begin{frame}[fragile]
  \frametitle{Binomial Coefficients}
  {\smaller

    \begin{block}{Definition}
      Binomial Coefficients are the number series that correspond to
      the coefficients of the expansion of a binomial:

      \medskip

      Binom(3) = $(a+b)^3$ = $1a^3 + 3ab^2 + 3ab^2 + b^3$ = $\{1,3,3,1\}$

      \medskip

      We are usually interested in the $k^{th}$ coefficient of the
      $n^{th}$ binomial:

      \medskip

      $C(n,k) = C(3,2) = \{1,$ \alert{3} $,3,1\} = 3$

    \end{block}

    Pascal's Triangle gives us a good representation of C(n,n):
\begin{verbatim}
0  1  0  0  0  0  0  0  0  0
0  1  1  0  0  0  0  0  0  0
0  1  2  1  0  0  0  0  0  0
0  1  3  3  1  0  0  0  0  0
0  1  4  6  4  1  0  0  0  0
0  1  5  10 10 5  1  0  0  0
0  1  6  15 20 15 6  1  0  0
0  1  7  21 35 35 21 7  1  0
0  1  8  28 56 70 56 28 8  1
\end{verbatim}
  }
\end{frame}

\begin{frame}
  \frametitle{Uses for the Binomial Coefficient}

  The value of $C(n,k)$ tells us how many ways we can choose $n$
  items, $k$ at a time.

  \bigskip

  Some use cases:
  \begin{itemize}
  \item \structure{Probabilities:} What is the probability of winning
    a loto when you choose 5 numbers out of 60? $1/C(60,5)$
  \item \structure{Grids:} How many ways are there to go from the
    bottom left end of a $mn$ grid to the top right, if you can only
    go up and right? $C(m+n,n)$
  \end{itemize}
\end{frame}


\begin{frame}
  \frametitle{Calculating the Binomial Coefficient}
  {\smaller

    \begin{block}{Closed form of C(n,k)}
      \begin{equation*}
        C(n,k) = \frac{n!}{(n-k)!k!}
      \end{equation*}

      \alert{Problem}: Multiplying factorials tends to generate huge numbers
      even for small $n$ and $k$.
    \end{block}

    \begin{block}{Recurrence for C(n,k)}
      \begin{itemize}
      \item C(n,0) = C(n,n) = 1;
      \item C(n,k) = C(n-1,k-1) + C(n-1,k)
      \end{itemize}

      Using a memoization table will cut the calculation time by
      half. In this case, top-down DP will usually be faster than
      bottom-up.
    \end{block}
  }
\end{frame}

\subsection{Catalan}

\begin{frame}
  \frametitle{Another useful sequence: Catalan Numbers}
  {\smaller
  \begin{block}{The Catalan sequence}
    \begin{equation*}
      C(n) = 1, 1, 2, 5, 14, 42, 132, 429, 1430
    \end{equation*}
  \end{block}

  \begin{exampleblock}{The Recurrence}
    \begin{equation*}
    C(n) = \sum^{n-1}_{k=0}C(k)C(n-1-k)
    \end{equation*}
  \end{exampleblock}

  \begin{exampleblock}{Closed Form}
    \begin{equation*}
      C(n) = \frac{1}{n+1}\binom{2n}{n}
    \end{equation*}
  \end{exampleblock}
  }
\end{frame}

\begin{frame}
  \frametitle{Catalan Numbers -- Uses}
  \begin{itemize}
    \item Number of ways that you can match $n$ parenthesis.\\
      C(3):((())),()(()),(())(),()()(),(()())

      \medskip

    \item Number of ways that you can triangulate a poligon with $n+2$ sides
    \item Number of monotonic paths on an $nxn$ grid that do not pass above
      the diagonal.
    \item Number of distinct binary trees with $n$ vertices
    \item Etc...
  \end{itemize}
\end{frame}

\subsection{Integer Partition}
\begin{frame}
  \frametitle{Integer Partition}
  \begin{block}{}
    f(5,5) = (5),(4,1),(3,2),(3,1,1),(2,2,1),(2,1,1,1),(1,1,1,1,1)
  \end{block}
  \begin{block}{Definition and calculation}
    $f(n,k)$ -- number of ways that we can sum $n$, using integers
    equal or less than $k$.

    \bigskip

    \structure{Recurrence:}
    \begin{itemize}
    \item $f(n,k) = f(n-k,k) + f(n, k+1)$
    \item $f(1,1) = 1$; $f(n,k) = 0$ if $k > n$
    \end{itemize}
  \end{block}
\end{frame}

\begin{frame}
  \frametitle{Ad Hoc Example: Probability problems}

  {\smaller
    \begin{block}{Dice Throwing}
      If you have $n$ dice, what is the chance of rolling a total above $m$?
    \end{block}

    \begin{itemize}
    \item \structure{Example:} For $n=3$, $m=16$, what is the probability?
    \end{itemize}
  }
\end{frame}

\begin{frame}
  \frametitle{Ad Hoc Example: Probability problems}

  {\smaller
    \begin{block}{Dice Throwing}
      If you have $n$ dice, what is the chance of rolling a total above $m$?
    \end{block}

    \begin{itemize}
    \item \structure{Example:} For $n=3$, $m=16$, the chance is $10/216$

      \bigskip

    \item All combinations of 3 dice: $6*6*6 = 216$
    \item Combinations above 16:
    \end{itemize}

    \begin{columns}[T]
      \column{0.3\textwidth}
      \begin{itemize}
      \item 6,6,6
      \item 6,6,5
      \item 6,5,6
      \item 5,6,6
      \end{itemize}
      \column{0.3\textwidth}
      \begin{itemize}
      \item 6,5,5
      \item 5,6,5
      \item 5,5,6
      \end{itemize}
      \column{0.3\textwidth}
      \begin{itemize}
      \item 4,6,6
      \item 6,4,6
      \item 6,6,4
      \end{itemize}
    \end{columns}

    \medskip

    \begin{itemize}
    \item What algorithm do you use?
    \end{itemize}
  }
\end{frame}

\begin{frame}
  \frametitle{Ad Hoc example: Probabilty Problems}

  {\smaller
  \begin{block}{The dice problem}
    If I have $n$ dice, what is the chance of rolling a total above $m$?
  \end{block}

  \medskip

  Solving with DP

  \medskip

  \begin{itemize}
  \item For $n=0$, we have only one result: $r=0$
  \item For $n=1$, we have 6 results: $r = \{1,2,3,4,5,6\}$
  \item The result for $n=i$ and $r_{n-1}=k$ is $r_n = k + \{1,2,3,4,5,6\}$

    \bigskip

  \item With a state table (dice,result), we can count the number of
    dice combination above a certain value;

  \end{itemize}
  }
\end{frame}

\begin{frame}[fragile]
  \frametitle{Ad Hoc example: Probability Problems}
  \begin{exampleblock}{Example Code}
{\small
\begin{verbatim}
int count(int dice_left, int score_left) {
   if (score_left < 1) return 1;
   if (dice_left == 0) return 0;
   if (result[dice_left][score_left] != -1)
      return result[dice_left][score_left];
   int sum = 0;
   for (int i = 0; i < 6; i++)
      sum += count(dice_left-1, score_left-(i+1))
   result[dice_left][score_left] = sum;
   return sum;
}

prob = count(n,m)/6**n;

\end{verbatim}
}
  \end{exampleblock}
\end{frame}

% TODO: Hare and tortoise algorith,

% TODO expand this, add more sequences
% TODO add a section on HOW to derive closed forms (from maths for CS class)


\subsection{Conclusion}
\begin{frame}
  \frametitle{Class Summary}
  \begin{itemize}
  \item Math Problems
  \item Java's Big Integer class
  \item Primality
  \item Modulo arithmetic
  \item GCD and Diophantine Equations
  \item Combinatorics
  \end{itemize}

  \begin{block}{}
    Next week: Geometry problems!
  \end{block}
\end{frame}

% \begin{frame}
  \frametitle{This Week's Problems}
  \begin{itemize}
  \item Dominator;
  \item Knight in a War grid;
  \item Wetlands in Florida;
  \item Battleships;
  \item Pick up Sticks;
  \item Place the Guards;
  \item Street Directions;
  \item Dominos;
  \item Freckles;
  \item Artic Network;
  \end{itemize}
\end{frame}

\begin{frame}
  \frametitle{Problem Hints (0)}
  \begin{itemize}
  \item All the problems this week (and next week!) include graphs,
    and probably need BFS and/or DFS;

    \medskip

  \item Prepare a ``template'' of an Adjacency list and DFS/BFS, and
    put it in the code before starting;

    \medskip

  \item Try to draw the problem on paper before coding;

    \medskip

  \item Remember to test ``tricky'' cases: Graphs with 1 node,
    disconnected graphs, self-edges, multi-edges;
  \end{itemize}
\end{frame}

\begin{frame}
  \frametitle{Problem Hints (1)}
  {\smaller
  \begin{block}{Dominator}
    \begin{itemize}
    \item Remember: A node is not dominated by anyone if it is not connected to the root (node 0);
    \item Basic algorithm discussed in class: Calculate all nodes
      reachable from root. Then remove one node at a time, and node which ones are not reachable anymore;
    \item If removing node $i$ makes node $j$ not reachable, then $i$ dominates $j$.
    \item To ``remove'' a node, modify the DFS(root,i) so that it returns if $i$ is reached;
    \end{itemize}
  \end{block}
  }
\end{frame}

\begin{frame}
  \frametitle{Problem Hints (2)}
  {\smaller
  \begin{block}{Knight in a War Grid}
    \begin{itemize}
    \item The problem only wants to know which squares are reachable,
      it is not worried about minimum distance;
    \item Be careful, $M$ or $N$ can be zero!
    \item Be careful, if $M == N$, the graph becomes multigraph!
    \item This graph is implicit, the connections are given by the
      knight step, the board size, and the impossible squares;
    \end{itemize}
  \end{block}
  }
\end{frame}

\begin{frame}
  \frametitle{Problem Hints (3)}
  {\smaller
  \begin{block}{Wetlands of Florida}
    \begin{itemize}
      \item Make a graph with 0 and 1 indicating water or no water;
      \item Flood-fill the graph at the requested location;
      \item Multiple-case input is a bit hard to read, make sure to test that;
    \end{itemize}
  \end{block}
  \begin{block}{Battleships}
    \begin{itemize}
      \item Scan the graph (double fors). 
      \item For each unvisited 'x' or '@', flood fill the ship (mark
        visited) and add the ship;
      \item A ship with only @'s should not be counted.
    \end{itemize}
  \end{block}}
\end{frame}

\begin{frame}
  \frametitle{Problem Hints (4)}
  {\smaller
  \begin{block}{Pick up sticks}
    \begin{itemize}
    \item The input gives you directed nodes. 
    \item Try to build a topological order (follow the class code)
    \item Any order is fine. If you find a cycle, print ``impossible''
    \end{itemize}
  \end{block}
  \begin{block}{Palace Guards}
    \begin{itemize}
    \item Each junction is a node, each street is an edge. 
    \item We have junctions with guards and without guards. (No guard can be near each other)
    \item There is a solution if the graph is bipartite!
    \item How do you calculate the smallest number of guards?
    \end{itemize}
  \end{block}}
\end{frame}

\begin{frame}
  \frametitle{Problem Hints (5)}
  {\smaller
  \begin{block}{Street Directions}
    \begin{itemize}
    \item We have to convert two way streets to one way streets 
    \item Undirected graph to directd graph. 
    \item When is a 2-way street \alert{necessary}?
    \item How can you generate 1 way streets?
    \item Hint: you need to draw the graph on paper
    \end{itemize}
  \end{block}
  \begin{block}{Dominos}
    \begin{itemize}
    \item The dominos falling is a directed graph. 
    \item Each domino that falls, we visit one node.
    \item How many nodes do we need to start, to visit all nodes?
    \end{itemize}
  \end{block}}
\end{frame}

\begin{frame}
  \frametitle{Problem Hints (6)}
  \begin{block}{Freckles}
    \begin{itemize}
    \item The problem requires the minimum ink (cost) among all freckles;
    \item This is straight up MST code;
    \item Be careful when rounding up values;
    \end{itemize}    
  \end{block}
  \begin{block}{Arctic Network}
    \begin{itemize}
    \item Also wants to calculate the MST (minimum radio power necessary);
    \item However, we can use $S$ ``satellite'' links, which cost 0;
    \item Remember that two stations need a satellite link to talk;
    \end{itemize}
  \end{block}
\end{frame}

%
% \begin{frame}
%   \frametitle{To Learn More}
%
%   Euler Project: Mathematical questions using computers:
%
%   \url{http://projecteuler.net}
% \end{frame}

\end{document}
