\documentclass{beamer}

\usepackage{amssymb,amsmath}
\usepackage{graphicx}
\usepackage{url}
\usepackage{color}
\usepackage{relsize}		% For \smaller
\usepackage{url}			% For \url
\usepackage{epstopdf}	% Included EPS files automatically converted to PDF to include with pdflatex
\usepackage{pagenote}[continuous,page]

%For MindMaps
% \usepackage{tikz}%
% \usetikzlibrary{mindmap,trees,arrows}%

%%% Color Definitions %%%%%%%%%%%%%%%%%%%%%%%%%%%%%%%%%%%%%%%%%%%%%%%%%%%%%%%%%
%\definecolor{bordercol}{RGB}{40,40,40}
%\definecolor{headercol1}{RGB}{186,215,230}
%\definecolor{headercol2}{RGB}{80,80,80}
%\definecolor{headerfontcol}{RGB}{0,0,0}
%\definecolor{boxcolor}{RGB}{186,215,230}

%%% Save space in lists. Use this after the opening of the list %%%%%%%%%%%%%%%%
%\newcommand{\compresslist}{
%	\setlength{\itemsep}{1pt}
%	\setlength{\parskip}{0pt}
%	\setlength{\parsep}{0pt}
%}

%\setbeameroption{show notes on top}

% You should run 'pdflatex' TWICE, because of TOC issues.

% Rename this file.  A common temptation for first-time slide makers
% is to name it something like ``my_talk.tex'' or
% ``john_doe_talk.tex'' or even ``discrete_math_seminar_talk.tex''.
% You really won't like any of these titles the second time you give a
% talk.  Try naming your tex file something more descriptive, like
% ``riemann_hypothesis_short_proof_talk.tex''.  Even better (in case
% you recycle 99% of a talk, but still want to change a little, and
% retain copies of each), how about
% ``riemann_hypothesis_short_proof_MIT-Colloquium.2000-01-01.tex''?

\mode<presentation>
{
  % A tip: pick a theme you like first, and THEN modify the color theme, and then add math content.
  % Warsaw is the theme selected by default in Beamer's installation sample files.

  %%%%%%%%%%%%%%%%%%%%%%%%%%%% THEME
  %\usetheme{Madrid}		% No subsection
  \usetheme{AnnArbor}  % Subsection on top, no color


  %\usetheme{Antibes}
  %\usetheme{Bergen}
  %\usetheme{Berkeley}		% bem bacana - menu esquerdo
  %\usetheme{Berlin}
  %\usetheme{Boadilla}
  %\usetheme{boxes}
  %\usetheme{CambridgeUS}		% bem bacana - menu superior
  %\usetheme{Copenhagen}
  %\usetheme{Darmstadt}
  %\usetheme{default}
  %\usetheme{Dresden}
  %\usetheme{Frankfurt}
  %\usetheme{Goettingen}
  %\usetheme{Hannover}		% bem bacana - menu esquerdo
  %\usetheme{Ilmenau}
  %\usetheme{JuanLesPins}
  %\usetheme{Luebeck}
  %\usetheme{Malmoe}
  %\usetheme{Marburg}		% bem bacana - menu direito
  %\usetheme{Montpellier}
  %\usetheme{PaloAlto}		% bem bacana - menu esquerdo
  %\usetheme{Pittsburgh}
  %\usetheme{Rochester}		%bacana
  %\usetheme{Singapore}
  %\usetheme{Szeged}
  %\usetheme{Warsaw}

  %%%%%%%%%%%%%%%%%%%%%%%%%%%% COLOR THEME
  %\usecolortheme{default}		% branco, azul clarinho
  \usecolortheme{crane}		% Very yellow (ok)

  %\usecolortheme{albatross}		% azul escuro, massa
  %\usecolortheme{beetle}		% cinza, menu azul
  %\usecolortheme{dolphin}		% azul e branco, legal
  %\usecolortheme{dove}			% cinza e branco, feio
  %\usecolortheme{fly}			% todo cinza, horrível
  %\usecolortheme{lily}			% parece o default
  %\usecolortheme{orchid}		% azul e branco, ok
  %\usecolortheme{rose}			% branco e violeta-claro, bonito
  %\usecolortheme{seagull}		% cinza, feio
  %\usecolortheme{seahorse}		% nhé, meio feio
  %\usecolortheme{sidebartab}		% Azul, branco, destaque na tab, interessante
  %\usecolortheme{structure}		% bichado
  %\usecolortheme{whale}		% Azul e branco, bem bonito

  %%%%%%%%%%%%%%%%%%%%%%%%%%%% OUTER THEME
  \useoutertheme{default}
  %\useoutertheme{infolines}
  %\useoutertheme{miniframes}
  %\useoutertheme{shadow}
  %\useoutertheme{sidebar}
  %\useoutertheme{smoothbars}
  %\useoutertheme{smoothtree}
  %\useoutertheme{split}
  %\useoutertheme{tree}

  %%%%%%%%%%%%%%%%%%%%%%%%%%%% INNER THEME
  \useinnertheme{circles}
  %\useinnertheme{default}
  %\useinnertheme{inmargin}
  %\useinnertheme{rectangles}
  %\useinnertheme{rounded}

  %%%%%%%%%%%%%%%%%%%%%%%%%%%%%%%%%%%

  \setbeamercovered{invisible} % or whatever (possibly just delete it)
  % To change behavior of \uncover from graying out to totally
  % invisible, can change \setbeamercovered to invisible instead of
  % transparent. apparently there are also 'dynamic' modes that make
  % the amount of graying depend on how long it'll take until the
  % thing is uncovered.

}


% Get rid of nav bar
\beamertemplatenavigationsymbolsempty

% Use short top
%\usepackage[headheight=12pt,footheight=12pt]{beamerthemeboxes}
%\addheadboxtemplate{\color{black}}{
%\hskip0.5cm
%\color{white}
%\insertshortauthor \ \ \ \
%\insertframenumber \ \ \ \ \ \ \
%\insertsection \ \ \ \ \ \ \ \ \ \ \ \ \ \ \ \ \  \insertsubsection
%\hskip0.5cm}
%\addheadboxtemplate{\color{black}}{
%\color{white}
%\ \ \ \
%\insertsection
%}
%\addheadboxtemplate{\color{black}}{
%\color{white}
%\ \ \ \
%\insertsubsection
%}

% Insert frame number at bottom of the page.
% \usefoottemplate{\hfil\tiny{\color{black!90}\insertframenumber}}

%% makes the ppagenote command for figure references at the end.

\usepackage[english]{babel}
%qq\usepackage[latin1]{inputenc}
\usepackage{CJKutf8}
\usepackage{subfigure}

\usepackage{times}
\usepackage[T1]{fontenc}

\makepagenote
\renewcommand{\notenumintext}[1]{}
\newcommand{\ppagenote}[1]{\pagenote[Page \insertframenumber]{#1}}

\title[Programming Challenges]{GB20602 - Programming Challenges}
\author[Claus Aranha]{Claus Aranha\\{\footnotesize caranha@cs.tsukuba.ac.jp}}
\institute[U. Tsukuba]{University of Tsukuba, Department of Computer Sciences}

\usepackage{tikz}
\usetikzlibrary{arrows,shapes}
\tikzstyle{vertex}=[circle,fill=black!25,minimum size=10pt,inner sep=0pt]
\tikzstyle{blue vertex}=[circle,fill=blue!100,minimum size=10pt,inner sep=0pt]
\tikzstyle{red vertex}=[circle,fill=red!100,minimum size=10pt,inner sep=0pt]
\tikzstyle{edge} = [draw,thick,-]
\tikzstyle{red edge} = [draw, line width=5pt,-,red!50]
\tikzstyle{black edge} = [draw, line width=5pt,-,black!20]
\tikzstyle{weight} = [font=\smaller]

\title[GB21802]{GB21802 - Programming Challenges}
\subtitle[]{Week 8 - Computational Geometry}
\author[Claus Aranha]{Claus Aranha\\{\footnotesize caranha@cs.tsukuba.ac.jp}}
\institute{College of Information Science}
\date{2015-06-17,20\\{\tiny Last updated \today}}
% TODO: Fix many small typos in code

\begin{document}

\section{Introduction}
\subsection{Title}
\begin{frame}
\maketitle
\end{frame}

\subsection{Notes and Warnings}

\begin{frame}
  \frametitle{Last Week Results}
  \begin{block}{Week 6 - Graph II}
    {\smaller
      \begin{columns}[T]
        \column{0.5\textwidth}
        \begin{itemize}
        \item Division -- 16/31
        \item What Base is This? -- 6/31
        \item Divisibility of Factors -- 11/31
        \item Triangle Counting -- 11/31
        \item Help my Brother (II) -- 4/31
        \item Marbles -- 1/31
        \item Ocean Deep! Make it Shallow! -- 9/31
        \item Winning Streak -- 0/31
        \end{itemize}
        \column{0.5\textwidth}
        \begin{itemize}
        \item  14 people: 0 problems; 
        \item  6 people: 1-2 problems;
        \item  6 people: 3-4 problems;
        \item  4 people: 5-6 problems;
        \item  1 people: 7-8 problems!
        \end{itemize}
      \end{columns}
    }
  \end{block}
\end{frame}

\begin{frame}
  \frametitle{Special Notes}

\end{frame}

\subsection{Characteristics}
\begin{frame}
  \frametitle{Topic of the Week -- Computational Geometry}
  {\small
  
    \begin{itemize}
    \item \structure{Computational Geometry} problems are generally
      considered to be difficult, both in terms of understanding the
      solution, and programming the solution;

      \vfill

    \item One trick for these problems is to \structure{prepare} a
      large library of basic geometric operations (distances,
      intersections, angle operations, etc); \\

      \begin{itemize}
        \item {\smaller Focus of this class is the implementation of these
          operations.}
      \end{itemize}

      \vfill

    \item Special attention is needed to deal with \alert{degeneracies};
    \end{itemize}
  }
\end{frame}

\begin{frame}
  \frametitle{Degeneracies: Special cases}

  {\smaller
    \begin{block}{}
      Two types of degeneracies: \structure{Special cases} and
      \structure{Precision errors}
    \end{block}

    \bigskip

    (some) Special cases:
    \begin{itemize}
    \item Lines parallel to the vertical axis
    \item Colinear Lines
    \item Overlapping Segments
    \item Concave polygons
    \item Etc...
    \end{itemize}

    \medskip

    Good implementations should deal with common special cases.
  }
\end{frame}


\begin{frame}
  \frametitle{Degeneracies: Precision errors}

  Representation of \structure{floating point} numbers in computers
  has a limited precision. So for multiple operations on very small
  numbers, we may start to see calculation errors.

  \medskip

  Some ways to avoid floating point precision errors:

  \begin{itemize}
  \item Whenever possible, convert the float numbers to integers
  \item Never compare ``float x == float y''. 
  \item Instead, use this: ``fabs(x - y) < EPS'' (float EPS$ = 0.00000001$)
  \end{itemize}

  %% TODO: Add some degeneracy image examples
\end{frame}



\section{Points and Lines}
\subsection{Points}
\begin{frame}[fragile,singleslide]
  \frametitle{Point Representation}  
  {\smaller

    \begin{columns}
      \column{0.6\textwidth}
      Points are the building blocks of geometric objects.
      In C/C++, we can represent them using a struct with two members:
      \column{0.4\textwidth}
      \includegraphics[width=.65\textwidth]{../img/geom2}
    \end{columns}

    \begin{exampleblock}{}
\begin{verbatim}
// When possible, use int coordinates
struct point_i { int x, y;
  point_i() { x = y = 0; }
  point_i(int _x, int _y) : x(_x), y(_y) {}};

// Floating point variation
struct point { double x, y;
  point() { x = y =0.0;}
  point(double _x, double _y) : x(_x), y(_y) {}};
\end{verbatim}
    \end{exampleblock}
  }
\end{frame}

\begin{frame}[fragile,singleslide]
  \frametitle{Point Operations}

  {\smaller
    To compare two points, or test for equality, we can 
    overload the \emph{equal} or \emph{less} operator in the point struct.

    \begin{exampleblock}{}
\begin{verbatim}
struct point { double x, y;
   point() { x = y = 0.0;
   point(double _x, double _y) : x(_x), y(_y) {}

   // override less than operator -- useful for sorting
   bool operator < (point other) const { 
      if (fabs(x - other.x) > EPS)
         return x < other.x;
      return y < other.y; }

   // override equal operator, takes EPS into account
   bool operator == (point other) const {
      return (fabs(x - other.x) < EPS &&  
             (fabs(y - other.y) < EPS)); }
   }
\end{verbatim}
    \end{exampleblock}
  }
\end{frame}

\begin{frame}[fragile,singleslide]
  \frametitle{Point: Euclidean Distance}
  {\smaller

    \begin{exampleblock}{}
\begin{verbatim}
#define hypot(dx,dy) sqrt(dx*dx + dy*dy)

double dist(point p1, point p2) {
  return hypot(p1.x - p2.x, p1.y - p2.y);
}
\end{verbatim}
    \end{exampleblock}

    \begin{center}
      \includegraphics[width=0.5\textwidth]{../img/geom1}
    \end{center}
    %% TODO: Add image for Euclidean distance
  }
\end{frame}
  
\begin{frame}[fragile,singleslide]
  \frametitle{Point: Rotation around origin}
  {\smaller
    \begin{exampleblock}{}
\begin{verbatim}
#define PI           3.14159265358979323846  /* pi */
#define DEG_to_RAD(X) (X*PI)/180.0

// theta is in degrees
point rotate(point p, double theta) {
   double rad = DEG_to_RAD(theta);
   return point(p.x * cos(rad) - p.y * sin(rad),
                p.x * sin(rad) + p.y * cos(rad));}   
\end{verbatim}
    \end{exampleblock}
    \begin{center}
      \includegraphics[width=0.8\textwidth]{../img/rotation_halim}
    \end{center}
  }
\end{frame}

\subsection{Lines}

\begin{frame}[fragile,singleslide]
  \frametitle{Line Representation}

  {\small
  How to represent a line?

  \begin{itemize}
  \item \structure{Two points}. Problem: cannot generalize for other
    points of the line easily;
  \item \structure{y = mx + c}. Problem: cannot handle vertical lines (m is infinite)
  \item \structure{ax + by + c = 0}. Better representation for ``most'' cases.
  \end{itemize}

  \begin{exampleblock}{}
\begin{verbatim}
struct line { double a,b,c; };

void pointsToLine(point p1, point p2, line &l) {
   if (fabs(p1.x - p2.x) < EPS { 
      l.a = 1.0; l.b = 0.0; l.c = -p1.x; }
   else {
      l.a = -(double) (p1.y-p2.y)/(p1.x-p2.x);
      l.b = 1.0; l.c = -(double) (l.a*p1.x) - p1.y;}
}
\end{verbatim}
  \end{exampleblock}
  }
\end{frame}

\begin{frame}[fragile,singleslide]
  \frametitle{Line: Parallel and Identical lines}
  {\smaller

    \begin{columns}
      \column{0.6\textwidth}
      \begin{itemize}
      \item Two lines are parallel if their coefficients $(a, b)$ are the same;
      \item Two lines are identical if all coefficients $(a,b,c)$ are the same;
      \item Remember that we force $b$ to be 0 or 1;
      \end{itemize}
      \column{0.4\textwidth}
      \includegraphics[width=.8\textwidth]{../img/geom3}
    \end{columns}
  \begin{exampleblock}{}
\begin{verbatim}
bool areParallel(line l1, line l2) {
   return (fabs(l1.a-l2.a) < EPS) && 
          (fabs(l1.b-l2.b) < EPS); }

bool areSame(line l1, line l2) {
   return areParallel(l1,l2) && 
          (fabs(l1.c - l2.c) < EPS); }
\end{verbatim}    
  \end{exampleblock}
  }
\end{frame}

\begin{frame}[fragile,singleslide]
  \frametitle{Line: Intersection} 
  {\smaller 
    
    If two lines are not parallel, then they will intersect at a
    point. This point (x,y) is found by solving the system of two
    linear equations: 

    \begin{equation*}
      a_1x+b_1y+c_1 = 0 \text{ and } a_2x+b_2y+c_2 = 0
    \end{equation*}

    \vfill

    \begin{exampleblock}{}
\begin{verbatim}
bool areIntersect(line l1, line l2, point &p) {
   if (areParallel(l1,l2)) return False;

   p.x = (l2.b * l1.c - l1.b * l2.c) / 
         (l2.a * l1.b - l1.a * l2.b);
   if (fabs(l1.b) > EPS) // Testing for vertical case
      p.y = -(l1.a * p.x + l1.c);
   else
      p.y = -(l2.a * p.x + l2.c);
   return true; }}
\end{verbatim}
    \end{exampleblock}

  }
\end{frame}

\subsection{Segments}
\begin{frame}[fragile,singleslide]
  \frametitle{Segments and Vectors}
  {\smaller
    \begin{columns}
      \column{0.7\textwidth}
      \begin{itemize}
      \item A \structure{Line Segment} is a line limited by two points and finite length;
      \item A \structure{Vector} is a segment with an associated direction;
      \item Often vectors are represented by a single point
        (the other assumed to be the origin);
      \end{itemize}
      \column{0.3\textwidth}
      \includegraphics[width=.95\textwidth]{../img/geom4}
    \end{columns}
    \begin{exampleblock}{}
\begin{verbatim}
struct vec { double x, y; 
    vec(double _x, double _y) : x(_x), y(_y) {} };

vec toVec(point a, point b) { 
    return vec(b.x - a.x, b.y - a.y); }
vec scale(vec v, double s) { 
    return vec(v.x * s, v.y * s); }
point translate(point p, vec v) { 
    return point(p.x + v.x , p.y + v.y); }
\end{verbatim}
    \end{exampleblock}
  }
\end{frame}

\begin{frame}
  \frametitle{Distance between point and line}
  \begin{block}{}
    Given a point $p$ and a line $l$, the distance between the point
    and the line is the distance between $p$ and the $c$, the closest
    point in $l$ to $p$.

    \bigskip

    We can calculate the position of $c$ by taking the projection of
    $\bar{ac}$ into $l$ ($a,b$ are points in $l$).
  \end{block}

  \begin{center}
    \includegraphics[width=0.8\textwidth]{../img/geom5}
  \end{center}
\end{frame}

\begin{frame}[fragile,singleslide]
  \frametitle{Distance between point and line}

  {\smaller
  \begin{exampleblock}{}
\begin{verbatim}
double dot(vec a, vec b) { 
   return (a.x * b.x + a.y * b.y); }
double norm_sq(vec v) { 
   return v.x * v.x + v.y * v.y; }

// Calculates distance of p from line, given 
// a,b different points in the line.
double distToLine(point p, point a, point b, point &c) {
  // formula: c = a + u * ab
  vec ap = toVec(a, p), ab = toVec(a, b);
  double u = dot(ap, ab) / norm_sq(ab);
  c = translate(a, scale(ab, u));
  // translate a to c
  return dist(p, c); }
\end{verbatim}
  \end{exampleblock}

}
\end{frame}

\begin{frame}[fragile,singleslide]
  \frametitle{Distance between segment and line} 
  % TODO: Add an image explaining the u conditional

  {\smaller 
    If we have a \structure{segment} $ab$ instead of a line, the
    procedure to calculate the distance is similar, but we need to
    test if the intersection point falls in the segment.
    
    \begin{exampleblock}{}
\begin{verbatim}
double distToLineSegment(point p, point a, 
                         point b, point &c) {
  vec ap = toVec(a, p), ab = toVec(a, b);
  double u = dot(ap, ab) / norm_sq(ab);
  
  if (u < 0.0) { c = point(a.x, a.y); // closer to a
                 return dist(p, a); }
  if (u > 1.0) { c = point(b.x, b.y); // closer to b
                 return dist(p, b); }

  return distToLine(p, a, b, c); }
\end{verbatim}
    \end{exampleblock}
  }
\end{frame}

\begin{frame}[fragile,singleslide]
  \frametitle{Angles between segments}
  {\smaller

    \begin{exampleblock}{angle between two segments ao and ob}
\begin{verbatim}
#import <cmath>

double angle(point a, point o, point b) { // in radians
vec oa = toVector(o, a), ob = toVector(o, b);
return acos(dot(oa, ob)/sqrt(norm_sq(oa)*norm_sq(ob)));}
\end{verbatim}
    \end{exampleblock}

\structure{Left/Right test}: We can calculate the position of point
$p$ in relation to a line $l$ using the \structure{cross product}.

\smallskip

Take $q,r$ points in $l$. Magnitude of the cross product $pq$ x $pr$
being positive/zero/negative means that $p \rightarrow q \rightarrow
r$ is a left turn/collinear/right turn.

\begin{exampleblock}{}
\begin{verbatim}
double cross(vec a, vec b) { 
  return a.x * b.y - a.y * b.x; } 
bool ccw(point p, point q, point r) { 
  return cross(toVec(p, q), toVec(p, r)) > 0; } 
collinear(point p, point q, point r) { 
  return fabs(cross(toVec(p, q), toVec(p, r))) < EPS;
\end{verbatim}
\end{exampleblock}
}
\end{frame}

\subsection{Problem Example}
\begin{frame}
  \frametitle{Problem Example: UVA -- Intersection}
  % TODO: Add image for this problem
  % TODO: Add code for this problem
  {\small
    \begin{block}{Summary}
    Given two points $p_1$ and $p_2$, and a rectangle, test whether
    the segment $p_1p_2$ intersects the rectangle.
    \end{block}

    Strategy
    \uncover<2->{
      \begin{itemize}
      \item Test if points $p_1$ or $p_2$ are in the rectangle (easy tests first)
      \item Test if $p_1p_2$ intersects with any side of the rectangle. 
      \item ``Hard'' Way:
        \uncover<3->{
          {\smaller
          \begin{itemize}
          \item Find the intersection between lines $p_1p_2$, and top/bottom/left/right
          \item Test if the intersection point is in line $p_1p_2$;
          \item Test if the intersection point is in the rectangle;
          \end{itemize}}
        }
      \uncover<4>{\item There is an easier way that takes into account vertical/horizontal sides}

      \end{itemize}
    }

  }
\end{frame}

\begin{frame}
  \frametitle{Problem Example: UVA -- Waterfalls}
  % TODO: Add image for this problem
  % TODO: Add code for this problem
  {\small
    \begin{block}{Summary}
      Given a list of water sources, and a list of segments, calculate the position that 
      each water source will arrive at the bottom.
    \end{block}

    Strategy:
    \begin{itemize}
    \item For each water source, calculate all the segments that intersect it (easy because vertical line)
    \item For each segment, calculate the intersection point - get the highest one.
    \item New position of the water source is the lowest point of that segment.
      
      \bigskip

    \uncover<2>{\item \alert{Problem:} No limit of segments or water sources. How do you avoid TLE?}
    \end{itemize}
  }
\end{frame}

\section{Circles}
\subsection{Circles}
\begin{frame}[fragile,singleslide]
  \frametitle{Circles}
  {\smaller
  \begin{itemize}
  \item A circle is defined by its center $(a,b)$ an its radius $r$
  \item The circle contains all points such $(x,y)$ such as $(x-a)^2+(y-b)^2 \leq r^2$
  \end{itemize}

  \begin{exampleblock}{}
\begin{verbatim}
int insideCircle(point_i p, point_i c, int r) {
   int dx = p.x-c.x, dy = p.y-c.y;
   int Euc = dx*dx + dy*dy, rSq = r*r;
   return Euc < rSq ? 0 : Euc == rSq ? 1 : 2; 
   // 0 - inside, 1 - border, 2- outside
}
\end{verbatim}
  \end{exampleblock}
  }
\end{frame}


\begin{frame}
  \frametitle{Circles (2)}
  {\smaller

    \begin{center}
      % TODO: Make my own image
      \includegraphics[width=0.65\textwidth]{../img/circle_halim}
    \end{center}

    \begin{itemize}
      \item If you are not given $\pi$, use $pi = 2*$acos$(0.0)$;
      \item Diameter: $D=2r$; Perimeter/Circumference: $C=2\pi r$; Area: $A=\pi r^2$;
      \item To calculat the \structure{Arc} of an angle $\alpha$ (in Degrees), $\frac{\alpha}{360}*C$;
    \end{itemize}
  }
\end{frame}

\begin{frame}
  \frametitle{Circles (3)}
  {\smaller
    \begin{center}
      % TODO: Make my own image
      \includegraphics[width=0.65\textwidth]{../img/circle_halim}
    \end{center}
  
  \begin{itemize}
  \item A \structure{chord} of a circle is a segment composed of two points in the circle's border. A circle with radius $r$ and angle $\alpha$ degrees has a chord of length sqrt$(2r^2(1-\cos{\alpha}))$
  \item A \structure{Sector} is the area of the circle that is enclosed by two radius and and arc between them. Area is: $\frac{\alpha}{360}A$
  \item A \structure{Segment} is the region enclosed by a chord and an arc.

  \end{itemize}
  }
\end{frame}


\subsection{Problem Example}
\begin{frame}
  \frametitle{Problem Example: Area}

  \begin{block}{Summary}
    Given 4 circles, determine the proportion of points that fall in all four circles.
  \end{block}
  
\end{frame}

% Circles

\section{Triangles}
\subsection{Triangles}
\begin{frame}
  \frametitle{Triangles!}
\end{frame}
% Triangles


\section{Polygons}
\begin{frame}
  \frametitle{Polygons!}
\end{frame}
% Polygons

%% Intro and representation

%% Area of a polygon

%% Checking if a polygon is convex

%% Checking if a point is in a polygon

%% Cutting a polygon

%% Grahan Scan
%% Grahan Scan without using atan (Andrew's Monotone Chain Algorithm)

%% TODO: Add problem examples to each section -- book only deals with libraries

\section{Conclusion}

\subsection{Problem Discussion}
\begin{frame}
  \frametitle{Problem Discussion}
  \begin{itemize}
  \item Sunny Mountains
  \item Bright Lights
  \item Rope Crisis in Ropeland
  \item Bounding Box
  \item Soya Milk
  \item SCUD Bursters
  \item Trash Removal
  \item The Sultan's Problem
  \end{itemize}
\end{frame}

\subsection{Conclusion}
\begin{frame}
  \frametitle{Class Summary}
  Computational Geometry
  \begin{itemize}
  \item Basic Concepts
  \item Triangles
  \item Circles
  \item Polygons
  \end{itemize}

  \begin{block}{}
    Final Week: String Problems!
  \end{block}
\end{frame}

\end{document}

