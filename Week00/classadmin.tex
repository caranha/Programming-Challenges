
\section{Course System}
\subsection{Outline}

\begin{frame}
  \centering
  {\huge
    Course Organization
  }
\end{frame}

\begin{frame}
  \frametitle{Outline}
  \begin{enumerate}
    \item Course Schedule
    \item Course Materials
    \item How to submit problems
    \item Grading
    \item Office Hours and Teacher Communication
  \end{enumerate}
\end{frame}

\subsection{Course Schedule}
\begin{frame}{What you will do every week}
  \begin{description}[Next Monday]
    \item[Monday] -- Get materials from \alert{manaba}. Check the problems;
    \medskip
    \item[Tuesday] -- 3C205/3C206: Solve the problems, Ask questions to the teacher;
    \medskip
    \item[Entire Week] -- Submit the problems on \alert{kattis};
    \medskip
    \item[Next Tue] -- Homework Deadline
  \end{description}
\end{frame}

\begin{frame}{Course Dates and Deadlines}
  \begin{block}{Course Dates}
    \begin{itemize}
    \item 4/18, 4/25, 5/02, 5/09, 5/16, 5/30, 6/6, 6/13, 6/20, 6/27;
    \item No final exam;
    \end{itemize}
  \end{block}
  \begin{block}{Deadlines}
    \begin{itemize}
      \item Deadline for homework: {\bf Every Tuesday, 23:00}
    \end{itemize}
  \end{block}
\end{frame}

\subsection{Course Materials}
\begin{frame}{Where to find the material?}
  \begin{block}{manaba}
    \begin{itemize}
      \item Official place for lecture material and videos;
      \item Use Forum for questions;
      \item Please read announcements; Please answer surveys;
    \end{itemize}
  \end{block}
  \begin{block}{github}
    \begin{itemize}
      \item Lecture materials is also available on github:
      \item URL: \url{https://caranha.github.io/Programming-Challenges/}
      \item Not-official. Includes material from last year.
      \item manaba is the official version
    \end{itemize}
  \end{block}
\end{frame}

\begin{frame}{Websites to submit homework}
  \begin{block}{KATTIS online judge}
    \begin{itemize}
      \item If you are \alert{Taking Credit} for the class: \url{https://tsukuba.kattis.com}
      \item If you are \alert{Not Taking Credit} for the class: \url{https://open.kattis.com}
        \medskip
        
      \item Create an account using your university e-mail (sXXXXXXX@u.tsukuba.ac.jp)
        \begin{itemize}
        \item The name can be anything.
        \item If you don't use your university e-mail, I will not be able to count the homework.
        \end{itemize}\medskip
        
      \item (Kattis Demo) See material on Manaba for more information.
    \end{itemize}
  \end{block}
\end{frame}

\begin{frame}{About Course Language}
  \begin{block}{Natural Language}
    \begin{itemize}
      \item Materials and Homework: English
      \item Video and manaba: Japanese
      \item E-mail, feedback: English/Japanese;
      \item If you want to help me translate the homework, contact me!
    \end{itemize}
  \end{block}

  \begin{block}{Programming Language}
    \begin{itemize}
      \item The Judge accepts: C, C++, Java, Python, Ruby;
      \item The teacher helps with: C, C++, Java, Python;
      \item If you want to use another language, contact me;
    \end{itemize}
  \end{block}
\end{frame}

\begin{frame}{Reference Books}
  Textbook:
  \begin{itemize}
    \item {\bf textbook:} Steven Halim, Felix Halim,"Competitive Programming", 4th edition. \url{https://cpbook.net/}
  \end{itemize}
  \bigskip

  Other books:
  \begin{itemize}
    \item Steven S. Skiena, Miguel A. Revilla,"Programming Challenges", Springer, 2003
    \item 秋葉拓哉、 岩田陽一、 北川宜稔,『プログラミングコンテストチャレンジブック』
    \item 渡部有隆、『オンラインチャレンジではじめるC/C++プログラミング入門、Online Programming Challenge!』 (ISBN978-4-8399-5110-8)
    \item 渡部有隆、『プログラミングコンテスト攻略のためのアルゴリズムとデータ構造』(ISBN978-4-8399-5295-2)
  \end{itemize}
\end{frame}

\subsection{Grading}
\begin{frame}{Grading Rules}{Base Grade}

  Your {\bf base grade} is based on the {\bf number of accepted homework programs} you submit:

  \begin{itemize}
    \item {\bf C grade:} 2+ accepted problems every week;
    \item {\bf B grade:} 3+ accepted problems every week;
    \item {\bf A grade:} 5+ accepted problems every week;
    \item {\bf A+ grade:} 7+ accepted problems every week;
  \end{itemize}\bigskip

  \begin{alertblock}{Important!!}
    \begin{itemize}
      \item Every week means "X problems every week" (not ``average'');
      \item You can submit problems late, with a penalty;
    \end{itemize}
  \end{alertblock}
\end{frame}

\begin{frame}{Grading Rules}{Late Penalty}
  You can submit problems late. But there is a penalty.\bigskip

  If the number of {\bf total late programs} $\ge$ 25\% of total programs, your grade will lower 1 step.\bigskip

  {\bf You will not fail the course for late programs.}

  \begin{exampleblock}{Example:}
    \begin{itemize}
      \item Student (1) submitted 45 problems, minimum 5 problems per week. 5 problems are late. $5 \le 45*0.25$, no penalty. Grade A.
      \item Student (2) submitted 46 problems, minimum 5 problems per week. 16 problems are late. $16 \ge 46*0.25$, {\bf penalty}. Grade B.
      \item Student (3) submitted 24 problems, minimum 2 problems per week. 10 problems are late. $10 \ge 24*0.25$, {\bf penalty}. \alert{Grade C (will not fail)}.
    \end{itemize}
  \end{exampleblock}
\end{frame}


\begin{frame}{Grading}{Plagiarism}
  The assignments are \alert{individual}. You must write your
  programs by yourself.

  \begin{exampleblock}{You can do this}
    \begin{itemize}
    \item Ask for ideas to your friends;
    \item Ask for ideas in the MANABA forum;
    \item Ask for help with a bug;
    \end{itemize}
  \end{exampleblock}

  \begin{alertblock}{You can NOT do this}
    \begin{itemize}
    \item Copy a solution from the internet;
    \item Copy a solution from your friends;
    \item Give your code to a friend;
    \end{itemize}
  \end{alertblock}

  Students who do plagiarism will fail the course, and suffer penalties from the university.
\end{frame}
