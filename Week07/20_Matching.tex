

\section{String Matching}

\begin{frame}
  \begin{center}
    {\bf Part I: String Matching}
  \end{center}
\end{frame}

\subsection{Basics}

\begin{frame}[fragile]{The String Matching Problem}
  \begin{block}{Definition}
    Given a string $T$ (also called {\bf text}), we want to test if the substring $P$ (also called {\bf pattern}) exists in $T$.
    \bigskip

    If $P$ exists in $T$, we want to know the {\bf index} of the start of $P$ in $T$.
  \end{block}\bigskip

  Example:
\begin{verbatim}
T: STEVEN EVENT

P: EVE            indexes: 2 and 7
P: EVENT          indexes: 7
P: EVENING        indexes: -1 or NULL
\end{verbatim}
\end{frame}

\begin{frame}{String Matching and Libraries}

  \begin{block}{String Matching with the standard functions}
    \begin{itemize}
      \item C/C++: strstr(T,P) or T.find(P)
      \item Python: T.find(P)
      \item Java: T.indexOf(P)
    \end{itemize}
  \end{block}

  Using the standard library is usually bug-free, but sometimes you need to code string search by hand:
  \begin{itemize}
    \item Specific Matching Function (ex: "1" == "I", "0" == "O");
    \item Match in multiple directions (graph, grind);
    \item Match multiple strings at once;
    \item etc...
  \end{itemize}\bigskip
\end{frame}

\begin{frame}[fragile]{String Matching: Complete Search}

  For every character $T_i$, test if $P$ begins at that position.\bigskip

\begin{verbatim}
for (int i = 0; i < |T|; i++)
  bool match = true;
  for (int j = 0; j < |P| && match; j++)
    if (i+j >= |T| || P[j] != T[i+j])
      match = false;
  if (match)
    printf("Match P at index %d\n", i);
\end{verbatim}

{\bf Number of Steps}:
  \begin{itemize}
    \item Average case: $O(|T|)$ -- For natural T and small P;
    \item Worst case: $O(|T|\times|P|)$;
    \begin{itemize}
      \item T = AAAAAAAAAAAAB
      \item P = AAAAAAAB
    \end{itemize}
  \end{itemize}
\end{frame}

\subsection{Knuth-Morris-Pratt}
\begin{frame}[fragile]
  \frametitle{The Knuth-Morris-Pratt (KMP) Algorithm}

  \begin{itemize}
    \item Complete Search can be very expensive if the prefix of $P$ happens many times in $T$.
    \item In 1977, Knuth, Morris and Pratt developed an algorithm that {\bf uses these prefixes} to realize fast string matching.
  \end{itemize}

  \begin{block}{Basic Idea}
    \begin{itemize}
    \item The KMP algorithm identifies "borders" in the partial match between $P$ and $T$.
    \item These borders are characterized by identical prefixes and sufixes in the T-P match.
    \item The algorithm uses these matches to advance the indexes of $T$ and $P$, greatly reducing the number of comparisons.
  \end{itemize}
  \end{block}
  The KMP algorithm is O(P+T).
\end{frame}

\begin{frame}[fragile]{KMP Algorithm -- Simulation}

{\smaller
\begin{verbatim}
              1         2         3         4         5
    012345678901234567890123456789012345678901234567890
T = I DO NOT LIKE SEVENTY SEV BUT SEVENTY SEVENTY SEVEN
P = SEVENTY SEVEN
// for i from 0 to 13, KMP works like full search

                  SEVENTY SEVEN
// Here, the collision is at i=25, j = 11, But because "SEV" is
// a "border", i stays the same and j is rewinded to 3

                                  SEVENTY SEVEN
// Here we find a match with i=43, j=13; SEVEN is a border, so j
// is rewinded to 5, and i is kept the same. The algorithm
// continues matching at i=44, j=5 ("T")

                                          SEVENTY SEVEN
// KMP finds a second match
\end{verbatim}}

\end{frame}


\begin{frame}[fragile]{KMP Algorithm -- Rewind Array}

\begin{block}{}
To avoid repeated matches, the KMP algorithm builds a {\bf rewind table} $b$ (back).

\begin{verbatim}
     0 1 2 3 4 5 6 7 8 9 0 1 2 3
P =  S E V E N T Y   S E V E N \0
b = -1 0 0 0 0 0 0 0 0 1 2 3 4 5
\end{verbatim}

Following the table $b$, we know that if we find a mismatch at $j = 11$, then we need to rewing $j$ to $b[11] = 3$ to continue matching.\bigskip

The text index $i$, on the other hand, will stay the same, and go forward by 1 if $b[j] = -1$.
    \end{block}
\end{frame}

\begin{frame}[fragile]{KMP Algorithm -- PseudoCode}

  {\smaller
  \begin{exampleblock}{}
\begin{verbatim}
char T[MAX_N], P[MAX_N];    int b[MAX_N], n, m;

void kmpPreprocess() {                         // Create the Back Array
  int i = 0, j = -1; b[0] = -1;
  while (i < m) {
     while (j >= 0 && P[i] != P[j]) j = b[j];
     i++; j++;
     b[i] = j; }}

void kmpSearch() {                             // Search the substring
  int i = 0, j = 0;
  while (i < n) {
     while (j >= 0 && T[i] != P[j]) j = b[j];
     i++; j++;
     if (j == m) {
        printf("P is found at index %d in T\n", i - j);
        j = b[j]; }}}
\end{verbatim}
  \end{exampleblock}
  }
\end{frame}

\subsection{Z algorithm}

\begin{frame}[fragile]{String Matching with the Z-Algorithm}
  Another alogirthm that performs string matcfhing in linear time is the {\bf Z algorithm}.\bigskip

  The Z algorithm first makes a {\bf Search String} $S = P + 'S' + T$.\\
  The Z algorithm next constructs a {\bf Z array} of "prefix lengths".\\
  For every index $i\in S$, $Z[i]$ is the size of the prefix of $S$ that begins in $i$.\bigskip

\begin{verbatim}
T = AASABAABAAT, P = AAB, S = P + '$' + T

... Build Z Array ...
S   = AAB$AASABAABAAT
Z[S]= X10021010310210
               ^
               String matched here. Z[i] = Len(P)
\end{verbatim}
\end{frame}

\begin{frame}[fragile]{Z-Algorithm -- Pseudocode}
  \begin{block}{}
{\smaller
\begin{verbatim}
void Zarray(string S, int Z[]) {
    int n = S.length(); int L, R, k;
    L = R = 0;             // Prefix counters
    for (int i = 1; i < n; i++) {
        if (i > R) {       // Full search of prefix
            L = R = i;
            while (R < n && S[R] == S[R-L]) R++;
            Z[i] = R-L; R--;
        } else {           // Inside prefix candidate
            k = i-L;
            if (Z[k] < R-i+1) Z[i] = Z[k]; // no extension
            else {                         // prefix extension
                L = i;
                while (R < n && S[R] == S[R-L]) R++;
                Z[i] = R-L; R--;
}   }   }   }
\end{verbatim}}
  \end{block}

{\smaller {\bf Simulation:}
\url{https://personal.utdallas.edu/~besp/demo/John2010/z-algorithm.htm}}

\end{frame}

\begin{frame}{Z algorithm or KMP algorithm?}
  Should you use the Z algorithm or the KMP algorithm?\bigskip

  \begin{itemize}
    \item Both algorithms have the same time complexity: $O(T+P)$\bigskip

    \item Which algorithm is easier to understand?
    \begin{itemize}
      \item KMP calculates a recursive suffix state machine for P;
      \item Z-algorithm calculates a substring size array for T;
    \end{itemize}\bigskip
  \end{itemize}
\end{frame}
