\documentclass{beamer}

\usepackage{amssymb,amsmath}
\usepackage{graphicx}
\usepackage{url}
\usepackage{color}
\usepackage{relsize}		% For \smaller
\usepackage{url}			% For \url
\usepackage{epstopdf}	% Included EPS files automatically converted to PDF to include with pdflatex
\usepackage{pagenote}[continuous,page]

%For MindMaps
% \usepackage{tikz}%
% \usetikzlibrary{mindmap,trees,arrows}%

%%% Color Definitions %%%%%%%%%%%%%%%%%%%%%%%%%%%%%%%%%%%%%%%%%%%%%%%%%%%%%%%%%
%\definecolor{bordercol}{RGB}{40,40,40}
%\definecolor{headercol1}{RGB}{186,215,230}
%\definecolor{headercol2}{RGB}{80,80,80}
%\definecolor{headerfontcol}{RGB}{0,0,0}
%\definecolor{boxcolor}{RGB}{186,215,230}

%%% Save space in lists. Use this after the opening of the list %%%%%%%%%%%%%%%%
%\newcommand{\compresslist}{
%	\setlength{\itemsep}{1pt}
%	\setlength{\parskip}{0pt}
%	\setlength{\parsep}{0pt}
%}

%\setbeameroption{show notes on top}

% You should run 'pdflatex' TWICE, because of TOC issues.

% Rename this file.  A common temptation for first-time slide makers
% is to name it something like ``my_talk.tex'' or
% ``john_doe_talk.tex'' or even ``discrete_math_seminar_talk.tex''.
% You really won't like any of these titles the second time you give a
% talk.  Try naming your tex file something more descriptive, like
% ``riemann_hypothesis_short_proof_talk.tex''.  Even better (in case
% you recycle 99% of a talk, but still want to change a little, and
% retain copies of each), how about
% ``riemann_hypothesis_short_proof_MIT-Colloquium.2000-01-01.tex''?

\mode<presentation>
{
  % A tip: pick a theme you like first, and THEN modify the color theme, and then add math content.
  % Warsaw is the theme selected by default in Beamer's installation sample files.

  %%%%%%%%%%%%%%%%%%%%%%%%%%%% THEME
  %\usetheme{Madrid}		% No subsection
  \usetheme{AnnArbor}  % Subsection on top, no color


  %\usetheme{Antibes}
  %\usetheme{Bergen}
  %\usetheme{Berkeley}		% bem bacana - menu esquerdo
  %\usetheme{Berlin}
  %\usetheme{Boadilla}
  %\usetheme{boxes}
  %\usetheme{CambridgeUS}		% bem bacana - menu superior
  %\usetheme{Copenhagen}
  %\usetheme{Darmstadt}
  %\usetheme{default}
  %\usetheme{Dresden}
  %\usetheme{Frankfurt}
  %\usetheme{Goettingen}
  %\usetheme{Hannover}		% bem bacana - menu esquerdo
  %\usetheme{Ilmenau}
  %\usetheme{JuanLesPins}
  %\usetheme{Luebeck}
  %\usetheme{Malmoe}
  %\usetheme{Marburg}		% bem bacana - menu direito
  %\usetheme{Montpellier}
  %\usetheme{PaloAlto}		% bem bacana - menu esquerdo
  %\usetheme{Pittsburgh}
  %\usetheme{Rochester}		%bacana
  %\usetheme{Singapore}
  %\usetheme{Szeged}
  %\usetheme{Warsaw}

  %%%%%%%%%%%%%%%%%%%%%%%%%%%% COLOR THEME
  %\usecolortheme{default}		% branco, azul clarinho
  \usecolortheme{crane}		% Very yellow (ok)

  %\usecolortheme{albatross}		% azul escuro, massa
  %\usecolortheme{beetle}		% cinza, menu azul
  %\usecolortheme{dolphin}		% azul e branco, legal
  %\usecolortheme{dove}			% cinza e branco, feio
  %\usecolortheme{fly}			% todo cinza, horrível
  %\usecolortheme{lily}			% parece o default
  %\usecolortheme{orchid}		% azul e branco, ok
  %\usecolortheme{rose}			% branco e violeta-claro, bonito
  %\usecolortheme{seagull}		% cinza, feio
  %\usecolortheme{seahorse}		% nhé, meio feio
  %\usecolortheme{sidebartab}		% Azul, branco, destaque na tab, interessante
  %\usecolortheme{structure}		% bichado
  %\usecolortheme{whale}		% Azul e branco, bem bonito

  %%%%%%%%%%%%%%%%%%%%%%%%%%%% OUTER THEME
  \useoutertheme{default}
  %\useoutertheme{infolines}
  %\useoutertheme{miniframes}
  %\useoutertheme{shadow}
  %\useoutertheme{sidebar}
  %\useoutertheme{smoothbars}
  %\useoutertheme{smoothtree}
  %\useoutertheme{split}
  %\useoutertheme{tree}

  %%%%%%%%%%%%%%%%%%%%%%%%%%%% INNER THEME
  \useinnertheme{circles}
  %\useinnertheme{default}
  %\useinnertheme{inmargin}
  %\useinnertheme{rectangles}
  %\useinnertheme{rounded}

  %%%%%%%%%%%%%%%%%%%%%%%%%%%%%%%%%%%

  \setbeamercovered{invisible} % or whatever (possibly just delete it)
  % To change behavior of \uncover from graying out to totally
  % invisible, can change \setbeamercovered to invisible instead of
  % transparent. apparently there are also 'dynamic' modes that make
  % the amount of graying depend on how long it'll take until the
  % thing is uncovered.

}


% Get rid of nav bar
\beamertemplatenavigationsymbolsempty

% Use short top
%\usepackage[headheight=12pt,footheight=12pt]{beamerthemeboxes}
%\addheadboxtemplate{\color{black}}{
%\hskip0.5cm
%\color{white}
%\insertshortauthor \ \ \ \
%\insertframenumber \ \ \ \ \ \ \
%\insertsection \ \ \ \ \ \ \ \ \ \ \ \ \ \ \ \ \  \insertsubsection
%\hskip0.5cm}
%\addheadboxtemplate{\color{black}}{
%\color{white}
%\ \ \ \
%\insertsection
%}
%\addheadboxtemplate{\color{black}}{
%\color{white}
%\ \ \ \
%\insertsubsection
%}

% Insert frame number at bottom of the page.
% \usefoottemplate{\hfil\tiny{\color{black!90}\insertframenumber}}

%% makes the ppagenote command for figure references at the end.

\usepackage[english]{babel}
%qq\usepackage[latin1]{inputenc}
\usepackage{CJKutf8}
\usepackage{subfigure}

\usepackage{times}
\usepackage[T1]{fontenc}

\makepagenote
\renewcommand{\notenumintext}[1]{}
\newcommand{\ppagenote}[1]{\pagenote[Page \insertframenumber]{#1}}

\title[Programming Challenges]{GB20602 - Programming Challenges}
\author[Claus Aranha]{Claus Aranha\\{\footnotesize caranha@cs.tsukuba.ac.jp}}
\institute[U. Tsukuba]{University of Tsukuba, Department of Computer Sciences}

\usepackage{tikz}
\usetikzlibrary{arrows,shapes}
% Latex Graph Example:
% https://www.overleaf.com/5297501zrjzfm#/16716638/

% TODO: Silly Makefile

\tikzstyle{vertex}=[circle,fill=black!25,minimum size=10pt,inner sep=0pt]
\tikzstyle{blue vertex}=[circle,fill=blue!100,minimum size=10pt,inner sep=0pt]
\tikzstyle{red vertex}=[circle,fill=red!100,minimum size=10pt,inner sep=0pt]
\tikzstyle{edge} = [draw,thick,-]
\tikzstyle{red edge} = [draw, line width=5pt,-,red!50]
\tikzstyle{black edge} = [draw, line width=5pt,-,black!20]
\tikzstyle{weight} = [font=\smaller]

\title[GB21802]{GB21802 - Programming Challenges}
\subtitle[]{Week 7 - Math Problems}
\author[Claus Aranha]{Claus Aranha\\{\footnotesize caranha@cs.tsukuba.ac.jp}}
\institute{College of Information Science}
\date{2015-06-10,13\\{\tiny Last updated \today}}

\begin{document}

\section{Introduction}
\subsection{Title}
\begin{frame}
\maketitle
\end{frame}

\subsection{Notes and Warnings}

\begin{frame}
  \frametitle{Last Week Results}
  \begin{block}{Week 6 - Graph II}
    {\smaller
      \begin{columns}[T]
        \column{0.5\textwidth}
        \begin{itemize}
        \item From Dusk Until Dawn -- 1/31
        \item Wormholes -- 11/31
        \item Mice and Maze -- 6/31
        \item Degrees of Separation -- 5/31
        \item Avoiding your Boss -- 4/31
        \item Arbitrage -- 0/31
        \item Software Allocation -- 3/31
        \item Sabotage -- 0/31
        \item Little Red Riding Hood -- 6/31
        \item Gopher II -- 3/31
        \end{itemize}
        \column{0.5\textwidth}
        \begin{itemize}
        \item 19 people: 0 problems;
        \item 3 people: 1 problem;
        \item 3 people: 2 problems;
        \item 2 people: 3-4 problems;
        \item 2 person: 6-7 problems!
        \end{itemize}
      \end{columns}
    }
  \end{block}
\end{frame}

\begin{frame}
  \frametitle{Special Notes}

  {\small
    \begin{block}{ASPS for single weighted graphs}

      Apparently, this is still an open problem!

      \begin{itemize}
      \item $v$ times BFS: $O(ve+v^2)$
      \item A paper (2009) claims $O(v^2\text{log}v)$:\\
        {\tiny 
        \url{http://waset.org/publications/8870/all-pairs-shortest-paths-problem-for-unweighted-graphs-in-o-n2-log-n-time}}\\
        (pseudocode included!)
      \item This is better for dense graphs ($e \rightarrow v^2$), but
        for sparse graphs does not make a difference;
      \end{itemize}
    \end{block}
  }
\end{frame}

\subsection{Outline}
\begin{frame}
  \frametitle{Math problems in programming Competitions} {\smaller
    \structure{Math problems} have a wide variety of forms, just like
    Graph problems. However, unlike graph problems, \structure{the
      programming part is easy}, and \alert{the formulation is hard}.

    \bigskip

    A sample of math topics in programming challenges:
    \begin{itemize}
    \item \structure{Ad-hoc:} Simulation, Probability;
    \item \structure{Big Num:} Simple problems with $n > 1000000000000000000000$      
    \item \structure{Number Theory:} Primality, Divisibility, Modulo arithmetic;
    \item \structure{Combinatorics:} Counting, closed forms, recurrences;
    \end{itemize}
    
    \bigskip

    In this lecture, we will just scratch the surface, focusing on
    examples. Experience is the best teacher!
  }
\end{frame}

\section{Ad Hoc}
\subsection{Intro}
\begin{frame}
  \frametitle{Ad-hoc Maths Problems}

  \begin{block}{What is ad-hoc?}
    \structure{ad-hoc} means ``single purpose''. In other words, you 
    need to improvise a solution useful for only one (or few) problems.

    Ad-hoc problems u
    

  % What is Ad-hoc? -- No rule, improvisation

  % Ad-hoc problems usually require just a little bit of elementary 
  % Maths and a little bit of programming. Common sub categories are:

  \end{block}


  % Simulation (Brute Force): execute an simple formula on a set of numbers,
  % for example, how many of the elements in S are smaller than X?

  % Finding patterns or formulas: Unlike simulation problems, trying every
  % Number in this one will give you TLE. (Usually they have huge inputs)
  % In this case, you need to find a formula to solve. For example, 
  % Set S has all square numbers (1,4,9,16). For an integer X (X < 10^17),
  % How many numbers in S are smaller than X? (answer: SQRT(x-1))

  % Number Systems or sequences: The ``opposite'' of the previous one.
  % Given a series of numbers x1,x2...xi, calculate x_n for a very big n

  % Logarithmic, Exponentiation, Base

  % Polinomials (sum/multiplication/division of polynomials)
\end{frame}

\begin{frame}
  \frametitle{Some Programming Hints}

  Calculating Logb(a): import cmath, log(a)/log(b)

  Counting Digits: (int) floor(1+log10((double)a))

  Square root n of a: pow((double) a, 1/(double) n)
\end{frame}

\subsection{Example Problem} 
\begin{frame}
  \frametitle{Example Problem: Probability}

  Throw x dice, what is the chance of result > m?

  Chance = \#desired outcomes / \#all outcomes

  This can be solved by DP!

  (roundoff errors, simplifying fractions)

\end{frame}


\section{BigNum}
\subsection{Bignum}
\begin{frame}
  \frametitle{Dealing with big numbers}

  C++ unsigned int = unsigned long = $2^{32}$ (9-10 digits)
  C++ unsigned long long: $2^{64}$ (19-20 digits)
  Factorial of 21: $>$ 20 digits

  For math problems with relatively small $n$, we need bignum;

  The ``DIY'' approach is to transform the numbers in strings, and
  implement digit by digit operators. The programmer efficient
  approach is to use the JAVA BigInteger class
\end{frame}

\begin{frame}
  \frametitle{Java Bignum Class}
  BigInteger.add(BI)
  BigInteger.subtract(BI)
  BigInteger.multiply(BI)
  BigInteger.divide(BI)
  BigInteger.pow(int)
  BigInteger.mod(BI)
  BigInteger.remainder(BI)
  BigInteger.divideAndRemainder(BI)
\end{frame}

\begin{frame}[fragile,singleslide]
  \frametitle{Problem Example - 10925 - Krakovia}
  \begin{block}{Problem Description}
  \end{block}
  \begin{exampleblock}{}
\begin{verbatim}
\end{verbatim}
  \end{exampleblock}
\end{frame}

\begin{frame}
  \frametitle{Java BigInteger superpowers}

  .toString(int radix) -- converts base

  .isProbablePrime(int certainty) -- probabilistic primality test
  $1-\frac{1}{2}^{\text{certainty}}$ (for ``small'' primes, we can use
  the Sieve of Erasthothenes which we will see later)

  .gcd(BI)

  .modPow(BI exponent, BI m)  
\end{frame}



\section{Number Theory}
\begin{frame}
  \frametitle{Number Theory} 
  The field of Number theory studies the properties of integers and
  sets. Some problems in field include primality and modular arithmetic.

  An understanding of number theory is important to avoid brute 
  force attacks to certain problems, or to pre-process data in large
  problems.
\end{frame}

\subsection{Prime Numbers}
\begin{frame}
  \frametitle{Number Theory: Primality} 
  
  Prime numbers are numbers >= 1 that are only divisible by 1 and
  themselves. There is a huge use for prime numbers, including 
  cryptography.

  Naive calculation of prime number: For i in 1:N, test i//N

  Better calculation of prime number: For i in 1:sqrtN, test i//N

  Even better calculation of prime number: For i in primes[1:sqrtN] test i//N

  Can we calculate the primes[1:sqrtN] fast? pi(x) = x/logx (prime number theorem)

  \hfill {\tiny \url{https://primes.utm.edu/howmany.html}}
\end{frame}


\begin{frame}[fragile,singleslide]
  \frametitle{Sieve of Eratosthenes}

  Idea

  Code
  
  Complexity of primality testing: O(NloglogN)
\end{frame}

\begin{frame}
  \frametitle{Finding Prime Factors}

  Prime factors: is the list of all primes that divide N

  Naive approach: If you have a list of primes, try to divide each
  prime in that list by N

  Divide and conquer approach: Find the smallest prime factor i of
  N. Now find the smallest prime factor i' of N//i, recursively.

  This is slow if the number has few, big prime factors, but very fast
  if the number has many, small primes. This is the principle of our
  cryptography!
\end{frame}

\begin{frame}
  \frametitle{Working with Prime Factors}

  Prime factors are the building blocks of all integers. Therefore,
  sometimes we can work with prime factors instead of working with
  huge integers.


  Example: Problem 10139: Factorvision

  Problem description: Does $m$ divides $n!$? (m,n $< 2^{31}-1$).
  Factor of 22 is already bignum, we cannot work with the factors directly!


  Solution: Calculate the prime factors of $m$ and $n!$, and see if $n!$ contains $m$ 

\end{frame}

\subsection{Modulo Operations}
\begin{frame}
  \frametitle{Modulo Operation} 

  {\smaller
  We can use \structure{modulo arithmetic} to operate on very large
  numbers without calculating the entire number.

  Remember that:
  \begin{enumerate}
  \item $(a+b)\%s = ((a\%s)+(b\%s)+s)\%s$
  \item $(a*b)\%s = ((a\%s)*(b\%s))\%s$
  \item $(a^n)\%s = ((a^{n/2}\%s)*(a^{n/2}\%s)*(a^{n\%2}\%s))\%s$
  \end{enumerate}
  
  \begin{block}{UVA 10176 -- Ocean Deep!}
    Test if a 100-digit binary is divisible by 131071.

    \medskip

    We can calculate the module of each (binary) digit using the
    recurrence in (3), and module sum the result without ever touching
    the entire $2^{100}$ number.
  \end{block}}
\end{frame}

\subsection{GCD/LCM}
\begin{frame}[fragile, singleframe]
  \frametitle{Euclid Algorithm and Extended Euclid Algorithm}

  {\smaller  
    \begin{itemize}
    \item The \structure{Euclid Algorithm} gives us the maximum common divisor $D$ of $a$ and $b$;
    \item The \structure{Extended Euclid Algorithm} also gives us $x,y$ so that $ax+by = D$;
    \item Both are extremely simple to code:
    \end{itemize}
    
    \begin{exampleblock}{}
\begin{verbatim}
int gcd(int a, int b) {return (a == 0?b:gcd(b%a,a));}
int x, y;
int egcd(int a, int b) { 
   if (a==0) {x = 0; y = 1; return b;} // stop condition
   int d = egcd(b%a, a); int tx = x; // gcd recurrence
   x = y - (b/a)*tx; y = tx; return d; } // update x,y
\end{verbatim}
    \end{exampleblock}
}
\end{frame}

\begin{frame}
  \frametitle{Using EGCD: The Diophantine Equation}
  {\smaller
    \begin{block}{Problem Example (variations of this problem are common)}
      You have 839 yen. \alert{X}hoco candy costs 25 yen,
      \alert{Y}anilla candy costs 18 yen. How many candies can we buy?
    \end{block}

    \bigskip

    The equation $xA+yB=C$ is called the \structure{Linear Diophantine
      Equation}. It has infinite solutions if GCD(A,B)|C, but none if
    it does not.

    \bigskip   

    The first solution ($x_0,y_0$) can be derived from the extended
    GCD, and other solutons can be found from:
    expressed as:
    \begin{itemize}
    \item $x = x_0 + (b/d)n$
    \item $y = y_0 - (a/d)n$
    \end{itemize}
    Where $d$ is GCD(A,B) and $n$ is an integer.
  }
\end{frame}

\begin{frame}
  \frametitle{Using EGCD: The Diophantine Equation}
  {\smaller
    \begin{block}{Problem Example (variations of this problem are common)}
      You have 839 yen. \alert{X}hoco candy costs 25 yen,
      \alert{Y}anilla candy costs 18 yen. How many candies can we buy?
    \end{block}
   
    \begin{itemize}
    \item \structure{EGCD} gives us: $x=-5, y=7, d=1$ or $25(-5)+18(7) = 1$
    \item Multiply both sides by 839: $25(-4195)+18(5873) = 839$
    \item So: $x_n = -4195 + 18n$ and $y_n = 5873 - 25n$
    \item We have to find $n$ so that both $x_n,y_n$ are $> 0$.
    \item $-4195 + 18n \geq 0$ and $5873 - 25n \geq 0$
    \item $n \geq 4195/18$ and $5873/25 \geq n$
    \item $4195/18 \leq n \leq 5873/25$
    \item $233.05 \leq n \leq 234.92$
    \end{itemize} 
  }
\end{frame}


\section{Combinatorics}
\subsection{Counting and Closed Forms}
\begin{frame}
  \frametitle{Combinatorics problems} 

  Combinatorics is the branch of mathematics concerning the study of
  \structure{countable discrete structures}.

  Combinatory problems usually involving finding out the
  \structure{recurrence} of a set (recursive function that construct
  the set) or the \structure{closed form} of a set (formula that
  calculates the nth element of the set)

  Depending on the recurrence or formula, the need of bignums is not
  uncommon either. Also, often the recurrence can be sped-up by using
  DP.

  Let's see some examples.  
\end{frame}

\subsection{Specific Sequences}
\begin{frame}
  \frametitle{Fibonacci Numbers}
  
  fib(0) = 0, fib(1) = 1, fib(n>2) = fib(n-1)+fib(n-2)

  Usually this is implemented based on a O(n) table, as a complete recursion is very slow.

  However, there is an approximation that is O(1): calculate the
  closest integer to $\frac{(phi^n - (-phi)^{-n})}{\sqrt{5}}$ where phi is the golden ratio
  $\frac{1+\sqrt{5}}{2}$ (this is not accurate for very large fibonacci numbers).
\end{frame}

\begin{frame}
  \frametitle{Funny Fibonacci}
% Zeckendorf's theorem: every positive integer can be written in a unique way as a sum of 
% one or more distinc fibonacci numbers which are not consecutive.(can find this by greedy form)

% Pisano period: the last one/two/three/four digits repeat with a period of 60/300/1500/15000
% What are the last two digits of the 1211313628th fibonacci number?
\end{frame}

\begin{frame}
  \frametitle{Binomial Coefficients}
  $^nC_k$ the number of ways that $n$ items can be selected $k$ at a time, also the coefficients of 
  the $(x+y)^n$ polinomial.

  A single value of the binomial can be calculated as
  $\frac{n!}{(n-k)!k!}$. This formula is problematic, however. Not
  only factorials are very large for small $n$, multiplyign factorials
  have a good chance of exploding.

  Some tricks to avoid using bignum
  * If k > n-k. then exchange k and n-k.
  * Try to divide before multiplyign
  * top down dynamic programming: C(n,0) = C(n,n) = 1
  C(n,k) = C(n-1,k-1) + C(n-1,k). 

  In this case, the recursive formula is more useful than the closed form.
\end{frame}

\begin{frame}
  \frametitle{Catalan Numbers}

  Cat(n) = 1, 1, 2, 5, 14, 42, 132, 429, 1430

  They are calculated as: 
  Cat(n) = $\frac{(^{2n}C_n)}{(n+1)}$, Cat(0) = 1

  Or

  Cat(n+1) = $\frac{(2n+2)(2n+1)}{(n+2)(n+1)}$Cat(n)

\end{frame}

\begin{frame}
  \frametitle{Catalan Numbers -- Uses}
  \begin{itemize}
    \item Number of ways that you can match $n$ parenthesis.
    \item Number of ways that you can triangulate a poligon with $n+2$ sides
    \item Number of monotonic paths on an $nxn$ grid that do not pass above
      the diagonal.
    \item Number of distinct binary trees with $n$ vertices
    \item Etc...
  \end{itemize}
\end{frame}


\section{Conclusion}
\subsection{Conclusion}
\begin{frame}
  \frametitle{Class Summary}
  \begin{itemize}
  \item Math Problems!
  \end{itemize}
\end{frame}

\begin{frame}
  \frametitle{This Week's Problems}
  {\smaller
  \begin{itemize}
  \item UVA Problems
  \end{itemize}}
\end{frame}

\begin{frame}
  \frametitle{To Learn More}
  Euler Project
\end{frame}

\end{document}

