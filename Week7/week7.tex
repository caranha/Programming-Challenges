\documentclass{beamer}

\usepackage{amssymb,amsmath}
\usepackage{graphicx}
\usepackage{url}
\usepackage{color}
\usepackage{relsize}		% For \smaller
\usepackage{url}			% For \url
\usepackage{epstopdf}	% Included EPS files automatically converted to PDF to include with pdflatex

%For MindMaps
% \usepackage{tikz}%
% \usetikzlibrary{mindmap,trees,arrows}%

%%% Color Definitions %%%%%%%%%%%%%%%%%%%%%%%%%%%%%%%%%%%%%%%%%%%%%%%%%%%%%%%%%
%\definecolor{bordercol}{RGB}{40,40,40}
%\definecolor{headercol1}{RGB}{186,215,230}
%\definecolor{headercol2}{RGB}{80,80,80}
%\definecolor{headerfontcol}{RGB}{0,0,0}
%\definecolor{boxcolor}{RGB}{186,215,230}

%%% Save space in lists. Use this after the opening of the list %%%%%%%%%%%%%%%%
%\newcommand{\compresslist}{
%	\setlength{\itemsep}{1pt}
%	\setlength{\parskip}{0pt}
%	\setlength{\parsep}{0pt}
%}

%\setbeameroption{show notes on top}

% You should run 'pdflatex' TWICE, because of TOC issues.

% Rename this file.  A common temptation for first-time slide makers
% is to name it something like ``my_talk.tex'' or
% ``john_doe_talk.tex'' or even ``discrete_math_seminar_talk.tex''.
% You really won't like any of these titles the second time you give a
% talk.  Try naming your tex file something more descriptive, like
% ``riemann_hypothesis_short_proof_talk.tex''.  Even better (in case
% you recycle 99% of a talk, but still want to change a little, and
% retain copies of each), how about
% ``riemann_hypothesis_short_proof_MIT-Colloquium.2000-01-01.tex''?

\mode<presentation>
{
  % A tip: pick a theme you like first, and THEN modify the color theme, and then add math content.
  % Warsaw is the theme selected by default in Beamer's installation sample files.

  %%%%%%%%%%%%%%%%%%%%%%%%%%%% THEME
  %\usetheme{AnnArbor}
  %\usetheme{Antibes}
  %\usetheme{Bergen}
  %\usetheme{Berkeley}		% bem bacana - menu esquerdo
  %\usetheme{Berlin}
  %\usetheme{Boadilla}
  %\usetheme{boxes}
  %\usetheme{CambridgeUS}		% bem bacana - menu superior
  %\usetheme{Copenhagen}
  %\usetheme{Darmstadt}
  %\usetheme{default}
  %\usetheme{Dresden}
  \usetheme{Frankfurt}
  %\usetheme{Goettingen}
  %\usetheme{Hannover}		% bem bacana - menu esquerdo
  %\usetheme{Ilmenau}
  %\usetheme{JuanLesPins}
  %\usetheme{Luebeck}
  %\usetheme{Madrid}		%bacana
  %\usetheme{Malmoe}
  %\usetheme{Marburg}		% bem bacana - menu direito
  %\usetheme{Montpellier}
  %\usetheme{PaloAlto}		% bem bacana - menu esquerdo
  %\usetheme{Pittsburgh}
  %\usetheme{Rochester}		%bacana
  %\usetheme{Singapore}
  %\usetheme{Szeged}
  %\usetheme{Warsaw}

  %%%%%%%%%%%%%%%%%%%%%%%%%%%% COLOR THEME
  %\usecolortheme{albatross}		% azul escuro, massa
  %\usecolortheme{beetle}		% cinza, menu azul
  %\usecolortheme{crane}		% branco e amarelo, massa
  \usecolortheme{default}		% branco, azul clarinho
  %\usecolortheme{dolphin}		% azul e branco, legal
  %\usecolortheme{dove}			% cinza e branco, feio
  %\usecolortheme{fly}			% todo cinza, horrível
  %\usecolortheme{lily}			% parece o default
  %\usecolortheme{orchid}		% azul e branco, ok
  %\usecolortheme{rose}			% branco e violeta-claro, bonito
  %\usecolortheme{seagull}		% cinza, feio
  %\usecolortheme{seahorse}		% nhé, meio feio
  %\usecolortheme{sidebartab}		% Azul, branco, destaque na tab, interessante
  %\usecolortheme{structure}		% bichado
  %\usecolortheme{whale}		% Azul e branco, bem bonito

  %%%%%%%%%%%%%%%%%%%%%%%%%%%% OUTER THEME
  \useoutertheme{default}
  %\useoutertheme{infolines}
  %\useoutertheme{miniframes}
  %\useoutertheme{shadow}
  %\useoutertheme{sidebar}
  %\useoutertheme{smoothbars}
  %\useoutertheme{smoothtree}
  %\useoutertheme{split}
  %\useoutertheme{tree}

  %%%%%%%%%%%%%%%%%%%%%%%%%%%% INNER THEME
  \useinnertheme{circles}
  %\useinnertheme{default}
  %\useinnertheme{inmargin}
  %\useinnertheme{rectangles}
  %\useinnertheme{rounded}

  %%%%%%%%%%%%%%%%%%%%%%%%%%%%%%%%%%%

  \setbeamercovered{invisible} % or whatever (possibly just delete it)
  % To change behavior of \uncover from graying out to totally
  % invisible, can change \setbeamercovered to invisible instead of
  % transparent. apparently there are also 'dynamic' modes that make
  % the amount of graying depend on how long it'll take until the
  % thing is uncovered.

}


% Get rid of nav bar
\beamertemplatenavigationsymbolsempty

% Use short top
%\usepackage[headheight=12pt,footheight=12pt]{beamerthemeboxes}
%\addheadboxtemplate{\color{black}}{
%\hskip0.5cm
%\color{white}
%\insertshortauthor \ \ \ \ 
%\insertframenumber \ \ \ \ \ \ \ 
%\insertsection \ \ \ \ \ \ \ \ \ \ \ \ \ \ \ \ \  \insertsubsection
%\hskip0.5cm}
%\addheadboxtemplate{\color{black}}{
%\color{white}
%\ \ \ \ 
%\insertsection
%}
%\addheadboxtemplate{\color{black}}{
%\color{white}
%\ \ \ \ 
%\insertsubsection
%}

% Insert frame number at bottom of the page.
% \usefoottemplate{\hfil\tiny{\color{black!90}\insertframenumber}} 

\usepackage[english]{babel}
\usepackage[latin1]{inputenc}
\usepackage{subfigure}

\usepackage{times}
\usepackage[T1]{fontenc}

\usepackage{tikz}
\usetikzlibrary{arrows,shapes}
% Latex Graph Example:
% https://www.overleaf.com/5297501zrjzfm#/16716638/

% TODO: Silly Makefile

\tikzstyle{vertex}=[circle,fill=black!25,minimum size=10pt,inner sep=0pt]
\tikzstyle{blue vertex}=[circle,fill=blue!100,minimum size=10pt,inner sep=0pt]
\tikzstyle{red vertex}=[circle,fill=red!100,minimum size=10pt,inner sep=0pt]
\tikzstyle{edge} = [draw,thick,-]
\tikzstyle{red edge} = [draw, line width=5pt,-,red!50]
\tikzstyle{black edge} = [draw, line width=5pt,-,black!20]
\tikzstyle{weight} = [font=\smaller]

\title[GB21802]{GB21802 - Programming Challenges}
\subtitle[]{Week 7 - Math Problems}
\author[Claus Aranha]{Claus Aranha\\{\footnotesize caranha@cs.tsukuba.ac.jp}}
\institute{College of Information Science}
\date{2015-06-10,13\\{\tiny Last updated \today}}

\begin{document}

\section{Introduction}
\subsection{Title}
\begin{frame}
\maketitle
\end{frame}

\subsection{Notes and Warnings}

\begin{frame}
  \frametitle{Last Week Results}
  \begin{block}{Week 6 - Graph II}
    {\smaller
      \begin{columns}[T]
        \column{0.5\textwidth}
        \begin{itemize}
        \item From Dusk Until Dawn -- 4/31
        \item Wormholes -- 18/31
        \item Mice and Maze -- 11/31
        \item Degrees of Separation -- 9/31
        \item Avoiding your Boss -- 5/31
        \item Arbitrage -- 3/31
        \item Software Allocation -- 4/31
        \item Sabotage -- 1/31
        \item Little Red Riding Hood -- 9/31
        \item Gopher II -- 4/31
        \end{itemize}
        \column{0.5\textwidth}
        \begin{itemize}
        \item 11 people: 0 problems;
        \item 8 people: 1-2 problems;
        \item 7 people: 3-4 problems;
        \item 3 people: 5-6 problems;
        \item 2 people: 7+ problems!
        \end{itemize}
      \end{columns}
    }
  \end{block}
\end{frame}

\begin{frame}
  \frametitle{Special Notes}

  {\small
    \begin{block}{ASPS for single weighted graphs}

      Apparently, this is still an open problem!

      \begin{itemize}
      \item $v$ times BFS: $O(ve+v^2)$
      \item A paper (2009) claims $O(v^2\text{log}v)$:\\
        {\tiny 
        \url{http://waset.org/publications/8870/all-pairs-shortest-paths-problem-for-unweighted-graphs-in-o-n2-log-n-time}}\\
        (pseudocode included!)
      \item This is better for dense graphs ($e \rightarrow v^2$), but
        for sparse graphs does not make a difference;
      \end{itemize}
    \end{block}
  }
\end{frame}

\subsection{Outline}
\begin{frame}
  \frametitle{Math problems in programming Competitions} {\smaller
    \structure{Math problems} have a wide variety of forms, just like
    Graph problems. However, unlike graph problems, \structure{the
      programming part is easy}, and \alert{the formulation is hard}.

    \bigskip

    A sample of math topics in programming challenges:
    \begin{itemize}
    \item \structure{Ad-hoc:} Simulation, Probability;
    \item \structure{Big Num:} Simple problems with $n > 1000000000000000000000$      
    \item \structure{Number Theory:} Primality, Divisibility, Modulo arithmetic;
    \item \structure{Combinatorics:} Counting, closed forms, recurrences;
    \end{itemize}
    
    \bigskip

    In this lecture, we will just scratch the surface, focusing on
    examples. Experience is the best teacher!
  }
\end{frame}

\section{Ad Hoc}
\subsection{Intro}
\begin{frame}
  \frametitle{Ad-hoc Maths Problems}

  {\small
  \begin{block}{What is ad-hoc?}
    \structure{ad-hoc} means ``single purpose''. In other words, you 
    need to improvise a solution useful for only one (or few) problems.

    \bigskip

    Ad-hoc problems usually require only a bit of elementary maths and
    a bit of programming;
  \end{block}

  Common categories:
  \begin{itemize}
  \item \structure{Simulation (Brute Force)}: Execute one or more simple formulas
    on a set of numbers and report the result;
  \item \structure{Finding patterns/formulas}: Similar to simulation, but trying
    to execute it directly will result in TLE. You have to find a
    function that summarizes the formula;
  \end{itemize}}
\end{frame}

\begin{frame}
  \frametitle{Ad Hoc example: Probability problems}

  {\smaller 
    ``What is the chance of X happening'' is a common ad-hoc
    problem. The general formula is ``prob = events of interest /
    total events'';
    
    \medskip

    \begin{block}{The dice problem}
      If I have $n$ dice, what is the chance of rolling a total above $m$?
    \end{block}

    \begin{itemize}
    \item \structure{Example:} For $n=3$ dice, and $m=16$, the chance is $10/216$
    \end{itemize}
    \begin{columns}
      \column{0.3\textwidth}
      \begin{itemize}
      \item 6,6,6
      \item 6,6,5
      \item 6,5,6
      \item 5,6,6
      \end{itemize}
      \column{0.3\textwidth}
      \begin{itemize}
      \item 6,5,5
      \item 5,6,5
      \item 5,5,6
      \end{itemize}
      \column{0.3\textwidth}
      \begin{itemize}
      \item 4,6,6
      \item 6,4,6
      \item 6,6,4
      \end{itemize}
    \end{columns}


  }
\end{frame}

\begin{frame}
  \frametitle{Ad Hoc example: Probabilty Problems}

  {\smaller
  \begin{block}{The dice problem}
    If I have $n$ dice, what is the chance of rolling a total above $m$?
  \end{block}
 
  \begin{itemize}
  \item For $n=0$, we have only one result: $r=0$
  \item For $n=1$, we have 6 results: $r = \{1,2,3,4,5,6\}$
  \item The result for $n=i$ and $r_{n-1}=k$ is $r_n = k + \{1,2,3,4,5,6\}$ 

    \bigskip

  \item Can you start to see a bottom up DP here?
  \item With a state table (dice,result), we can count the number of
    dice combination above a certain value;

  \end{itemize}
  }
\end{frame}

\begin{frame}[fragile,simpleslide]
  \frametitle{Ad Hoc example: Probability Problems}
  \begin{exampleblock}{Example Code}
{\small
\begin{verbatim}
int count(int dice_left, int score_left) {
   if (score_left < 1) return 1;
   if (dice_left == 0) return 0;
   if (result[dice_left][score_left] != -1)
      return result[dice_left][score_left];       
   int sum = 0;
   for (int i = 0; i < 6; i++)
      sum += count(dice_left-1, score_left-(i+1))
   result[dice_left][score_left] = sum;
   return sum;
}

prob = count(n,m)/6**n;

\end{verbatim}
}
  \end{exampleblock}
\end{frame}


\begin{frame}
  \frametitle{Other hints for ad hoc problems}


  \begin{itemize}
  \item Calculating $\text{log}_ba$: \#import<cmath>; log(a)/log(b);

    \medskip

  \item Counting Digits: (int) floor(1+log10((double)a));
    
    \medskip

  \item $n^{th}$ root of $a$: pow((double) a, 1/(double) n);
  \end{itemize}

\end{frame}


\section{BigNum}
\subsection{Bignum}
\begin{frame}
  \frametitle{Dealing with big numbers}

  {\smaller
  \begin{block}{}
    Many math problems require the handling of numbers much bigger
    than C++'s long long.
  \end{block}

  \begin{itemize}
  \item C++ unsigned int = unsigned long = $2^{32}$ (9-10 digits)
  \item C++ unsigned long long = $2^{64}$ (19-20 digits)

    \medskip

  \item Factorial of 21: $> 20$ digits!
  \end{itemize}

  \begin{block}{}
    The ``DIY'' approach is to transform big numbers into strings, and
    implement digit operations.
    
    \bigskip

    The \structure{programmer efficient} approach is to use the JAVA
    BigInteger class.    
  \end{block}  
  }
\end{frame}

\begin{frame}[fragile,singleslide]
  \frametitle{Java BigInteger Class}

  {\smaller
  \begin{block}{}
    Java's \structure{BigInteger} class handles arbitrarily big
    numbers: Input, Output, and many operations.
  \end{block}

  \begin{exampleblock}{}
\begin{verbatim}
import java.util.Scanner;
import java.math.BigInteger;
class Main {
    public static void main(String[] args) {
    Scanner sc = new Scanner(System.in);
    int caseNo = 1;
    while (true) {
    int N = sc.nextInt(), F = sc.nextInt();
    if (N == 0 && F == 0) break;
    BigInteger sum = BigInteger.ZERO; 
        for (int i = 0; i < N; i++) {
            BigInteger V = sc.nextBigInteger(); 
            sum = sum.add(V);
}}}}
\end{verbatim}
\end{exampleblock}}
\end{frame}

\begin{frame}
  \frametitle{Java BigInteger algebraic functions}

  \begin{itemize}
    \item BigInteger.add(BI): \hfill $A+BI$
    \item BigInteger.subtract(BI): \hfill $A-BI$
    \item BigInteger.multiply(BI): \hfill $A*BI$
    \item BigInteger.divide(BI): \hfill $A/BI$
    \item BigInteger.pow(int): \hfill $A^{BI}$
    \item BigInteger.mod(BI): \hfill $A\%BI$
    \item BigInteger.remainder(BI): \hfill $A - A//BI$
    \item BigInteger.divideAndRemainder(BI): \hfill $(A//BI,A-A//BI)$
  \end{itemize}
\end{frame}

\begin{frame}
  \frametitle{Problem Example - 10925 - Krakovia}
  \begin{block}{Problem Description}
    Given a number of costs, and a number of friends,
    divide the total cost among all friends.
  \end{block}

  \bigskip

  \begin{itemize}
  \item Super simple problem, but requires BigInteger;
  \item Use this problem to familiarize yourself with writing JAVA
    code;
  \end{itemize}
\end{frame}

\begin{frame}
  \frametitle{Java BigInteger superpowers}

  \begin{block}{}
    The class BigInteger implements many other functions that 
    might be useful in mathematical problems:
  \end{block}


  \begin{itemize}
  \item .toString(int radix) -- Outputs the Big Integer, can also be
    used to convert base

  \item .isProbablePrime(int certainty) -- probabilistic primality
    test. The chance of the test being correct is $1-\frac{1}{2}^{\text{certainty}}$\\
    \medskip

  \item .gcd(BI)
  \item .modPow(BI exponent, BI m)  
  \end{itemize}
\end{frame}



\section{Number Theory}
\subsection{outline}
\begin{frame}
  \frametitle{Number Theory} {\small 

    The field of Number theory studies the properties of integers and
    sets. Some problems in field include \structure{primality} and
    \structure{modular arithmetic}.

    \bigskip
    
    An understanding of number theory is important to avoid brute
    force attacks to certain problems, or to pre-process data in large
    problems.}
\end{frame}

\subsection{Prime Numbers}
\begin{frame}
  \frametitle{Number Theory: Primality} 
  {\smaller
  Prime numbers are numbers >= 1 that are only divisible by 1 and
  themselves. There is a huge use for prime numbers, including 
  cryptography.

  \begin{itemize}
    
  \item Naive calculation of prime number: For each $i \in 1..N$, test i|N
    
  \item Better calculation of prime number: For each $i \in 1..\sqrt{N}$, test i|N
        
  \item Even better calculation: For each \alert{i is prime} $\in 1..\sqrt{N}$, test i|N\\

    \medskip

    number of primes up to x: $\pi(x) =$ x/logx (prime number theorem)
    
    \bigskip

    \begin{block}{}
    How can we calculate all the primes between 1 and $\sqrt{N}$ fast? 
    \end{block}

  \end{itemize}}

  \hfill {\tiny \url{https://primes.utm.edu/howmany.html}}
\end{frame}


\begin{frame}[fragile,singleslide]
  \frametitle{Sieve of Eratosthenes}

  {\smaller
  \begin{block}{Main Idea}
    Start with a set of numbers (possible primes) between 2 and $\sqrt{N}$.\\
    For each confirmed prime, remove all multiples of that number from the list.
  \end{block}

  \begin{exampleblock}{}
\begin{verbatim}
def sieve(k):                 ## Find all primes up to k
   primes = []
   sieve = [1]*(k+1)    ## all numbers start in the list
   sieve[0] = sieve[1] = 0             ## except 0 and 1
   for i in range(k+1):                          ## O(N)
      if (sieve[i] == 1):
         primes.append(i)             ## new prime found
         j = i*i   ## why can i start from i*i, not i*2?
         while (j < k+1):                  ## O(loglogN)
            sieve[j] = 0
            j += i                      ## next multiple
   return primes
\end{verbatim}
  \end{exampleblock}
  
  Complexity the sieve: $O(N\text{loglog}N)$ (Also, this is a kind of DP!)
  }
\end{frame}

% Prime number: Sosuu
% Prime factors: Soinsuu
% Factorization: Insuu Bunkai
\begin{frame}
  \frametitle{Finding Prime Factors}
  
  {\smaller 

    Any natural number $N$ can be expressed as a \structure{unique}
    set of prime numbers: 

    \begin{equation*}
      N=1p_1^{e_1}p_2^{e_2}\ldots p_n^{e_n}
    \end{equation*}

    These are the \structure{Prime Factors} of $N$. From this set, we
    can also obtain the set of \structure{Factors} of $N$ (all numbers
    $i$ such as $i|N$.

    \medskip
    
    Factorization is a key issue in \structure{cryptography}

    \begin{block}{Very Naive approach}
      For every $i \in 1..\sqrt{N}$, test $i|N$ and isPrime(i).\\
      \smallskip
      \hfill Very Costly!
    \end{block}

    \begin{block}{Naive approach}
      Calculate a list of primes, for each prime $i$, test if $i|N$.\\
      \smallskip
      \hfill Will not find all non-prime factors!
    \end{block}
}
\end{frame}

\begin{frame}[fragile,singleslide]
  \frametitle{Prime factorization: Divide and conquer approach}
  
  {\smaller
  \begin{block}{Idea}
    The prime factorization of $N$ is equal to the union of $p_i$ and
    the prime factorization of $N/p_i$, where $p_i$ is the smallest
    prime factor of $N$.

    \bigskip

    The set of all factors is composed of all combinations of the set
    of prime factors (including repetitions).  
  \end{block}

  \begin{exampleblock}{}
\begin{verbatim}
def primefactors(n):
   primes = sieve(int(np.sqrt(n))+1)
   c = 0, i = n, factors = []
   while i > 1:
      if (i%primes[c] == 0):
          i = i/primes[c]
          factors.append(primes[c])
      else:
          c = c+1
   return factors
\end{verbatim}
  \end{exampleblock}}
\end{frame}


\begin{frame}
  \frametitle{Working with Prime Factors: 10139 -- Factovisors}

  {\smaller
    \begin{block}{Problem description}
      Calculate whether $m$ divides $n!$ ($1 \leq m,n \leq 2^{31}-1$)
    \end{block}

    Factorial of 22 is already bigint! But we can break down these numbers into their 
    factors, which are all $\leq 2^{30}$.

    \begin{itemize}
    \item $F_m$: primefactors(m)
    \item $F_{n!}$: $\cup$(primefactors(1), primefactors(2)...,primefactors(n))
    \end{itemize}

    Having the factor sets, $m$ divides $n!$ if $F_m \subset F_{n!}$.

    \bigskip

    Examples:
    \begin{itemize}
    \item $m = 48$ and $n=6$\\
      $F_m = \{2,2,2,2,3\} F_{n!} = \{2,3,2,2,5,2,3\}$
  
  \medskip

    \item $m = 25$ and $n = 6$\\
      $F_m = \{5,5\} F_{n!} = \{2,3,2,2,5,2,3\}$

    \end{itemize}
  }
\end{frame}

\subsection{Modulo Operations}
\begin{frame}
  \frametitle{Modulo Operation} 

  {\smaller
  We can use \structure{modulo arithmetic} to operate on very large
  numbers without calculating the entire number.

  \bigskip

  Remember that:
  \begin{enumerate}
  \item $(a+b)\%s = ((a\%s)+(b\%s)+s)\%s$
  \item $(a*b)\%s = ((a\%s)*(b\%s))\%s$
  \item $(a^n)\%s = ((a^{n/2}\%s)*(a^{n/2}\%s)*(a^{n\%2}\%s))\%s$
  \end{enumerate}

  }
\end{frame}

\begin{frame}
  \frametitle{Modulo Operation -- UVA 10176, Ocean Deep!}
  {\smaller
  \begin{block}{Problem summary}
    Test if a binary number $n$ (up to 100000 digits) is divisible by 131071
  \end{block}

  \begin{itemize}
  \item The problem wants to know if $n\%13107 == 0$
  \item But $n$ is too big!

  \item Use the recurrence in the previous slide to break down each
    digit to a reasonable value.
  \end{itemize}}

\end{frame}

% Saidaikōyakusū
\subsection{GCD/LCM}
\begin{frame}[fragile, singleframe]
  \frametitle{Euclid Algorithm and Extended Euclid Algorithm}

  {\smaller  
    \begin{itemize}
    \item \structure{Euclid Algorithm} gives us the greatest common divisor $D$ of $a,b$;
    \item \structure{Extended Euclid Algorithm} also gives us $x,y$ so that $ax+by = D$;
    \item Both are extremely simple to code:
    \end{itemize}
    
    \vfill

    \begin{exampleblock}{}
\begin{verbatim}
int gcd(int a, int b) {return (a == 0?b:gcd(b%a,a));}

int x, y;
int egcd(int a, int b) { 
   if (a==0) 
      {x = 0; y = 1; return b;}        // stop condition
   int d = egcd(b%a, a); 
   int tx = x;                         // gcd recurrence
   x = y - (b/a)*tx; y = tx; return d; }   // update x,y
\end{verbatim}
    \end{exampleblock}
}
\end{frame}

\begin{frame}
  \frametitle{Using EGCD: The Diophantine Equation}
  {\smaller
    \begin{block}{Problem Example (variations of this problem are common)}
      You have 839 yen. \alert{X}hoco candy costs 25 yen,
      \alert{Y}anilla candy costs 18 yen. How many candies can we buy?
    \end{block}

    \bigskip

    The equation $xA+yB=C$ is called the \structure{Linear Diophantine
      Equation}. It has infinite solutions if GCD(A,B)|C, but none if
    it does not.

    \bigskip   

    The first solution ($x_0,y_0$) can be derived from the extended
    GCD, and other solutons can be found from:
    expressed as:
    \begin{itemize}
    \item $x = x_0 + (b/d)n$
    \item $y = y_0 - (a/d)n$
    \end{itemize}
    Where $d$ is GCD(A,B) and $n$ is an integer.
  }
\end{frame}

\begin{frame}
  \frametitle{Using EGCD: The Diophantine Equation}
  {\smaller
    \begin{block}{Problem Example (variations of this problem are common)}
      You have 839 yen. \alert{X}hoco candy costs 25 yen,
      \alert{Y}anilla candy costs 18 yen. How many candies can we buy?
    \end{block}
   
    \begin{itemize}
    \item \structure{EGCD} gives us: $x=-5, y=7, d=1$ or $25(-5)+18(7) = 1$
    \item Multiply both sides by 839: $25(-4195)+18(5873) = 839$
    \item So: $x_n = -4195 + 18n$ and $y_n = 5873 - 25n$
    \item We have to find $n$ so that both $x_n,y_n$ are $> 0$.
    \item $-4195 + 18n \geq 0$ and $5873 - 25n \geq 0$
    \item $n \geq 4195/18$ and $5873/25 \geq n$
    \item $4195/18 \leq n \leq 5873/25$
    \item $233.05 \leq n \leq 234.92$
    \end{itemize} 
  }
\end{frame}


\section{Combinatorics}
\subsection{Counting and Closed Forms}
\begin{frame}
  \frametitle{Combinatorics problems} 

  {\smaller
    \begin{block}{Definition}
      Combinatorics is the branch of mathematics concerning the study of
      \structure{countable discrete structures}.
    \end{block}

    Combinatory problems involve understanding a sequence, and
    figuring one of:

    \medskip 

    \begin{itemize}
    \item \structure{Recurrence}: A formula that calculates the
      $n^{th}$ member of a sequence, based on the value of previous members;
      
    \item \structure{Closed form}: A formula that calculates the
      $n^{th}$ member of a sequence independently from other members;
    \end{itemize}

  \bigskip

  It is not uncommon to use \structure{Dynamic Programming} or
  \structure{Bignum} to solve combinatoric related problems.
  }
\end{frame}

\begin{frame}
  \frametitle{Example: Triangular Numbers}
  {\smaller
  \begin{block}{Definition}
    The triangular numbers is the sequence where the $n^{th}$ value is
    composed of the sum of all integers from $1$ to $n$
  \end{block}

  \begin{itemize}
  \item S(1) = 1
  \item S(2) = 1+2 = 3
  \item S(3) = 1+2+3 = 6
  \item $\ldots$
  \item S(7) = 1+2+3+4+5+6+7 = 28
  \end{itemize}
  }

  What are the recurrence and the closed form for this sequence?
\end{frame}

\begin{frame}
  \frametitle{Example: Triangular Numbers}
  {\smaller
    \begin{itemize}
    \item S(1) = 1, S(2) = 3, S(3) = 6
    \end{itemize}

    \begin{block}{Recurrence}
      The recursive form of a sequence:
      \begin{equation*}
        S(n) = S(n-1)+n; S(1) = 1
      \end{equation*}
    \end{block}
    \begin{block}{Closed Form}
      The non-recursive form of a sequence:
      \begin{equation*}
        S(n) = \frac{n(n+1)}{2}
      \end{equation*}
    \end{block}
    \alert{Problem:} Calculate the first triangle number with more
    than 500 factors!  }
\end{frame}

\subsection{Fibonacci}

\begin{frame}
  \frametitle{A more famous sequence: Fibonacci Numbers}

  {\smaller
  \begin{block}{Definition -- very famous sequence}
    Each number is the sum of the two numbers before it.

    \medskip

    F() = 0,1,1,2,3,5,8,13,21,34...    
  \end{block}

  \begin{block}{The recurrence is well known}
  \begin{equation*}
  F(0) = 0, F(1) = 1, F(n) = F(n-1)+F(n-2)
  \end{equation*}
  When implementing the recurrence, don't forget the memoization
  table!
  \end{block}

  \begin{block}{Closed Form}
    The Fibonacci numbers also have a less well known
    \structure{closed form}:
    
    \begin{equation*}
      F(n) = \frac{1}{\sqrt{5}}\left(\left(\frac{1+\sqrt{5}}{2}\right)^n-\left(\frac{1-\sqrt{5}}{2}\right)^n\right)
    \end{equation*}
  \end{block}
    Square roots introduce floating point errors. What is the maximum
    $n$ this can calculate with less than 0.1 error?
  }
\end{frame}

\begin{frame}[fragile,singleslide]
  \frametitle{Fibonacci Facts}
  {\smaller
  \begin{block}{Zeckendorf's theorem}
    Every positive integer can be written in a \structure{unique way} as a sum of
    one or more distinct fibonacci numbers, which are not consecutive.

\begin{verbatim}
def zeckenfy(n):
    fibs = []
    f = greatest fib =< n; fibs.append(f)
    fibs.append(zeckenfy(n-f))
    return fibs
\end{verbatim}
  \end{block}

  \begin{block}{Pisano's period} 
    The last digits of the Fibonacci sequence repeat!

    \medskip

    The last one/two/\alert{three/four} digits repeat with a period of
    60/300/\alert{1500/15000}.

    F(6) = 8\\ 
    F(66) = 27777890035288\\
    F(366) = 1380356705549181797202918793682511 3333650564850089197542855968899086435571688

  \end{block}
  }
\end{frame}

\subsection{Binom}

\begin{frame}[fragile,singleframe]
  \frametitle{Binomial Coefficients}
  {\smaller

    \begin{block}{Definition}
      Binomial Coefficients are the number series that correspond to
      the coefficients of the expansion of a binomial:

      \medskip

      Binom(3) = $(a+b)^3$ = $1a^3 + 3ab^2 + 3ab^2 + b^3$ = $\{1,3,3,1\}$ 

      \medskip

      We are usually interested in the $k^{th}$ coefficient of the
      $n^{th}$ binomial:
      
      \medskip

      $C(n,k) = C(3,2) = \{1,$ \alert{3} $,3,1\} = 3$

    \end{block}

    Pascal's Triangle gives us a good representation of C(n,n):
\begin{verbatim}
0  1  0  0  0  0  0  0  0  0
0  1  1  0  0  0  0  0  0  0
0  1  2  1  0  0  0  0  0  0
0  1  3  3  1  0  0  0  0  0
0  1  4  6  4  1  0  0  0  0
0  1  5  10 10 5  1  0  0  0
0  1  6  15 20 15 6  1  0  0
0  1  7  21 35 35 21 7  1  0
0  1  8  28 56 70 56 28 8  1
\end{verbatim}
  }
\end{frame}

\begin{frame}
  \frametitle{Uses for the Binomial Coefficient}

  The value of $C(n,k)$ tells us how many ways we can choose $n$
  items, $k$ at a time.

  \bigskip

  Some use cases:
  \begin{itemize}
  \item \structure{Probabilities:} What is the probability of winning
    a loto when you choose 5 numbers out of 60? $1/C(60,5)$
  \item \structure{Grids:} How many ways are there to go from the
    bottom left end of a $mn$ grid to the top right, if you can only
    go up and right? $C(m+n,n)$
  \end{itemize}
\end{frame}


\begin{frame}
  \frametitle{Calculating the Binomial Coefficient}
  {\smaller

    \begin{block}{Closed form of C(n,k)}
      \begin{equation*}
        C(n,k) = \frac{n!}{(n-k)!k!}
      \end{equation*}

      \alert{Problem}: Multiplying factorials tends to generate huge numbers
      even for small $n$ and $k$.
    \end{block}

    \begin{block}{Recurrence for C(n,k)}
      \begin{itemize}
      \item C(n,0) = C(n,n) = 1;
      \item C(n,k) = C(n-1,k-1) + C(n-1,k)
      \end{itemize}

      Using a memoization table will cut the calculation time by
      half. In this case, top-down DP will usually be faster than
      bottom-up.
    \end{block}
  }
\end{frame}

\subsection{Catalan}

\begin{frame}
  \frametitle{Another useful sequence: Catalan Numbers}
  {\smaller
  \begin{block}{The Catalan sequence}
    \begin{equation*}
      C(n) = 1, 1, 2, 5, 14, 42, 132, 429, 1430
    \end{equation*}
  \end{block}

  \begin{exampleblock}{The Recurrence}
    \begin{equation*}
    C(n) = \sum^{n-1}_{k=0}C(k)C(n-1-k)
    \end{equation*}
  \end{exampleblock}

  \begin{exampleblock}{Closed Form}
    \begin{equation*}
      C(n) = \frac{1}{n+1}\binom{2n}{n}
    \end{equation*}
  \end{exampleblock}
  }
\end{frame}

\begin{frame}
  \frametitle{Catalan Numbers -- Uses}
  \begin{itemize}
    \item Number of ways that you can match $n$ parenthesis.\\
      C(3):((())),()(()),(())(),()()(),(()())

      \medskip
      
    \item Number of ways that you can triangulate a poligon with $n+2$ sides
    \item Number of monotonic paths on an $nxn$ grid that do not pass above
      the diagonal.
    \item Number of distinct binary trees with $n$ vertices
    \item Etc...
  \end{itemize}
\end{frame}

\subsection{Integer Partition}
\begin{frame}
  \frametitle{Integer Partition}
  \begin{block}{}
    f(5,5) = (5),(4,1),(3,2),(3,1,1),(2,2,1),(2,1,1,1),(1,1,1,1,1)
  \end{block}
  \begin{block}{Definition and calculation}
    $f(n,k)$ -- number of ways that we can sum $n$, using integers
    equal or less than $k$.

    \bigskip

    \structure{Recurrence:}
    \begin{itemize}
    \item $f(n,k) = f(n-k,k) + f(n, k+1)$
    \item $f(1,1) = 1$; $f(n,k) = 0$ if $k > n$ 
    \end{itemize}
  \end{block}
\end{frame}


% TODO expand this, add more sequences
% TODO add a section on HOW to derive closed forms (from maths for CS class)


\section{Problems}
%% TODO: Problem Discussion and example problems

\subsection{Problem Discussion}
\begin{frame}
  \frametitle{Problem Discussion (1)}

  \begin{block}{Division (7)}
    For $a,b,t$, determine whether $\frac{t^a-1}{t^b-1}$ is integer
    with less than 100 digits.

    \bigskip

    \begin{itemize}
    \item What are the special cases?
    \item What language did you use to solve it?
    \end{itemize}
  \end{block}

  \begin{block}{What base is this? (1)}
    Given two numbers $a,b$, determine what pair of bases (1..35) can
    be used to make $a==b$.

    \begin{itemize}
    \item Naive search: Try all bases (35x35)
    \item Are there any special cases?
    \item Can we reduce the number of bases?
    \end{itemize}
  \end{block}
\end{frame}

\begin{frame}
  \frametitle{Problem Discussion (2)}
  \begin{block}{Divisibility of Factors (4)}
    Given $N$ and $D$, how many factors of $N!$ are divisible by $D$?

    \begin{itemize}
    \item We already know how to calculate the set of prime factors.
    \item A factor $f$ of $N!$ is divisible by $D$, if $f$ contains
      all prime factos of $D$.
    \end{itemize}
  \end{block}

  \begin{block}{Triangle Counting (3)}
    If we have $n$ sticks of size $1,2,3,\ldots,n$, how many different
    triangles can we create?

    \begin{itemize}
    \item Triangle inequality: $a+b > c$
    \item How many triangles can we create with $n = 3$? $n = 4$? $n =
      m+1$?
    \end{itemize}
  \end{block}
\end{frame}

\begin{frame}[fragile,singleslide]
  \frametitle{Problem Discussion (3)}
  \begin{block}{Help My Brother (0)}
    Calculate the median of the ``line'' composed of fibonacci numbers:
\begin{verbatim}
0 / 1 / 2 3 / 4 5 6 / 7 8 9 10 11
\end{verbatim}
\begin{itemize}
\item What is the first number of line $n$? What is the last?
\item How to you calculate the median of an unitary sequence?
\item Note: Fibonacci above 60 are VERY BIG, use tables!
\end{itemize}
  \end{block}
  \begin{block}{Marbles (0)}
    \begin{itemize}
    \item Diaphantine equation, but now with a cost associated for $a$ and $b$.
    \item Calculate the feasible range for $n$, then find $n$ with minimal cost.
    \end{itemize}
  \end{block}
\end{frame}

\begin{frame}
  \frametitle{Problem Discussion (4)}
  \begin{block}{Ocean Deep, Make it Shallow! (1)}
    Use modulo operation to break a binary number, and test it for
    divisibility with 131071
  \end{block}
  \begin{block}{Winning Streak (0)}
    Calculate the probability for all the events, and use the formula 
    to calculate the expected length of the winning streak.
  \end{block}
\end{frame}

\section{Conclusion}
\subsection{Conclusion}
\begin{frame}
  \frametitle{Class Summary}
  \begin{itemize}
  \item Math Problems
  \item Java's Big Integer class
  \item Primality
  \item Modulo arithmetic
  \item GCD and Diophantine Equations
  \item Combinatorics
  \end{itemize}

  \begin{block}{}
    Next week: Geometry problems!
  \end{block}
\end{frame}

\begin{frame}
  \frametitle{This Week's Problems}
  {\smaller
  \begin{itemize}
  \item Division
  \item What Base Is This
  \item Divisibility of Factors
  \item Triangle Counting
  \item Help My Brother
  \item Marbles
  \item Ocean Deep!
  \item Winning Streak
  \end{itemize}}
\end{frame}

\begin{frame}
  \frametitle{To Learn More}

  Euler Project: Mathematical questions using computers:

  \url{http://projecteuler.net}
\end{frame}

\end{document}

